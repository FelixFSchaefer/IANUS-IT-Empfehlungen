\chapter{Forschungsmethoden}
	\abschnittsautor{F. Schäfer, M. Trognitz}
	\label{methoden}
Die Ergebnisse aus verschiedenen in den Altertumswissenschaften angewandten Methoden setzen sich oft aus mehreren verschiedenen Dateien unterschiedlicher Formate zusammen. Dadurch gehen die Ansprüche an die Dokumentation und die erforderlichen Metadaten zu den einzelnen Dateiformaten über die des Kapitels zu Dateiformaten hinaus und unterscheiden sich von Methode zu Methode. 

Sofern spezifische Dateiformate eine besondere Relevanz besitzen, sind diese entweder hier beschrieben oder es wird auf die entsprechenden Abschnitte im Kapitel Dateiformate auf Seite \pageref{dateiformate} verwiesen.

Da die Kapitel der einzelnen Forschungsmethoden von verschiedenen Spezialisten verfasst wurden, können sie deshalb und in Abhängigkeit von der jeweils beschriebenen Methode in Inhalt, Länge und Gliederung sehr unterschiedlich ausfallen.

Die vorhandenen Inhalte decken nur einen kleinen Teil der angewandten Forschungsmethoden in den Archäologien und Altertumswissenschaften ab. Weitere Inhalte sind geplant, wobei dafür auch Autoren gesucht werden. Die folgende Auflistung soll einen Eindruck der möglichen Inhalte vermitteln, wobei weitere Vorschläge und Ergänzungen willkommen sind.

3D-Scanning · Anthropologie · Archäobotanik · Archäometrie · Archäozoologie · Ausgrabung · Bauforschung · Datierungsmethoden · Geoarchäologie · Numismatik · Oberflächenbegehung (Survey) · RTI · Textanalyse

Bis wir Ihnen ausführlichere Hinweise zur Verfügung stellen können, können Sie sich auf den Seiten des \href{http://guides.archaeologydataservice.ac.uk/g2gp}{Archaeology Data Service informieren}, wo insbesondere folgende Inhalte zu Forschungsmethoden zur Verfügung stehen.

Dendrochronolgie $\cdot$ Fernerkundung $\cdot$ GIS $\cdot$ Geophysik $\cdot$ Laserscanning $\cdot$ Photogrammetrie $\cdot$ UAV Survey $\cdot$ Unterwassersurvey

\section{Geodäsie}
\abschnittsautor{K. Heine, U. Kapp}
Bitte beachten Sie, dass die Inhalte dieses Abschnittes den Inhalten des IT-Leitfadens des DAIs von 2011 entsprechen.
\begin{center}
\tib{\rule{0.9\textwidth}{0.2mm}}\vspace{3mm}
\end{center}
Alle Protokolle von Messungsrohdaten, mit Ergebnissen der Netzmessungen, der Ausgleichung und Festpunktkoordinaten inkl. Einmessungsskizzen müssen im \gls{ASCII}-Format (\gls{TXT}, \gls{CSV}, DAT) gespeichert werden.
	\input{methoden/geodaesie}

\section{Georeferenzierung}
\abschnittsautor{K. Heine, U. Kapp}
	\input{methoden/georeferenzierung}

\section{Oberflächen- (DOM) und Geländemodellierung (DGM)}
\abschnittsautor{K. Lambers, M. Reindel}
	\input{methoden/oberflaechenmodellierung}

	
\section{Rekonstruktion}
\abschnittsautor{A. Müller, U. Wulf-Rheidt}
	%
% Rekonstruktion
%

%%%%%%%%%%%%%%%%%%%%%%%%%%%%%%%%%%%%%%%%%%%%%%%%%%%%%%%%%%%%%%%%%%%%%%%%%%%%%%%
%%DAI%%ALT%%DAI%%ALT%%ALT%%ALT%%ALT%%ALT%%ALT%%ALT%%ALT%%ALT%%ALT%%ALT%%ALT%%%%
%%%%%%%%%%%%%%%%%%%%%%%%%%%%%%%%%%%%%%%%%%%%%%%%%%%%%%%%%%%%%%%%%%%%%%%%%%%%%%%
Bitte beachten Sie, dass die Inhalte dieses Abschnittes den Inhalten des IT-Leitfadens des DAIs von 2011 entsprechen.
\begin{center}
\tib{\rule{0.9\textwidth}{0.2mm}}\vspace{3mm}
\end{center}

\paragraph{3D-Modelle}
Einfache 3D-Modelle sind relativ schnell angelegt und können zur visuellen Kommunikation einen erheblichen Beitrag leisten. Nach Auswertung erster Forschungs- oder Grabungsergebnisse entstehen oft erste, vereinfachte Massenmodelle, die z. B. der Erläuterung von Rekonstruktionsideen dienen können. Es ist dann kein Problem, die erstellten Volumina in georeferenzierte Systeme, wie beispielsweise Google Earth zu integrieren oder in präzisere Geländemodelle einzubauen. Im Verlaufe des Projektes wird das 3D-Modell in der Regel immer differenzierter ausgearbeitet und es werden mehr Datensätze und Informationen angehängt, in dem einzelne Gebäudeteile und Räume differenziert dargestellt, Bauteile ausgearbeitet, u.U. Bilder, Geodaten, Befund-Daten und Literaturhinweise angehängt und schließlich die zur Visualisierung notwendigen Texturen und Lichtquellen sowie dynamische und statische Schnitte generiert werden. Derart komplexe Aufgaben erfordern eine gut durchdachte Datenverwaltung, da ihr Verwaltungs- und Pflegeaufwand exponentiell zur Datenmenge anwächst. Deshalb ist es wichtig, sich vor Anlegen eines 3D-Modells über die Ziele, die verwirklicht werden sollen, klar zu werden. Wenn z.B. ein Modell nur benötigt wird, um einige Perspektiven für eine Publikation zu visualisieren, genügt es, die Kubatur des Gebäudes zu erstellen, sie zu texturieren, Lichtquellen zu setzen und Ansichtsdefinitionen festzulegen. Dementsprechend gering bleibt der Sicherungs- und Dokumentationsaufwand. Je genauer der Auftraggeber eines 3D-Modells die Ziele, zur Verwendung des Modells definiert, desto geringer kann der Aufwand der Datenerstellung und Datenhaltung gehalten werden. 

Deshalb sind Mindestanforderungen bei der Auftragsvergabe nur auf sehr niedrigem Niveau möglich. Zurzeit beschränken sich diese Forderungen in der Regel auf die Sicherung in einem gut dokumentierten Format wie z.B. dxf oder Wavefront OBJ-Dateien, mit allen Nachteilen, die sich dabei aus einer eventuellen Datenkonvertierung ergeben.

\paragraph{Praxis}
Es ist derzeit verbreitet, die Wahl der Software dem Zufall zu überlassen: Softwarekenntnisse des Bearbeiters, das Vorhandensein von Software oder die Frage, welche Software preiswert zu erwerben ist, bestimmen meist die Wahl.

Das bedeutet zwar nicht, dass eine spätere Entscheidung für eine andere Software zu totalem Datenverlust führen muss, denn grundsätzlich ist die Konvertierung beinahe jedes Zeichenformates in ein anderes möglich. Dies ist jedoch häufig mit erheblichen Informationsverlusten oder/und gewaltigem Arbeits- und Kostenaufwand verbunden.

Wurde ein Modell beispielsweise in einer Visualisierungssoftware wie Rhino, Maya oder 3ds Max erstellt, so haben die Objekte hauptsächlich Metainformationen hinsichtlich Geometrie und Textur. Mit dieser Software sind sehr hochwertige Bilder und Animationen generierbar, die Verwaltung hinsichtlich bautechnisch sinnvoller  Einheiten wie Wände und Fenster ist dagegen nicht vorgesehen und ist deshalb schwierig, zeitaufwändig und umständlich.

Architektursoftware bietet erhebliche Vorteile hinsichtlich der Datenverwaltung, da  Objekte von vornherein in üblichen Klassifizierungen, wie Wänden, Decken, Stützen, Böden und Öffnungen usw. verarbeitet werden können. Die Eingabe ist aufwändig, da sie parametrisch erfolgt. Sie bietet aber den Vorteil der Nachvollziehbarkeit. Außerdem besteht die Möglichkeit, zwischen verschiedenen Maßstäben hin- und herzuschalten und standardmäßig 2D-Schnitte und -Grundrisse darzustellen bzw. zu exportieren. Nachteile liegen vor allem im Preis der Anschaffung und der Notwendigkeit regelmäßig upzudaten, verlustfreier Datentransfer ist - wenn überhaupt - meist nur innerhalb der Produktgruppe des selben Software- Entwicklers möglich. Standardisierte Objekte wie Wände sind starr definiert und können nicht frei modelliert werden, eine verformungsgetreue Darstellung ist damit nicht möglich. Allerdings ist in den meisten Fällen eine verformungsgetreue Nachbildung des Baukörpers im Modell nicht nötig und verkompliziert unnötig die Geometrie und vergrößert die Datenbanken.

Offene Systeme wie AutoCad, VectorWorks und MicroStation bieten viele Vorteile hinsichtlich der Konvertierbarkeit und haben Schnittstellen zu beinahe jeder Art von Software. Sie ermöglichen jedoch meist keine normierten Klassifizierungen (Wände, Türen etc), es sei denn, es wird eine spezifische Applikation (z.B. für Architektur oder GIS) verwendet. Diese Vorgabe muss daher vor Anlegen des 3D-Modells festgelegt werden. Diese Systeme haben neben den genannten Vorteilen vor allem den Vorzug, weit verbreitet, gut dokumentiert und als Schulungsversionen relativ kostengünstig zu sein. Nachteile liegen in der Datenverwaltung und dem Bearbeitungskomfort.

Ein wichtiges Kriterium stellt ein gut dokumentiertes und leicht lesbares Format dar, das sämtliche Erfordernisse des Projekts mit einem Datenformat erfüllt. Ein gut dokumentiertes Format ist dabei unbedingt einem proprietären Binärformat, welches außer von einer kleinen Community nicht genutzt wird, vorzuziehen.

Wegen ihrer Verbreitung und den ubiquitär verfügbaren Importern sollte als erstes die Verwendung der ASCII-Version von DXF oder eines reinen Wavefront OBJ-Files in Betracht gezogen werden.

Zudem sollte das Datenformat rekursives Laden von Daten ermöglichen, also die Zerlegung eines Modells in mehrere Dateien, die dann wiederum an ein datenbankunabhängiges Bauwerksinformationssystem angehängt werden können.


\subparagraph{Auswahl geeigneter Software}Bei den 3D-Architekturmodellen im archäologischen Kontext ist es sinnvoll, drei Kategorien zu unterscheiden:

\begin{itemize}
	\item 1. Arbeitsmodelle
	\item 2. Visualisierungsmodelle
	\item 3. Interaktive Modelle
\end{itemize}


Arbeitsmodelle müssen maßstäblich und ggfls. auch georeferenziert sein. Dabei können sehr unterschiedliche Fragestellungen im Vordergrund stehen, z. B. stadträumliche oder topografische Beziehungen, Massenermittlungen, bei Einzelbefundrekonstruktionen aufgrund eines Versturzes z.B auch Lagebestimmungen, statische und konstruktive Bedingungen, bis hin zu einfachen Licht- und- Schatten- Untersuchungen. Aus solchen Modellen sollten maßstäbliche und vermaßbare technische Pläne generierbar sein, die der Kommunikation exakt nachvollziehbarer Zusammenhänge wie Höhen, Längen, Breiten und Massen dienen. Die Geometrie kann hierbei stark vereinfacht sein, grundsätzlich gilt dabei, dass Genauigkeit und Nachvollziehbarkeit vor Schönheit stehen sollte. Je nach gewünschter (mess- und abfragbarer) Informationsdichte des Modells sollte sorgfältig überlegt werden, welche Software bei welchem Arbeitsaufwand die gewünschten Standards bietet. Je nach Komplexität der angestrebten Informationsdichte besonders bei Rekonstruktionsmodellen ist zusätzlich zu beachten, dass auch der Zeichner zumindest architektonische Vorkenntnisse haben sollte.

Geeignete Software sind vor allem Architekturprogramme, wie z.B. Archicad, AutocadArchitectural, Nemetschek, Spirit usw. aber auch offene Systeme wie Microstation, Vectorworks und AutoCad. Architekturprogramme bringen logische Klassifizierungen, wie Wände, Stützen, Unterzüge, Treppen, Fenster, Dächer etc. mit sich, die Eingabe erfolgt dabei parametrisch und alle Parameter sind in Projekt-Datenbanken erfasst. Ob solche Datenbanken nur programmintern abgefragt oder wie sie in geeignete übergreifende Systeme exportiert werden können, muss sorgfältig geklärt werden.  

Mit allen diesen Programmen lassen sich auch mehr oder weniger realistische Bilder generieren, häufig jedoch nur von mäßiger Qualität. Die Modellierung hochkomplexer geometrischer Körper, wie z. B. eines korinthischen Kapitells, ist jedoch kaum möglich. 

Bei Visualisierungsmodellen ist vor allem Art und Qualität der angestrebten Visualisierung zunächst zu klären. Dabei gilt ,je höher die Qualität, desto höher der Arbeitsaufwand und die aufzuwendenden Kosten. Von einfachen Modellen ohne Textur bis hin zu fotorealistischen Bildern ist heute alles möglich. Für fotorealistische Bilder ,wenn diese nicht als animierte Szenen gewünscht werden, kann auch überlegt werden, ob eine Kombination von Einzelbildsynthese und Bildmontage in einem Bildbearbeitungsprogramm (z.B. Photoshop) einfacher und schneller zum Ergebnis führt.

Zur Anwendung können sehr einfache, z.T. auch Freewareprogramme kommen, wenn das Interesse in erster Linie der Geometrie gilt und diese eine überschaubare Polygon - Anzahl umfasst. Sollen anspruchsvolle, realistische oder gar fotorealistische Szenen erstellt werden, ist die Wahl einer professionellen Visualisierungs- Software ratsam. Da Visualisierungsprogramme in erster Linie der Erstellung maßstabloser, möglichst ansprechender Bilder dienen, ist präzises, maßstäbliches Zeichnen nur bedingt möglich. Die Definition der Körper erfolgt bei den einfacheren Programmen über ihre absolute Geometrie, bei aufwändigen Programmen wie Rhino, Cinema4d, MAYA oder 3dsMAx über Parameter für die Ausgangsgeometrie und ggfls. verschiedene Modifikatoren für Geometrie und Animation.

In Visualisierungsprogrammen ist - mit entsprechendem Aufwand - die Herstellung jeder denkbaren Qualität von Bildern, sowohl hinsichtlich der Auflösung wie der Detailqualität und des Realismus, möglich. Außerdem können Filmszenen generiert werden, die z.B. für die Visualisierung der Funktionsweise von Maschinen, für dynamische Schnitte oder die Darstellung von Bauabläufen sehr sinnvoll sein können. 

Da das Erstellen fotorealistischer Bilder mit 3d-Software hochkomplex ist, ist eine Verknüpfung mit bau-, geo- oder grabungsspezifischen Datenbanken nicht sinnvoll. Es gilt ganz allgemein der Satz: Schönheit und/oder visuelle Aussage vor Genauigkeit. Solche Modelle dienen allein der visuellen Kommunikation über Standbilder und Filme. 

Interaktive Modelle können mit Datenbanken verknüpft werden. Diese sind ohne Programmierkenntnisse bislang kaum sinnvoll erstellbar. Es stellt sich dabei  - abgesehen von der Qualität der verknüpften Datenbanken und ihren Fragestellungen - vor allem das Problem der Ladezeiten. Deshalb sind besondere Render- Engines unverzichtbar, die Geometrie und Texturen mit intelligenten Algorithmen reduzieren, je nachdem, ob sich diese im Vorder-, Mittel- oder Hintergrund befinden. 

In der Regel werden solche Modelle von einem Team aus Zeichnern und Programmierern erstellt. Wegen der sehr spezifischen Anforderungen an die Benutzeroberflächen und den Workflow ist auch an Personen mit Kenntnissen in der Erstellung von 3d-Spielen und/oder Webdesigner zu denken. 

Allgemein gilt die Prämisse: Workflow, einfache Steuerung, Übersichtlichkeit und intuitive Verständlichkeit der Oberfläche. Komplexe Inhalte können über Links abgefragt werden und sollten nicht auf der Oberfläche platziert sein.

\subparagraph{Auswahl der Hardware} Bei der Erstellung von 3d- Modellen sind naturgemäß sehr hohe Anforderungen an die grafische Rechenleistung zu stellen. Aufwändige Visualisierungsarbeiten erfordern eine entsprechend gute Ausstattung mit Hard- und Software, die teuer ist. Beim Echtzeit- Rendering im virtuellen 3d- Raum müssen riesige Datenmengen in sehr kurzer Zeit bearbeitet werden. Dabei sind Einzelplatz- Rechner meist auf die Render- Parameter, wie sie für eine gute Performance in 3d- Spielen erforderlich sind, ausgelegt. Sie sind für die Produktion hochwertig texturierter Einzelbilder oder Filme nicht geeignet. Man spricht heutzutage von Render- Farmen und Netzwerk- Rendering. Das bedeutet, das Rechenleistungen auf viele Prozessoren in einem Netzwerk verteilt werden können. Dabei stellt sich die Frage, ob solche Konstellationen vom Administrator zu gewährleisten sind. Außerdem muss die Software Netzwerkrendering unterstützen. Multiple Grafikprozessoren, Multikern- Hauptprozessoren und adäquat ausgelegte Arbeitsspeicher sind sinnvoll (z.B. 128 GB). Dabei ist darauf zu achten, dass die Hardware auch vom Betriebssystem unterstützt wird (speziell beim Arbeitsspeicher: Windows hat hier bislang die größten Nachteile) und von der eingesetzten 3d- Software gleichermaßen sinnvoll verwaltet werden kann. Riesige Festplatten sind nur bei der Erstellung von Filmen notwendig. Hier empfiehlt sich die Auslagerung auf externe Medien. Wegen der umfangreichen Paletten ist ein großer Monitor mit zusätzlichem Kontrollmonitor unverzichtbar. 


%##############################################################
\paragraph{Quellen}
\begin{flushleft}
Da Software- und Hardwaremärkte sehr dynamisch sind, ist es empfehlenswert, sich in aktuellen, Zeitschriften zu informieren, z.B. \url{http://utils.exeter.ac.uk/german/media/archited.html} und Messen, wie die Bautec zu besuchen. 

Außerdem kann man sich auf Wikipedia.de eine gute Orientierung über die diversen Probleme bei der Erstellung von 3d- Modellen verschaffen. Dort finden sich auch Links zu weiterführender Literatur die i.d. Regel recht aktuell sind: \url{http://de.wikipedia.org/wiki/} Kategorie:Bildsynthese, sehr empfehlenswert ist auch der Artikel: \url{http://de.wikipedia.org/wiki/Grafik-Engine} 

Viele Software- Hersteller unterhalten eigene Foren oder verweisen zumindest auf ihren Seiten auf User- Foren, z.B.: \url{http://www.autodesk.de/}. 

%XXX Alter Inhalt; Kommentare vor URL sind grob eingetragen
Gerade im Bereich der 3d-Modellierung gibt es viele einschlägige Foren, z.B.:
3D-Ring: \urllist{http://www.3d-ring.de/forum.php}
3Dlinks: \urllist{http://www.3dlinks.com/} 
Grafiker: \urllist{http://www.grafiker.de/forum/forumdisplay.php?f=39}
3Dcenter: \urllist{http://www.forum-3dcenter.org/vbulletin/} 
3Dcafe: \urllist{http://www.3dcafe.com/}

Zur spezifischen Problemstellung interaktiver oder Web- basierter Modelle: 
ki-smile: \urllist{http://www.ki-smile.de/kismile/view145,11,1424.html} 
garybeene: \urllist{http://www.garybeene.com/3d/3d-pure.htm}

Quality management systems - Requirements(Iso- Norm) Edition: 4 ; Stage: 60.60 ; TC 176/SC 2 ICS: 03.120.10 Beuth- Verlag,Berlin

nestor edition 5: Langzeiterhaltung von 3D-Röntgen-Computertomographien in der Archäologischen Denkmalpflege. Sayer-Degen, Leila (2012) \url{http://files.dnb.de/nestor/edition/05-sayer-degen.pdf}
\end{flushleft}

	
\newpage
\section{Reflectance Transformation Imaging (RTI)}
\abschnittsautor{M. Trognitz}
	Reflectance Transformation Imaging (RTI) ist eine computergestützte Fotografiermethode, mit der von einem Objekt mehrere Bilder mit fixierter Kameraposition und variablen Beleuchtungspositionen gemacht werden. Anschließend werden die Aufnahmen vom Computer zu einem Polynomial Texture Map (PTM) zusammengerechnet. In der resultierenden Datei kann die Position der Lichtquelle verändert werden, um beispielsweise die Oberfläche des aufgenommenen Objekts im Schräglicht untersuchen zu können. 

RTI wurde erst 2001 entwickelt und hat inzwischen eine weite Verbreitung für die Dokumentation von Objekten mit flachen Oberflächen, wie etwa Felsbilder, Inschriften oder Münzen, gefunden, da hochauflösende Ergebnisse mit vergleichsweise günstiger Ausrüstung erzielt werden können.

\subparagraph{Langzeitformate}
Die Aufnahmen für RTI erfolgen mit einem Fotoapparat, weshalb man es bei der Archivierung hauptsächlich mit Rastergrafiken zu tun hat. Ausführliche Hinweise zu geeigneten Langzeitformaten sind in dem Kapitel Rastergrafiken ab Seite \pageref{rastergrafiken} zu finden. 

Zusätzlich müssen die resultierenden softwareabhängigen Dateien für die Archivierung berücksichtigt werden. Die Formate für RTI wurden in den Hewlett-Packard-Laboratories entwickelt und von Cultural Heritage Imaging (CHI) erweitert und ergänzt.

\begin{center}
	\begin{tabular}{l L{0.2\textwidth} p{0.6\textwidth}}
		\toprule
		\multicolumn{2}{l}{Ausgangsformat} & Begründung \\ \midrule
		\multirow{2}{*}{\color{ForestGreen} \LARGE \checkmark} & DNG & Originale Aufnahmen in RAW, sollten als DNG gespeichert werden. \\ 
		& TIFF & Originale Aufnahmen, die nicht in RAW aufgenommen wurden, sollten als TIFF gespeichert werden. Vor der Prozessierung erheblich bearbeitete Aufnahmen (z.B. Freistellung des Objektes), sollten in einem gesonderten Ordner ebenfalls als TIFF gespeichert werden.\\ 
 		\bottomrule
		\bottomrule
	\end{tabular}
\end{center}

\begin{center}
	\begin{tabular}{l L{0.2\textwidth} p{0.6\textwidth}}
		\toprule
		\multicolumn{2}{l}{Prozessierungsformat} & Begründung \\ \midrule
		\multirow{3}{*}{\color{ForestGreen} \LARGE \checkmark} & JPG & Programme zur Erzeugung von RTI-Dateien benötigen meist JPG-Dateien. Auch die von den Programmen erzeugten Zwischenbilder werden als JPG gespeichert. Für diese Dateien ist die Archivierung im JPG-Format zulässig.\\ 
		& XML & Alle Einstellungen werden in einer automatisch erzeugten XML-Datei gespeichert.\\ 
		& LP & In der textbasierten LP-Datei (light position file) werden die Positionen der Lichtquelle für jedes einzelne Bild gespeichert. Sie wird automatisch erzeugt, kann aber auch manuell erstellt werden. \\ 
 		\bottomrule
		\bottomrule
	\end{tabular}
\end{center}

\begin{center}
	\begin{tabular}{l L{0.2\textwidth} p{0.6\textwidth}}
		\toprule
		\multicolumn{2}{l}{Ergebnisformat} & Begründung \\ \midrule
		\multirow{3}{*}{\color{ForestGreen} \LARGE \checkmark} & PTM & Das Format Polynomial Texture Map wurde in den Hewlett-Packard-Laboratories entwickelt und speichert für jedes Pixel eine Funktion, mit der die Farbwerte des Pixels in Abhängigkeit der Lichtquelle errechnet werden können.\\ 
		& RTI & Das RTI-Format wurde von CHI entwickelt und bietet im Vergleich zu PTM erweiterte Funktionalitäten im \emph{RTIViewer} von CHI.\\ 
		& XMP & Der \emph{RTIViewer} von CHI erlaubt das Annotieren der Dateien, um beispielsweise besondere Bildbereiche hervorzuheben. Diese Informationen werden in einer textbasierten XMP-Datei gespeichert.\\ 
 		\bottomrule
		\bottomrule
	\end{tabular}
\end{center}

\subparagraph{Ordnerstruktur}
Zwar kann eine PTM- oder RTI-Datei ohne zusätzliche Dateien verwendet werden, jedoch sollte für die Nachnutzbarkeit und Nachvollziehbarkeit das gesamte Datenpaket archiviert werden. Dabei sollten alle Daten nicht alle in einem einzigen Ordner, sondern in einer sinnvollen Ordnerstruktur abgelegt werden. In der folgenden Tabelle wird eine solche Struktur vorgeschlagen und auch die entsprechenden Dateitypen darin aufgelistet.

Die Benennung der Ordner ist hier als Vorschlag zu betrachten. Dabei muss berücksichtigt werden, dass einige Ordner und Dateien von der Software erzeugt werden und diese auch so belassen werden sollten, um wieder mit dem Programm geöffnet werden zu können. Die Struktur orientiert sich an den Vorgaben von Cultural Heritage Imaging (CHI) und der von dem Programm \emph{RTIBuilder} erzeugten Dateistruktur.

Die gesamten Aufnahmen sollten in einem übergeordneten Verzeichnis mit dem Projektnamen gespeichert werden. Für jedes aufgenommene Objekt muss ein eigener Unterordner angelegt werden, der idealerweise eine eindeutige Bezeichnung des Objektes, wie etwa die Inventarnummer, in der Benennung enthält. In dem übergeordneten Verzeichnis können Dokumentationsdateien abgelegt werden.

\begin{center}
	\begin{tabular}{l l l l r p{0.52\textwidth}}
		\toprule
		\multicolumn{6}{l}{Ordnerstruktur und enthaltene Dateiformate} \\ \midrule
		\multicolumn{4}{l}{\includegraphics[width=0.4cm]{bilder/OrdnerIconZu.png} \hspace*{0.04cm} Projekt RTI-Aufnahmen} & & \multirow{25}{*}{\begin{minipage}{0.52\textwidth}\textit{In dem übergeordneten Verzeichnis sind alle Aufnahmen enthalten. Außerdem können hier auch Dokumentationsdateien abgelegt werden. Pro Aufnahme entsteht ein Ordner. Neben den verschiedenen Unterordnern enthält dieser auch eine Projektdatei im XML-Format.\\
		'original-captures' enthält die unveränderten Originalaufnahmen im DNG- oder TIFF-Format.\\
		'edited-captures' (optional) enthält überarbeitete Aufnahmen im TIFF-Format.\\
		'jpeg-exports' enthält die Bilder, aus denen das Ergebnis erzeugt wird. Die JPG-Bilder werden entweder aus den Dateien in 'original-captures' oder in 'edited-captures' erzeugt.\\
		'cropped-files' wird automatisch erzeugt. Er enthält JPG-Dateien, die bei der Verwendung der Zuschnittsfunktion als Zwischenergebnis gespeichert werden.\\
		'assembly-files' wird automatisch erzeugt. Er enthält JPG-Dateien, welche die Ausleuchtung des Objektes repräsentieren und eine LP-Datei, in der die Position der Lichtquellen gespeichert werden.\\
		'finished-files' wird automatisch erzeugt und enthält die resultierenden RTIs und PTMs. Werden die Dateien im \emph{RTIViewer} annotiert, entstehen zusätzlich XMP-Dateien.
		}\end{minipage}}\\
		& \multicolumn{3}{l}{\includegraphics[width=0.4cm]{bilder/DateiIcon.png} \hspace*{0.04cm} dokumentation} &  & \\
		& \multicolumn{3}{l}{\includegraphics[width=0.4cm]{bilder/OrdnerIconZu.png} \hspace*{0.04cm} Inventarnummer 001} &  & \\
		& \multicolumn{3}{l}{\includegraphics[width=0.4cm]{bilder/OrdnerIconAuf.png} \hspace*{0.04cm} Inventarnummer 002} &  & \\
		& & \multicolumn{2}{l}{\includegraphics[width=0.4cm]{bilder/DateiIcon.png}\hspace*{0.04cm} .xml} &  & \\
		& & \multicolumn{2}{l}{\includegraphics[width=0.4cm]{bilder/OrdnerIconAuf.png} \hspace*{0.04cm} original-captures} &  & \\
		& & & \includegraphics[width=0.4cm]{bilder/DateiIcon.png} \hspace*{0.04cm} .dng / .tiff &  & \\
		& & \multicolumn{2}{l}{\includegraphics[width=0.4cm]{bilder/OrdnerIconAuf.png} \hspace*{0.04cm} edited-captures} &  & \\
		& & & \includegraphics[width=0.4cm]{bilder/DateiIcon.png} \hspace*{0.04cm} .tiff &  & \\
		& & \multicolumn{2}{l}{\includegraphics[width=0.4cm]{bilder/OrdnerIconAuf.png} \hspace*{0.04cm} jpeg-exports} &  & \\
		& & & \includegraphics[width=0.4cm]{bilder/DateiIcon.png} \hspace*{0.04cm} .jpg &  & \\
		& & \multicolumn{2}{l}{\includegraphics[width=0.4cm]{bilder/OrdnerIconAuf.png} \hspace*{0.04cm} cropped-files} &  & \\
		& & & \includegraphics[width=0.4cm]{bilder/DateiIcon.png} \hspace*{0.04cm} .jpg &  & \\
		& & \multicolumn{2}{l}{\includegraphics[width=0.4cm]{bilder/OrdnerIconAuf.png} \hspace*{0.04cm} assembly-files} &  & \\
		& & & \includegraphics[width=0.4cm]{bilder/DateiIcon.png} \hspace*{0.04cm} .jpg &  & \\
		& & & \includegraphics[width=0.4cm]{bilder/DateiIcon.png} \hspace*{0.04cm} .lp &  & \\
		& & \multicolumn{2}{l}{\includegraphics[width=0.4cm]{bilder/OrdnerIconAuf.png} \hspace*{0.04cm} finished-files} &  & \\
		& & & \includegraphics[width=0.4cm]{bilder/DateiIcon.png} \hspace*{0.04cm} .ptm &  & \\
		& & & \includegraphics[width=0.4cm]{bilder/DateiIcon.png} \hspace*{0.04cm} .rti &  & \\
		& & & \includegraphics[width=0.4cm]{bilder/DateiIcon.png} \hspace*{0.04cm} .xmp &  & \\
		& \multicolumn{3}{l}{\includegraphics[width=0.4cm]{bilder/OrdnerIconZu.png} \hspace*{0.04cm} Inventarnummer 003} &  & \\
		& & & & & \\
		& & & & & \\
		& & & & & \\
		& & & & & \\
		& & & & & \\
 		\bottomrule 
		\bottomrule
	\end{tabular}
\end{center}

\subparagraph{Dokumentation} Neben den allgemeinen minimalen Angaben zu Einzeldateien, wie sie in dem Abschnitt Metadaten in der Anwendung ab Seite \pageref{Metadaten-anwendung} beschrieben sind, erfordern RTI-Daten Angaben, die Aufschluss über die Aufnahmemethode, deren Umstände und über die Prozessierung der Daten geben. Zusätzlich können weitere Angaben zu dem Projekt als ganzes notwendig sein, um etwa genauere Angaben zu den aufgenommenen Objekten zu machen. Hinweise hierfür sind ebenfalls in dem Abschnitt Metadaten in der Anwendung ab Seite \pageref{Metadaten-anwendung} beschrieben.

Speziell für die freihändige RTI-Aufnahme mittels Highlight-RTI gibt es von Cultural Heritage Imaging einen \emph{Shooting-Log}, der in angepasster Form auch für andere Aufnahmemethoden verwendet werden kann. Eine Vorlage hierfür wird online in den IT-Empfehlungen zur Verfügung gestellt. Es empfiehlt sich diese Übersicht schon während der Aufnahme der einzelnen Objekte auszufüllen.

\begin{center}
	\begin{tabular}{L{0.3\textwidth} p{0.6\textwidth}} 
		\toprule
		Metadatum & Beschreibung \\ \midrule
		Kamera und Objektiv & Modell und Name der verwendeten Kamera sowie Details zum verwendeten Objektiv.\\
		Kamerafilter & Auflistung aller für die Aufnahme verwendeten Filter, wie etwa Graufilter oder Infrarotfilter.\\
		Aufnahmemethode & Angabe der Methode, die für die RTI-Aufnahme verwendet wurde. Beispielsweise feste Kuppel oder Highlight-RTI (H-RTI).\\
		Radius oder Schnurlänge & Radius der verwendeten Kuppel oder des Armes. Wenn H-RTI angewandt wurde, muss die Länge der Schnur angegeben werden. Die Angabe erfolgt in cm.\\
		Größe der Sphäre & Durchmesser in cm oder mm der verwendeten Sphäre, wenn H-RTI angewandt wurde.\\
		Lichtquelle & Anzahl und Typ der Lichtquelle, sowie deren Einstellungen. Beispielsweise Canon Speedlite 580 EX II, 1/4 Leistung.\\
		Projektordner & Name des Verzeichnisses in dem alle Aufnahmeordner gespeichert werden.\\
		Verzeichnisname & Der Name des Verzeichnisses in dem die originalen Aufnahmen, prozessierten Bilder und Ergebnisdateien liegen.\\
		Objektbezeichnung & Eindeutige Bezeichnung des Objektes, wie beispielsweise die Inventar- oder Fundnummer. \\
		Beschreibung & Kurze Beschreibung des Objektes.\\
		Farbkarte & Wenn eine Farbkarte verwendet wurde, soll angegeben werden, ob sie in jedem Bild vorhanden ist oder nur in einem separaten Bild. Wenn ein Farbprofil verwendet wurde, kann dieses hier angegeben werden.\\
		Testbilder & Angabe der Dateinamen aller nicht für die Erstellung der RTI-Datei verwendeten Bilder, die jedoch weiterhin beibehalten werden, wie beispielsweise das Bild mit der Farbkarte.\\
		Aufnahmeteam & Vor- und Nachnamen der Personen, die an der Aufnahme beteiligt waren.\\
		Rechte & Lizenz der Aufnahme, z.B. CC-BY 3.0\\
		Notizen & Weitere Angaben zum Aufnahmeaufbau und besonderen Vorkommnissen.\\
		Software & Name und Version der zur Erzeugung der RTI-Daten verwendeten Software, beispielsweise \emph{RTIBuilder} Version 2.0.2.\\
		Anzahl der Bilder & Zahl der für das finale RTI verwendeten Bilder.\\
		Weitere Dateien & Liste weiterer Dateien, wie beispielsweise des Shooting-Logs oder Dokumentation der verwendeten Software.\\
	  \bottomrule
	\end{tabular}
\end{center}


\paragraph{Vertiefung}
Reflection Transformation Imaging (RTI) erzeugt Polynomial Texture Maps (PTMs), in denen für jedes Pixel Informationen über die Reflexionseigenschaften der Oberfläche gespeichert werden. Die erzeugten interaktiven Dateien werden aus einer Vielzahl von Bildern errechnet, wobei in jedem Bild die Kameraposition unverändert ist, während die Richtung, aus der das Licht kommt, jeweils wechselt.

Mit speziellen Viewer-Programmen können die Daten angesehen werden und die aufgenommenen Objekte interaktiv aus jeder Richtung beleuchtet werden. Da es sich aber nicht um eine 3D-Datei handelt, können die Objekte nicht bewegt werden. Jedoch bieten die Programme zusätzlich unterschiedliche Algorithmen, um bestimmte Oberflächeneigenschaften, wie Kanten oder Risse, optisch hervorzuheben, die unter natürlichen Bedingungen nicht oder nur schwer erkennbar sind.

Für die Erstellung von RTI-Daten können unterschiedliche Aufnahmemethoden verwendet werden, die unterschiedliche Ausrüstung voraussetzen, wobei das Grundprinzip dahinter jeweils immer das gleiche ist.

\begin{figure}[!hpt]
  \begin{center}
    \includegraphics[width=0.85\textwidth]{bilder/rti_TA01029}
  \end{center}
  \caption{Ausschnitt der Rückseite der Grabstele TA 1029 mit reichsaramäischer Inschrift des Verstorbenen (Tayma, Saudi-Arabien). Links ein Foto. In der Mitte der gleiche Ausschnitt als RTI mit 'Specular Enhancement', rechts mit 'Static Multi Light'. (Foto: Mirco Cusin; RTI: Martina Trognitz, Max Haibt; DAI-Orient-Abteilung)}
\end{figure} 

\subparagraph{Polynomial Texture Map}
Ein Polynomial Texture Map (PTM) ist eine Repräsentationsform von Bildern mit Hilfe von Funktionen, statt einzelner Farbwerte. Im Gegensatz zu Rastergrafiken werden für die einzelnen Pixel von PTMs nicht nur feste Farbwerte gespeichert, sondern zusätzlich eine Funktion, die mit Hilfe der Parameter $l_u$ und $l_v$ die Leuchtdichte der Oberfläche berechnet. Die Leuchtdichte bestimmt wie das menschliche Auge eine Oberfläche wahrnimmt, also ob sie besonders hell, dunkel, spiegelnd oder matt erscheint.

%lu und lv noch einmal kontrollieren, so ganz kann das gerade nicht stimmen, was im ersten satz steht
Die Parameter $l_u$ und $l_v$ spezifizieren, wo sich eine punktförmige Lichtquelle befindet. Wird die Lichtquelle bewegt, ändern sich die Parameter und somit auch der errechnete Farbwert für den jeweiligen Pixel. Somit kann ein PTM simulieren, wie ein Objekt bei wechselnder Beleuchtungsrichtung aussieht.

Ganz vereinfacht lässt sich das Prinzip hinter RTI in der untenstehenden Abbildung visualisieren. Für jeden Punkt einer dreidimensionalen Oberfläche kann eine Normale bestimmt werden. An dieser Normale wird einfallendes Licht im gleichen Winkel reflektiert (Reflexionsgesetz). Da die Kamera sich in einer festen Position befindet, die Richtung aus der das Licht kommt ebenfalls bekannt ist und es einen Bilderstapel mit unterschiedlichen Lichtpositionen gibt, kann für jeden Bildpunkt eine Oberflächennormale berechnet werden. Wird die fertige RTI-Datei betrachtet und die Beleuchtung geändert, können die angezeigten Farbwerte der einzelnen Punkte anhand des gespeicherten Farbwertes, der errechneten Normalen und dem Einfallswinkel berechnet und ausgegeben werden.

\begin{figure}[!hpt]
  \begin{center}
    \includegraphics[width=0.9\textwidth]{bilder/rti_ptmAbstrakt}
  \end{center}
  \caption{Rechts sind auf einer dreidimensionalen Oberfläche für vier Punkte die Normalen (rot) eingezeichnet, an denen einfallendes Licht (gelb) reflektiert wird (grün). Links ist diese Obefläche als PTM dargestellt, in dem für die Punkte jeweils die Oberflächennormalen gespeichert sind. So entsteht ein optischer 3D-Effekt, der an sich aber nicht in der Datei hinterlegt ist.}
\end{figure}

\begin{wrapfigure}{r}{0.435\textwidth}
  \begin{center} \vspace{-0.6cm}
    \includegraphics[width=0.435\textwidth]{bilder/rti_aufnahme}
  \end{center}
  \caption{Oben links eine Kuppel für RTI-Aufnahmen. Rechts ein gebogener Arm, der um das Objekt rotiert wird. Unten ist in den Aufbau von CHI eine imaginäre Kuppel projiziert. (Hewlett-Packard-Laboratories und CHI)}
\end{wrapfigure} 
\subparagraph{Aufnahmemethoden}
Das Grundprinzip für RTI-Aufnahmen besteht in der immer gleich bleibenden Position der Kamera in Bezug auf das Objekt und der wechselnden Position der Lichtquelle. Bei der Lichtquelle muss der Abstand zu dem Objekt immer der gleiche sein, um eine konstante Lichtintensität zu gewährleisten, und zusammen betrachtet sollten die jeweiligen Positionen so gewählt werden, dass sie gleichmäßig um das Objekt herum verteilt sind. Für die Positionierung der Lichtquellen wurden unterschiedliche Methoden und Ausrüstungsteile entwickelt, die in der nebenstehenden Abbildung zusammengefasst dargestellt sind.

Bei der Aufnahme mit einer Kuppel (engl. \emph{dome}) wird eine Kuppel so über das Objekt platziert, dass dieses in der Mitte liegt. Die Kamera wird ebenfalls mittig der Kuppel direkt über dem Objekt platziert. An der Kuppel können mehrere Lichtquellen, wie etwa LEDs befestigt sein, wobei bei jedem Bild jeweils nur eine eingeschaltet wird. Es gibt auch Kuppeln, die statt fest eingebauter Lichtquellen nur Markierungen für eine manuelle Platzierung der Lichtquelle haben. Die Position der Lichtquellen ergeben sich aus den Abmessungen der Kuppel und müssen in dem Programm eingetragen werden.

Eine vereinfachte Form der Kuppel sind gebogene Arme, mit daran befestigten Lichtquellen, wie etwa externe Blitzgeräte. Dieser Arm kann um das Objekt herum rotiert werden, so dass am Ende der Aufnahme eine kuppelförmige Verteilung der Lichter erzielt wird. Auch hier kann die Position der Lichtquellen aus den Abmessungen des Armes und dem Rotationswinkel berechnet werden.

Eine sehr flexible Methode, Highlight-RTI, wurde von Cultural Heritage Imaging entwickelt. Hierfür wird nur ein Blitzgerät (oder eine andere Lichtquelle) benötigt, an dem eine Schnur befestigt wird, mit deren Hilfe man das Gerät so ausrichten kann, dass der Abstand immer gleich ist und das Licht auch direkt auf das Objekt gerichtet ist. Auf diese Weise kann man sich an einer imaginären Kuppel entlang orientieren. Um die Position der Lichtquelle errechnen zu können, wird neben das Objekt eine schwarz glänzende Kugel platziert, die auch auf jedem Bild zu sehen sein muss. Aus den Reflexionspunkten des Lichtes an der Kugel kann das Programm dann die Richtung, aus der das Licht kommt, berechnen.


\paragraph{Praxis} In diesem Abschnitt werden Programme vorgestellt, mit denen RTI-Daten erstellt und betrachtet werden können.

\subparagraph{Erstellung von RTI-Daten}
Für die Aufnahme und Erstellung von RTI-Daten gibt es zwei frei verfügbare Programme. Das eine ist das \emph{PTM Fitter Programm}, welches an den Hewlett-Packard-Laboratories entwickelt wurde und zugleich auch das erste Programm ist, dass diese Art von Daten erstellen konnte. Es kann auf allen gängigen Betriebssystemen verwendet werden. Das zweite Programm, \emph{RTIBuilder}, stammt von CHI und ist für Windows und Mac verfügbar. CHI stellt weitere Materialien zur Erstellung von RTI-Daten zur Verfügung und deren Webangebot dient als zentrale Anlaufstelle für alles rund um RTI. Sie bieten ebenfalls ein Forum für aktive Anwender an.

\begin{flushleft}
	RTIBuilder: \urllist{http://culturalheritageimaging.org/What_We_Offer/Downloads/Process/index.html}
	Weiteres Material von CHI: \urllist{http://culturalheritageimaging.org/What_We_Offer/Downloads/}
	PTM Fitter Program: \urllist{http://www.hpl.hp.com/research/ptm/downloads/download.html}
\end{flushleft}
	
	
\subparagraph{Ansicht von RTI-Daten}
RTI-Daten können nur mit speziellen Viewer-Programmen betrachtet werden. Von den Hewlett-Packard-Laboratories gibt es die frei verfügbare \emph{PTM Viewer Application}, die auf Windows, Mac und Linux verwendet werden kann. Das Team von CHI bietet den \emph{RTIViewer} an, der ebenfalls frei verfügbar ist und auf Windows und Mac läuft. Er bietet erweiterte Funktionalitäten, wie beispielsweise die Annotation von Dateien.

Für eine Präsentation der Daten auf Webseiten kann der \emph{WebRTIViewer} empfohlen werden, der an dem Visual Computing Laboratory von CNR-ISTI in Pisa entwickelt wurde.

\begin{flushleft}
	PTM Viewer Application: \urllist{http://www.hpl.hp.com/research/ptm/downloads/download.html}
	RTIViewer: \urllist{http://culturalheritageimaging.org/What_We_Offer/Downloads/View/index.html}
	WebRTIViewer: \urllist{http://vcg.isti.cnr.it/rti/webviewer.php}
\end{flushleft}


%##############################################################
\paragraph{Quellen}
\begin{flushleft}
Cultural Heritage Imaging: \urllist{http://culturalheritageimaging.org/}

Hewlett-Packard-Laboratories, PTM: \urllist{http://www.hpl.hp.com/research/ptm/}

S. Duffy -- P. Bryan -- G. Earl -- G. Beale -- H. Pagi -- E. Kotouala, Multi-light Imaging for Heritage Applications (2013) \urllist{https://www.historicengland.org.uk/images-books/publications/multi-light-imaging-heritage-applications/}

T. Malzbender -- D. Gelb -- H. Wolters, Polynomial Texture Maps, in: Proceedings of ACM Siggraph 2001 (2001) \urllist{http://www.hpl.hp.com/research/ptm/papers/ptm.pdf}

\quelltyp{Formatspezifikationen}
PTM: \urllist{http://www.hpl.hp.com/research/ptm/downloads/PtmFormat12.pdf}
RTI: \urllist{http://forums.culturalheritageimaging.org/index.php?app=core&module=attach&section=attach&attach_id=81}

\quelltyp{Tools und Programme}
	RTIBuilder: \urllist{http://culturalheritageimaging.org/What_We_Offer/Downloads/Process/index.html}
	Weiteres Material von CHI: \urllist{http://culturalheritageimaging.org/What_We_Offer/Downloads/}
	PTM Fitter Program: \urllist{http://www.hpl.hp.com/research/ptm/downloads/download.html}
	PTM Viewer Application: \urllist{http://www.hpl.hp.com/research/ptm/downloads/download.html}
	RTIViewer: \urllist{http://culturalheritageimaging.org/What_We_Offer/Downloads/View/index.html}
	WebRTIViewer: \urllist{http://vcg.isti.cnr.it/rti/webviewer.php}
\end{flushleft}


\newpage
\section{Satellitenmessungen}%%Ursprünglich: Satellitenbilder und subsubsection
\abschnittsautor{S. Reinhold}
	\input{methoden/satellitenmessungen}


%%%%%%%%%%%%%%%%%%%%%
%% Alte Gliederung %%
%%%%%%%%%%%%%%%%%%%%%
%\section{Anthropologoie}
%\subsection{Genetik (Genotyp)}
%\subsection{Morphologie und Metrik (Phänotyp)}
%\subsection{Physiologie (Ökotyp)}
%\subsection{Verhaltensforschung (Ethotyp)}

%\section{Archäobotanik}
%\subsection{Makroreste}
%\subsection{Pollenanalyse (Palynologie)}

%\section{Archäozoologie}

%\section{Ausgrabung}
%\section{Bauforschung}
%\subsection{Bauaufnahme}

%\section{Datierung}
%\subsection{Andere Methoden}
%\subsection{C14-Datierung}
%\subsection{Dendrochronologie}
%\subsection{OSL/TL-Datierung}
%\subsection{Uran-Ungleichgewichtsmethoden}

%\section{Geoarchäologie}
%\subsection{Biomarker}
%\subsection{Bodenuntersuchungen}
%\subsection{Geologische Untersuchungen}
%\subsection{Geomorphologie}
%\subsection{Sedimentuntersuchungen (Bohrkerne)}

%(Geodatenanalyse)
%\subsection{Geocodierung und Geoparsing}
%\subsection{Geomorphometrie}
%\subsection{Visualisierung und Kartierung}


%\section{Geophysik}
%\subsection{Geoelektrik}
%\subsection{Geomagnetik}
%\subsection{Georadar}

%\section{Materialaufnahme}
%\subsection{3D-Scanning} (DFG Digitalisierung hat auch einen Abschnitt dazu)
%\subsection{Nahbereichsphotogrammetrie}
%\subsection{RTI}

%\section{Materialwissenschaftliche Archäometrie}
%\subsection{Anorganische Materialien}
%\subsubsection{Gesteine}
%\subsubsection{Metalle}
%\subsubsection{Silikatische Materialien}

%\subsection{Methoden}
%\subsubsection{Bildgebende Verfahren}
%\subsubsection{Chemische Verfahren}
%\subsubsection{Isotopenuntersuchung}
%\subsubsection{Physikalische Verfahren}

%\subsection{Organische Materialien}
%
%\section{Numismatik}
%\section{Oberflächenbegehung (Survey)}
%\section{Statistische Daten}

%\section{Textwissenschaften und Schriftzeugnisse}
%%In this section we cover the main kinds of text providing for each of them an overview of problems, existing standards, existing tools and recommended best practices. Also for each type of text we suggest possible workflows with different levels of depth of text encoding. The first step will be always creating a metadata catalog for the texts we are dealing with, etc.
%%
%%\begin{itemize}
%%	\item goal: to provide a set of recommendations and best practices to follow when encoding textual sources.
%%	\item Also we want to put the currently available standards in the context of the specific problems and perspectives of the disciplines that IANUS aims to cater for, such as for instance Archaeology.
%%	\item The way texts have been treated to date has lead to a perceivable dichotomy between the communities of those who deal with texts and those who deal with objects, which often are also text-bearing objects.
%%	\item more than just saying which standards should be used, in this document we want to state clearly what are the problems and challenges we are faced with when encoding textual sources electronically
%%	\item identifying the intended use of data and the target user community
%%	\item open challenges are:
%%	\begin{itemize}
%%		\item theoretical problems related to texts
%%		\begin{itemize}
%%			\item encoding text and the problem of interpretation
%%			\item function/role of the editor, shift from print to digital era
%%		\end{itemize}	
%%	\end{itemize}	
%%	\begin{itemize}
%%		\item interoperability of encoding formats
%%		\begin{itemize}
%%			\item interoperability between texts
%%			\item interoperability between tools that analyse and manipulate texts
%%		\end{itemize}	
%%	\end{itemize}	
%%\end{itemize}
%
%\subsection{Sprachen und Zeichensysteme}
%%Matteo Romanello
%%Markus Schnöpf -> siehe Vortrag 2014 DCSB
%
%%Auszeichnungssprachen vertiefen; SGML, XML; sowie TEI ausführen
%%Texttechnologie?
%
%
%%the choice of an encoding format is not merely a technical one, but has both theoretical and practical implications and consequences. These decisions are inevitable, therefore it is important to be aware of such implications in order to make an informed decision.
%%
%%Character encoding; maybe OCR; 
%
%\subsubsection{Ägyptisch}
%\subsubsection{Akkadisch}
%\subsubsection{Aramäisch}
%\subsubsection{Griechisch}
%\subsubsection{Hebräisch}
%\subsubsection{Hethitisch}
%\subsubsection{Latein}
%\subsubsection{Sumerisch}
%
%\subsection{Antike Objekte mit Texten}
%\subsubsection{Gefäße}
%\subsubsection{Inschriften}
%%EpiDoc guidelines (\url{http://www.stoa.org/epidoc/gl/latest/})
%
%%list of projects using EpiDoc: \url{http://www.stoa.org/epidoc/gl/latest/app-bibliography.html}
%
%%EAGLE project: \url{http://www.eagle-network.eu/}
%
%\subsubsection{Manuskripte}
%%	\url{http://www.digipal.eu/}
	%
%%	\url{http://www.homermultitext.org/}
%
%%	\url{http://homermultitext.blogspot.de/}
	%
%\subsubsection{Papyri}
%%	\url{http://www.papyri.info/}
	%
%\subsubsection{Siegel}
%\subsubsection{Tontafeln}
%
%\subsection{Literarische Schriftquellen}
%%	Digital Latin Library: \url{http://www.apaclassics.org/index.php/research/digital_latin_library_project}
%
%%	Classical Works Knowledge Base:	\url{http://www.cwkb.org/}
%
%%	Perseus Catalog:	\url{http://catalog.perseus.org/}
%
%\subsection{Veröffentlichung und Archivierung}
%%Mit Rücksicht auf Werkzeuge und Plattformen
%%\begin{itemize}
%%	\item SoSOL for papyri and inscriptions (Philologist  perseid, as presented at DH2012)
%%	\item Arachne's TEI/OCR editor
%%	\item Open Journal System (OJS) for journals
%%	\item CTS repositories and CITE suite
%%\end{itemize}
%
%\subsubsection{Wissenschaftliche Editionen}
%%\begin{itemize}
%%	\item the term scholarly edition is intentionally rather general: the idea is not to limit this category to any discipline in particular. 
%%	\item	a set of general problems related to scholarly editions are to be identified
%%	\item then there are specific kinds of texts which challenge the print-based model of scholarly edition, for instance fragmentary texts
%%	\item \url{http://wiki.tei-c.org/index.php/Critical_Apparatus_Workgroup}
%%	\item \url{http://www.i-d-e.de/aktivitaeten/reviews/kriterien-version-1}
%%\end{itemize}
%
%\subsubsection{Schnittstellen zu Textsammlungen}
%%\begin{itemize}
%%	\item OAI-PMH as sort of a minimum to expose a collection of resources
%%	\item then there exist more sophisticated protocols to access text collections such as the CTS protocol
%%\end{itemize}
%\subsubsection{Kollaborative Arbeitsumgebungen}
%%emphasis on platforms and licenses that allow the users to contribute to the texts and to improve them (OCR corrections, levels of annotation, etc.)
%\subsubsection{Publikation}
%%XML Print (\url{https://sites.google.com/a/budabe.eu/xmlprint_de/home}) DFG-funded project, developed by UTrier, and related to TextGrid. Goal is to make easier to produce (high quality/precision) print output out of XML/TEI encoded texts.
%\subsubsection{Lizenzierung}
%%\begin{itemize}
%%	\item as soon as data are published online, a license must be attached to them. Possibly an open one. 
%%	\item Put the open licensing in the context of crowdsourcing and the virtuous cycle of open access; new projects and initiatives can be built on top of openly licensed materials
%%\end{itemize}