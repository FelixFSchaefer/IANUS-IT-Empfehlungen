Textdokumente stellen in der altertumswissenschaftlichen Forschung einen häufig vertretenen Dateityp dar. In Artikeln, Berichten, Anträgen, Tagebüchern, Notizen oder Dokumentationen sind wichtige Informationen enthalten, deren Fortbestehen und Lesbarkeit gewährleistet sein müssen. Auch Beschreibungen von anderen Dateien und Datensätzen oder des gesamten Projektes können als Textdokumente vorliegen.

Die Mehrzahl der Dokumente besteht aus strukturiertem Text, nämlich Sätzen, Absätzen, Seiten, Fußnoten und Kapiteln, und kann Formatierungsangaben, wie verschiedene Schriftgrößen, Fett- oder Kursivschreibung enthalten. Zusätzlich können Medien, wie Bilder, Tabellen oder Videos in die Dokumente integriert sein. 

Da dasselbe Dokument auf verschiedenen Systemen unterschiedlich dargestellt werden kann, kann die Speicherung von Textdokumenten problematisch sein. Insbesondere wenn bestimmte Formatierungen von Textelementen mit einer Bedeutung verbunden sind und die Authentizität des Erscheinungsbildes, also das Aussehen des Dokumentes wichtig ist, ist bei der Speicherung besondere Aufmerksamkeit erforderlich.


\subparagraph{Langzeitformate} Finalisierte Dokumente mit Formatierungsangaben können im Format PDF/A gespeichert werden. Dieses Format erlaubt eine konsistente Darstellung des Dokumentes auf verschiedenen Systemen, verhindert aber auch eine nachträgliche Bearbeitung. Nähere Informationen sind im Abschnitt über PDF-Dokumente ab Seite \pageref{pdf-dokumente} zu finden.

Textdokumente mit Formatierungsangaben, bei denen auch weiterhin eine Bearbeitung möglich sein soll, sollten in einem offenen auf XML basierenden Format gespeichert werden, wie beispielsweise DOCX oder ODT. Ersteres ist das Standardformat, das in Microsoft Word seit 2007 verwendet wird und auch von Microsoft entwickelt wurde. Letzteres ist das Format für Textdokumente, welches in OpenOffice oder LibreOffice verwendet wird. ODT ist ein Teil vom OpenDocument Format (ODF) und wurde von einem technischen Komitee unter der Leitung der \emph{Organization for the Advancement of Structured Information Standards} (OASIS) entwickelt. Die Darstellung von DOCX- oder ODT-Dokumenten kann jedoch von System zu System unterschiedlich ausfallen, wenn beispielsweise bestimmte Schriftarten fehlen. Gegebenfalls kann das Dokument parallel im Format PDF/A gespeichert werden. 

Bearbeitbare Textdokumente ohne Formatierungsangaben werden am besten als TXT-Datei gespeichert. Neben diesem einfachen, reinen Textformat (\emph{plain text}) gibt es weitere textbasierte Formate, die auf eine bestimmte Weise strukturiert sind oder eine Auszeichnungssprache verwenden. Es handelt sich dabei um sogenannte Textdateien, die im Gegensatz zu binären Formaten darstellbare Zeichen enthalten und in Abhängigkeit ihrer Strukturierung unterschiedliche Dateiformate beschreiben. Beispielsweise werden mit Hilfe von CSV-Dateien Tabellen oder mit PLY-Dateien 3D-Inhalte gespeichert. Diese Formate werden in den Abschnitten "`Tabellen"' und "`3D und Virtual Reality"' ab Seite \pageref{tabellen} bzw. \pageref{3D} behandelt. Für textuelle Inhalte gibt es spezialisierte Formate, wie beispielsweise SGML-, XML- oder HTML-Dateien. Die Archivierung dieser Dateien, die weit verbreiteten Konventionen folgen, ist unproblematisch, bedarf jedoch zusätzlicher Dateien, die die verwendete Struktur beschreiben, wie beispielsweise die sogenannte Dokumenttypdefinition (DTD, Document Type Definition) oder ein XML Schema (XSD, XML Schema Definition). Auch andere Textdateien mit spezieller Strukturierung können archiviert werden, wenn die Struktur im Dokument oder in einer separaten Datei erläutert und mitarchiviert wird.

Alle Textdokumente sollten Unicode für die Zeichenkodierung verwenden, wobei UTF-8 ohne BOM besonders empfohlen wird, falls keine speziellen Anforderungen dagegen sprechen. Wenn die verwendete Zeichenmenge es erlaubt, ist ASCII ebenfalls geeignet.

Hinweis: Eingebettete Bilder oder andere Medien sollten zusätzlich separat gespeichert werden. Außerdem muss beachtet werden, dass Links oder dynamische Inhalte, nicht immer dauerhaft erhalten bleiben.

\begin{center}
	\begin{longtable}{l L{0.2\textwidth} p{0.6\textwidth}}
			\toprule 
		\multicolumn{2}{l}{Format} & Begründung \\
		\midrule \endfirsthead
		\multicolumn{3}{l}{\footnotesize Fortsetzung der vorhergehenden Seite}\\
		\toprule
		\multicolumn{2}{l}{Format} & Begründung \\ \midrule \endhead
		\bottomrule \multicolumn{3}{r}{{\footnotesize Fortsetzung auf der nächsten Seite}} \\
		\endfoot
		\bottomrule 
		\endlastfoot
		
		\multirow{5}{*}{\color{ForestGreen} \LARGE \checkmark} & PDF/A & Wenn neben dem Inhalt auch das Aussehen des Dokumentes erhalten bleiben soll und die Bearbeitung des Dokuments abgeschlossen ist, eignet sich PDF/A am besten. Nähere Informationen sind in dem Abschnitt über PDF-Dokumente ab Seite \pageref{pdf-dokumente} zu finden.\\
		  & ODT & ODT basiert auf XML und ist Teil vom OpenDocument Format. Damit können bearbeitbare Dokumente mit Formatierungsangaben gespeichert werden. ODF verwendet standardmäßig UTF-8 und erlaubt das Einbetten von Fonts.\\
		  & DOCX & DOCX ist das auf XML basierende Format von Microsoft, das ebenfalls bearbeitbare Dokumente mit Formatierungsangaben speichern kann. DOCX verwendet standardmäßig UTF-8 und erlaubt das Einbetten von TrueType-Fonts.\\
		  & TXT und \emph{plain text} & Das Format eignet sich für reinen Text ohne Formatierungsangaben, wie Kursiv-, Fettschreibung oder Schriftgrößen. Die Zeichen sollten in UTF-8 ohne BOM kodiert sein.\\ 
		  & strukturierter Text & Alle anderen textbasierten Formate, wie beispielsweise valide SGML-, XML- oder HTML-Dateien können ebenfalls archiviert werden. Für SGML und XML ist zusätzlich die DTD-Datei oder ein XML Schema erforderlich. Anders strukturierte textbasierte Dateien benötigen eine Erläuterung der Struktur innerhalb der Datei oder als zusätzliche separate Datei. Die Zeichen sollten in UTF-8 ohne BOM kodiert sein.\\ \cmidrule(r){1-3}
		\multirow{2}{*}{$\color{BurntOrange} \thicksim$} & RTF & RTF ist ein proprietäres Format von Microsoft für den Datenaustausch, das von vielen Programmen unterstützt wird. Wegen möglichen Kompatibilitätsproblemen sollte DOCX oder ODT bevorzugt werden.\\ 
			& SXW & SXW ist ein Vorgängerformat von ODT, weshalb letzteres auch bevorzugt werden sollte.\\ \cmidrule(r){1-3}
		\multirow{2}{*}{\LARGE \boldmath$\color{BrickRed} \times$}& DOC & Das DOC-Format von Microsoft eignet sich nicht zur Archivierung, da es proprietär ist und die Inhalte nicht textbasiert gespeichert werden.\\
			& PDF & Für die Archivierung wurde speziell das Format PDF/A entwickelt, weshalb dieses verwendet werden sollte.\\
 		\bottomrule    
	\end{longtable}
\end{center}


\subparagraph{Dokumentation} Metadaten für Textdokumente können in vielen Fällen direkt in das Dokument eingetragen werden. Beispielsweise als Deckblatt oder in dafür vorgesehenen Teilen von strukturierten Dokumenten. Zusätzlich können einige Informationen als Dokumenteigenschaften in der Datei gespeichert werden.

Neben den allgemeinen Angaben zu Einzeldateien, wie sie in dem Abschnitt Metadaten in der Anwendung ab Seite \pageref{Metadaten-anwendung} gelistet sind, benötigen Textdokumente insbesondere Angaben zur verwendeten Zeichenkodierung und eine Auflistung der Sprachen. 

Falls das Dokument publiziert wurde und eine ISBN oder einen anderen persistenten Identifikator erhalten hat, müssen diese neben den allgemeinen Angaben zur Publikation ebenfalls angegeben werden. Eingebettete Medien, wie Bilder oder Tabellen mit Formeln, sollten separat gespeichert und archiviert werden und in einer Liste weiterer Dateien aufgeführt werden. 

Wenn das Aussehen wichtig ist und ein Format verwendet wird, welches das Einbetten von Schriftarten nicht ermöglicht, müssen die verwendeten Schriftarten explizit genannt werden.

Die hier angegebenen Metadaten sind als minimale Angabe zu betrachten und ergänzen die angegebenen Metadaten für Projekte und Einzeldateien in dem Abschnitt Metadaten in der Anwendung ab Seite \pageref{Metadaten-anwendung}.


\begin{center}
		\begin{longtable}{L{0.3\textwidth} p{0.6\textwidth}}
			\toprule 
		Metadatum & Beschreibung \\
		\midrule \endfirsthead
		\multicolumn{2}{l}{\footnotesize Fortsetzung der vorhergehenden Seite}\\
		\toprule
		Metadatum & Beschreibung \\ \midrule \endhead
		\bottomrule \multicolumn{2}{r}{{\footnotesize Fortsetzung auf der nächsten Seite}} \\
		\endfoot
		\bottomrule 
		\endlastfoot
		
		Zeichenkodierung & Welches Zeichenkodierung wird verwendet? \\
		Sprache & In welchen Sprachen ist das Dokument verfasst? Sprachkennungen nach ISO 639 angeben.\\
		Identifikator & Wenn das Dokument bereits veröffentlicht wurde und eine ISBN oder einen anderen persistenten Identifikator erhalten hat, sollte dieser angegeben werden.\\
		weitere Dateien & Liste von eingebetteten Medien, die zusätzlich separat gespeichert wurden. Liegt eine Dokumentationsdatei für das Dokument vor, muss diese ebenfalls genannt werden.\\
		Schriftarten & Angabe der verwendeten Schriftarten (Fonts), für Dokumente ohne eingebettete Fonts.\\
	  \bottomrule
	\end{longtable}
\end{center}

Weitere Metadaten sind methodenabhängig und können in den jeweiligen Abschnitten nachgelesen werden.