\hyphenation{
Au-dio-stream
Bit-ra-te
FFmpeg
}
Videos sind in der altertumswissenschaftlichen Forschung vergleichsweise wenig verbreitet und dienen vor allem der Dokumentation von Arbeitsabläufen, Visualisierung von Ergebnissen oder publikumswirksamen Medienauftritten.

Eine digitale Videodatei enthält sowohl visuelle als auch auditive Inhalte. Diese sind jeweils in einem eigenen Format, dem sogenannten Codec, gespeichert und werden in einem Containerformat zusammengeführt. Je nachdem welches Containerformat verwendet wird, können weitere Informationen, wie Metadaten oder Untertitel gespeichert werden. Zu beachten ist, dass es sich bei manchen Containerformaten gleichzeitig um Codecs handeln kann. Die Digitalisierung von analogem Film- und Videomaterial ist nicht Gegenstand dieser Empfehlungen.

\subparagraph{Langzeitformate} Bei der Auswahl für die Langzeitarchivierung von digitalen Videos muss nicht nur ein passendes Format, sondern auch ein geeigneter Codec gefunden werden. Dabei gelten für die Auswahl des Codecs und des Containerformats die Kriterien, die für die Wahl von Dateiformaten ab Seite \pageref{dateiformate} beschrieben werden: Es sollte sich um einen weit verbreiteten, möglichst nicht proprietären und offen dokumentierten Standard handeln, der verlustfreie oder gar keine Kompression anwendet. Außerdem sollte darauf geachtet werden, dass das gewählte Format die Speicherung aller relevanten Elemente des Videos, wie beispielsweise Untertitel, unterstützt.

Das Containerformat Matroska (MKV) erfüllt alle Anforderungen an ein Archivformat. Es wird seit 2003 explizit als offenes Containerformat entwickelt, das modernen Ansprüchen genügt und viele verschiedene Codecs enthalten sowie zusätzliche Elemente, wie Untertitel, speichern kann. Das Format basiert auf einer binären Variante von XML, nämlich EBML (Extensible Binary Meta Language), was eine zukünftige flexible Erweiterung erlaubt, jedoch auch sicherstellt, dass ältere Programme weiterhin damit umgehen können. Zusätzlich ist das Format fehlertolerant und kann bis zu einem gewissen Grad auch beschädigte Dateien wiedergeben. Zusammen mit den Codecs FFV1 für Video und FLAC für Audio wird eine Standardisierung bei der IETF angestrebt, weshalb diese Kombination empfohlen wird.

Motion JPEG 2000 (MJ2) ist ein speziell für die Archivierung entwickeltes Containerformat, das unter ISO/IEC 15444-3:2007 zertifiziert und als MIME-Type video/mj2 registriert ist. Es verwendet einen eigenen Codec, JPEG 2000, der jedes Einzelbild verlustfrei mit JPEG 2000 komprimiert. Zugehörige auditive Inhalte werden von Motion JPEG 2000 ebenfalls unterstützt. Das Format ist rechenintensiv und nicht vollständig lizenzfrei, weshalb es besser sein kann den JPEG 2000 Codec in einem anderen Containerformat, wie beispielsweise MXF zu verwenden. Außerdem ist zu Beachten, dass MJ2 auch verlustbehaftete Varianten kennt, die für die Archivierung nicht verwendet werden sollten. Motion JPEG 2000 darf nicht mit dem MPEG-Containerformat oder dem verlustbehafteten Motion JPEG (M-JPEG) verwechselt werden.

Die Familie der MPEG-Containerformate wird von der Moving Picture Experts Group (MPEG) entwickelt und ISO/IEC zertifiziert. Es gibt mehrere Generationen von Standards, darunter MPEG-1 (MPG), MPEG-2 (MPG) oder MPEG-4 (MP4), die entsprechend unterschiedliche Eigenschaften aufweisen. Die Standards beschreiben nicht nur Containerformate, sondern auch Codecs für Video und Audio. MPEG-1 ist seit 1991 unter ISO/IEC 11172 zertifiziert und beschreibt einen Standard für verlustbehaftete Komprimierung von Audio- und Videodaten, der eine gute Qualität und weite Verbreitung hat. MPEG-1 ist insbesondere für das Audioformat MP3 bekannt, eignet sich aber nicht für die Langzeitarchivierung. Der ebenfalls verlustbehaftete Nachfolger, MPEG-2, ist unter ISO/IEC 13818 zertifiziert und hat eine weite Verbreitung in dem Containerformat MPEG-TS (TS) für Rundfunkübertragungen oder auf DVDs gefunden. MPEG-2 kann nur als Langzeitformat für Dateien verwendet werden, die ursprünglich in diesem Format entstanden sind und nicht mehr bearbeitet werden. 

Der unter ISO/IEC 14496 zertifizierte MPEG-4-Standard weist eine höhere Effizienz in der Komprimierung und eine höhere Qualität der Videodaten als seine Vorgänger auf. Er verwendet den Codec H.264/MPEG-4 AVC, der meistens verlustbehaftet verwendet wird, jedoch auch eine kaum verwendete verlustfreie Variante hat. Dieser Codec bietet eine hohe Bildqualität, weshalb er eine weite Verbreitung auf Blu-ray Discs, als Aufnahmeformat oder für das Streaming über das Internet gefunden hat. Teil 14 der MPEG-4-Spezifikation definiert das Containerformat MP4, das auch als MIME-Type video/mp4 registriert ist. Wird dieses Format mit dem H.264-Codec verwendet, kann es zur Langzeitarchivierung verwendet werden, wenn dies entweder dem Ursprungsformat der Videodatei entspricht oder verlustfreie Kompression verwendet wird.

Das Containerformat Material eXchange Format (MXF) ist ein offenes Format, das für den Austausch von auditiven und visuellen Inhalten entwickelt wurde und durch eine Reihe von Standards der Society of Motion Picture and Television Engineers (SMPTE) beschrieben wird. MXF wird vor allem im Bereich Kino und Fernsehen verwendet, wo jeweils spezialisierte Varianten verwendet werden, die nicht immer vollständig untereinander kompatibel sind. MXF ist als MIME-Type application/mxf registriert. Zusammen mit der verlustfreien Variante des Codecs JPEG 2000 stellt MXF eine für die Langzeitarchivierung geeignete Variante dar.

Es gibt eine große Zahl an weiteren Containerformaten, wie etwa MOV und ASF/WMV. Sie eignen sich aber nicht für die langfristige Videoarchivierung, da es sich dabei um proprietäre Formate handelt. Bei Audio Video Interleave (AVI) handelt es sich ebenfalls um ein proprietäres Format, es wird jedoch von einigen Archiven als Archivformat verwendet, da es sich um ein einfaches und robustes Format mit einer großen Verbreitung handelt. Die Formate Ogg und Flash wurden vor allem für das Streaming von Videos entwickelt und sind aufgrund der verwendeten verlustbehafteten Codecs nicht als Langzeitformate geeignet.

Die Wahl des Containerformates hängt von dem zu verwendenden Codec ab, da nicht jeder Codec in jedem Container verwendet werden kann. Bekannte und verbreitete Codecs werden nachfolgend beschrieben. 

Der Nachteil bei unkomprimierten Videodateien ist, dass sie sehr viel Speicherplatz (teilweise mehrere GB pro Filmminute) beanspruchen und auch für die Wiedergabe entsprechend leistungsfähige Hardware benötigen. Eine Alternative ist die Verwendung von Codecs mit verlustfreier Kompression, wie etwa FFV1, oder HuffYUV und Lagarith. Der FFmpeg Video Codec 1 (FFV1) wurde im Rahmen des FFmpeg-Projektes entwickelt, ist offen dokumentiert, hat eine weite Verbreitung gefunden und komprimiert sehr gut und schnell. Diese Eigenschaften machen ihn zu einem empfehlenswerten Codec für die Langzeitarchivierung. HuffYUV und dessen Abspaltung Lagarith wurden eigens für Windows-Systeme entwickelt, weshalb diese nicht für die langfristige Archivierung empfohlen werden können. 

Einige Codecs können sowohl verlustfrei als auch verlustbehaftet komprimieren. Dazu gehören: Dirac/Schroedinger, JPEG 2000, H.264/MPEG-4 AVC und H.265/MPEG-H (HEVC). Dirac/Schroedinger wurde beim BBC entwickelt, ist jedoch nicht sehr performant und wird von nur wenigen Programmen unterstützt, weshalb er aktuell nicht für die Archivierung empfohlen werden kann. Bei der Verwendung von JPEG 2000 muss darauf geachtet werden, dass die verlustfrei komprimierende Variante verwendet wird, da dieser Standard auch verlustbehaftet komprimieren kann. H.264/MPEG-4 AVC kann in einer verlustfreien Variante verwendet werden, wenn das entsprechende Programm dies unterstützt. Die verlustbehaftete Variante von H.264/MPEG-4 AVC kann nur für die Archivierung von Dateien verwendet werden, die in diesem Codec entstanden sind und nicht mehr bearbeitet werden. Seit 2013 gibt es den Codec High Efficiency Video Coding (HEVC, auch H.265/MPEG-H), der als Nachfolger für H.264 gedacht ist, jedoch noch nicht vollständig spezifiziert ist. Daher und aus patentrechtlichen Gründen ist dieser Codec nicht für die Archivierung zu empfehlen.

Neben den oben erwähnten Codecs gibt es weitere, jedoch verlustbehaftete Codecs, wie M-JPEG, Theora und DV. Im Gegensatz zu JPEG 2000 ist der Codec Motion JPEG (M-JPEG) nicht für die langfristige Archivierung von Videodaten geeignet, da dieser keine verlustfreie Komprimierung unterstützt. Theora ist der für Ogg-Dateien entwickelte Codec. DV ist ein Codec für digitale Videos auf Videokasetten, der mit der Ablösung von Kasetten durch andere Speichermedien obsolet wird. Theora und DV sind keine geeigneten Codecs für die Langzeitarchivierung.

Hinweise auf geeignete Audioformate und dazugehöriger Metadaten sind in dem Kapitel Audio ab Seite \pageref{audio} zu finden.

\pagebreak
\begin{center}
	\begin{longtable}{l L{0.2\textwidth} p{0.6\textwidth}}
			\toprule 
		\multicolumn{2}{l}{Format} & Begründung \\
		\midrule \endfirsthead
		\multicolumn{3}{l}{\footnotesize Fortsetzung der vorhergehenden Seite}\\
		\toprule
		\multicolumn{2}{l}{Format} & Begründung \\ \midrule \endhead
		\bottomrule \multicolumn{3}{r}{{\footnotesize Fortsetzung auf der nächsten Seite}} \\
		\endfoot
		\bottomrule 
		\endlastfoot
		
		{\color{ForestGreen} \LARGE \checkmark} & Matroska (MKV) & Ein offenes Containerformat, das eine große Bandbreite von Codecs und ergänzenden Inhalten unterstützt. Für die Archivierung können die Codecs FFV1 für Video und FLAC für Audio empfohlen werden. Weitere geeignete Codecs für Matroska sind H.264/MPEG-4 AVC und MPEG-2. \\ \cmidrule(r){1-3}
		\multirow{4}{*}{$\color{BurntOrange} \thicksim$} & Motion JPEG 2000 (MJ2) & Motion JPEG 2000 verwendet den Codec JPEG 2000 und ist unter ISO/IEC 15444-3:2007 zertifiziert. \\
			& MP4 & Der unter ISO/IEC 14496 zertifizierte MPEG-4-Standard verwendet den Codec H.264/MPEG-4 AVC, der meistens verlustbehaftet verwendet wird, jedoch auch verlustfrei komprimieren kann. Das Containerformat MP4 ist als MIME-Type video/mp4 registriert. Wird dieses Format mit dem H.264-Codec verwendet, kann es zur Langzeitarchivierung verwendet werden, wenn dies entweder dem Ursprungsformat der Videodatei entspricht oder verlustfreie Kompression verwendet wird.\\ 
			& MXF & Das Containerformat Material eXchange Format (MXF) ist ein offenes Format, das durch eine Reihe von Standards der SMPTE beschrieben wird. MXF ist als MIME-Type application/mxf registriert. Zusammen mit der verlustfreien Variante des Codecs JPEG 2000 stellt MXF eine für die Langzeitarchivierung geeignete Variante dar.\\ 
			& MPEG-2 & MPEG-2 ist unter ISO/IEC 13818 zertifiziert und komprimiert verlustbehaftet. MPEG-2 kann nur als Langzeitformat für Dateien verwendet werden, die ursprünglich in diesem Format entstanden sind und nicht mehr bearbeitet werden. \\ \cmidrule(r){1-3}
		\multirow{6}{*}{\LARGE \boldmath$\color{BrickRed} \times$}& MPEG (weitere Varianten) & Mit Ausnahme von MPEG-4 und MPEG-2 können andere MPEG-Varianten nicht für die Langzeitarchivierung empfohlen werden.\\
			& AVI & Audio Video Interleave ist ein von Microsoft entwickeltes proprietäres Format mit einer weiten Verbreitung. Es eignet sich nicht für die Langzeitarchivierung.\\
		  & MOV & Ein verbreitetes proprietäres Format von Apple, das in QuickTime verwendet wird und nicht für die Langzeitarchivierung geeignet ist.\\ 
			& ASF/WMV & Ein proprietäres Format von Microsoft, das nicht für die Langzeitarchivierung geeignet ist.\\ 
			& Ogg & Ein von Xiph entwickeltes und offenes Format für das Streaming von Videos, das jedoch nicht für die Archivierung geeignet ist.\\
			& Flash & Ein von Macromedia und dann von Adobe entwickeltes Format für das Streaming von Videos. Es ist nicht für die Langzeitarchivierung geeignet. \\
 		\bottomrule    
	\end{longtable}
\end{center}


\subparagraph{Dokumentation} Neben den allgemeinen minimalen Angaben zu Einzeldateien, wie sie in dem Abschnitt Metadaten in der Anwendung ab Seite \pageref{Metadaten-anwendung} gelistet sind, werden für Videos weitere Angaben benötigt, die insbesondere technische Details dokumentieren.

Die technischen Angaben zu Bildformat, Seitenverhältnis, Bildfrequenz, Bitrate und Codec werden zur korrekten Wiedergabe der Datei benötigt. Angaben zu Länge, Tonkanälen, Profilen und weiteren Inhalten sind zur Prüfung auf Vollständigkeit der Datei erforderlich.

Bereits eingebettete Metadaten, wie beispielsweise Exif, oder Bestandteile im Containerformat sollten behalten und archiviert werden. Am besten werden sie in eine eigene Text- oder XML-Datei transferiert und getrennt gespeichert.

\begin{center}
	\begin{longtable}{L{0.3\textwidth} p{0.6\textwidth}}
			\toprule 
		Metadatum & Beschreibung \\
		\midrule \endfirsthead
		\multicolumn{2}{l}{\footnotesize Fortsetzung der vorhergehenden Seite}\\
		\toprule
		Metadatum & Beschreibung \\ \midrule \endhead
		\bottomrule \multicolumn{2}{r}{{\footnotesize Fortsetzung auf der nächsten Seite}} \\
		\endfoot
		\bottomrule 
		\endlastfoot

		Länge & Dauer des Videos. Diese Angaben sollten konform zu ISO 8601 erfolgen. Beispiel: P3Y6M4DT12H30M5S (3 Jahre, 6 Monate, 4 Tage, 12 Stunden, 30 Minuten und 5 Sekunden) oder T2H2M (2 Stunden und 2 Minuten)\\
		Beteiligte & Angabe der Beteiligten, wie etwa Autor, Regisseur, Darsteller, Interviewpartner etc.\\
		Bildgröße und Seitenverhältnis & Maße des Bildes gemessen in Pixeln, z.B. $1280px \times 720px$, und Angabe des Seitenverhältnisses für die korrekte Darstellung im Verhältnis Breite zu Höhe, z.B.: 16:9\\ 
		Bildfrequenz & Angabe der Bildfrequenz in Bildern pro Sekunde, z.B. 25\\
		Bitrate Video & Angabe der Datenrate in Bits pro Sekunde, z.B. 863 kbps\\
		Farbraum & Angabe des verwendeten Farbraums, z.B. YUV \\
		Farbtiefe & z.B. 8 bit oder 10 bit \\
		Farbunterabtastung & Angabe der verwendeten Farbunterabtastung, z.B. 4:2:0 \\
		Videocodec & Name und Version des verwendeten Videocodecs und welche Kompression verwendet wird\\
		Audiostreams & Anzahl der enthaltenen Audiostreams \\
		Audiocodec & Angaben zu den verwendeten Audiocodecs je enthaltenen Audiostreams\\
		Bitrate Audio &  Angabe der Datenrate in Bits pro Sekunde je enthaltenen Audiostreams, z.B. 666 kbps\\
		Abtastrate & Angabe der Abtastrate in Hertz je enthaltenen Audiostreams, z.B. 44.1 kHz\\
		Tonkanäle & Angabe über Anzahl der Tonkanäle je enthaltenen Audiostreams \\
		Profil & Wenn ein bestimmtes Videoprofil für ein Containerformat verwendet wurde, bitte angeben.\\
		Weitere Inhalte & Angabe über weitere Inhalte, die in dem Containerformat enthalten sind oder als zusätzliche Datei vorliegen, wie beispielsweise Untertitel\\
		Aufnahmegerät & Herstellername und Modell des Aufnahmegeräts (z.B. eine Kamera)\\
		Software & Name und Versionsnummer der Software, mit der das Video aufgenommen, erstellt oder bearbeitet wurde, wie z.B. Adobe Premiere Pro CC (2015.1)\\
	  \bottomrule
	\end{longtable}
\end{center}

Weitere Metadaten sind methodenabhängig und können in den jeweiligen Abschnitten nachgelesen werden.