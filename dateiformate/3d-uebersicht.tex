Im Gegensatz zu statischen zweidimensionalen Bildern können dreidimensionale Repräsentationen von Objekten aus jeder Richtung betrachtet, skaliert und rotiert werden. Ein Punkt in einem 3D-Modell wird von seiner Lage auf der x-, der y- und der z-Achse eines kartesischen Koordinatensystems beschrieben, wobei die z-Achse in diesem Zusammenhang üblicherweise die Tiefe, seltener die Höhe, angibt.

Virtual Reality (Virtuelle Realität) bezeichnet digitale dreidimensionale Welten, mit welchen in Echtzeit interagiert werden kann.

3D-Inhalte können auf unterschiedliche Weise entstehen: durch manuelle Modellierung, wie Rekonstruktionen von Gebäuden, durch Aufnahme, wie etwa einem 3D-Scan von Objekten, oder durch automatisierte Berechnung aus Fotos, wie etwa Photogrammetrie oder Structure from Motion. Von der Entstehungsweise hängen weitere Angaben für die Dokumentation ab, die über die hier angegebenen hinausgehen. Zusätzlichen Angaben sind in den jeweiligen Abschnitten in dem Kapitel Forschungsmethoden ab Seite \pageref{methoden} zu finden, wobei vor allem die Abschnitte Bauforschung, Geodäsie, Geodatenanalyse und Materialaufnahme von Interesse sind. Außerdem bieten die Ergebnisse des Projektes 3D ICONS umfangreiche Informationen zur Dokumentation von 3D-Aufnahmemethoden und die anschließende Verarbeitung. 

\subparagraph{Langzeitformate} 3D-Inhalte sollten in einem offen dokumentierten, textbasierten Format (ASCII) gespeichert werden. Dies ermöglicht bei Bedarf die Rückentwicklung unabhängig von einem bestimmten Programm.

Das zu verwendende Format hängt von der Entstehungsweise und dem Zweck des 3D-Inhaltes ab, da unterschiedliche Formate unterschiedliche Eigenschaften und Elemente speichern, wie beispielsweise Geometrie, Texturen, Lichtquellen oder Standpunkt und Bildausschnitt (auch Viewport genannt). Eine Übersicht über die Speichereigenschaften von den hier empfohlenen 3D-Formaten wird in Tabelle \ref{tab:3Deigenschaften} auf Seite \pageref{tab:3Deigenschaften} gegeben.

Das vom Web3D Consortium entwickelte Format X3D (eXtensible 3D Graphics) ist seit 2006 unter ISO/IEC 19775/19776/19777 zertifiziert und eignet sich sowohl zur Speicherung von einzelnen 3D-Modellen, als auch komplexer 3D-Inhalte, wie etwa Virtual Reality. Es ist das Nachfolgeformat von dem seit 2007 unter ISO/IEC 14772-1 zertifizierten VRML-Format und sollte diesem daher vorgezogen werden. 
	
Ein weiteres für komplexe 3D-Inhalte geeignetes Format ist das von der Khronos Group entwickelte COLLADA (collaborative design activity, DAE), das seit 2012 unter ISO 17506 standardisiert ist.

Das U3D-Format bietet einen ähnlichen Funktionsumfang wie X3D und COLLADA und ist insbesondere für das Teilen von 3D-Inhalten in PDF"=Dokumenten gedacht. Es ist nicht für die Langzeitarchivierung geeignet. 

Die Formate OBJ, PLY und STL eignen sich nicht für komplexe Szenen mit Lichtquellen oder gar Animationen, bieten aber alle Eigenschaften, um die visuellen Oberflächeneigenschaften eines 3D-Objektes zu speichern.

Aus dem CAD-Bereich stammt das Format DXF, welches neben 2D-Inhalten auch 3D-Inhalte speichern kann. Dieses Format sollte nur verwendet werden, wenn die 3D-Inhalte mit CAD-Software erstellt wurden. In der Industrie werden als alternative Formate auch IGES und STEP verwendet.

Nicht für die Langzeitarchivierung geeignet sind programmspezifische oder binäre Formate, wie beispielsweise FBX, 3DS, MAX, SKP, BLEND, PRC oder NXS.

Um zukünftigen Nutzern einen schnellen Überblick über die 3D-Inhalte zu bieten, ist die zusätzliche Speicherung von Bild- oder Videodateien, die einen ersten Eindruck des Modells vermitteln, zu empfehlen.

Hinweis: Bei der Konvertierung von einem 3D-Format in ein anderes können bestimmte Informationen verloren gehen, wenn sie von dem Zielformat nicht unterstützt werden. Zusätzlich zu dem gewählten Archivierungsformat sollten die originalen Quelldateien, verwendete Texturdateien, Visualisierungen und alle weiteren Dateien, die für die Benutzung und das Verständnis des 3D-Modells relevant sind aufgehoben werden. In den entsprechenden Abschnitten in dem Kapitel Forschungsmethoden ab Seite \pageref{methoden} sind nähere Details zu finden.

Die Abkürzungen in der folgenden Tabelle geben die unterstützten Eigenschaften der 3D-Formate an: DG~= Drahtgittermodell, P~= Parametrisch, F~= Farbe, X~= Textur mittels Bild, B~= Bumpmapping, M~= Material, V~= Viewport und Kamera, L~= Lichtquellen, T~= Transformationen, G~= Gruppierung. Diese Begriffe werden in dem Abschnitt Vertiefung ab Seite \pageref{3d:vertiefung} erläutert.

\begin{center}
	\begin{longtable}{l L{0.2\textwidth} p{0.6\textwidth}}
			\toprule 
		\multicolumn{2}{l}{Format} & Begründung \\
		\midrule \endfirsthead
		\multicolumn{3}{l}{\footnotesize Fortsetzung der vorhergehenden Seite}\\
		\toprule
		\multicolumn{2}{l}{Format} & Begründung \\ \midrule \endhead
		\bottomrule \multicolumn{3}{r}{{\footnotesize Fortsetzung auf der nächsten Seite}} \\
		\endfoot
		\bottomrule 
		\endlastfoot
		
		\multirow{4}{*}{\color{ForestGreen} \LARGE \checkmark} & X3D & Das X3D-Format wurde vom Web3D Consortium entwickelt und ist seit 2006 ISO-zertifiziert. Es darf nicht mit dem proprietären Format 3DXML verwechselt werden. Speichert: DG, P, F, X, B, M, V, L, T, G und Animationen.\\
		  & COLLADA & COLLADA (\emph{.dae}) wurde von der Khronos Group entwickelt und ist seit 2012 ISO-zertifiziert. Es basiert auf XML und speichert: DG, P, B-Rep, F, X, B, M, V, L, T, G und Animationen. \\
		  & OBJ & Das offene OBJ-Format wurde von Wavefront Technologies entwickelt und hat eine weite Verbreitung gefunden. Material oder Texturen werden in gesonderten MTL- oder JPG-Dateien gespeichert, die auch archiviert werden müssen. Speichert: DG, P, F, X, B, M und G.\\ 
		  & PLY & Das Polygon File Format (acuh Stanford Triangle Format) ist ein einfaches Format, das an der Universität Stanford für Daten von 3D-Scannern entwickelt wurde. Mittels Erweiterungen kännten weitere Eigenschaften gespeichert werden, die aber nicht von jedem Programm unterstützt werden. Für die Langzeitarchivierung sollte die ASCII-Version verwendet werden. Speichert: DG, F, X, B und M.\\
		%\cmidrule(r){1-3}
		\multirow{3}{*}{$\color{BurntOrange} \thicksim$} & VRML & Die Virtual Reality Modeling Language ist das Vorgängerformat von X3D. Die jüngste Version wurde 1997 unter dem Namen VRML97 veröffentlicht. Speichert: DG, P, F, X, B, M, V, L, T, G und Animationen.\\ 
		  & STL & Das Format STL wurde von 3D Systems entwickelt. Es steht für \emph{stereolithography} oder \emph{Standard Tessellation Language} und findet weite Verbreitung im Bereich von 3D-Druckern und digitaler Fabrikation. Die ASCII-Variante des STL-Formats kann nur die Geometrie eines 3D-Modells speichern. Die Binärvariante des Formates ist weniger speicherintensiv und kann mit einer entsprechenden Erweiterung auch Farben des 3D-Modells speichern, ist aber nicht für die Langzeitarchivierung geeignet.\\
		  & DXF & Das von Autodesk entwickelte Drawing Interchange Format stammt aus dem CAD-Bereich und sollte nur für 3D-Inhalte verwendet werden, die mit CAD-Software erstellt wurden. Speichert: DG, P, CSG, B-Rep, F und G.\\
		\cmidrule(r){1-3}
		\multirow{8}{*}{\LARGE \boldmath$\color{BrickRed} \times$}& U3D & Das Universal 3D Format ist ein 2005 von der Ecma standardisiertes 3D-Format, das vom 3D Industry Forum mit Mitgliedern wie Intel und Adobe Systems entwickelt wurde. Dieses Format ist nur für die Integration von 3D-Modellen in ein 3D-PDF relevant. Speichert: DG, F, X, B, V, L, T, G und Animationen.\\ 
		 & FBX & Ein proprietäres Format von Autodesk, für den Datenaustausch mit anderen kommerziellen 3D-Programmen.\\
		 & 3DS, MAX  & Proprietäre binäre Formate von AutoDesk.\\
		 & SKP & Natives Format von Google SketchUp.\\
		 & BLEND & Natives binäres Format von Blender.\\
		 & PRC & PRC (Product Representation Compact) ist ein stark komprimiertes Format für den CAD-Bereich und für die Verwendung in 3D-PDFs, das unter ISO 14739-1 standardisiert ist.\\
		 & PDF & 3D-PDFs mit eingebetteten Modellen in U3D oder PRC eignen sich für einen unkomplizierten Datenaustausch, jedoch nicht für die Langzeitarchivierung. \\
		 & NXS & Nexus ist ein von CNR-ISTI offen entwickeltes Format, für die Web-Visualisierung von 3D-Modellen.\\
		\bottomrule    
	\end{longtable}
\end{center}


%PRC is a highly compressed format that facilitates the storage of different representations of a 3D model. For example, you can save only a visual representation that consists of polygons (a tessellation), or you can save the model's exact geometry (B-rep data). Varying levels of compression can be applied to the 3D CAD data when it is converted to the PRC format using the 3D PDF Converter plugin for Acrobat. The 3D data stored in PRC format in a PDF is interoperable with Computer-Aided Manufacturing (CAM) and Computer-Aided Engineering (CAE) applications.


\subparagraph{Dokumentation} Für die Dokumentation von 3D-Inhalten sollten die Leitsätze der \emph{\href{www.londoncharter.org}{Londoner Charta}} berücksichtigt und eingehalten werden. Dabei ist insbesondere \emph{"`Leitsatz 4: Dokumentation"'} wichtig, dessen erster Absatz lautet:  

"`\emph{Es sollen genügend Informationen dokumentiert und weitergegeben werden, um das Verstehen und Bewerten der computergestützten Visualisierungsmethoden und -ergebnisse in Bezug auf die Zusammenhänge und Absichten, für die sie eingesetzt werden, zu ermöglichen}"'

Dazu gehören unter anderem die Dokumentation der Forschungsquellen, die zur Erstellung des 3D-Modells herangezogen wurden, die Dokumentation der Prozesse, die im Verlauf der Modellerstellung durchlaufen wurden, die Dokumentation der angewendeten Methoden und eine Beschreibung der Abhängigkeitsverhältnisse zwischen den unterschiedlichen Bestandteilen. 

Wenn das Dateiformat es erlaubt, sollte ein Teil der Metadaten zusätzlich dort integriert werden. Allerdings kann der Großteil der Metadaten nur extern abgelegt werden.

Die hier angegebenen Metadaten sind als minimale Angabe zu betrachten und ergänzen die angegebenen Metadaten für Projekte und Einzeldateien in dem Abschnitt Metadaten in der Anwendung ab Seite \pageref{Metadaten-anwendung}.

\begin{center}
	\begin{longtable}{L{0.3\textwidth} p{0.6\textwidth}}
			\toprule 
		Metadatum & Beschreibung \\
		\midrule \endfirsthead
		\multicolumn{2}{l}{\footnotesize Fortsetzung der vorhergehenden Seite}\\
		\toprule
		Metadatum & Beschreibung \\ \midrule \endhead
		\bottomrule \multicolumn{2}{r}{{\footnotesize Fortsetzung auf der nächsten Seite}} \\
		\endfoot
		\bottomrule 
		\endlastfoot
		
		Anzahl Eckpunkte & Aus wie vielen Eckpunkten besteht das 3D-Modell? \\
		Anzahl Polygone & Wie viele Polygone hat das 3D-Modell?  \\
		Geometrietyp & Welcher Geometrietyp wird verwendet (Drahtgittermodell, parametrisch, CSG, B-Rep etc.)? \\
		Maßstab & Welcher Maßstab liegt vor bzw. was stellt eine Einheit dar? \\
		Koordinatensystem & Wird ein geographisches Koordinatensystem oder ein beliebiges verwendet? \\
		Modellstatus & Handelt es sich bei dem Modell um das ursprünglich erzeugte und unverarbeitete Modell (den Master) oder ist es ein Modell, das mittels weiterverarbeitenden Schritten, wie Füllen von Löchern, Vereinfachung oder Glättung aus dem Master erzeugt wurde?\\
		Detaillierungsgrad (LOD) oder Auflösung & Wie detailliert ist das Modell oder welche Auflösung wurde beim 3D-Scan verwendet? \\
		Ebenen & Werden Ebenen verwendet? Wie viele? \\
		Farbe und Textur & Werden Farben oder Texturen verwendet? Wie werden diese gespeichert? Müssen zusätzliche Texturdateien archiviert werden?\\
		Material & Informationen über die Materialeigenschaften des Modells und inwieweit sie physikalisch korrekt sind. \\
		Licht & Informationen über die Lichtquellen und inwieweit sie physikalisch korrekt sind.\\
		Shader & Werden spezielle, erweiterte Shader verwendet? \\
		Animation & Ist das Modell animiert? Welche Art von Animation wird verwendet (Keyframe, motion capture)? \\
		Externe Dateien & Eine Liste aller externen Dateien, die für das Öffnen der Datei notwendig sind (z.B. Texturen)\\

	  \bottomrule
	\end{longtable}
\end{center}
Weitere Metadaten sind methodenabhängig und können in den jeweiligen Abschnitten nachgelesen werden.