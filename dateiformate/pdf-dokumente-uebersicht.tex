Das PDF (Portable Document Format) wurde 1993 von Adobe Systems entwickelt, um den Datenaustausch zu erleichtern. Es ist ein plattformunabhängiges, offenes Dateiformat, das 2008 mit der Version 1.7 als ISO-Standard zertifiziert wurde und seitdem von der ISO weiter gepflegt wird.  

Der große Vorteil von PDFs liegt darin, dass Dateiinhalte unabhängig vom Betriebssystem, dem ursprünglichen Anwendungsprogramm und der Hardwareplattform unverändert dargestellt werden. Das Aussehen von Dokumenten wird, wie bei einem analogen Ausdruck, eingefroren und kann somit angezeigt werden, wie es ursprünglich vom Autor intendiert war.  Gleichzeitig sind die Möglichkeiten zur nachträglichen Bearbeitung begrenzt, wodurch eine große Authentizität gewährleistet wird.

Für das Öffnen einer PDF-Datei gibt es verschiedene freie Anwendungen, worin ein Grund für die große Verbreitung und Akzeptanz von PDFs liegt. Viele Programme können Dateien direkt im PDF-Format speichern oder exportieren. Darüber hinaus lassen sich mit Hilfe von zusätzlich installierten Druckertreibern PDF-Dateien aus allen Programmen heraus erzeugen.

\subparagraph{Langzeitformate} Nicht jede Datei mit der Dateierweiterung .pdf  ist gleichermaßen für die Langzeitarchivierung geeignet. Für diesen Zweck  wurde das PDF/A-Format entwickelt und als ISO-Standard zertifiziert. Es beschreibt in welcher Form bestimmte Elemente in einer PDF-Datei enthalten sein müssen und welche nicht erlaubt sind. Die erste Version der PDF/A-Norm wurde nachträglich um zwei weitere, aufeinander aufbauende Normteile ergänzt. Wird das PDF/A-Format für die Langzeitarchivierung einer Datei verwendet, die nicht nur textuelle Informationen enthält, sollten nach Möglichkeit auch die ursprünglichen Ausgangsdateien in archivierungstauglichen Formaten entweder als separate Dateien oder integriert in das Container-Format PDF/A-3 mitarchiviert werden.

\begin{center}
	\begin{tabular}{l p{0.2\textwidth} p{0.6\textwidth}}
		\toprule
		\multicolumn{2}{l}{Format} & Begründung \\ \midrule
		\multirow{3}{*}{\color{ForestGreen} \LARGE \checkmark} & PDF/A-1 & \multirow{2}{*}{\parbox{0.6\textwidth}{PDF/A ist gezielt als stabiles, offenes und standardisiertes Format für die Langzeitarchivierung unterschiedlicher Ausgangsdateien entwickelt worden.}}\\
		& PDF/A-2 & \\
		& &\\
		\cmidrule(r){1-3}
		\multirow{1}{*}{$\color{BurntOrange} \thicksim$} & PDF/A-3 & PDF/A-3 ist nur dann für die Langzeitarchivierung geeignet, wenn alle eingebetteten Dateien in einem anerkannten Archivformat vorliegen.\\
		\cmidrule(r){1-3}
		\multirow{1}{*}{\LARGE \boldmath$\color{BrickRed} \times$}& andere PDF-Varianten & Viele gängige PDF-Varianten sind nicht für die Langzeitarchivierung geeignet. Stattdessen sollten entweder die Ausgangsdateien in einem passenden Format archiviert oder eine Migration in ein PDF/A-Format vorgenommen werden.\\
		\bottomrule
		\bottomrule
	\end{tabular}
\end{center}

\newpage
\subparagraph{Dokumentation}
Welche Metadaten für ein PDF-Dokument relevant sind, hängt von dessen Inhalt ab. Bei PDF-Dateien, die nur Text enthalten, sind meist weniger Informationen erforderlich, als bei PDFs, die Bilder, GIS-Karten oder 3D-Modelle enthalten. Die Metadaten können unmittelbar in einem PDF-Dokument im XMP-Format gespeichert werden.

Die hier angegebenen Metadaten sind als minimale Angabe zu betrachten und ergänzen die angegebenen Metadaten für Projekte und Einzeldateien in dem Abschnitt Metadaten in der Anwendung ab Seite \pageref{Metadaten-anwendung}.

\begin{center}
	\begin{tabular}{l p{0.6\textwidth}}
		\toprule
		Metadatum & Beschreibung \\ \midrule
Titel & Titel des Dokuments, nicht der Dateiname\\
Autor & Name des Verfassers oder Erstellers der Datei; gegebenfalls Name der Einrichtung\\
Stichwörter & Schlagworte, wie z.B. Periode, Fundstelle oder charakteristische Merkmale. Wenn vorhanden, angemessene Thesauri verwenden\\
Beziehungen & Dateien oder Ressourcen, die mit dem PDF zusammenhängen, insbesondere der Name der Originaldatei, aus der heraus ein PDF erstellt wurde\\
Anwendung & Programm, mit dem der Inhalt ursprünglich erzeugt wurde\\
Datum & Datum der Erstellung oder letzten Änderung der Datei\\
Copyright-Angaben & Angaben zur Person oder Einrichtung, die das Copyright oder die Lizenzrechte an der Datei oder deren Inhalt besitzt\\
\multicolumn{2}{l}{Zusätzliche Metadaten}\\
Kurzbeschreibung & Kurzbeschreibung über den Inhalt des Dokuments\\
Sprache & Sofern schriftliche Inhalte vorhanden sind, die Sprache angeben. Sprachkennungen nach ISO 639 angeben.\\
Verfasser der Metadaten & Name der Person, welche die Metadaten ausgefüllt hat\\
 		\bottomrule
		\bottomrule
	\end{tabular}
\end{center}

Weitere Metadaten sind abhängig vom Inhalt und der Methoden und können in den jeweiligen Abschnitten im Kapitel Forschungsmethoden ab Seite \pageref{methoden} nachgelesen werden.