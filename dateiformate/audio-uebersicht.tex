\hyphenation{
Au-dio-stream
Bit-ra-te
FFmpeg
WAVE
}
Tonaufnahmen werden in der altertumswissenschaftlichen Forschung eher selten erzeugt. Sie werden für die Aufnahme von Interviews, rekonstruierten Musikinstrumenten im Rahmen der Musikarchäologie oder akustischen Eigenschaften von archäologischen Orten erstellt. Sogar Hörbücher altertumswissenschaftlichen Inhalts existieren mittlerweile.

Eine Audiodatei enthält auditive Inhalte, die durch zusätzliche Komponenten wie Kapiteleinteilungen angereichert sein können. Die zusammenfassende Speicherung aller Inhalte erfolgt in einem Containerformat. Die auditiven Inhalte selbst werden nochmals in einem eigenen Format, dem sogenannten Codec, gespeichert. Je nachdem welches Containerformat verwendet wird, können weitere Informationen, wie Metadaten oder Transkriptionen gespeichert werden.


\subparagraph{Langzeitformate} Da es sich bei den Dateiformaten für digitale Audiodateien um Containerformate handelt, muss bei der Auswahl für die Langzeitarchivierung nicht nur ein passendes Format, sondern auch ein geeigneter Codec gefunden werden. Dabei gelten für die Auswahl des Codecs und des Containerformats die Kriterien, die für die Wahl von Dateiformaten ab Seite \pageref{dateiformate} beschrieben werden: Es sollte sich um einen weit verbreiteten, möglichst nicht proprietären und offen dokumentierten Standard handeln, der verlustfreie oder gar keine Kompression anwendet. Außerdem sollte darauf geachtet werden, dass das gewählte Format die Speicherung aller relevanten Elemente der Audiodatei, wie etwa Kapiteleinteilungen, unterstützt.

Da viele Containerformate für Audio nur ein Codec-Format speichern können, werden diese in der Regel unter einem Namen zusammengefasst.

Für die Speicherung von unkomprimierten Audiodaten hat sich lineares PCM (Lineare Puls-Code-Modulation, auch LPCM) als Standard durchgesetzt, das von verschiedenen Containerformaten unterstützt und auch von der IASA (International Association of Sound and Audiovisual Archives) empfohlen wird. Der einzige Nachteil ist der große Speicherplatzbedarf mit etwa 10 MB pro Minute.

Eine verlustfrei komprimierende Alternative für die Langzeitarchivierung stellt der Free Lossless Audio Codec (FLAC) dar, der offen dokumentiert und frei verfügbar ist. FLAC wird beispielsweise von dem Containerformat Matroska unterstützt oder kann als eigenständiges Format verwendet werden. 

Das Waveform Audio File Format (WAVE) wurde von Microsoft und IBM als Teil des Resource Interchange File Format (RIFF) entwickelt. Es ist ein offen dokumentiertes aber proprietäres Format, das mehrere Audio-Codecs unterstützt. Aufgrund der weiten Verbreitung wird dieses Format jedoch zusammen mit linearem PCM von der IASA empfohlen und hat sich als Standard de facto durchgesetzt.

Die European Broadcast Union (EBU) hat auf WAVE aufbauend das Format BWF entwickelt, das zusätzlich die Einbettung von Metadaten unterstützt. Für die Langzeitarchivierung sollte dieses Format mit linearem PCM verwendet werden. Für Dateien, die größer als 4 GB sind, hat die EBU das Format RF64/MBWF entwickelt, das zusätzlich die Speicherung von mehr als zwei Tonkanälen erlaubt.

Das Containerformat Matroska (MKA) wird seit 2003 explizit als offenes Containerformat entwickelt, das modernen Ansprüchen genügt und viele verschiedene Codecs enthalten sowie zusätzliche Elemente, wie Kapiteleinteilungen, speichern kann. Das Format basiert auf einer binären Variante von XML, nämlich EBML (Extensible Binary Meta Language), was eine zukünftige flexible Erweiterung erlaubt, jedoch auch sicher stellt, dass ältere Programme weiterhin damit umgehen können. Zusätzlich ist das Format fehlertolerant und kann bis zu einem gewissen Grad auch beschädigte Dateien wiedergeben. Wenn die Audiodaten in linearem PCM oder FLAC gespeichert werden, kann Matroska für die Langzeitarchivierung empfohlen werden.

Aus der Familie der MPEG-Containerformate, die von der Moving Picture Experts Group (MPEG) entwickelt werden und ISO/IEC zertifiziert sind, stammen das MP3- und das AAC-Format. MP3 (MPEG-1 Audio Layer III) wurde schon in MPEG-1 spezifiziert, das seit 1991 unter ISO/IEC 11172 zertifiziert ist. Es handelt sich dabei um einen Audiocodec mit verlustbehafteter Komprimierung, der jedoch eine weite Verbreitung gefunden hat und in einem gleichnamigen Containerformat gespeichert wird.

Advanced Audio Coding (AAC) ist ein Audiocodec, der als Nachfolger von MP3 im Rahmen von MPEG-2 und MPEG-4 spezifiziert wurde und unter ISO/IEC 13818-7 und 14496-3 standardisiert ist. Auch AAC komprimiert verlustbehaftet und kann in einem gleichnamigen Container gespeichert werden oder beispielsweise auch in dem Containerformat MP4. MP3 und AAC sollten nur dann als Langzeitformat für Dateien verwendet werden, wenn diese ursprünglich in dem Format entstanden sind.

Das von Apple entwickelte Audio Interchange File Format (AIFF) ist nicht für die Langzeitarchivierung geeignet, weil es proprietär ist und hauptsächlich nur auf Apple-Systemen Verbreitung gefunden hat. Das von Microsoft entwickelte Windows Media Audio (WMA) ist ebenfalls ein proprietäres Format, das sich wegen der verwendeten verlustbehafteten Kompression nicht für die Langzeitarchivierung eignet. Das Format Ogg wurde vor allem für das Streaming entwickelt und ist aufgrund der verwendeten verlustbehafteten Codecs Opus und Vorbis nicht als Langzeitformat geeignet.
		

\begin{center}
	\begin{tabular}{l L{0.2\textwidth} p{0.6\textwidth}}
		\toprule
		\multicolumn{2}{l}{Format} & Begründung \\ \midrule
		\multirow{3}{*}{\color{ForestGreen} \LARGE \checkmark} & FLAC & Free Lossless Audio Codec ist ein verlustfrei komprimierender Codec, der offen dokumentiert und frei verfügbar ist.\\
			& WAVE & Waveform Audio File Format wurde von Microsoft und IBM entwickelt und ist offen dokumentiert aber proprietär. Die Audiodaten sollten als lineares PCM gespeichert werden.\\
			& BWF & Das Format BWF baut auf WAVE auf und wurde von der EBU entwickelt, um zusätzlich die Einbettung von Metadaten zu unterstützen. Die Audiodaten sollten als lineares PCM gespeichert werden.\\ 
		\cmidrule(r){1-3}
		\multirow{4}{*}{$\color{BurntOrange} \thicksim$} & Matroska & Ein offenes Containerformat, das eine große Bandbreite von Codecs und ergänzenden Inhalten unterstützt. Die Audiodaten sollten als lineares PCM oder FLAC gespeichert werden.\\
			& RF64/MBWF & Wurde von der EBU aus BWF entwickelt, um Dateien zu speichern, die größer als 4 GB sind oder mehr als zwei Tonkanäle beinhalten. Die Audiodaten sollten als lineares PCM gespeichert werden.\\
			& AAC/MP4 & Advanced Audio Coding ist der Nachfolger von MP3 und unter ISO/IEC 13818-7 und 14496-3 standardisiert. Er kann unter anderem in den Formaten AAC oder MP4 gespeichert werden. Die Daten werden verlustbehaftet komprimiert, weshalb er nur für Dateien verwendet werden darf, die ursprünglich in diesem Format entstanden sind.\\
			& MP3 & MPEG-1 Audio Layer III ist unter ISO/IEC 11172 zertifiziert und verwendet verlustbehaftete Kompression. MP3 kann nur als Langzeitformat für Dateien verwendet werden, die ursprünglich in diesem Format entstanden sind.\\
		\cmidrule(r){1-3}
		\multirow{3}{*}{\LARGE \boldmath$\color{BrickRed} \times$} & AIFF & Das Audio Interchange File Format (AIFF) von Apple ist nicht für die Langzeitarchivierung geeignet, weil es proprietär ist. \\
			& WMA & Windows Media Audio ist ein von Microsoft entwickeltes Format mit verlustbehafteter Kompression.\\
			& Ogg & Ein von Xiph entwickeltes und offenes Format, das jedoch nicht für die Archivierung geeignet ist, da die Codecs Opus und Vorbis verlustbehaftet komprimieren.\\
 		\bottomrule
		\bottomrule
	\end{tabular}
\end{center}


\subparagraph{Dokumentation} Neben den allgemeinen minimalen Angaben zu Einzeldateien, wie sie in dem Abschnitt Metadaten in der Anwendung ab Seite \pageref{Metadaten-anwendung} gelistet sind, werden für Audiodateien weitere Angaben benötigt, die insbesondere technische Details dokumentieren. 

Die technischen Angaben zu Codec, Bitrate, Abtastrate und Abtasttiefe werden zur korrekten Wiedergabe der Datei benötigt und vermitteln einen ersten Eindruck über die Qualität der Datei. Angaben zu Länge, Tonkanälen und weiteren Inhalten sind zur Prüfung auf Vollständigkeit der Datei erforderlich.

Bereits eingebettete Metadaten, wie beispielsweise Exif, oder Bestandteile im Containerformat sollten behalten und archiviert werden. Am besten werden sie in eine eigene Text- oder XML-Datei transferiert und getrennt gespeichert.

\begin{center}
	\begin{tabular}{L{0.3\textwidth} p{0.6\textwidth}} 
		\toprule
		Metadatum & Beschreibung \\
		\midrule
		Beteiligte & Angabe der Beteiligten, wie etwa Autor, Komponist, Interpret, Interviewpartner etc.\\
		Länge & Dauer der Audiodatei. Diese Angaben sollten konform zu ISO 8601 erfolgen. Beispiel: P3Y6M4DT12H30M5S (3 Jahre, 6 Monate, 4 Tage, 12 Stunden, 30 Minuten und 5 Sekunden) oder T2H2M (2 Stunden und 2 Minuten)\\
		Audiocodec & Angabe des verwendeten Audiocodecs\\
		Tonkanäle & Angabe der Anzahl der Tonspuren und Benennung des Systems, z.B. 5 (Dolby 5.1)\\
		Bitrate & Angabe der Datenrate in Bits pro Sekunde, z.B. 666 kbps\\
		Abtastrate & Angabe der Abtastrate in Hertz, z.B. 44.1 kHz \\
		Abtasttiefe & Angabe der Anzahl der Quantisierungsstufen als Bittiefe, z.B. 16 bit\\
		Weitere Inhalte & Angabe über weitere Inhalte die in dem Containerformat enthalten sind oder als zusätzliche Datei vorliegen, wie beispielsweise Transkriptionen\\
		Aufnahmegerät & Herstellername und Modell des Aufnahmegeräts (z.B. eines Analog-Digital-Umsetzers oder einer Kamera)\\
		Software & Name und Versionsnummer der Software mit der die Audiodatei aufgenommen, erstellt oder bearbeitet wurde, wie z.B. Audacity 2.1.2\\
	  \bottomrule
		\bottomrule
	\end{tabular}
\end{center}

Weitere Metadaten sind methodenabhängig und können in den jeweiligen Abschnitten nachgelesen werden.