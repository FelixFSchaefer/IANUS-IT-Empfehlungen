\hyphenation{
Wech-sel-ob-jek-ti-ven
}
Bei Rastergrafiken, auch Pixelgrafiken, handelt es sich um digitale Bilder, die mittels rasterförmig angeordneter Bildpunkte, den Pixeln, beschrieben werden. Jedem Pixel ist dabei ein Farbwert zugeordnet. Rastergrafiken haben eine fixe Größe und sind im Gegensatz zu Vektorgrafiken nicht beliebig skalierbar. 

Zu den Rastergrafiken gehören: Digitale Fotografien jeder Art, Satellitenbilder, digitalisierte Bilder (Scans), Screenshots sowie digitale Originalbilder und -grafiken.

\subparagraph{Langzeitformate} Alle als Rastergrafiken vorliegenden Roh- und Urfassungen (Master) von Bildern sind in angemessener Qualität und unkomprimiert im baseline TIFF- oder DNG-Format abzuspeichern. Für georeferenziertes digitales Bildmaterial ist zwecks Erhalt der Referenzdaten das Format GeoTIFF zu verwenden.

Nur für Grafiken, \emph{nicht} für Fotos, eignet sich auch das PNG-Format. Allerdings ist jederzeit das TIFF-Format vorzuziehen. JPEG, bzw. JPG, eignet sich nicht zur Langzeitarchivierung, da es keine verlustfreie Komprimierung anbietet.

Es muss darauf geachtet werden, dass die Bildgröße und -auflösung der originalen Datei erhalten bleibt, wenn in andere Formate umgewandelt wird. Außerdem muss bei der Konvertierung darauf geachtet werden, dass eine verlustfreie Komprimierung verwendet wird. Auch Farbtiefe und Farbraum sollten nach der Konvertierung erhalten bleiben.

Bilder, die Ebenen enthalten, müssen vorher auf eine Ebene reduziert werden. Bei Bedarf, sollten die verschiedenen Ebenen und Komponenten als einzelne Dateien abgespeichert werden. 

%Vorschlag einer Hinweisbox
%\begin{center}
%	\fboxrule0.5mm
% 	\fcolorbox{ianusGrau}{white}{\parbox[c]{0.95\textwidth}{{\scshape\large\color{ianusBlau}Achtung} Wiederholtes Bearbeiten und Abspeichern, führt zu einer allmählichen Abnahme der Qualität. Dieser sogenannte \gls{Generationsverlust} tritt insbesondere bei der verlustbehafteten Bildkomprimierung auf.}}
%\end{center}

Hinweis: Wiederholtes Bearbeiten und Abspeichern, führt zu einer allmählichen Abnahme der Qualität. Dieser sogenannte Generationsverlust tritt insbesondere bei der verlustbehafteten Bildkomprimierung auf.

\begin{center}
	\begin{longtable}{l L{0.2\textwidth} p{0.6\textwidth}}
			\toprule 
		\multicolumn{2}{l}{Format} & Begründung \\
		\midrule \endfirsthead
		\multicolumn{3}{l}{\footnotesize Fortsetzung der vorhergehenden Seite}\\
		\toprule
		\multicolumn{2}{l}{Format} & Begründung \\ \midrule \endhead
		\bottomrule \multicolumn{3}{r}{{\footnotesize Fortsetzung auf der nächsten Seite}} \\
		\endfoot
		\bottomrule 
		\endlastfoot
		
		\multirow{3}{*}{\color{ForestGreen} \LARGE \checkmark} & Baseline TIFF, unkomprimiert & TIFF ist quasi ein Standardformat für die digitale Langzeitarchivierung von Bilddateien und unterstützt auch die Speicherung von Metadaten im Exif-Format. Obwohl das Format auch Kompression verwenden kann, kommt für die Langzeitarchivierung nur die unkomprimierte Form in Frage. \\
		  & DNG & Das von Adobe entwickelte Digital Negative Format ist ein offenes Format, das für die Langzeitarchivierung geeignet ist. Damit können RAW-Dateien und deren Metadaten (Exif oder IPTC-NAA) gelesen und gespeichert werden. Außerdem können über XMP weitere Metadaten eingespeist werden.\\ 
		  & GeoTIFF & GeoTIFF basiert auf TIFF und ist besonders für georeferenzierte Bilddaten geeignet, da somit die Referenzdaten erhalten bleiben.\\ \cmidrule(r){1-3}
		\multirow{1}{*}{$\color{BurntOrange} \thicksim$} & PNG & PNG ist eine verlustfreie Alternative zu dem GIF-Format, welches eine verlustbehaftete Kompression verwendet. Es bietet eine Farbtiefe von 32 Bit, einen Alphakanal für Transparenz und verlustfreie Kompression. Allerdings wird nur RGB als Farbraum unterstützt und es können keine Exif-Daten gespeichert werden. Das Format eignet sich \emph{nicht} für digitale Fotos. \\ \cmidrule(r){1-3}
		\multirow{2}{*}{\LARGE \boldmath$\color{BrickRed} \times$}& JPEG & Trotz einiger Vorzüge, eignet sich das JPEG-Format \emph{nicht} für die Langzeitarchivierung, da es \emph{keine} verlustfreie Komprimierung bietet.\\
		 & GIF & GIF kann sowohl statische, als auch animierte Bilder speichern. Da es aber verlustbehaftet komprimiert, wird PNG als Alternativformat empfohlen. \\
 		\bottomrule    
	\end{longtable}
\end{center}


\subparagraph{Dokumentation}
\label{Metadaten-Rastergrafiken} Eingebettete Metadaten, wie beispielsweise Exif, IPTC-NAA oder XMP, sollten behalten und archiviert werden. Am besten werden sie in eine eigene Text- oder XML-Datei transferiert und getrennt gespeichert.

Neben technischen Informationen, die sich hauptsächlich mit der Erstellung der Rastergrafik befassen, sollten vor allem auch beschreibende und administrative Metadaten über das Bild gespeichert werden.

Hinweis: Werden digitale Aufnahmen mit Programmen bearbeitet, welche die eingebetteten Metadaten ignorieren, gehen diese verloren. 

Die hier angegebenen Metadaten sind als minimale Angabe zu betrachten und ergänzen die angegebenen Metadaten für Projekte und Einzeldateien in dem Abschnitt Metadaten in der Anwendung ab Seite \pageref{Metadaten-anwendung}.

\begin{center}
		\begin{longtable}{L{0.3\textwidth} p{0.6\textwidth}}
			\toprule 
		Metadatum & Beschreibung \\
		\midrule \endfirsthead
		\multicolumn{2}{l}{\footnotesize Fortsetzung der vorhergehenden Seite}\\
		\toprule
		Metadatum & Beschreibung \\ \midrule \endhead
		\bottomrule \multicolumn{2}{r}{{\footnotesize Fortsetzung auf der nächsten Seite}} \\
		\endfoot
		\bottomrule 
		\endlastfoot
		
		Identifikator & Name der Datei, z.B. grabung01.tif \\
		Bildunterschrift & Der Titel oder eine passende Bildunterschrift des Bildes \\
		Beschreibung & Beschreibung des Bildes  \\
		Urheber & Name des Fotografen oder Erstellers \\
		Datum & Datum der Erstellung oder letzten Änderung des Bildes\\
		Rechte & Details zum Urheberrecht \\
		Schlagworte & Schlagworte, wie z.B. Periode, Fundstelle oder charakteristische Merkmale. Wenn vorhanden, angemessene Thesauri verwenden\\
		Ort & Ortsinformationen zu dem Bild. Möglichst in einem standardisierten Format angeben, wie z.B. Lat/Long oder Schlagworte aus einem geeigneten Thesaurus, z.B. Getty Thesaurus of Geographic Names oder GeoNames \\
		Dateiformat \& Version & z.B. Baseline TIFF 6.0 \\
		Dateigröße & Größe der Datei in Bytes \\
		Bildgröße & Maße des Bildes gemessen in Pixeln, z.B. $400px \times 700px$ \\
		Auflösung & Bildauflösung, gemessen in Punkten pro Zoll (dpi) \\
		Farbraum & Der in dem Bild verwendete Farbraum, z.B. RGB oder Graustufen \\
		Farbtiefe & z.B. 24 bit oder 8 bit \\
		Aufnahmegerät & Beispielsweise Details zur Kamera oder dem Scanner \\
		Software & Software mit der das Bild aufgenommen, erstellt oder bearbeitet wurde, wie z.B. Adobe Photoshop CS3 \\ 
 		\bottomrule    
	\end{longtable}
\end{center}

Weitere Metadaten sind methodenabhängig und können in den jeweiligen Abschnitten nachgelesen werden.