Um Datenverlust vorzubeugen, ist es unerlässlich eine geeignete Sicherungsstrategie zu verwenden. Dabei muss zwischen einer kurzfristigen Speicherung und einer mittelfristigen Sicherung unterschieden werden. Ersteres meint, wie Daten während der Erstellung und Bearbeitung gespeichert werden. Letzteres bezieht sich auf einen längeren Speicherzeitraum, der durchaus auch mehrere Monate betragen kann, stellt also ein klassisches Backup der Daten dar.

Die Sicherungsstrategie legt fest, wie die Datensicherung erfolgen soll und berücksichtigt folgende Fragen:
\begin{itemize}
	\item Wer ist für die Datensicherung verantwortlich?
	\item Wer hat Zugriff auf die gesicherten Daten?
	\item Wann und wie oft soll die Datensicherung durchgeführt werden?
	\item Auf welche Weise soll gesichert werden?
	\item Welche Daten sollen gesichert werden?
	\item Welche Speichermedien sollen verwendet werden?
	\item Wie viele Sicherungskopien sollen angelegt werden?
	\item Wo sollen die Sicherungen aufbewahrt und wie sollen sie geschützt werden?
	\item Wie soll der Transport der Sicherungskopien erfolgen?
	\item Wie lange soll eine Sicherung aufbewahrt werden?
\end{itemize}

Die Sicherungsstrategie kann mit den zu verwendenden Richtlinien zur Dateiablage eng verzahnt sein, um etwa einen nahezu automatisierten Sicherungsvorgang zu ermöglichen.

Die richtige Speicherstrategie beugt zwar möglichen Datenverlusten durch Hardware-, Software- oder menschliche Fehler vor, stellt aber noch \emph{keine} Archivierung der Daten dar. Eine Archivierung ist auf Langfristigkeit ausgelegt und impliziert immer eine bewusste Auswahl und umfassende Dokumentation der Daten, da eine Nachnutzung derselben das Ziel ist. Während der Projektlaufzeit kann bereits ein eigener Archivordner angelegt werden, worin beispielsweise finale Dateien oder unprozessierte Rohdaten abgelegt werden können, um die spätere Auswahl der zu archivierenden Daten vorzubereiten und zu vereinfachen.

Weiterführende Hinweise zur Datensicherung sind auf den Seiten des Bundesamtes für Sicherheit in der Informationstechnik zu finden, die sich sowohl an Einsteiger\footnote{\url{https://www.bsi-fuer-buerger.de/BSIFB/DE/MeinPC/Datensicherung/Sicherungsmethoden/sicherungsmethoden\_node.html}} als auch an Experten\footnote{\url{https://www.bsi.bund.de/DE/Themen/ITGrundschutz/ITGrundschutzKataloge/Inhalt/\_content/baust/b01/b01004.html}} richten. 