\subsection{Datenmanagementplan}
Die einzelnen in einem Datenmanagementplan zu berücksichtigenden Aspekte werden im Folgenden kurz skizziert, wobei diese Liste keinen Anspruch auf Vollständigkeit erhebt, sondern vor allem als Anregung für verschiedene Themenfelder gedacht ist. Zu einigen Aspekten folgen ausführlichere Abschnitte im Anschluss an diese Übersicht. 

Die Aufgaben können grob in drei Phasen gegliedert werden:
\begin{itemize}
    \item Planungsphase (ab Seite \pageref{dmp-planung})
    \item Durchführungsphase (ab Seite \pageref{dmp-durchfuehrung})
    \item Abschlussphase (ab Seite \pageref{dmp-abschluss})
\end{itemize}

\label{dmp-planung}\paragraph{Planungsphase}
In der Planungsphase eines Projektes ist es im Hinblick auf den Datenmanagementplan insbesondere notwendig, die hierfür erforderlichen Ressourcen einzuplanen. Diese stehen in direkter Abhängigkeit von dem allgemeinen Rahmendaten und den anzuwendenden Methoden. Sie umfassen sowohl den technischen, als auch den personellen Aufwand. Neben der allgemeinen Festlegung von Verantwortlichkeiten empfiehlt es sich, für den Datenmanagementplan als ganzes oder spezifischer Teile und Datenarten Datenverantwortliche zu benennen.

\subparagraph{Rahmendaten und administrative Angaben}
\begin{itemize}
    \item Welche allgemeinen Informationen und Rahmenbedingungen des Projektvorhabens kontextualisieren das Vorhaben und die Daten? (z.B. Eckdaten des Projektes wie Titel, Namen der Verantwortlichen, Partner, Methoden und Laufzeit)
    \item Was sind, zusammengefasst, die Ziele und das Vorhaben?
    \item Wer sind die Projektträger und Finanzgeber?
\end{itemize}
Dazu kann auch die Liste für Projektbezogene Metadaten ab Seite \pageref{Metadaten-anwendung} konsultiert werden.

\subparagraph{Verantwortlichkeiten}
\begin{itemize}
    \item Wie werden die Rollen und Zuständigkeiten beim Datenmanagement eingeteilt?
    \item Wer beaufsichtigt die Einhaltung des Datenmanagementplans und der daraus resultierenden Vorgaben?
    \item Wer ist für Hard- und Software zuständig?
    \item Wer kümmert sich um die Sicherung der Daten und Backups?
    \item Wer sind institutionelle Ansprechpartner?
    \item Wer sind sonstige Ansprechpartner?
    \item Wer kümmert sich um die Integrität der Daten?
    \item Wie werden die Verantwortlichkeiten kommuniziert?
    \item Ist für eine Kontinuität bei den Verantwortlichkeiten gesorgt?
    \item Wer bekommt welche Berechtigungen für welche Daten?
    \item Gibt es für bestimmte Datenarten Datenverantwortliche?
\end{itemize}

\subparagraph{Rechtliche Aspekte}
\begin{itemize}
    \item Welche Daten sind urheberrechtlich geschützt?
    \item Welche Daten fallen unter Datenschutz?
    \item Wie werden die Rechte am geistigen Eigentum für die Daten von Beginn an dokumentiert?
    \item Gibt es Anforderungen und Einschränkungen für eine Veröffentlichung der Daten?
    \item Mit welchen Lizenzen sollen die Daten für Dritte zur Verfügung gestellt werden?
\end{itemize}

\subparagraph{Methoden}
\begin{itemize}
    \item Welche Methode oder Grabungsmethode wird angewandt?
    \item Hat die Fundstelle oder das Projektvorhaben Einfluss auf die zu verwendenden Methoden?
    \item Gibt es Richtlinien oder Best-Practices für die eingesetzten Methoden?
    \item Welche Dokumentationsmethoden kommen zum Einsatz?
    \item Beeinflusst die zu verwendende Methode die Datenmenge?
\end{itemize}

\subparagraph{Vorgaben, Richtlinien und Standards}
\begin{itemize}
    \item Gibt es Gesetze, Vorschriften der Institution, der Projektträger, der Geldgeber, der externen Partner, der zuständigen Landesämter, die eingehalten werden müssen?
    \item Kann eine bestehende institutionelle Infrastruktur zur Organisation, Verwaltung und Speicherung der Daten genutzt werden?
    \item Sind externe Standards und Richtlinien für den Umgang mit den Daten bekannt?
    \item Welche Qualitätsvorgaben sind für die verschiedenen Datenarten notwendig?
    \item Müssen eigene Vorgaben definiert werden?
    \item Gibt es Checklisten zur Kontrolle der Einhaltung von Vorgaben?
\end{itemize}

\subparagraph{Kosten und Ressourcen}
\begin{itemize}
    \item Wie hoch werden die anfallenden Kosten für Personal und Technik eingeschätzt?
    \item Berücksichtigt die Kostenkalkulation die Kosten in Abhängigkeit der zu erwartenden Datenmenge?
    \item Fallen weitere Kosten für externe Partner oder Dienstleister an?
    \item Muss mit zusätzlichen Kosten für spezielle Anwendungen, Werkzeuge, Systeme etc. gerechnet werden?
    \item Wie sind die Kosten für die Publikation der Ergebnisse zu erwarten?
    \item Welche Folgekosten sind nach Projektende zu erwarten, beispielsweise für die Archivierung der Forschungsdaten?
    \item Welche weiteren Ressourcen werden benötigt?
    \item Bei reproduzierbaren Daten: Sind die Kosten für eine Aufbewahrung höher als für eine Wiederbeschaffung?
    \item Welche Kosten fallen einmalig an, welche regelmäßig?
    \item Können Kosten durch regelmäßige Aufgaben oder durch eine frühzeitige Berücksichtigung von bestimmten Aufgaben verringert werden? (z.B. Dokumentation, Auswahl für das Archiv)
    \item Wer trägt welche Kosten?
    \item Kann der Standard während der gesamten Laufzeit gehalten werden?
\end{itemize}
Weitere Hinweise zur Planung von Kosten und Ressourcen sind in der Broschüre “Einstieg ins Forschungsdatenmanagement in den Geowissenschaften” des EWIG-Projektes ab Seite 11 zu finden. (http://doi.org/10.2312/lis.14.01)


\label{dmp-durchfuehrung}\paragraph{Durchführungsphase}
Während der Durchführung eines Projektes werden die in der Planungsphase gesammelten und erstellten Vorgaben umgesetzt und deren Einhaltung aktiv kontrolliert, sowie bei Bedarf angepasst. 

Eine zeitnah nach Plan durchgeführte Erstellung, Verarbeitung, Analyse und Dokumentation von Daten und Arbeitsabläufen vermeidet eine lange und kostenintensive Abschlussphase. 

\subparagraph{Externe Partner oder Dienstleister}
\begin{itemize}
    \item Mit welchen externen Partnern soll kooperiert werden?
    \item Welche Dienstleister sollen in Anspruch genommen werden?
    \item Welche Auflagen entstehen dadurch und sind diese mit den eigenen Vorgaben vereinbar?
    \item Wie erfolgt der Datenaustausch?
    \item Bei wem verbleiben die Rechte an den Daten?
    \item Wie werden die verschiedenen Vorgaben und Richtlinien an externe Partner kommuniziert?
\end{itemize}

\subparagraph{Hard- und Software}
\begin{itemize}
    \item Welche Hard- und Software steht zur Verfügung?
    \item Werden spezielle Geräte oder Programme benötigt?
    \item Erfüllen die Systeme die vorgegebenen Auflagen, etablierte Standards und die Anforderungen an ein nachhaltiges Datenmanagement?
    \item Können kostenpflichtige Programme durch frei verfügbare Programme (sogenannte Open-Source-Software) ersetzt werden?
\end{itemize}

\subparagraph{Datentypen und Datenformate}
\begin{itemize}
    \item Welche Daten werden verwendet oder erzeugt? (z.B. Beobachtungs- und Messdaten, prozessierte Daten etc.)
    \item In welchen Formaten werden die Daten erzeugt und in welchen sollen sie gesichert werden?
    \item Welches Dateiformat ist für die Archivierung geeignet?
    \item Gibt es verbreitete Standards, die bei der Wahl des Formats zu beachten sind?
    \item Können offene Formate verwendet werden oder müssen proprietäre verwendet werden und hat das Implikationen für die verwendete Hard- und Software?
\end{itemize}
Weitere Informationen sind in den Kapiteln "`Dateiformate"' ab Seite \pageref{dateiformate} und "`Forschungsmethoden"' ab Seite \pageref{methoden} zu finden.

\subparagraph{Nachnutzung vorhandener Daten}
\begin{itemize}
    \item Gibt es bereits vorhandene Daten, die nachgenutzt werden können?
    \item Wurde nach Datenbeständen im Besitz der eigenen Institution oder von Dritten recherchiert?
    \item Wie sind deren Zugriffsmöglichkeiten und Urheberrechte?
    \item Ist die Qualität der Daten ausreichend? (z.B. geeignete Formate, ausreichend Dokumentation)
    \item Wie wird die Integration zwischen bereits bestehenden und neuen Daten organisiert?
    \item Sind auch analoge Quellen zu berücksichtigen?
\end{itemize}

\subparagraph{Erzeugung und Prozessierung von Daten}
\begin{itemize}
    \item Welche Daten müssen neu erzeugt werden oder können bestehende nachgenutzt werden, um das Projektziel zu erreichen?
    \item Handelt es sich um einmalige Daten oder können sie reproduziert werden?
    \item Welche Rolle spielen die Daten innerhalb des Projektes? (z.B. Dokumentation, Publikation, Nachnutzung etc.)
    \item Gibt es sensible oder besonders schützenswerte Daten?
    \item Kann man Anforderungen potentieller Nachnutzer bei der Datenerzeugung mitberücksichtigen?
\end{itemize}

\subparagraph{Datenmenge}
\begin{itemize}
    \item Wie groß ist die zu erwartende Datenmenge für die Gesamtdauer des Projekts?
    \item Hat das Folgen für die Speicherung und Sicherung der Daten? (z.B. längere Backup-Zeiten)
    \item Ergeben sich daraus besondere Anforderungen an die technische Infrastruktur? (z.B. mehr Speicherplatz)
    \item Fallen verschiedene Bearbeitungsstufen mit verschiedenen Versionen an, die gegebenenfalls ein Versionierungssystem erfordern?
\end{itemize}
Weitere Informationen zur Versionierung sind in dem Abschnitt "`Versionskontrolle"' ab Seite \pageref{versionskontrolle} zu finden.

\subparagraph{Dateispeicherung und -sicherung}
\begin{itemize}
    \item Welche Maßnahmen zur Dateispeicherung und -sicherung sind während des Projekts notwendig?
    \item Auf welcher Hardware sollen die Daten gesichert werden? (z.B. Server oder Festplatten)
    \item Wie oft, womit und durch wen werden Backups durchgeführt?
    \item Wer ist für die Datenspeicherung und -sicherung verantwortlich?
    \item Gibt es ein Disaster Management?
    \item Wurden die Maßnahmen zur Dateiwiederherstellung geprobt?
\end{itemize}
Weitere Informationen sind im Abschnitt "`Dateispeicherung und -sicherung"' ab Seite \pageref{dateispeicherung} zu finden.

\subparagraph{Dateiverwaltung}
\begin{itemize}
    \item Wie sollen Dateien benannt, geordnet und versioniert werden?
    \item Gibt es Benennungsregeln für die Dateien?
    \item Kann auf auf bestehende Vorgaben und Systeme für die Dateiverwaltung zurückgegriffen werden?
    \item Sind Verzeichnisstrukturen logisch nachvollziehbar und selbsterklärend?
    \item Ist die Datenablage dokumentiert?
    \item Wie ist der Umgang mit verschiedenen Dateiversionen geplant?
    \item Ist der Einsatz eines Versionierungssystems notwendig?
    \item Werden Daten, die für eine künftige Nachnutzung geeignet sind oder archiviert werden sollen, gesondert verwaltet?
\end{itemize}
Weitere Informationen sind im Abschnitt "`Dateiverwaltung"' ab Seite \pageref{dateiverwaltung} zu finden.

\subparagraph{Dokumentation}
\begin{itemize}
    \item Wie sollen die Daten beschrieben werden, damit sie kurz- und langfristig lesbar und verständlich sind?
    \item Welche Informationen sind zur Dokumentation der Forschungsdaten notwendig?
    \item Zu welchem Zeitpunkt muss die Dokumentation geschehen?
    \item Gibt es Vorgaben oder Standards dafür?
    \item Werden Veränderungen und Aktualisierungen der Daten dokumentiert?
    \item Wie sollen Metadaten abgelegt und gespeichert werden?
    \item Werden Anpassungen an der Projektstruktur und dem Datenmanagementplan dokumentiert?
    \item Werden Ausnahmen dokumentiert?
    \item Gibt es Werkzeuge, die den Dokumentationsprozess unterstützen?
\end{itemize}
Weitere Informationen sind im Abschnitt "`Dokumentation"' ab Seite \pageref{Metadaten-allgemein} zu finden.

\subparagraph{Qualitätssicherung}
\begin{itemize}
    \item Welche Kriterien sind hinsichtlich vorhandener Standards zur Qualitätssicherung zu beachten?
    \item Wie werden Vorgaben zu Formaten und zur Datenbearbeitung eingehalten?
    \item Sind die Daten genau, konsistent und authentisch?
    \item Sind die Daten vollständig?
    \item Wie steht es um die Integrität der Daten?
    \item Sind die Daten verständlich Dokumentiert und geht aus der Dokumentation hervor: \emph{Wer} hat \emph{wann}, zu welchem \emph{Zweck}, \emph{was} und \emph{womit} gemacht?
    \item Wird eine Qualitätskontrolle durchgeführt?
    \item Gibt es Checklisten zur unterstützung der Qualitätskontrolle?
    \item Welche Maßnahmen gegen ein versehentliches Löschen oder eine Manipulation der Daten werden getroffen?
\end{itemize}

\subparagraph{Datenaustausch}
\begin{itemize}
    \item Wie ist der Datenaustausch zwischen den Projektbeteiligten geplant?
    \item Welche technische Infrastruktur ist für den Datenaustausch erforderlich?
    \item Sind gesetzliche Vorgaben oder andere Einschränkungen zu beachten?
    \item Wie soll auf die Daten zugegriffen werden?
    \item Welche Nutzungsrechte liegen für die Daten vor?
\end{itemize}


\label{dmp-abschluss}\paragraph{Abschlussphase}
In der das Projekt abschließenden Phase müssen besonders Entscheidungen sowohl zur mittelfristigen Datenaufbewahrung, als auch zur langfristigen Archivierung der im Projekt generierten Daten und Ergebnisse getroffen werden. Neben der Klärung der organisatorischen und rechtlichen Rahmenbedingungen muss hierbei spezielles Augenmerk auf die Regelung der Zugänglichkeit zu den langzeitarchivierten Daten und der Nachnutzung gelegt werden. So wird sichergestellt, dass über den Projektabschluss hinaus die Daten langfristig zur Verfügung stehen. 

\subparagraph{Mittelfristige Datenaufbewahrung}
\begin{itemize}
    \item Welche Gründe gibt es für eine Aufbewahrung der Daten?
    \item Liegen Vorgaben zur Aufbewahrungsdauer der Daten vor?
    \item Liegen Vorgaben zu Aufbewahrungsorten der Daten vor?
    \item Welche Daten sollen oder müssen aufbewahrt werden, welche Daten sollen oder können gelöscht werden?
    \item Sind sie selbst Hersteller der Daten?
    \item Ist die Aufbewahrung von Fremddaten zwingend notwendig?
    \item Wie lange sollen die Daten aufbewahrt werden?
    \item Wie und wo sollen die Daten aufbewahrt werden?
    \item Wer ist für die Aufbewahrung der Daten verantwortlich?
    \item Sind Kosten zu erwarten?
\end{itemize}

\subparagraph{Vorbereitung für die langfristige Archivierung}
\begin{itemize}
    \item Welche Daten sollen archiviert werden?
    \item Liegen Kriterien für die Auswahl der Daten vor?
    \item Gibt es eine passende Archivlösung?
    \item Wurde bereits Kontakt mit dem Archiv aufgenommen?
    \item Welche Besonderheiten gilt es zu beachten? (z.B. eine gesonderte Aufbereitung der Daten)
    \item Wann und durch wen werden die Daten übergeben?
\end{itemize}

\subparagraph{Zugänglichkeit und Nachnutzung}
\begin{itemize}
    \item Wie sollen die Daten zugänglich sein?
    \item Welche Zusatzinformationen sind für das Verständnis der Daten notwendig?
    \item Wer darf die Daten nutzen, welche Lizenzen sollen verwendet werden?
    \item Gibt es Einschränkungen für den Zugriff auf und die Nutzung der Daten?
    \item Auf welche Weise sollen die Daten zur Verfügung gestellt werden?
\end{itemize}

\subparagraph{Projektabschluss}
\begin{itemize}
    \item Liegt die Dokumentation vollständig und den Vorgaben entsprechend vor?
    \item Ist eine gesonderte abschließende Dokumentation erforderlich?
    \item Ist der Datenmanagementplan in die Dokumentation integriert?
    \item Ist die Nachnutzbarkeit der Forschungsdaten auch nach Projektende gewährleistet?
    \item Wie ist die mittelfristige Aufbewahrung der Daten geregelt?
    \item Wie ist die langfristige Archivierung der Daten geregelt?
    \item Wie ist die Zugänglichkeit der Daten geregelt?
\end{itemize}

%XXX Grafik zur Visualisierung vom DMP: Mit Waben oder Puzzleteilen

\subsection{Weiterführende Informationen}
\paragraph{Quellen}
\begin{flushleft}
Archaeology Data Service, Planning for the Creation of Digital Data \urllist{http://guides.archaeologydataservice.ac.uk/g2gp/CreateData_1-0}
Archaeology Data Service, Data Management and Sharing Plans \urllist{http://archaeologydataservice.ac.uk/advice/DataManagementPlans}
Archaeology Data Service, DataTrain. Open Access Post-Graduate Teaching Materials in Managing Research Data in Archaeology \urllist{http://archaeologydataservice.ac.uk/learning/DataTrain}
S. Büttner --  H.-C. Hobohm -- L. Müller (Hrsg.) Handbuch Forschungsdatenmanagement (Bad Honnef 2011)\abstand
DCC (Hrsg.) Data Management Plans \urllist{http://www.dcc.ac.uk/resources/data-management-plans}
Helmholtz-Zentrum Potsdam - Deutsches GeoForschungsZentrum GFZ -- Bibliothek und Informationsdienste -- Institut für Meteorologie der FU Berlin -- Konrad-Zuse-Zentrum für Informationstechnik Berlin (Hrsg.) Einstieg ins Forschungsdatenmanagement in den Geowissenschaften (2014), DOI: 10.2312/lis.14.01 \urllist{http://doi.org/10.2312/lis.14.01}
HU Berlin (Hrsg.)  Datenmanagementplan. Anleitung zur Erstellung eines Datenmanagementplans (DMP) \urllist{https://www.cms.hu-berlin.de/de/dl/dataman/arbeiten/dmp_erstellen}
U. Jakobsson, Data Management Planning. What it is and how to do it (2013) \urllist{http://www.ariadne-infrastructure.eu/fre/Events/ARIADNE-Workshop-EAA-2013/Agenda-Presentations/DataManagementPlanning_SND_Ujakobsson_04092013}
H. Neuroth -- A. Oßwald -- R. Scheffel -- S. Strathmann -- M. Jehn (Hrsg.) nestor Handbuch. Eine kleine Enzyklopädie der digitalen Langzeitarchivierung. Version 2.0 (2009) Kap.3:15 \abstand
K. Perrin -- D. H. Brown -- G. Lange -- D. Bibby -- A. Carlsson -- A. Degraeve -- M. Kuna -- Y. Larsson -- S. U. Pálsdóttir -- B. Stoll-Tucker -- C. Dunning -- A. Rogalla von Bieberstein, A Standard and Guide to Best Practice for Archaeological Archiving in Europe, EAC Guidelines 1 (2013) \urllist{http://archaeologydataservice.ac.uk/arches/Wiki.jsp?page=The\%20Standard\%20and\%20Guide\%20to\%20Best\%20Practice\%20in\%20Archaeological\%20Archiving\%20in\%20Europe}
UK Data Archive (Hrsg.) Create \& Manage Data. Planning for Sharing \urllist{http://data-archive.ac.uk/create-manage/planning-for-sharing}
Uni Marburg (Hrsg.) Projekt "Kompetenzzentrum Forschungsdatenmanagement und -archivierung" \urllist{https://www.uni-marburg.de/projekte/forschungsdaten}
V. van den Eynden -- L. Corti -- M. Woollard -- L. Bishop -- L. Horton, Managing and Sharing Data. Best Practice for Researchers (Essex 2011) \urllist{http://ukdataservice.ac.uk/manage-data/handbook.aspx}
WissGrid (Hrsg.) Leitfaden Forschungsdaten-Management (2011) \urllist{http://www.wissgrid.de/publikationen/deliverables/wp3/WissGrid-oeffentlicher-Entwurf-Leitfaden-Forschungsdaten-Management.pdf}

\quelltyp{Beispiele für Datenmanagementpläne}
DataONE (Hrsg.) Data Management Planning \urllist{https://www.dataone.org/data-management-planning}
DCC (Hrsg.) Data plan guidance and examples \urllist{http://www.dcc.ac.uk/resources/data-management-plans/guidance-examples}
MITLibraries (Hrsg.) Write a Data Management Plan \urllist{http://libraries.mit.edu/data-management/plan/write/\#Resources}

\quelltyp{Checklisten und Tools für Datenmanagementpläne}
DCC (Hrsg.) Checklist for a Data Management Plan (2014) \urllist{http://www.dcc.ac.uk/webfm_send/1279}
DCC (Hrsg.) DMPOnline \urllist{https://dmponline.dcc.ac.uk/}
J. Richards, Online Resources for Data Management Planning (2013) \urllist{http://www.ariadne-infrastructure.eu/fre/Events/ARIADNE-Workshop-EAA-2013/Agenda-Presentations/DataManagementPlanning_Online_Resources_ADS_JRichards_04092013}
RDMO (Hrsg.) Research Data Management Organiser \urllist{http://rdmorganiser.github.io/software/}
WissGrid (Hrsg.) Checkliste zum Forschungsdaten-Management (2011) \urllist{http://www.wissgrid.de/publikationen/deliverables/wp3/WissGrid-oeffentlicher-Entwurf-Checkliste-Forschungsdaten-Management.pdf}
\end{flushleft}