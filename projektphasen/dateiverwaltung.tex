\hyphenation{
Schnitt-stel-le
}
Der tägliche Umgang mit digitalen Daten wird durch eine effiziente Dateiverwaltung erheblich erleichtert. Aussagekräftige Dateinamen, die auch für Dritte verständlich sind, sorgen dafür, dass die Dateien gefunden und deren Inhalt verstanden wird. Einheitliche Dateinamensstrukturen erhöhen die Lesbarkeit. Die konsequente Einhaltung von Versionierungsangaben sorgt dafür, dass immer mit der richtigen Dateiversion gearbeitet wird und eine selbsterklärende Ordnerstruktur hilft dabei auch in großen Projekten bestimmte Dateien wiederzufinden.

Die Konzipierung der Dateiverwaltung muss schon zu Beginn des Projektes erfolgen, damit die Regeln von Anfang an angewendet werden können. Die Niederschrift der durch Beispiele ergänzten Benennungsregeln und deren Weitergabe dient im laufenden Betrieb als wichtiges Nachschlagewerk für die konsistente und konsequente Einhaltung derselben. Dies ermöglicht eine effizientere Arbeitsweise.

Im laufenden Projektbetrieb sollte die Einhaltung der Benennungsregeln kontrolliert und gegebenfalls angepasst werden.


\subsection{Dateiablage}
Mit Dateiablage ist hier vor allem die Ordnerstruktur gemeint. Für die Benennung der Ordner gelten die gleichen Regeln, wie für die Dateibenennung, die im Abschnitt Dateibenennung ab Seite \pageref{dateibenennung} thematisiert werden. Lediglich die Dateinamenserweiterung wird bei Ordnernamen nicht verwendet. Die Dateiablage sollte selbsterklärend sein und unpräzise Namen wie etwa \emph{"`In Arbeit"'} vermieden werden.

Wichtig ist, dass die Dateiablage logisch und hierarchisch aufgebaut ist, damit andere Nutzer die gewünschten Informationen finden und einordnen können. Dies bedeutet unter anderem, dass die Inhalte und Unterordner auch tatsächlich thematisch in den übergeordneten Ordner hineinpassen. Beispielsweise erwartet man in einem Ordner mit dem Namen \emph{"`Fotos"'} diverse Fotos, die bei einer entsprechenden Menge vielleicht noch auf verschiedene Unterordner, wie zum Beispiel \emph{"`Plana"'} und \emph{"`Profile"'}, verteilt sind.   

Ist aber in dem Ordner \emph{"`Fotos"'} ein Unterordner mit dem Namen \emph{"`Zeichnungen"'} enthalten, in dem verschiedene digitalisierte Zeichnungen abgelegt sind, führt das zu Schwierigkeiten. Der Unterordner wird wahrscheinlich nur schwer wiedergefunden werden, da der Name des übergeordneten Ordners einen anderen Inhalt verspricht.

Tritt der Fall ein, dass eine Datei oder ein Ordner thematisch in mehrere verschiedene übergeordnete Ordner passen würde, können verschiedene Lösungswege zur Anwendung kommen. Beispielsweise könnte eine Kopie abgelegt werden, was allerdings problematisch ist, wenn eine der Kopien verändert wird und diese nicht mit der anderen abgeglichen wird.

Auch von Dateiverknüpfungen ist abzuraten, da sie meist nicht betriebssystemübergreifend funktionieren und im schlechtesten Fall nur auf dem Rechner, mit dem sie erstellt wurden, funktionieren. Wird die Datei verschoben, umbenannt oder gelöscht, so wird die Dateiverknüpfung ins Leere führen.

Eine bessere Methode ist das Anlegen einer Textdatei mit einem Hinweis auf den Ablageort der Datei oder des Ordners. In diesem Fall muss aber darauf geachtet werden, dass die gesuchte Datei sich auch tatsächlich an dem angegebenen Ort befindet. Bei Verschieben oder Löschen müsste die Textdatei also auch immer berücksichtigt und angepasst werden. 

Am besten ist es, nur eine Instanz einer Datei oder eines Ordners zu haben. Bei Zuordnungsschwierigkeiten kann es helfen, die Datei oder den Ordner in der hierarchischen Struktur höher anzusiedeln.

Eine Textdatei kann auch verwendet werden, um die vorhandene Ordnerstruktur zu erklären oder auf Besonderheiten hinzuweisen. Beispielsweise kann in einem Ordner mit Fotos, deren Metadaten in einer Datenbank abgespeichert sind, mit Hilfe einer Textdatei verdeutlicht werden, wo die Metadaten zu finden sind. Damit die Textdatei auch als eine Hilfedatei erkannt wird, kann man ihr den Namen \emph{"`README"'} oder \emph{"`LIESMICH"'} geben. Die Schreibung mit Großbuchstaben erhöht die Auffälligkeit der Datei. Durch eine vorangestellte Null oder einen vorangestellten Unterstrich kann sie auch in der Sortierung an den Anfang gestellt werden.

Mittels eines Dokumentenmangaementsystems, kann eine Dateiablage auch datenbankgestützt Verwaltet werden, was jedoch einen höheren technischen Aufwand erfordert. Der Vorteil ist, dass Dateien nach beliebigen Kriterien sortiert und gesucht werden können. Jedoch kann eine mit einem Dokumentenmanagementsystem erstellte Dateiablage ohne dieses System unter Umständen nicht mehr verständlich sein, da für die Dateien abstrakte Bezeichnungen vom System vergeben werden.

\subsection{Empfehlungen für eine Ordnerstruktur}
Die Organisation der Ordnerstruktur kann sich anhand der angewandten Prozesse oder an den Ergebnissen orientieren. Im ersten Fall bildet die Struktur zeitliche und methodische Prozesse der Daten ab, was zu einer besseren Nachvollziehbarkeit der Arbeitsschritte führt. Im zweiten Fall orientiert sich die Ordnerstruktur an den fachlichen Ergebnissen, was die inhaltliche Nachvollziehbarkeit verbessert.

Bei der Planung einer Ordnerstruktur spielen weitere Kriterien eine Rolle. Abhängig von dem Umfang und der Dauer des Projektes, kann sich die Ordnerstruktur beispielsweise am Thema, dem Material, dem Jahr, dem Bearbeiter oder den einzelnen Arbeitsschritten orientieren. Dabei sollte beachtet werden, dass die Dateiablage nicht zu verzweigt ist, da die maximale Pfadlänge, die sich aus allen enthaltenden Ordnernamen und dem Dateinamen zusammensetzt, in Windows auf 260 Zeichen begrenzt ist.

Ganz allgemein kann die oberste Hierarchie der Dateiablage nach folgenden Kriterien unterteilt werden:
\begin{itemize}
	\item Ort
	\item Fundplatz, Monumente oder Denkmäler
	\item Aktivität
	\item Projekt
\end{itemize}

Für weitere Hierarchieebenen können folgende Kriterien berücksichtigt werden:
\begin{itemize}
	\item Verfahren, wie Prospektion, Voruntersuchung oder Hauptuntersuchung
	\item Arbeitsschritte mit Bezug auf Zeit, Typ und Kennung
	\item Fachliche Inahlte, wie Befunde, Funde, Proben, Bauwerke, Berichte oder Tagebuch
	\item Methodik, wie Vermessung, Foto, Zeichnung oder Photogrammetrie
	\item Räumliche Spezifizierung, wie Planum, Schnitt oder Surveyfläche
	\item Administration
\end{itemize}

Zusätzlich sollte eine Unterscheidung zwischen originalen Daten (Rohdaten), sekundären und finalen Daten erfolgen, um Prozesse transparent abzubilden. Dies ist auch für die zu archivierenden Daten relevant, da bei der Auswahl der Daten überholte oder temporäre Dateien und Dubletten in der Regel nicht berücksichtigt werden.

Anwendungsgebiete mit mehreren voneinander abhängigen Dateikomplexen, wie zum Beispiel 3D-Scanning oder RTI-Fotografie, erfordern eigene Dateiablagen. Diese werden in den entsprechenden Abschnitten im Kapitel "`Forschungsmethoden"' ab Seite \pageref{methoden} beschrieben.