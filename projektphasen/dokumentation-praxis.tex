\subsection{Metadaten in der Anwendung}
\label{Metadaten-anwendung}
\paragraph{Projektbezogene Metadaten}
Die wichtigsten Metadaten, die für die Beschreibung eines Projektes oder einer Dokumentensammlung erforderlich sind, werden in der folgenden Tabelle abgebildet und knapp definiert. Sie geben einen Überblick über einen größeren, zusammenhängenden Datenbestand und beschreiben den fachlichen Kontext in dem dieser entstanden ist. Vergleichbar einem Bibliothekskatalog liegt die Hauptfunktion dieser Metadaten darin, dass externe Personen ein Projekt oder eine dessen digitale Daten über Web-Portale und Suchmaschinen finden und einordnen können. Darüber hinaus enthalten sie rechtliche Informationen, die für den weiteren Umgang mit den Daten wichtig sind. 

Die hier vorgestellten Eigenschaften basieren auf dem Dublin Core Metadata Schema und den Angaben, wie sie vom ADS in den UK und tDAR in den USA erhoben werden, um einen zukünftigen Austausch zu vereinfachen. Ein darauf aufbauendes ausführliches Metadatenschema, das auch die Grundlage für die Archivierung von Projektdaten bei IANUS bildet, sowie ausgefüllte Beispielformulare sind in dem Kapitel Archivierung von Forschungsdaten in IANUS ab Seite \pageref{archivierungIANUS} zu finden.

\begin{center}
	\begin{longtable}{L{0.25\textwidth} p{0.68\textwidth}}
		\toprule
		Bezeichnung & Kurzdefinition\\ \midrule \endfirsthead
		\multicolumn{2}{l}{\footnotesize Fortsetzung der vorhergehenden Seite}\\
		\toprule
		Bezeichnung & Kurzdefinition\\ \midrule \endhead
		\bottomrule \multicolumn{2}{r}{{\footnotesize Fortsetzung auf der nächsten Seite}} \\
		\endfoot
		\bottomrule 
		\endlastfoot

		Identifizierung -- Projekttitel & Verbindliche Kurzbezeichnung des Projektes.\\
		Identifizierung -- Alternativtitel & Ggf. alternative Titel für ein Projekt.\\
		Identifizierung -- Projektnummer(n) & Nummern oder Kennungen, die z.B. innerhalb der durchführenden Organisation oder von Mittelgebern verwendet werden, um das Projekt eindeutig identifizieren zu können.\\
		Kurzbeschreibung & Knappe Angaben zur Fragestellung, zum Verlauf und Ergebnis des Projektes sowie Skizzierung der Datensammlung (insgesamt ca. 100-300 Worte)\\
		Schlagworte -- Fachdisziplinen & Stichworte, die die beteiligten Disziplinen und Fächer benennen. Sofern die Stichworte auf publizierten Standards oder internen Thesauri beruhen, müssen diese mitangegeben werden.\\
		Schlagworte -- Inhalt & Stichworte, die den Inhalt der Datensammlung benennen., z. B. zu Materialgruppen, Fundstellen-Klassifizierung, Quellenarten,  Kulturgruppen etc. Sofern die Stichworte auf publizierten Standards oder internen Thesauri beruhen, müssen diese mitangegeben werden.\\
		Schlagworte -- Methoden & Stichworte, die die eingesetzten Forschungsmethoden beschreiben. Sofern die Stichworte auf publizierten Standards oder internen Thesauri beruhen, müssen diese mitangegeben werden.\\
		Ausdehnung -- Geografisch-1 & Detaillierte Angaben zur räumlichen Ausdehnung oder zum Fundort des untersuchten Gegenstandes mittels geografischer Koordinaten. Die maximale Ausdehnung kann als Bounding Box angegeben werden.\\
		Ausdehnung -- Geografisch-2 & Sprachliche Beschreibung des untersuchten Gegenstandes mittels Ortsangaben mit Land, Stadt, Kreis, Straße, Gemarkung etc. Sofern Namen sich im Lauf der Zeit geändert haben, dies gesondert vermerken. Sofern eine Referenz zu einer Geo-Ressource oder einem Gazetteer existiert, sollte diese ebenfalls angegeben werden.\\
		Ausdehnung -- zeitlich & Chronologische Angaben zum untersuchten Gegenstand, entweder als Periodenbezeichnung und/oder mit groben/genauen Jahresangaben. Sofern die Stichworte auf publizierten Standards oder internen Thesauri beruhen, müssen diese mitangegeben werden.\\
		Primärforscher -- Person & Personen, die entweder für das Projekt als Ganzes, für das Datenmanagement oder für die Erzeugung bestimmter Datenarten zentral bzw. verantwortlich sind. Hier ist eine Kontaktadresse erforderlich und die aktuelle/letzte institutionelle Zugehörigkeit, damit die Personen bei Rückfragen erreicht werden kann.\\
		Eigentümer -- Organisation & Organisation, der die unter "`Primärforscher"' genannten Personen angehören, oder die nach Ausscheiden derselben für die Daten verantwortlich ist, im weitesten Sinne also Eigentümer der Daten ist. Hier ist eine Kontaktadresse erforderlich, damit die Organisation bei Rückfragen erreicht werden kann.\\
		Finanzierung & Nennung der Organisation(en) / (Dritt-)Mittelgeber, durch die das Projekt finanziert wurde. Es sollte jeweils der Zeitraum der Finanzierung angegeben werden.\\
		Veröffentlichung -- Projektdaten & Wenn die hier beschriebene Datensammlung des Projektes bereits an anderer Stelle veröffentlicht / online gestellt wurde, bitte entsprechende Angaben machen, z. B. durch Nennung der Organisationen, Datenarchive, Online-Ressourcen etc.\\
		Veröffentlichung -- Ergebnisse & Analoge oder digitale Publikationen zu Ergebnissen des Projektes oder zur Datensammlung des Projektes, ausführliche bibliographische Angaben (ohne fachspezifische Abkürzungen) unter Nennung des Verlages erforderlich.\\
		Dauer -- Projekt & Anfangs- und Enddatum des Projektes.\\
		Dauer -- Datenbestand & Anfangs- und Enddatum der Erzeugung oder Verarbeitung digitaler Daten im Rahmen des Projektes.\\
		Rechtliches -- Urheberrechte & Name des Inhabers der Urheber-, Nutzungs- und Verwertungsrechte; i. d. R. die Organisation, an der der Primärforscher beschäftigt war.\\
		Rechtliches -- Lizenzgeber & Angabe der Person, die i. d. R. als Vertretung für eine Organisation für die Lizenzierung von Daten zur Nachnutzung verantwortlich und berechtigt ist, einen Datenübergabevertrag abzuschließen.\\
		Rechtliches -- Datenschutz & Angaben, ob in der Datensammlung datenschutzrelevante Informationen enthalten sind. Wenn ja, in welchem Umfang.\\
		Quellen -- Ältere & Ältere Quellen oder existierende Ressourcen, auf denen die Daten aufbauen.\\
		Quellen -- Zugehörige & Sofern während des Projektes Informationen, Datensammlungen, (un-)publizierte Dokumente, Online-Ressourcen etc. verwendet oder erzeugt wurden, die nicht Teil der hier beschriebenen Datensammlung sind, aber für deren Verständnis wichtig sind, bitte entsprechende Angaben zu Art und Umfang dieser Quellen machen.\\
		Sprache & Die in den Dokumenten und Dateien verwendete(n) Sprache(n). Sprachkennungen nach ISO 639 angeben.\\
		Art der Daten & Kurzcharakterisierung der Daten, z. B. ob es sich um Rohdaten, verarbeitete Daten, Interpretationen, Ergebnisse, Abschlussberichte etc. handelt.\\
		Vollständigkeit & Aussagen zur Vollständigkeit der Projektdaten, z. B. ob bestimmte Datenarten noch fehlen und warum.\\
		Dateiformate & Auflistung der Dateiformate, die in der Datensammlung vorkommen, ggf. unter Nennung der verwendeten Programme und Zeichenkodierungen.\\
		Zugriffsrechte & Festlegung der gewünschten Zugriffsrechte für die Daten, sofern diese für den gesamten Projekt-Datenbestand gelten sollen; differenzierte Regelungen müssen auf Dateiebene vorgenommen werden.\\
		Signatur Metadaten & Angabe darüber, wer die o. g. Metadaten wann ausgefüllt hat.\\
		\bottomrule
	\end{longtable}
\end{center}

\paragraph{Dateibezogene Metadaten}
Bei dieser Art von Metadaten handelt es sich um technische und inhaltliche Informationen, die Nutzern verständlich machen, wie einzelne Dateien innerhalb eines Projektes oder einer Datensammlung beschaffen sind und welche Möglichkeiten der Nachnutzbarkeit sie beinhalten. 

Dateibezogene Metadaten sind abhängig von dem Format, dem Inhalt und der Methode, mit denen die Dateien erzeugt wurden. Beispielsweise sind für Rastergrafiken, die durch digitale Fotografie entstanden sind, andere Angaben erforderlich (Fotograf, Aufnahmedatum, Aufnahmeort, abgelichtetes Objekt etc.) als für Rasterdateien, die durch geophysikalische Messungen erzeugt wurden (Koordinaten, Messgerät, Genauigkeit, Datum etc.). Zusätzliche spezifische Angaben, die für bestimmte Dateiformate empfohlen werden, sind in den verschiedenen Kapiteln zu den jeweiligen Formaten ab Seite \pageref{dateiformate} beschrieben.

Es gibt jedoch dateibezogene Metadaten, die unabhängig von Format, Inhalt und Methode für alle Einzeldateien gleichermaßen relevant und notwendig sind. Zu diesen Metadaten gehören neben den Angaben zu Dateiname, Dateiformat, Dateiversion, Titel, Beschreibung und Ersteller auch Informationen zur verwendeten Soft- und Hardware, Versionierung, zu rechtlichen Aspekten und Verweise auf weitere relevante Dateien.

Auch wenn es in der Theorie wünschenswert ist, diese Angaben sowie die zugehörigen spezifischen Angaben zu Methoden und Dateiformaten für jede Datei einzeln zu erfassen, so zeigt die Praxis, dass es häufig ausreichend ist, einen Metadatensatz für Gruppen von Dateien anzulegen, wenn diese das gleiche Format oder die gleichen inhaltlichen Eigenschaften aufweisen. 

\begin{center}
	\begin{longtable}{L{0.25\textwidth} p{0.68\textwidth}}
		\toprule
		Bezeichnung & Kurzdefinition\\ \midrule \endfirsthead
		\multicolumn{2}{l}{\footnotesize Fortsetzung der vorhergehenden Seite}\\
		\toprule
		Bezeichnung & Kurzdefinition\\ \midrule \endhead
		\bottomrule \multicolumn{2}{r}{{\footnotesize Fortsetzung auf der nächsten Seite}} \\
		\endfoot
		\bottomrule 
		\endlastfoot
		
		Identifikator, Dateiname & Eindeutiger Name der Datei.\\
		Dateiformat & Format, in dem die Datei abgespeichert ist.\\
		Urheber & Name des Verfassers oder Erstellers der Datei.\\
		Titel & Titel der Datei, nicht der Dateiname.\\
		Beschreibung & Beschreibung des Inhalts der Datei. \\
		Schlagworte & Schlagworte, wie etwa Periode, Fundstelle oder charakteristische Merkmale. Wenn vorhanden, angemessene Thesauri verwenden.\\
		Software -- Dateierstellung & Software, mit der die Datei erstellt wurde.\\
		Hardware -- Dateierstellung & Hardware, mit der die Datei erstellt wurde, v. a. bei technischen Geräten wie Kameras, GPS-Geräten, Vermessungsinstrumenten, Laserscanner etc.\\
		Betriebssystem -- Dateierstellung & Betriebssystem, das verwendet wurde als die Datei erstellt wurde.\\
		Erstellungsdatum & Datum, an dem die Datei erstellt wurde. Datum und Zeit in UTC nach ISO 8601.\\
		Letzte Aktualisierung & Datum, an dem die Datei zuletzt bearbeitet wurde. Datum und Zeit in UTC nach ISO 8601.\\
		Dateiversion & Angabe der Versionsnummer der Datei.\\
		Weitere Dateien & Referenzen auf Dateien, die für das Verständnis einer anderen Datei zentral sind, insbesondere für zusammenhängende, komplexe Dateien oder wenn auf eine Ursprungsdatei verwiesen werden soll.\\
		Sprache & Sofern schriftliche Inhalte vorhanden sind, die Sprache angeben. Sprachkennungen nach ISO 639 angeben.\\
		Copyright-Angaben & Angaben zur Person oder Einrichtung, die das Copyright oder die Lizenzrechte an der Datei oder deren Inhalt besitzt.\\
		\bottomrule
	\end{longtable}
\end{center}

\paragraph{Methodenbezogene Metadaten}
Jede Fachdisziplin innerhalb der Altertumswissenschaften verfügt über spezifische Forschungsmethoden. Auch diese haben Einfluss auf Umfang, Art, Struktur und Inhalt digitaler Objekte, da sie bei verschiedenen Arbeitsweisen und technischen Geräten unterschiedlich ausfallen. Daher sollten methodenbezogene Metadaten ebenfalls dokumentiert werden, insbesondere wenn mehrere Zwischenstände einer Prozesskette archiviert und Dritten zur Verfügung gestellt werden sollen. Je nach Genauigkeit der Methodenbeschreibung kann sie sich sowohl auf eine Einzeldatei als auch auf mehrere Dateien gleichen Typs beziehen.

Die Angabe der methodenbezogenen Metadaten ist wichtig, um zu beschreiben, wie die Rohdaten in prozessierte Daten überführt wurden. Außerdem kann so verstanden werden, welchen Einfluss die angewandte Methode auf das Ergebnis hat, das Ergebnis folglich zu interpretieren ist und wo Fehler zu erwarten sind.

Zusätzliche spezifische Angaben, die für bestimmte Forschungsmethoden oder Prozesse empfohlen werden, sind in den einzelnen Kapiteln des Abschnittes Forschungsmethoden ab Seite \pageref{methoden} beschrieben.

\begin{center}
	\begin{tabular}{L{0.25\textwidth} p{0.68\textwidth}}
		\toprule
		Bezeichnung & Kurzdefinition\\ \midrule
		Prozessnummer & Eindeutige Nummer eines Prozesses oder einer Methode.\\
		Prozessbeschreibung & Beschreibung des Prozesses oder der Methode. Insbesondere Beschreibung der Ausgangssituation und der Zielvorstellung.\\
		Ausgangsformat(e) & Format der Dateien, die am Anfang eines gesamten Prozesses stehen und den Ausgangspunkt bilden.\\
		Zwischenformat(e) & Format der Dateien, die im Verlauf eines Prozesses erzeugt werden und den Ausgangspunkt für weitere Prozesse bilden.\\
		Zielformat(e) & Format der Dateien, die am Ende eines Prozesses erzeugt werden.\\
		Durchführender & Person(en), die den Prozess durchgeführt hat (haben).\\
		Prozessbeginn & Datum, an dem der Prozess begonnen wurde. Datum und Zeit in UTC nach ISO 8601.\\
		Prozessende & Datum, an dem der Prozess beendet wurde. Datum und Zeit in UTC nach ISO 8601.\\
		Software & Software, mit der der Prozess durchgeführt wurde.\\
		Hardware & Hardware, auf der der Prozess durchgeführt wurde.\\
 		\bottomrule
		\bottomrule
	\end{tabular}
\end{center}


%##################################################################################
\label{grabungsdokumentation}
\subsection{Grabungsdokumentation}
Speziell für die Dokumentation von Grabungen und anderen archäologischen Maßnahmen sind weitere Metadaten erforderlich, die in der folgenden Tabelle aufgelistet sind. Dabei sollten auch die verschiedenen angewandten Methoden berücksichtigt werden, da diese ebenfalls den Umfang und die Art der Dokumentation beeinflussen. Wichtig bei einer Grabungsdokumentation ist, dass nicht nur die digitalen Daten, sondern auch die analogen Daten mit einer Dokumentation versehen sind, um beispielsweise auch Abhängigkeiten zwischen den verschiedenen Daten festzuhalten. Zum Beispiel sollten Zeichnungen mit den Befundbeschreibungen, Fotos mit den Fundstellenplänen oder Objektbeschreibungen mit den jeweiligen Objekten verknüpft sein.

\begin{center}
	\begin{longtable}{L{0.25\textwidth} p{0.68\textwidth}}
		\toprule
		Bezeichnung & Kurzdefinition\\ \midrule \endfirsthead
		\multicolumn{2}{l}{\footnotesize Fortsetzung der vorhergehenden Seite}\\
		\toprule
		Bezeichnung & Kurzdefinition\\ \midrule \endhead
		\bottomrule \multicolumn{2}{r}{{\footnotesize Fortsetzung auf der nächsten Seite}} \\
		\endfoot
		\bottomrule 
		\endlastfoot

		Fundstellenart & Angabe über die Art der Fundstelle. Mehrfachangaben sind möglich, wenn beispielsweise ein sich mit einer Siedlung überlagerndes Gräberfeld beschrieben werden soll.\\
		Maßnahmenart & Angabe über die Art der Untersuchung. Werte können beispielsweise sein: Ausgrabung, Survey, Baustellenbegleitung etc.\\
		Anlass & Angabe des Anlasses, der zur Durchführung der Maßnahme führt. Werte können beispielsweise sein: Rettungsgrabung, Notgrabung, Forschungsgrabung, Lehrgrabung etc.\\
		Grabungsmethodik & Angabe der angewandten Grabungsmethodik. Werte können beispielsweise sein: Flächengrabung, Schichtengrabung, Wheeler-Kenyon-Methode etc.\\
		Verfahren & Angabe über die angewandten Arbeitsverfahren, die nicht zu Dokumentationsverfahren gehören. Werte können beispielsweise sein: Schlämmen, Bohrung, Sieben etc.\\
		Datierung & Datierung der Fundstelle, beispielsweise anhand des archäologischen Fundmaterials. Eingruppierung in eine Epoche als relative Zeitstellung und optional als absolute Zeitangabe. Bei mehrphasigen Fundstellen werden mehrere Zeiträume angegeben.\\
		Fundstellenstatus & Angaben zum Schutzstatus einer Fundstelle.\\
		Bodenbeschaffenheit & Angabe über die natürlichen Gegebenheiten und die Bodenbeschaffenheit, die Einfluss auf die Erhaltungsbedingungen von Objekten und Strukturen haben. Werte können beispielsweise sein: Feuchtboden, Mineralboden, Unter Wasser etc.\\
		Strukturen & Verschlagwortung von aufgefundenen archäologischen Strukturen, die sich aus den Befunden ergeben.\\
		Funde & Verschlagwortung von signifikanten Fundgattungen.\\
		Richtlinie & Angaben zur Grabungs- oder Dokumentationsrichtlinie, die für die Maßnahme vorgegeben war oder gewählt wurde.\\
		Dokumentationssystem & Angabe des verwendeten Dokumentationssystems. Werte können beispielsweise sein: Stellenkartensystem, Single Context Recording, spezielle institutionelle Systeme etc.\\
		Dokumentationsmethoden & Angabe der verwendeten Dokumentationsmethoden. Werte können beispielsweise sein: Text, Foto, Zeichnung, 3D-Scan, Vermessung etc. \\
		\bottomrule
	\end{longtable}
\end{center}

Die große Zahl an Dokumentationsverfahren und Forschungsmethoden führt dazu, dass eine umfassende Grabungsdokumentation aus einer großen Menge unterschiedlicher Dokumente besteht, die folgendes enthalten sollte:
\begin{itemize}
	\item Allgemeine Angaben
	\item Grabungsplan
	\item Grabungstagebuch und Grabungsprotokoll
	\item Befundblätter und Befundliste
	\item Fundzettel und/oder Fundmeldung
	\item Fund- und Probenliste
	\item Datenbanken
	\item Dokumentation der angewandten Methoden, wie:
	\begin{itemize}
		\item Vermessungsmethoden
		\item Fotografie
		\item Photogrammetrie
		\item 3D-Scans
		\item Luftbildaufnahmen
		\item Naturwissenschaftliche Beprobung
		\item Zeichnung
		\item Allgemeine Beschreibungen
	\end{itemize}
	\item Abschlussbericht
\end{itemize}

Wie eine Grabungsdokumentation aussehen und welchen Umfang sie haben soll, wird in zahlreichen Richtlinien und Vorgaben spezifiziert, die bei der Planung des Vorhabens bereits berücksichtigt werden sollten. Dabei handelt es sich meist um Vorgaben von Landesdenkmalämtern. Online verfügbare Vorgaben sind bei den weiterführenden Informationen unter Vorgaben zur Grabungsdokumentation ab Seite \pageref{Metadaten-ListeLDA} aufgelistet.

Um die Grabungsdokumentation zu erleichtern und auch sicher zu stellen, dass alle erforderlichen Metadaten angegeben werden, sollten Vorlagen für Formulare und Checklisten bereits vor der Durchführung der Maßnahme erstellt werden. In der online verfügbaren Fassung der IT-Empfehlungen sind zu diesem Kapitel passende Vorlagen zur freien Verwendung zu finden. Darunter befinden sich eine Befundliste, eine Fotoliste, eine Fundliste, eine Geräteliste, ein Probenverzeichnis, ein Restaurierungsverzeichnis, ein Zeichnungsverzeichnis und ein Dokument zur Urheberrechtsverwaltung. Ein weiterer Anhaltspunkt sind die bereits genannten Richtlinien, sowie das Werk "`Tabellen und Tafeln zur Grabungstechnik"' von Andreas Kinne und das Grabungstechnikerhandbuch des Verbandes der Landesarchäologen. 