Bereits vor Beginn eines Forschungsvorhabens steht die Konzeption und Planung des Projektes. Dazu gehört auch eine Beschreibung über den Umgang mit den resultierenden Forschungsdaten, der sogenannte Datenmanagementplan. Ein vollständiger Datenmanagementplan berücksichtigt alle Phasen des auf Seite \pageref{lebenszyklus} beschriebenen Lebenszyklus von Forschungsdaten. Er dient zunächst als Mittel zur strukturierten Reflexion über datenrelevante Aspekte eines Projektes und beantwortet grundsätzliche Fragen zu Verantwortlichkeiten, Maßnahmen zur Pflege und Verarbeitung der Daten, Standards und bereits vorhandenen Daten. Außerdem bietet er die Grundlage für Arbeitsabläufe (Workflows) im Umgang mit Forschungsdaten und für eine Kostenkalkulation.

Eine vollständige Dokumentation des Datenmanagements in einem Datenmanagementplan spart Zeit und Kosten, wenn beispielsweise Zusammenhänge beim Wechsel von Mitarbeitern hergestellt werden sollen, und beugt einem Datenverlust vor. Da die Nachnutzung von Forschungsdaten zunehmend an Bedeutung gewinnt, setzen Geldgeber oft einen Datenmanagementplan als Teil eines Förderantrags voraus.

Die konsequente Einhaltung aller im Datenmanagementplan gemachten Vorgaben während der gesamten Projektlaufzeit stellt sicher, dass die Daten auch von Dritten interpretiert und nachgenutzt werden können. Wird das Überführen der Daten in ein Archiv schon von Beginn des Projektes an vorbereitet und einkalkuliert ist zum Projektabschluss nur noch ein geringer Aufwand für die Übergabe in ein Archiv erforderlich, weil die Umformatierung, Neustrukturierung oder nachträgliche Dokumentation von Daten wegfällt.

Ein aktives Datenmanagement beugt insbesondere während der Planung und Durchführung eines Projektes späteren Zeit- und Budgetverlusten vor und stellt sicher, dass Daten am Ende in nachhaltigen Formaten, gut dokumentiert und gut strukturiert vorliegen.
% implementing data management measures during the planning and development stages of research will avoid later panic and frustration

\subsection{Übersicht der Aufgaben in den Projektphasen}
Mit Hilfe des zu Beginn eines Projektes erstellten Datenmanagementplans können sämtliche Prozesse während des Projekts strategisch umgesetzt werden, wobei die Umsetzung des Plans als laufender Vorgang zu verstehen ist. Wenn während eines Projektes Änderungen an dem Datenmanagementplan notwendig werden, sollten diese begründet, dokumentiert sowie Arbeitsprozesse und Ergebnisse angepasst werden. Außerdem muss dokumentiert werden wie die Änderungen sich auf bereits bestehende Daten auswirken.

In den unterschiedlichen Phasen des Datenlebenszyklus und des Projektes sind die Aspekte des Datenmanagementplans unterschiedlich stark zu berücksichtigen. Die folgende Tabelle veranschaulicht, wann im Lebenszyklus von Forschungsdaten welchen Aufgaben aus dem Datenmanagement besondere Aufmerksamkeit zuteil werden muss.

In Abhängigkeit der Projektgröße kann der Umfang der Aufgaben des Datenmanagementplanes und der Plan selbst skaliert werden, wobei jedoch Minimalanforderungen einzuhalten sind, um zuallererst die Verfügbarkeit der Daten im laufenden Projekt zu gewährleisten. Um die Planungen zu erleichtern und zu beschleunigen,  sollten bereits vorhandene institutionelle Vorgaben und Infrastrukturen genutzt werden.

\begin{table}[hbt]
\begin{tabular}{r!\tbg c!\tbg c!\tbg c!\tbg c!\tbg c!\tbg c!\tbg c!\tbg}
	\arrayrulecolor{ianusGrau}
 	\multicolumn{1}{r}{} & \multicolumn{1}{c}{\rot{Vorbereitung}} & \multicolumn{1}{c}{\rot{Erstellung}}& \multicolumn{1}{c}{\rot{Verarbeitung}} & \multicolumn{1}{c}{\rot{Analyse}} & \multicolumn{1}{c}{\rot{Archivierung}} & \multicolumn{1}{c}{\rot{Zugang}} & \multicolumn{1}{c}{\rot{Nachnutzung}}\\
	\cline{2-8}
	Rahmendaten\tib{*} & $\tib{\CIRCLE}$ & $\tib{\Circle}$ & $\tib{\Circle}$ & $\tib{\Circle}$ & $\tib{\RIGHTcircle}$ & $\tib{\Circle}$ & $\tib{\Circle}$\\
	\cline{2-8}
	Verantwortlichkeiten\tib{*} & $\tib{\CIRCLE}$ & $\tib{\Circle}$ & $\tib{\Circle}$ & $\tib{\Circle}$ & $\tib{\Circle}$ & $\tib{\Circle}$ & $\tib{\Circle}$\\
	\cline{2-8}
	Rechtliche Aspekte\tib{*} & $\tib{\CIRCLE}$ & $\tib{\Circle}$ & $\tib{\Circle}$ & $\tib{\Circle}$ & $\tib{\CIRCLE}$ & $\tib{\CIRCLE}$ & $\tib{\CIRCLE}$\\
	\cline{2-8}
	Methoden & $\tib{\CIRCLE}$ & $\tib{\CIRCLE}$ & $\tib{\RIGHTcircle}$ & $\tib{\RIGHTcircle}$ & $\tib{\Circle}$ & $\tib{\Circle}$ & $\tib{\Circle}$\\
	\cline{2-8}
	Vorgaben und Standards\tib{*} & $\tib{\CIRCLE}$ & $\tib{\CIRCLE}$ & $\tib{\CIRCLE}$ & $\tib{\Circle}$ & $\tib{\CIRCLE}$ & $\tib{\RIGHTcircle}$ & $\tib{\Circle}$\\
	\cline{2-8}	
	Kosten \& Ressourcen\tib{*} & $\tib{\CIRCLE}$ & $\tib{\CIRCLE}$ & $\tib{\CIRCLE}$ & $\tib{\Circle}$ & $\tib{\RIGHTcircle}$ & $\tib{\Circle}$ & $\tib{\Circle}$\\
	\cline{2-8}
	Externe Partner & $\tib{\CIRCLE}$ & $\tib{\CIRCLE}$ & $\tib{\RIGHTcircle}$ & $\tib{\Circle}$ & $\tib{\Circle}$ & $\tib{\RIGHTcircle}$ & $\tib{\CIRCLE}$\\
	\cline{2-8}
	Hard- und Software\tib{*} & $\tib{\CIRCLE}$ & $\tib{\CIRCLE}$ & $\tib{\RIGHTcircle}$ & $\tib{\RIGHTcircle}$ & $\tib{\Circle}$ & $\tib{\Circle}$ & $\tib{\Circle}$\\
	\cline{2-8}
	Datentypen und Datenformate & $\tib{\RIGHTcircle}$ & $\tib{\CIRCLE}$ & $\tib{\CIRCLE}$ & $\tib{\Circle}$ & $\tib{\CIRCLE}$ & $\tib{\Circle}$ & $\tib{\Circle}$\\
	\cline{2-8}
	Nachnutzung vorhandener Daten\tib{*} & $\tib{\RIGHTcircle}$ & $\tib{\RIGHTcircle}$ & $\tib{\CIRCLE}$ & $\tib{\CIRCLE}$ & $\tib{\Circle}$ & $\tib{\RIGHTcircle}$ & $\tib{\Circle}$\\
	\cline{2-8}
	Datenerzeugung \& -prozessierung\tib{*} & $\tib{\RIGHTcircle}$ & $\tib{\CIRCLE}$ & $\tib{\RIGHTcircle}$ & $\tib{\Circle}$ & $\tib{\RIGHTcircle}$ & $\tib{\RIGHTcircle}$ & $\tib{\Circle}$\\
	\cline{2-8}
	Datenmenge & $\tib{\CIRCLE}$ & $\tib{\CIRCLE}$ & $\tib{\CIRCLE}$ & $\tib{\CIRCLE}$ & $\tib{\RIGHTcircle}$ & $\tib{\Circle}$ & $\tib{\Circle}$\\
	\cline{2-8}
	Dateispeicherung und -sicherung\tib{*} & $\tib{\RIGHTcircle}$ & $\tib{\CIRCLE}$ & $\tib{\CIRCLE}$ & $\tib{\CIRCLE}$ & $\tib{\Circle}$ & $\tib{\Circle}$ & $\tib{\Circle}$\\
	\cline{2-8}
	Dateiverwaltung & $\tib{\RIGHTcircle}$ & $\tib{\CIRCLE}$ & $\tib{\CIRCLE}$ & $\tib{\CIRCLE}$ & $\tib{\RIGHTcircle}$ & $\tib{\Circle}$ & $\tib{\Circle}$\\
	\cline{2-8}
	Dokumentation\tib{*} & $\tib{\RIGHTcircle}$ & $\tib{\CIRCLE}$ & $\tib{\CIRCLE}$ & $\tib{\CIRCLE}$ & $\tib{\CIRCLE}$ & $\tib{\RIGHTcircle}$ & $\tib{\Circle}$\\
	\cline{2-8}
	Qualitätssicherung & $\tib{\RIGHTcircle}$ & $\tib{\CIRCLE}$ & $\tib{\CIRCLE}$ & $\tib{\CIRCLE}$ & $\tib{\CIRCLE}$ & $\tib{\Circle}$ & $\tib{\CIRCLE}$\\
	\cline{2-8}
	Datenaustausch & $\tib{\RIGHTcircle}$ & $\tib{\RIGHTcircle}$ & $\tib{\CIRCLE}$ & $\tib{\CIRCLE}$ & $\tib{\Circle}$ & $\tib{\CIRCLE}$ & $\tib{\Circle}$\\
	\cline{2-8}
	Mittelfristige Datenaufbewahrung & $\tib{\RIGHTcircle}$ & $\tib{\CIRCLE}$ & $\tib{\CIRCLE}$ & $\tib{\CIRCLE}$ & $\tib{\Circle}$ & $\tib{\Circle}$ & $\tib{\Circle}$\\
	\cline{2-8}
	Langfristige Archivierung\tib{*} & $\tib{\RIGHTcircle}$ & $\tib{\RIGHTcircle}$ & $\tib{\RIGHTcircle}$ & $\tib{\RIGHTcircle}$ & $\tib{\CIRCLE}$ & $\tib{\Circle}$ & $\tib{\CIRCLE}$ \\
	\cline{2-8}
	Zugänglichkeit und Nachnutzung & $\tib{\CIRCLE}$ & $\tib{\RIGHTcircle}$ & $\tib{\RIGHTcircle}$ & $\tib{\RIGHTcircle}$ & $\tib{\CIRCLE}$ & $\tib{\CIRCLE}$ & $\tib{\CIRCLE}$\\
	\cline{2-8}
	Projektabschluss & $\tib{\RIGHTcircle}$ & $\tib{\Circle}$ & $\tib{\Circle}$ & $\tib{\Circle}$ & $\tib{\RIGHTcircle}$ & $\tib{\RIGHTcircle}$ & $\tib{\Circle}$\\
	\cline{2-8}
\end{tabular}
\caption{Tabellarische Übersicht über die verschiedenen zu berücksichtigenden Aspekte eines Datenmanagementplans während unterschiedlicher Projektphasen und unter Beachtung des Lebenszyklus von Forschungsdaten. Besonders wichtige, als Minimalanforderung zu betrachtende Aspekte des Datenmanagementplans sind mit einem Stern gekennzeichnet.}
\label{tab:dmp-lebenszyklus}
\end{table}
\begin{table}[h!bt]
\begin{tabular}{cr}
	$\tib{\CIRCLE}$ & Aufgabe ist während der Phase relevant.\\
	$\tib{\RIGHTcircle}$ & Aufgabe ist während der Phase teilweise relevant.\\
	$\tib{\Circle}$ & Aufgabe ist während der Phase nicht relevant.\\
\end{tabular}
\end{table}