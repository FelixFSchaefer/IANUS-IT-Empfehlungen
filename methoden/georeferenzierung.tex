%
% Georeferenzierung
%
Bitte beachten Sie, dass die Inhalte dieses Abschnittes den Inhalten des IT-Leitfadens des DAIs von 2011 entsprechen.
\begin{center}
\tib{\rule{0.9\textwidth}{0.2mm}}\vspace{3mm}
\end{center}

GeoReferenzierung, d.h. Zuordnung von Koordinaten zu Bildpunkten auf Fotos und Plänen (Karten), die in Digitaler Form vorliegen.
Mit der GeoReferenzierung erreicht man gleichzeitig eine Entzerrung dieser Grundlagen. Vorbedingung ist ein Programm, mit dem man diese Arbeit durchführen kann. Jedes GIS (Geoinformationssystem) hat notwendigerweise ein Modul für GeoReferenzierung.

\paragraph{Einsatz von Hard- und Software in der Praxis}
Die GeoReferenzierung von Topographischen Karten ist in der Regel einfach, da Koordinaten normalerweise in Form eines Rasters auf den Karten vorhanden sind. Es wird komplizierter, wenn man mit diesen Koordinaten nicht arbeiten möchte. Dann wird es erforderlich, entweder vor Ort Vergleichspunkte (Referenzpunkte) zu messen oder mit anderen Programmen die vorhandenen Koordinaten in das gewünschte System zu transformieren. Viele GIS-Programme erlauben auch das. Ich habe Erfahrungen mit dem einfachen FUGAWI, das für die Bearbeitung der GPS-Daten der Handgeräte vorhanden ist.

Eine andere Möglichkeit, Referenzpunkte zu bekommen, bildet im Moment GOOGLE EARTH.  Besonders in den Bereichen mit hoher Auflösung bekommt man eine ausreichend genaue Informationen. Zu berücksichtigen ist natürlich der Maßstab, in dem man später arbeiten möchte. 

Die Anzahl der Referenzpunkte richtet sich nach dem Inhalt des Images. Liegt eine flächenfüllende Karte vor, reichen mindestens 4 Punkte in den Ecken der Karte. Zu empfehlen sind Kontrollpunkte; bei einer Karte würde ich die doppelte Anzahl an Punkten wählen, um auch bei der Entzerrung sicher zu sein.

Bei Fotos, die kein Raster vorgegeben haben, muss man daher mehr Punkte wählen, die das Foto bis an die Ränder der Darstellung bearbeiten können oder beschneidet das Image auf die Größe , die benötigt wird. Die Punkte bezieht man aus Messungen vor Ort oder aus einer Karte oder anderen Medien (GOOGLE EARTH). Es müssen auch Punkte innerhalb des Images bestimmt werden, um die Kontrolle über das Bild zu erhalten. 4 Punkte sind da nicht ausreichend, ich empfehle die doppelte Anzahl. Das richtet sich nach der Geometrie des Fotos.

Wenn verfügbar, sollten digitale Orthophotos bzw. Luftbildkarten an Stelle von Luftbildern verwendet werden, da diese bereits auf ein digitales Geländemodell entzerrt sind. Bei ebenem Gelände kann eine Entzerrung gemäß Tabelle 1 vorgenommen werden, ohne dass dabei größere Lageungenauigkeiten  hervorgerufen werden. Durch eine Entzerrung von Luftbildern von unebenem Gelände mit polynomialen Ansätzen höherer Ordnung können keine Unstetigkeiten im Gelände wie beispielsweise steile Felshänge o.ä. berücksichtigt werden. Hier ist eine Entzerrung auf ein digitales Geländemodell vorzunehmen (s. Tab. 1).

Natürlich mangelt es auch hier nicht an Programmen, die diese Arbeiten durchführen können. Jedes GIS Programm hat ein Modul für die Georeferenzierung von Rasterdaten. Die Ergebnisse sind untereinander austauschbar, da das gleiche Dateiformat benutzt werden kann. Man kann also ein "`world-tiff oder jpg"` in ein anderes GIS Programm exportieren. 

AutoCad Map oder der AutoCad Aufsatz PhoToPlan bieten auch die Möglichkeit der Affinen Transformation oder sogar der Polynominalen Entzerrung. Leider sind die neu entstandenen Rasterdaten nur in diesen Programmen verwendbar. Beim Export dieser Rasterdaten in GIS Programme müssen sie wieder neu referenziert werden.

Auch bei Anwendung der Digitalen Bildentzerrung bei der Bauaufnahme oder bei Grabungsflächen oder Grabungsprofilen kann eine Georeferenzierung durchgeführt werden. Die Referenzpunkte müssen eingemessen werden, wobei darauf zu achten ist, dass die Digitale Bildentzerrung nur 2-D einsetzbar ist. Um Strukturen mit unterschiedlicher Höhe mit diesem Verfahren zu bearbeiten, ist es notwendig, die Ebenen separat zu definieren und zu entzerren. PhoToPlan erlaubt die Montage eines Bildverbands, solange genügend verbindende Passpunkte vorhanden sind.


Die Tabelle mit der Übersicht der unterschiedlichen Entzerrungs-und Referenzierungsmethoden entnehmen Sie bitte Seite 28 des IT-Leitfadens des DAIs (\url{http://www.ianus-fdz.de/it-empfehlungen/sites/default/files/ianusFiles/IT-Leitfaden_Teil2_v100_DAI.pdf})


\paragraph{Quellen}
\begin{flushleft}
Alle Handbücher der GIS Programme, AutoCad Map, PhoToPlan etc. \urllist{http://www.giub.uni-bonn.de/gistutor/}
\end{flushleft}