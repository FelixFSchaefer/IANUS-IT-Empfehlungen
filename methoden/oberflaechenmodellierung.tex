%
% Oberflächenmodellierung
%
Bitte beachten Sie, dass die Inhalte dieses Abschnittes den Inhalten des IT-Leitfadens des DAIs von 2011 entsprechen.
\begin{center}
\tib{\rule{0.9\textwidth}{0.2mm}}\vspace{3mm}
\end{center}

Bei Geländemodellen ist zwischen primären und sekundären Geländemodellen zu unterscheiden:

a) primäre Geländemodelle

Aus Messung entstandene Geländemodelle sind über die Vermessungsdaten zu speichern.

b) sekundäre Geländemodelle

Aus primären Geländemodellen abgeleitete Geländemodelle (z.B. erworbene Geländemodelle) sind als Rasterdaten im TIFF-Format abzuspeichern.

\paragraph{Eine kurze Einführung}
Um Oberflächen maßstabsgetreu darstellen zu können, müssen sie vorher durch geeignete Vermessungen aufgenommen werden. Dies kann durch Handvermessung geschehen oder aber durch den Einsatz moderner Messinstrumente. Je nach Art und Größe des Untersuchungsobjektes kommen unterschiedliche Methoden und Techniken zur Anwendung. Die Dokumentation der Oberflächen all dieser Objekte erfolgt prinzipiell nach den gleichen Methoden. Die folgenden Ausführungen konzentrieren sich auf die Dokumentation und Darstellung von Geländeoberflächen.

Als Messinstrumente kommen für mittlere und große Objekte Theodolithen bzw. Tachymeter (mit elektronischer Distanzmessung) zum Einsatz. Moderne Messverfahren bedienen sich auch der Photogrammetrie (Vermessung in stereoskopisch aufgenommenen Bildern) und neuerdings des Laserscannings oder LIDAR (engl. light detection and ranging, s. u.). Allen Aufnahmeverfahren ist gemeinsam, dass sie Lagekoordinaten der vermessenen Objekte in einem zuvor definierten Vermessungssystem liefern. Für eine planimetrische Darstellung (ohne Höhenangaben) reichen Messwerte in X- und Y-Richtung aus. Sie ermöglichen zweidimensionale Darstellungen (2D). Will man jedoch die Oberfläche eines Objektes oder von Gelände räumlich darstellen, benötigt man auch die Höheninformation, das heißt einen Z-Wert. Die Grundlage einer räumlichen Darstellung einer Geländeoberfläche ist somit eine Sammlung von Punkten mit X-, Y- und Höheninformation in einem Koordinatensystem.

Im Gegensatz zu einer echten dreidimensionalen Darstellung (3D), in der X-, Y- und Höhenkoordinaten zu einem Volumenmodell verarbeitet werden, werden in sogenannten 2,5D-Darstellungen 2D-Daten mit einer zusätzlichen Höheninformationen versehen, die nicht als Koordinate, sondern als Attribut gespeichert wird. Dies bedeutet, dass jedem XY-Punkt nur ein Höhenwert Z zugeordnet werden kann, was die Modellierung von Senkrechten oder Überhängen ausschließt. Geoinformationssysteme, die ursprünglich für großflächige geographische Anwendungen entwickelt wurden, verarbeiten bis heute in der Regel nur 2,5D-Daten, während in CAD-Anwendungen, die für die Objektmodellierung entwickelt wurden, 3D-Informationen verwendet werden. Die Entwicklung geht jedoch in Richtung einer Konvergenz beider Systeme, so dass in Zukunft voll 3D-fähige GIS zum Standard werden dürften.

\subparagraph{Vermessung eines Grabungsgeländes}
Die Aufnahme von Messdaten mit dem Ziel, Gelände zu modellieren, spielt in der archäologischen Praxis zum einen bei der Dokumentation von Grabungsorten vor dem Beginn oder während der Ausgrabungen und zum anderen bei der Durchführung von Regionalstudien mit Hilfe von Geoinformationssystemen eine Rolle. Grabungsgelände werden in der Regel durch tachymetrische Vermessung aufgenommen, während die Topographie größerer Gebiete durch erworbene digitalen Kartendaten, Fernerkundungsdaten oder von Luftbildern erarbeitet wird.

Der erster Schritt bei der Vermessung eines Grabungsgeländes ist die Anlage eines Messnetzes. Dieses wird zumeist genordet, oft aber auch an den zu erwartenden Grabungsbefunden ausgerichtet. In der Regel wird zunächst ein Ausgangspunkt und eine Grundlinie angelegt und von dort das Messnetz aufgebaut. In anderen Fällen wird von dem Ausgangspunkt ein Polygonzug angelegt, von dem aus weitere Geländepunkte eingemessen werden.

Das Messnetz sollte an ein bestehendes übergeordnetes Messnetz angebunden werden. Bezugspunkte können vermarkte Messpunkte eines Landesvermessungsnetzes oder in abgelegeneren Regionen Triangulationspunkte von Kartierungsprojekten oder ähnliches sein. Die Koordinaten der Hauptmesspunkte können auch durch präzise GPS-Messungen bestimmt werden. Häufig werden bei der Anlage von Grabungen willkürliche Messpunkte festgelegt, die durch spätere Messungen an die übergeordneten Messnetze angebunden werden (siehe hierzu auch Kapitel 2).

Ausgehend von den Basispunkten des Messnetzes werden mit dem Tachymeter Geländepunkte aufgemessen. Bei der Auswahl der Punkte ist es wichtig sich zu vergegenwärtigen, welche Art und welcher Maßstab der Darstellung des Geländes angestrebt wird. Grundsätzlich sollten diejenigen Punkte aufgenommen werden, an denen sich die Geländeneigung oder der Verlauf einer gedachten Höhenlinie (Linie, die sich aus Punkten gleicher Höhe zusammensetzt) ändert. Das heißt, je unregelmäßiger die Geländeoberfläche ist und je detaillierter die Geländeoberfläche wiedergegeben werden soll, umso mehr Punkte pro gegebener Fläche sind aufzunehmen.

Die Punkte sind so dicht zu setzen, dass die relevanten Geländeformen in der graphischen Darstellung im angestrebten Maßstab wiedergegeben werden. Es ist einleuchtend, dass ein etwa 30 cm breiter Mauerzug, der sich im Gelände durch eine leichte Erhebung abzeichnet, nicht durch Punkte abgebildet werden kann, die im Abstand von 1 m gemessen wurden. Vielmehr muss zumindest der Beginn der Geländewölbung, der Scheitelpunkt über der Mauer und der Endpunkt der Geländewölbung vermessen werden. Bruchkanten müssen gesondert vermessen werden. Sie können bei der Auswertung der Messdaten mit Computerprogrammen speziell behandelt werden.

\subparagraph{Regionale Geländemodelle}
Ist das Untersuchungsgebiet größer als ein Grabungsgelände, werden dafür benötigte Geländemodelle üblicherweise nicht selbst erzeugt, sondern aus anderen Quellen erworben. Bezugsquellen für digitale Geländemodelle sind je nach Land staatliche Vermessungsämter, Universitäts- oder Forschungsinstitute für Geographie, Geodäsie, Kartographie oder Geophysik, Luft- und Raumfahrtagenturen oder das Militär.

Grundlage für die Erzeugung von Geländemodellen sind Fernerkundungsdaten, die entweder mit passiven Sensoren (Luft- und Satellitenbilder) oder mit aktiven Sensoren (Radar- oder Lidardaten) aufgenommen werden (siehe hierzu die Beiträge von Sabine Reinhold). Bilddaten werden mit photogrammetrischen Methoden ausgewertet. In überlappenden Bereichen benachbarter Bilder werden stereoskopisch homologe Punkte an der Erdoberfläche gemessen (engl. matching). Entweder werden ähnlich wie bei der terrestrischen Geländevermessung gezielt Einzelpunkte und ggfs. Bruchkanten gemessen, oder es werden bei automatisierten Verfahren Punkte in gleichmässigen Abständen gemessen. Photogrammetrische Auswertungen von Luftbildern bilden bis heute in vielen Ländern die Grundlage für die Produktion und Nachführung von topographischen Karten und für die Generierung von Geländemodellen.

Radar- und Lidardaten liefern demgegenüber direkte Informationen über die Distanz der gemessenen Punkte zum Sensor. Ein auf Radar (engl. radio detection and ranging) beruhendes Geländemodell der Shuttle Radar Topography Mission (SRTM) ist praktisch weltweit frei verfügbar und in vielen Gegenden die einfachste Möglichkeit, digitale Geländemodelle zu beziehen. Die geringe Auflösung von 90 m sowie Lücken und große Höhenfehler in gebirgigem Gelände sind jedoch limitierende Faktoren für die Verwendbarkeit. Vermessungen mit flugzeuggestütztem LIDAR (engl. light detection and ranging), bei dem die Erdoberfläche in hoher Dichte durch einen Laserstrahl gescannt wird, werden in Europa immer mehr zu einem Standardverfahren der Landesvermessung, sind jedoch in anderen Ländern oft noch nicht verfügbar.

Zur korrekten Georeferenzierung der Fernerkundungsdaten dienen GPS-Messungen, oft in einer Kombination von Positionsmessungen an Bord der Sensorplattform (z.B. GPS an Bord eines Flugzeuges gekoppelt mit INS, engl. inertial navigation system) mit Einmessungen von Kontrollpunkten am Boden, die in den Daten erkennbar sein müssen. Die GPS-Messungen dienen der Überführung der gemessenen Punkte von einem lokalen in ein globales Koordinatensystem (Georeferenzierung).

Erworbene Geländemodelle sind rechnerische Ableitungen von tatsächlich erhobenen Messdaten. Wie schon bei anderen Messdaten, sollten nach Möglichkeit auch die Rohdaten gespeichert werden. Beim Kauf von Standard-Produkten ist dies üblicherweise nicht möglich, wohl aber bei der Beauftragung von speziellen Geländemodellen, z.B. Lidar-Überfliegungen.

\subparagraph{Modellierung}
Zur Modellierung einer durchgehenden Oberfläche auf der Grundlage der vermessenen Punkte müssen die zwischen den Messpunkten liegenden Geländepunkte durch rechnerische Verfahren verknüpft werden. Dazu werden zunächst die Einzelpunkte zu einer aus Dreiecken bestehenden Oberfläche vermascht. Diese liegt im Vektorformat vor (engl. TIN: triangulated irregular network). Durch Interpolation werden die Vektordaten ggfs. in ein Rasterformat (z.B. Grid) mit einer festen Maschenweite überführt. Hierzu werden verschiedene mathematische Verfahren angewandt. Die Einzelheiten dieser Verfahren werden in den relevanten Computerprogrammen und Lehrbüchern erläutert, müssen jedoch vom Archäologen nicht im Detail nachvollzogen werden können. Wichtig ist jedoch, die Modellierungen mit den verschiedenen Verfahren zu testen, um entscheiden zu können, welche Methode die tatsächliche Geländeoberfläche am realistischsten darstellt. Die Maschenweite oder Zellengröße ist ein übliches Maß für die Auflösung eines Geländemodells. Während einige GIS-Programme (z.B. ArcGIS) auch TINs verarbeiten können, können viele andere nur mit Raster-Geländemodellen umgehen.

Je nach Platzierung der zu Grunde liegenden Messpunkte bilden Geländemodelle verschiedene Dinge ab. Als Oberbegriff dient die Bezeichnung Digitales Höhenmodell / DHM (engl. digital elevation model / DEM). Man unterscheidet zwei Typen von DHMs: 1. das Digitale Oberflächenmodell / DOM (engl. digital surface model / DSM), das die Erdoberfläche inklusive aller darauf befindlichen Objekte wie Häuser, Vegetation, etc. abbildet, und 2. das Digitale Geländemodell / DGM (engl. digital terrain model / DTM), das nur die Erdoberfläche abbildet. Bei der eigenen Vermessung des Grabungsgeländes wird man üblicherweise ein DGM erzeugen. Auch bei erworbenen Geländemodellen sind DGMs den DOMs vorzuziehen, da in archäologischen Untersuchungen üblicherweise die Geländegestalt ohne moderne Bebauung und Vegetation interessiert. Bei Geländemodellen mit geringer Auflösung wie dem SRTM-DHM sind die Unterschiede jedoch zu vernachlässigen.


\paragraph{Einsatz von Hard- und Software in der Praxis}
Für die Verarbeitung und die Interpolation der Messpunkte stehen unterschiedliche Computer-Programme zur Verfügung, darunter auch speziell auf die Messgeräte angepasste Software der Hersteller von Vermessungsgeräten (Zeiss, Leica, Wild, Sokkia, Trimble, etc.). In der archäologischen Praxis wird zur Verarbeitung und Umzeichnung von echten 3D-Messdaten zumeist das Programm AutoCAD verwendet (\url{http://www.autodesk.de}). Zur Generierung von Höhenlinien und 3D-Modellen können Zusatzmodule oder erweiterte Versionen erworben werden. Zur Modellierung von 2,5D-Geländeoberflächen kann auch jedes GIS-Programm verwendet werden.
Im Rahmen unserer Arbeiten wurden die Höhenlinien und Geländemodelle in der Regel mit dem Programm Surfer (\url{http://www.ssg-surfer.com/}) generiert. Surfer ist einfach zu handhaben und bietet eine Vielzahl von Möglichkeiten der Verarbeitung und Analyse archäologischer Daten. Unter anderem können auch Daten geophysikalischen Prospektionen einfach eingelesen und über die Geländemessdaten projiziert werden. Surfer ermöglicht auch die schnelle und einfache Überprüfung der Messergebnisse nach der Übertragung der Daten auf den Computer, wenn der Tachymeter nicht ohnehin mit einem Bildschirm ausgestattet ist. Die Koordinatenlisten lassen sich nachbearbeiten, Bruchkanten einfügen und nicht benötigte Messpunkte ausblenden. Die Höhenlinienpläne können als DXF-Dateien in AutoCAD exportiert werden. Schattierte oder farblich gestaltete Geländemodelle lassen sich in verschiedenen Graphikformaten (tiff, jpeg, bmp, etc.) exportieren und als solche ebenfalls in AutoCAD einbinden.

Erworbene Geländemodelle liegen üblicherweise in einem Rasterformat vor, in dem einer durch eine XY-Eckkoordinate definierten Zelle mit einer festgelegten Seitenlänge ein Höhenwert zugeordnet wird. Damit gleichen Raster-Geländemodelle im Prinzip Bilddaten, nur dass statt eines Farbwertes ein Höhenwert gespeichert wird. Deshalb können Geländemodelle z.B. auch als TIFF vorliegen. Ein anderes übliches Format ist ASCII; daneben gibt es proprietäre Formate wie z.B. ESRI Grid. Die Transformation von Raster-Geländemodellen von einem Format in ein anderes in einem GIS-Programm stellt üblicherweise kein unüberwindbares Problem dar, so dass hier kein Format empfohlen wird, zumal bei Standard-Produkten das Format oft bereits festgelegt ist.

Je nach gewähltem Ausgabemedium werden verschiedene Verfahren der Darstellung der Geländeoberflächen gewählt. Soll das Ergebnis auf Papier, also in 2D auf einem Plan als Arbeitsdokument oder für die Publikation ausgegeben werden, wird die Darstellung in Form von Höhenlinien, mit Schattierungen oder unterschiedlicher Farbgebung gewählt. AutoCAD-Dateien (*.dwg) lassen sich häufig nur mit Schwierigkeiten in andere vektorbasierte Programme (z.B. CorelDraw oder Adobe Illustrator) zur weiteren Bearbeitung exportieren. Für die Einbindung in Publikationen können AutoCAD-Pläne und -Zeichnungen jedoch als PDF-Dateien ausgegeben, in JPEG-Dateien umgewandelt und mit Bildbearbeitungsprogrammen ergänzt oder verändert werden. Am Computer können die räumlichen Modelle dargestellt, bewegt und von unterschiedlichen Standpunkten aus betrachtet werden.

GIS-Programme bieten die Möglichkeit, die in Geländemodellen enthaltene Höheninformation in verschiedener Form in 2D darzustellen, z.B. als Höhenlinien, Grauwerte oder Farbabstufungen. Das Layout von Karten ist in den verschiedenen Programmen unterschiedlich einfach zu handhaben, jedoch sind die grundlegenden Funktionen wie die automatische Platzierung eines Koordinatenrahmens, eines Maßstabes etc. überall vorhanden, so dass Karten im GIS-Programm erzeugt und entweder direkt ausgedruckt oder in ein Standard-Graphikformat (Vektor oder Raster) exportiert werden können.

\paragraph{Quellen}
\begin{flushleft}
Archaeology Data Service 2007: Guides to Good Practice. \urllist{http://ads.ahds.ac.uk/project/goodguides/g2gp.html}

Bill, Ralf 1999: Grundlagen der Geo-Informationssysteme, Band 2: Analysen, Anwendungen und neue Entwicklungen. 2. Aufl. Heidelberg: Wichmann. Kapitel 2.5: Digitales Geländemodell (DGM).

Conolly, James \& Mark Lake 2006: Geographical Information Systems in Archaeology (Cambridge Manuals in Archaeology). Cambridge: Cambridge University Press. Kapitel 6: Building Surface Models.

Howard, Phil 2007: Archaeological Surveying and Mapping: Recording and Depicting the Landscape. London: Routledge.

Wheatley, David \& Mark Gillings 2002: Spatial Technology and Archaeology: The Archaeological Applications of GIS. London: Taylor \& Francis. Kapitel 5: Digital Elevation Models.
\end{flushleft}