%Kommando für Titeltext
\newcommand{\titel}[1]{{\LARGE \sffamily #1}}

%Führende Null im Datum
\newcommand{\leadingzero}[1]{\ifnum #1<10 0\the#1\else\the#1\fi}

%Kurzer Befehl für das heutige Datum
\newcommand{\datum}{\the\day.\the\month.\the\year}

\newcommand{\HRule}{\rule{\linewidth}{0.4mm}}

%\begin{document}
\thispagestyle{empty}
\begin{titlepage}
%\setlength{\parskip}{2mm plus 1mm minus 1mm}

\begin{center}
% Upper part of the page
\begin{flushleft}
\includegraphics[width=0.13\textwidth]{../deckblattLogos/ianus.png}\\[3cm]
\end{flushleft}
% Title
\titel{Kurzfassung der IT-Empfehlungen für den nachhaltigen Umgang mit digitalen Daten in den Altertumswissenschaften}\\[0.2cm]
\tib{\HRule \\[1.0cm]}
\large
\emph{Herausgeber:}\\
\textsc{IANUS}\\[1cm]
Version 1.0.1.0\\
{\large \today}
\end{center}

% Bottom of the page
\vfill
\begin{flushright} \sffamily
\minipage{0.155\textwidth}
Koordination\\
\includegraphics[width=\textwidth]{../deckblattLogos/dai.jpg} 
\endminipage
\hspace{1cm}
\minipage{0.155\textwidth}
Förderung\\
\includegraphics[width=\textwidth]{../deckblattLogos/dfg.jpg}
\endminipage

\end{flushright}
\end{titlepage}