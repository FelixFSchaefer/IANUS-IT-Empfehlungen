%Um hyperref Warning: The PDF version number could not be set, abzuschalten
\pdfminorversion=4\relax
\pdfobjcompresslevel=0\relax

\documentclass[a4paper]{report}
\usepackage[usenames,dvipsnames]{xcolor}
\usepackage[ngerman]{babel}
\usepackage[utf8x]{inputenc} 
\usepackage[T1]{fontenc}
\inputencoding{utf8}
\makeatother
\usepackage{lmodern}           

%Bilder
\usepackage{graphicx}
%Nutzung: \includegraphics{eps; png; jpeg; gif; pdf}
%Text um Bilder herum
\usepackage{wrapfig}
\usepackage{adjustbox} %Boxen anpassen; vor allem für wrapfigures in itemize (bsp: vektor)
%Eigene Grafiken und Diagramme bauen
\usepackage{tikz}
\usetikzlibrary{positioning, matrix, arrows, shapes.geometric, fit, backgrounds, scopes}
%Damit mit b platzierte Grafiken nicht unter der Fußnote stehen
\usepackage[bottom]{footmisc}

%Tabellenspezifisches
\usepackage{array}
\usepackage{booktabs}
\usepackage{multirow}
\usepackage{colortbl}
\usepackage{caption}
\usepackage{subcaption}
\usepackage{longtable}

%Diverse Symbole
\usepackage{amssymb}
\usepackage[nointegrals]{wasysym}
%Ausschalten der Warnung:
%Latex Font Warning: Font shape `U/wasy/b/n' in size <7> not available
%(Font)              Font shape `U/wasy/m/n' tried instead on input line X
\wasyfamily
\DeclareFontShape{U}{wasy}{b}{n}{ <-10> ssub * wasy/m/n
 <10> <10.95> <12> <14.4> <17.28> <20.74> <24.88>wasyb10 }{}

\usepackage{ccicons}

%Zur Anpassung der Titelformatierung
\usepackage{titlesec}

%Für syntax highlighting von Codeschnipseln (z.B. XML)
\usepackage{listings}

%Für Listeneinstellungen
\usepackage{enumitem}


\usepackage[hyphens]{url}
%Pakete für PDF/A
%\usepackage[a-1a]{pdfx}
\usepackage[a-1b]{pdfx}
%\hypersetup{pdfencoding=unicode}
\input{glyphtounicode.tex}
\input{glyphtounicode-cmr.tex}
%\pdfglyphtounicode{CIRCLE}{0041} %0041=A
\pdfgentounicode=1
\usepackage[pdfa]{hyperref}
\usepackage{bookmark}

%\usepackage[acronym, toc, section, nonumberlist]{glossaries}
%\makeglossaries

\title{Kurzfassung der IT-Empfehlungen für den nachhaltigen Umgang mit digitalen Daten in den Altertumswissenschaften}
%\author{XX \and Xy \and Z}
\date{IANUS\\
Version 1.0.0.0\\ 
\today}

%Auch subsubsections sollen ins Inhaltsverzeichnis
\setcounter{secnumdepth}{3}
\setcounter{tocdepth}{3}

%Silbentrennung für das ganze Dokument (max. 300 Wörter!)
%Für die einzelnen Dateien kann noch mal eine Liste angelegt werden
\hyphenation{
al-ter-tums-wis-sen-schaft-lich-er
For-schungs-pro-jek-te
Aus-zeich-nungs-spra-che
Ge-ne-ra-tions-ver-lust
Digi-ta-li-sa-te
Digi-ta-li-sierung
un-kom-pri-mier-te
Da-tei-na-men
Da-tei-na-mens-er-wei-te-rung
PDF-Do-ku-men-ten
}

%compilieren des documentes mit nur einem der inkludierten files:
%\includeonly{dateiformate}
\begin{document}
%\newcommand{Achtung}[1]{\parbox[c]{1.1\width}{#1}}

%Für URL-Angaben in den Toollisten und den Quellangaben
\newcommand{\urllist}[1]{\newline \hspace*{0.08cm} {\parbox[c]{0.98\textwidth}{\itshape \url{#1}}\\ \vspace*{0.15cm}}}

%Für Abstand nach Quelle ohne URL in Quellangaben
%XXXHängender Absatz besser?
\newcommand{\abstand}{\\ \vspace*{0.15cm}}

%IANUS-Farben
\definecolor{ianusGrau}{cmyk}{0.07, 0, 0, 0.25}
\definecolor{ianusBlau}{cmyk}{0.851, 0.561, 0, 0.42}

\definecolor{ianusBlauHell}{cmyk}{0.851, 0.561, 0, 0.0}

%Kommando für Text in ianusBlau
\newcommand{\tib}[1]{{\color{ianusBlau}#1}}

%kurzes Kommando für vertikale Tabellenstreifen in ianusGrau
\newcommand{\tbg}{{\color{ianusGrau}\vrule}}

%Rotierter Text in Tabellen
\newcommand{\rot}[1]{\makebox[1em][l]{\rotatebox{45}{#1}}}

%Spalten bestimmter breite mit automatischen Textumbruch und links, zentriert oder rechts aligned
\newcolumntype{L}[1]{>{\raggedright\let\newline\\\arraybackslash\hspace{0pt}}p{#1}}
\newcolumntype{C}[1]{>{\centering\let\newline\\\arraybackslash\hspace{0pt}}p{#1}}
\newcolumntype{R}[1]{>{\raggedleft\let\newline\\\arraybackslash\hspace{0pt}}p{#1}}

%Formatierung der Überschriften
\newcommand{\hsp}{\hspace{20pt}}
%\titleformat{\chapter}[hang]{\Huge\bfseries}{\thechapter\hsp\textcolor{ianusGrau}{|}\hsp}{0pt}{\Huge\bfseries}
\titleformat{\chapter}[hang]{\huge\sffamily\color{ianusBlau}}{\thechapter\hspace{5pt}\textcolor{ianusGrau}{\bfseries |}\hspace{20pt}}{0pt}{\huge\sffamily\color{ianusBlau}}
\titlespacing*{\chapter}{0pt}{-\topskip}{8pt}

\titleformat*{\section}{\Large\sffamily\color{ianusBlau}}
\titlespacing*{\section} {0pt}{3.5ex plus 1ex minus .2ex}{0.8ex plus .2ex}
\titleformat*{\subsection}{\large\sffamily\color{ianusBlau}}
\titlespacing*{\subsection} {0pt}{3.25ex plus 1ex minus .2ex}{0.7ex plus .2ex}
\titleformat*{\subsubsection}{\sffamily\color{ianusBlau}}
\titlespacing*{\subsubsection}{0pt}{3.25ex plus 1ex minus .2ex}{0.6ex plus .2ex}
\titleformat{\paragraph}[hang]{\bfseries\sffamily\color{ianusBlau}}{\theparagraph}{1em}{}
\titlespacing*{\paragraph} {0pt}{3.25ex plus 1ex minus .2ex}{0.6ex plus .2ex}
\titleformat*{\subparagraph}{\bfseries\color{ianusBlau}}

%Überschrift in den Quellangaben
\newcommand{\quelltyp}[1]{\vspace*{0.2cm} {\bfseries\sffamily\color{ianusBlau} #1}\linebreak}

%Anpassung der Listenformatierung
\renewcommand{\labelitemi}{{\color{ianusGrau}{\bfseries -}}}
\renewcommand{\labelitemii}{{\color{ianusGrau}{\bfseries -}}}
\renewcommand{\labelitemiii}{{\color{ianusGrau}{\bfseries -}}}

\setlist[itemize]{leftmargin=\parindent} %Einzug
%\setlist[itemize,2]{leftmargin=0} %Einzug, zweite Listenebene
\setlist[itemize]{itemsep=0.2\itemsep} %Abstand zwischen items
\setlist[itemize,2]{topsep=-3pt} %Abstand oben für zweite Listenebene

%Anpassung der Angaben in Bildern (und Tabellen)
\renewcommand{\figurename}{Abb.}
%\renewcaptionname{ngerman}{\tablename}{Tab.}

%Kommando zur Angabe der Autoren
\newcommand{\abschnittsautor}[1]{\vspace{-1mm}{\footnotesize \itshape #1 \vspace{0.3cm}\newline}}


%\maketitle
\hypersetup{pageanchor=false}
%Kommando für Titeltext
\newcommand{\titel}[1]{{\LARGE \sffamily #1}}

%Führende Null im Datum
\newcommand{\leadingzero}[1]{\ifnum #1<10 0\the#1\else\the#1\fi}

%Kurzer Befehl für das heutige Datum
\newcommand{\datum}{\the\day.\the\month.\the\year}

\newcommand{\HRule}{\rule{\linewidth}{0.4mm}}

%\begin{document}
\thispagestyle{empty}
\begin{titlepage}
%\setlength{\parskip}{2mm plus 1mm minus 1mm}

\begin{center}
% Upper part of the page
\begin{flushleft}
\includegraphics[width=0.13\textwidth]{../deckblattLogos/ianus.png}\\[3cm]
\end{flushleft}
% Title
\titel{Kurzfassung der IT-Empfehlungen für den nachhaltigen Umgang mit digitalen Daten in den Altertumswissenschaften}\\[0.2cm]
\tib{\HRule \\[1.0cm]}
\large
\emph{Herausgeber:}\\
\textsc{IANUS}\\[1cm]
Version 1.0.1.0\\
{\large \today}
\end{center}

% Bottom of the page
\vfill
\begin{flushright} \sffamily
\minipage{0.155\textwidth}
Koordination\\
\includegraphics[width=\textwidth]{../deckblattLogos/dai.jpg} 
\endminipage
\hspace{1cm}
\minipage{0.155\textwidth}
Förderung\\
\includegraphics[width=\textwidth]{../deckblattLogos/dfg.jpg}
\endminipage

\end{flushright}
\end{titlepage}
\newpage
%Kommando für Text in ianusBlau, ohne Serifen
\newcommand{\tbsf}[1]{{\color{ianusBlau}\sffamily\normalsize#1}}

%Kommando für Text in ianusBlau, ohne Serifen in fett
\newcommand{\tbsff}[1]{{\color{ianusBlau}\sffamily\bfseries#1}}

%Breitere Zeilen in Tabellen
\renewcommand{\arraystretch}{1.2}

\thispagestyle{empty}
\urlstyle{same}

\begin{flushleft}
\tbsff{Herausgeber} IANUS -- Maurice Heinrich, Felix Schäfer, Martina Trognitz \vspace{2mm}

\tbsff{Autoren:} siehe Autorenverzeichnis \vspace{2mm}

\tbsff{Titel:} IT-Empfehlungen für den nachhaltigen Umgang mit digitalen Daten in den Altertumswissenschaften \vspace{2mm}

\tbsff{Sprache:} Deutsch \vspace{4mm}

%\tbsff{doi:10.13149/000.y47clt-t}

\end{flushleft}

\begin{center}
	\fboxrule0.3mm
 	\fcolorbox{ianusGrau}{white}{\parbox[c]{0.97\textwidth}{{\sffamily\color{ianusBlau}Zitierhinweis:}\\ IANUS (Hrsg.), IT-Empfehlungen für den nachhaltigen Umgang mit digitalen Daten in den Altertumswissenschaften (2017) [Version 1.0.1.0]
 	}}
\end{center}%\vspace{1mm}

\begin{flushleft}

\tbsff{Kontakt:}\\
ianus-fdz@dainst.de\\
\url{www.ianus-fdz.de}\\ \vspace{5mm}

\tbsff{Lizenz:} \vspace{3mm}

\setlength{\tabcolsep}{0.5pt}
\begin{tabular}{c c c c}
	\multirow{2}{*}{\raisebox{2.5mm}{\color{ianusBlau} \Huge \ccLogo}} 	& {\color{ianusBlau} \large \ccAttribution}		& {\color{ianusBlau} \large \ccShareAlike} \\
	 																									& \color{ianusBlau} \sffamily \scriptsize BY	& \color{ianusBlau} \sffamily \scriptsize SA \\
\end{tabular}

\vspace{2mm} Dieses Werk bzw. Inhalt ist lizenziert unter einer Creative Commons Namensnennung -- Weitergabe unter gleichen Bedingungen 4.0 International Lizenz. \\
\url{http://creativecommons.org/licenses/by-sa/4.0/}
\end{flushleft}

\vspace{2cm}

\begin{center}
	\begin{longtable}{!\tbg p{0.09\textwidth}!\tbg p{0.2\textwidth}!\tbg p{0.14\textwidth}!\tbg L{0.42\textwidth}!\tbg} %\small
		\arrayrulecolor{ianusGrau}
		\multicolumn{4}{l}{Versionshistorie}\\
		\multicolumn{1}{l}{\tbsf{Version}} & \multicolumn{1}{l}{\tbsf{Autor}} & \multicolumn{1}{l}{\tbsf{Datum}} & \multicolumn{1}{l}{\tbsf{Beschreibung}} \\
		\endfirsthead
		\multicolumn{4}{l}{\footnotesize Fortsetzung der vorhergehenden Seite}\\
		\multicolumn{1}{l}{\tbsf{Version}} & \multicolumn{1}{l}{\tbsf{Autor}} & \multicolumn{1}{l}{\tbsf{Datum}} & \multicolumn{1}{l}{\tbsf{Beschreibung}} \\ 
		\cline{1-4} 
		\endhead
		\cline{1-4}
		\multicolumn{4}{r}{{\footnotesize Fortsetzung auf der nächsten Seite}} \\
		\endfoot
		\cline{1-4}
		\endlastfoot
	\cline{1-4}
	0.9 & IANUS & 18.02.2014 & Erste veröffentlichte Fassung. \\
	\cline{1-4}
	0.9 & IANUS & 11.03.2014 & Änderung der Lizenzbedingungen. \\
	\cline{1-4}
	0.95 & IANUS & 05.02.2015 & Kapitel "`3D und Virtual Reality"' und "`Textdokumente"' neu. Anpassungen im Layout und kleinere Korrekturen. \\
	\cline{1-4}
	0.96 & IANUS & 27.05.2015 & Kapitel "`Datenmanagement"' neu, Ergänzung Versionskontrolle und Korrekturen. \\
	\cline{1-4}
	0.97 & IANUS & 09.12.2015 & Kapitel "`Tabellen"' neu, Ergänzungen im Glossar und Korrekturen. \\
	\cline{1-4}
	0.97.5 & IANUS & 21.07.2016 & Kapitel "`Video"', "`Audio"' und "`RTI"' neu, Ergänzungen im Glossar, Anpassungen im Kapitel "`Archivierung bei IANUS"' und Korrekturen. \\
	\cline{1-4}
	0.97.5 & IANUS & 13.09.2016 & Einarbeitung Reviews für "`Video"', "`Audio"' und "`RTI"', Ergänzungen im Glossar und Korrekturen. \\
	\cline{1-4}
	0.98 & IANUS & 22.11.2016 & Überarbeitung "`Datenmanagement"' und "`3D und Virtual Reality"', Ergänzung Bild in "`RTI"'.\\
	\cline{1-4}
	0.99 & D. Hagmann & 09.12.2016 & Kapitel "`Webseiten"' neu.\\
	\cline{1-4}
	1.0 & IANUS & 15.12.2016 & Kapitel "`Vektorgrafiken - Übersicht"', Kapitel "`Datenbanken - Übersicht, Vertiefung"' neu. Korrekturen.\\
	\cline{1-4}
	1.0.0.0 & IANUS & 29.03.2017 & Kapitel "`Vektorgrafiken - Vertiefung, Praxis"', Kapitel "`Datenbanken - Praxis"' neu. Ergänzungen im Glossar und Korrekturen.\\
	\cline{1-4}
	1.0.1.0 & IANUS & 18.11.2017 & Überarbeitung der Kapitel "`Dokumentation"' und "`Dateiverwaltung"'.\\
	\cline{1-4}
	 &  &  &  \\
\end{longtable}
 	
\end{center}
\hypersetup{pageanchor=true}
\tableofcontents
%\newpage

\definecolor{eclipseGreen}{RGB}{63,127,95}
%Einstellungen für Code-Blocks mit listings;
%Ab hier überall gültig
\lstset{
  basicstyle=\ttfamily\small,
  columns=fullflexible,
  showstringspaces=false,
	tabsize=2,
	linewidth=0.58\linewidth}
\lstdefinelanguage{XML}
{ morestring=[b]",
  morestring=[s]{>}{<},
  stringstyle=\color{black},
  identifierstyle=\color{ianusBlau},
  keywordstyle=\color{ianusGrau},
  morekeywords={typ, einheit, count, link, name, creationTime, version, encoding, path, creationDate}}
\lstdefinelanguage{HTML}
{ morecomment=[s]{<!--}{-->},
	commentstyle=\color{eclipseGreen},
  stringstyle=\color{black},
  identifierstyle=\color{ianusBlau},
  keywordstyle=\color{ianusGrau}
	}
\lstdefinelanguage{SQL}	
{	morecomment=[s]{/*}{*/},
	morecomment=[l]{--},
	commentstyle=\color{eclipseGreen},
	stringstyle=\color{black},
  %identifierstyle=\color{ianusBlau}%,
  keywordstyle=\color{ianusBlau},
	morekeywords={CREATE, TABLE, NOT NULL, SELECT, FROM, WHERE, UPDATE, SET, USER, PASSWORD}
	}

%Damit Sonderzeichen in den listings auch richtig gesetzt werden
\lstset{literate=
  {á}{{\'a}}1 {é}{{\'e}}1 {í}{{\'i}}1 {ó}{{\'o}}1 {ú}{{\'u}}1
  {Á}{{\'A}}1 {É}{{\'E}}1 {Í}{{\'I}}1 {Ó}{{\'O}}1 {Ú}{{\'U}}1
  {à}{{\`a}}1 {è}{{\`e}}1 {ì}{{\`i}}1 {ò}{{\`o}}1 {ù}{{\`u}}1
  {À}{{\`A}}1 {È}{{\'E}}1 {Ì}{{\`I}}1 {Ò}{{\`O}}1 {Ù}{{\`U}}1
  {ä}{{\"a}}1 {ë}{{\"e}}1 {ï}{{\"i}}1 {ö}{{\"o}}1 {ü}{{\"u}}1
  {Ä}{{\"A}}1 {Ë}{{\"E}}1 {Ï}{{\"I}}1 {Ö}{{\"O}}1 {Ü}{{\"U}}1
  {â}{{\^a}}1 {ê}{{\^e}}1 {î}{{\^i}}1 {ô}{{\^o}}1 {û}{{\^u}}1
  {Â}{{\^A}}1 {Ê}{{\^E}}1 {Î}{{\^I}}1 {Ô}{{\^O}}1 {Û}{{\^U}}1
  {œ}{{\oe}}1 {Œ}{{\OE}}1 {æ}{{\ae}}1 {Æ}{{\AE}}1 {ß}{{\ss}}1
  {ű}{{\H{u}}}1 {Ű}{{\H{U}}}1 {ő}{{\H{o}}}1 {Ő}{{\H{O}}}1
  {ç}{{\c c}}1 {Ç}{{\c C}}1 {ø}{{\o}}1 {å}{{\r a}}1 {Å}{{\r A}}1
  {€}{{\euro}}1 {£}{{\pounds}}1 {«}{{\guillemotleft}}1
  {»}{{\guillemotright}}1 {ñ}{{\~n}}1 {Ñ}{{\~N}}1 {¿}{{?`}}1
}	

%Einleitung
\chapter{Einleitung}
\abschnittsautor{E. Schneidenbach, F. Schäfer, M. Trognitz}
Die Anwendung von Informationstechnologien und digitaler Methoden für die Durchführung archäologischer und altertumswissenschaftlicher Forschungsprojekte wird  immer geläufiger und selbstverständlicher. Die von IANUS zusammengestellten IT-Empfehlungen sollen einen ersten Überblick über den nachhaltigen Umgang mit digitalen Forschungsdaten bieten und als Leitfaden dienen, um fachspezifische Mindeststandards im Bereich des Forschungsdatenmanagements zu etablieren. Eine konsequente Berücksichtigung dieser Empfehlungen trägt maßgeblich dazu bei, dass Forschungsdaten von Dritten, die nicht bei deren Erstellung und Verarbeitung beteiligt waren, heute und in Zukunft verstanden und nachgenutzt werden können. 

Das vorliegende Dokument stellt die Kurzversion der online verfügbaren IT-Empfehlungen für den nachhaltigen Umgang mit digitalen Daten in den Altertumswissenschaften (doi:\href{http://dx.doi.org/10.13149/000.111000-a}{10.13149/000.111000-a}) dar und konzentriert sich auf folgende Themen:
\begin{itemize}
	\item Eine Übersicht der zentralen Aspekte im Datenmanagement
	\item Hinweise zur Dokumentation, Verwaltung und Sicherung von Daten
	\item Geeignete Langzeitformate und Metadaten für einzelne Dateitypen
\end{itemize}

Weitere Inhalte sind in der Online-Version zu finden:
\begin{itemize}
	\item Ergänzende Hintergrundinformationen und praktische Tipps
	\item Hinweise zum Umgang mit in den Altertumswissenschaften angewandten digitalen Dokumentations- und Forschungsmethoden
	\item Archivierung von Forschungsdaten bei IANUS
	\item Glossar
\end{itemize}

Als allgemeine Empfehlungen lassen sich folgende Punkte zusammenfassen:
\begin{itemize}
	\item Bei der Verarbeitung von Forschungsdaten ist deren Lebenszyklus zu beachten. Dieser umfasst die Erstellung, Verarbeitung, Analyse, Archivierung, den Zugang und die Nachnutzung. Diese Phasen sollten so früh wie möglich berücksichtigt werden und durch ein detailliertes, nachvollziehbares und gut strukturiertes Datenmanagement unterstützt werden. Hierfür ist eine solide Planung sowie eine konsequente Umsetzung erforderlich, die auch kontinuierliche Überprüfungen und eventuelle Anpassungen in der Praxis beinhaltet.
	\item Daten sollten während und nach der Erzeugung strukturiert und umfassend dokumentiert werden.
	\item Es sollten offen dokumentierte, nicht-proprietäre und weit verbreitete Dateiformate bevorzugt werden.
	\item Bei der Organisation von Dateien und Ordnern sollten möglichst einfache, klare und allen Akteuren bekannte Strategien vereinbart werden.
	\item Bei der Datensicherung sollten eindeutige Zuständigkeiten und Arbeitsabläufe definiert werden. 
\end{itemize}


%Projektphasen
\chapter{Projektphasen}
\abschnittsautor{S. Jahn, F. Schäfer, M. Trognitz}
	\label{projektphasen}

Im Umgang mit digitalen Daten gelten für jede Phase eines Forschungsprojektes unterschiedliche Anforderungen und Bedingungen. Da Forschungsdaten einem Lebenszyklus unterliegen, haben Entscheidungen und Arbeitsschritte, die in einer bestimmten Phase getroffen werden, auch Auswirkungen auf die anderen Phase des Datenkreislaufs. Bereits bei Entwurf und Planung eines neuen Forschungsprojektes muss überlegt werden, welche Informationen bereits digital existieren, welche Arten von Dateien neu erzeugt werden müssen, welche Informationstechnologien zum Einsatz kommen sollen und wie das Management der Forschungsdaten gestaltet sein wird. 

Insofern gilt es einige Aspekte so früh wie möglich, häufig noch vor der Erstellung der ersten Datei, zu adressieren:
\begin{itemize}
\item Informationen über existierende Standards und Richtlinien einholen
\item Entscheidungen über die zu verwendenden Dateiformate und Softwareprogramme fällen
\item Verantwortliche für das Datenmanagement bestimmen
\item Kriterien für die laufende Dokumentation entwickeln
\item Konventionen für die Ablage, Benennung und Versionierung von Dateien festlegen
\item Strategien zur Speicherung und Sicherung definieren
\item Geeignete Infrastrukturen für die Archivierung auswählen und kontaktieren
\end{itemize}

Diese und weitere Fragen zur Vorbereitung, Durchführung und Nachbereitung eines Projektes werden im Folgenden anhand eines sogenannten Datenmanagementplans thematisiert. Der Datenmanagementplan bildet einen wichtigen Baustein zur Festlegung der Arbeitsabläufe im Umgang mit Forschungsdaten und für eine Kostenkalkulation.

Je mehr und je früher in den Einzelentscheidungen die Nachnutzung der Daten durch Dritte auch über die Lebensdauer eines Projektes hinaus berücksichtigt wird und die Langzeitarchivierung als kontinuierliche Maßnahme und nicht als letzter Schritt eines bereits abgeschlossenen Projektes begriffen wird, umso leichter gelingt es, einmal erhobene Daten auch für die Zukunft zu erhalten. Die Archivierung der Daten nach Abschluss eines Projektes wird erleichtert, wenn schon während der Durchführung eines Projektes auf die konsequente Erfüllung der im Datenmanagementplan genannten Aufgaben geachtet wird.

Vor oder kurz nach Beginn des Projektes erstellte und kommunizierte Richtlinien, ermöglichen es allen Projektmitarbeitern diese auch einzuhalten. Auf diese Weise wird die Handhabung der Projektdaten erheblich erleichtert, da geregelte Ordnerstrukturen und vorgegebene Namenskonventionen das Auffinden der Daten vereinfachen, die parallele Dokumentation das Verstehen und die Nachnutzung ermöglicht und Sicherungskopien das Risiko eines Datenverlustes reduzieren.

\newpage
\section{Datenmanagement}\label{datenmanagement}
	\abschnittsautor{M. Trognitz, D. Hagmann, J. Räther, S. Jahn}
	Bereits vor Beginn eines Forschungsvorhabens steht die Konzeption und Planung des Projektes. Dazu gehört auch eine Beschreibung über den Umgang mit den resultierenden Forschungsdaten, der sogenannte Datenmanagementplan. Ein vollständiger Datenmanagementplan berücksichtigt alle Phasen des auf Seite \pageref{lebenszyklus} beschriebenen Lebenszyklus von Forschungsdaten. Er dient zunächst als Mittel zur strukturierten Reflexion über datenrelevante Aspekte eines Projektes und beantwortet grundsätzliche Fragen zu Verantwortlichkeiten, Maßnahmen zur Pflege und Verarbeitung der Daten, Standards und bereits vorhandenen Daten. Außerdem bietet er die Grundlage für Arbeitsabläufe (Workflows) im Umgang mit Forschungsdaten und für eine Kostenkalkulation.

Eine vollständige Dokumentation des Datenmanagements in einem Datenmanagementplan spart Zeit und Kosten, wenn beispielsweise Zusammenhänge beim Wechsel von Mitarbeitern hergestellt werden sollen, und beugt einem Datenverlust vor. Da die Nachnutzung von Forschungsdaten zunehmend an Bedeutung gewinnt, setzen Geldgeber oft einen Datenmanagementplan als Teil eines Förderantrags voraus.

Die konsequente Einhaltung aller im Datenmanagementplan gemachten Vorgaben während der gesamten Projektlaufzeit stellt sicher, dass die Daten auch von Dritten interpretiert und nachgenutzt werden können. Wird das Überführen der Daten in ein Archiv schon von Beginn des Projektes an vorbereitet und einkalkuliert ist zum Projektabschluss nur noch ein geringer Aufwand für die Übergabe in ein Archiv erforderlich, weil die Umformatierung, Neustrukturierung oder nachträgliche Dokumentation von Daten wegfällt.

Ein aktives Datenmanagement beugt insbesondere während der Planung und Durchführung eines Projektes späteren Zeit- und Budgetverlusten vor und stellt sicher, dass Daten am Ende in nachhaltigen Formaten, gut dokumentiert und gut strukturiert vorliegen.
% implementing data management measures during the planning and development stages of research will avoid later panic and frustration

\subsection{Übersicht der Aufgaben in den Projektphasen}
Mit Hilfe des zu Beginn eines Projektes erstellten Datenmanagementplans können sämtliche Prozesse während des Projekts strategisch umgesetzt werden, wobei die Umsetzung des Plans als laufender Vorgang zu verstehen ist. Wenn während eines Projektes Änderungen an dem Datenmanagementplan notwendig werden, sollten diese begründet, dokumentiert sowie Arbeitsprozesse und Ergebnisse angepasst werden. Außerdem muss dokumentiert werden wie die Änderungen sich auf bereits bestehende Daten auswirken.

In den unterschiedlichen Phasen des Datenlebenszyklus und des Projektes sind die Aspekte des Datenmanagementplans unterschiedlich stark zu berücksichtigen. Die folgende Tabelle veranschaulicht, wann im Lebenszyklus von Forschungsdaten welchen Aufgaben aus dem Datenmanagement besondere Aufmerksamkeit zuteil werden muss.

In Abhängigkeit der Projektgröße kann der Umfang der Aufgaben des Datenmanagementplanes und der Plan selbst skaliert werden, wobei jedoch Minimalanforderungen einzuhalten sind, um zuallererst die Verfügbarkeit der Daten im laufenden Projekt zu gewährleisten. Um die Planungen zu erleichtern und zu beschleunigen,  sollten bereits vorhandene institutionelle Vorgaben und Infrastrukturen genutzt werden.

\begin{table}[hbt]
\begin{tabular}{r!\tbg c!\tbg c!\tbg c!\tbg c!\tbg c!\tbg c!\tbg c!\tbg}
	\arrayrulecolor{ianusGrau}
 	\multicolumn{1}{r}{} & \multicolumn{1}{c}{\rot{Vorbereitung}} & \multicolumn{1}{c}{\rot{Erstellung}}& \multicolumn{1}{c}{\rot{Verarbeitung}} & \multicolumn{1}{c}{\rot{Analyse}} & \multicolumn{1}{c}{\rot{Archivierung}} & \multicolumn{1}{c}{\rot{Zugang}} & \multicolumn{1}{c}{\rot{Nachnutzung}}\\
	\cline{2-8}
	Rahmendaten\tib{*} & $\tib{\CIRCLE}$ & $\tib{\Circle}$ & $\tib{\Circle}$ & $\tib{\Circle}$ & $\tib{\RIGHTcircle}$ & $\tib{\Circle}$ & $\tib{\Circle}$\\
	\cline{2-8}
	Verantwortlichkeiten\tib{*} & $\tib{\CIRCLE}$ & $\tib{\Circle}$ & $\tib{\Circle}$ & $\tib{\Circle}$ & $\tib{\Circle}$ & $\tib{\Circle}$ & $\tib{\Circle}$\\
	\cline{2-8}
	Rechtliche Aspekte\tib{*} & $\tib{\CIRCLE}$ & $\tib{\Circle}$ & $\tib{\Circle}$ & $\tib{\Circle}$ & $\tib{\CIRCLE}$ & $\tib{\CIRCLE}$ & $\tib{\CIRCLE}$\\
	\cline{2-8}
	Methoden & $\tib{\CIRCLE}$ & $\tib{\CIRCLE}$ & $\tib{\RIGHTcircle}$ & $\tib{\RIGHTcircle}$ & $\tib{\Circle}$ & $\tib{\Circle}$ & $\tib{\Circle}$\\
	\cline{2-8}
	Vorgaben und Standards\tib{*} & $\tib{\CIRCLE}$ & $\tib{\CIRCLE}$ & $\tib{\CIRCLE}$ & $\tib{\Circle}$ & $\tib{\CIRCLE}$ & $\tib{\RIGHTcircle}$ & $\tib{\Circle}$\\
	\cline{2-8}	
	Kosten \& Ressourcen\tib{*} & $\tib{\CIRCLE}$ & $\tib{\CIRCLE}$ & $\tib{\CIRCLE}$ & $\tib{\Circle}$ & $\tib{\RIGHTcircle}$ & $\tib{\Circle}$ & $\tib{\Circle}$\\
	\cline{2-8}
	Externe Partner & $\tib{\CIRCLE}$ & $\tib{\CIRCLE}$ & $\tib{\RIGHTcircle}$ & $\tib{\Circle}$ & $\tib{\Circle}$ & $\tib{\RIGHTcircle}$ & $\tib{\CIRCLE}$\\
	\cline{2-8}
	Hard- und Software\tib{*} & $\tib{\CIRCLE}$ & $\tib{\CIRCLE}$ & $\tib{\RIGHTcircle}$ & $\tib{\RIGHTcircle}$ & $\tib{\Circle}$ & $\tib{\Circle}$ & $\tib{\Circle}$\\
	\cline{2-8}
	Datentypen und Datenformate & $\tib{\RIGHTcircle}$ & $\tib{\CIRCLE}$ & $\tib{\CIRCLE}$ & $\tib{\Circle}$ & $\tib{\CIRCLE}$ & $\tib{\Circle}$ & $\tib{\Circle}$\\
	\cline{2-8}
	Nachnutzung vorhandener Daten\tib{*} & $\tib{\RIGHTcircle}$ & $\tib{\RIGHTcircle}$ & $\tib{\CIRCLE}$ & $\tib{\CIRCLE}$ & $\tib{\Circle}$ & $\tib{\RIGHTcircle}$ & $\tib{\Circle}$\\
	\cline{2-8}
	Datenerzeugung \& -prozessierung\tib{*} & $\tib{\RIGHTcircle}$ & $\tib{\CIRCLE}$ & $\tib{\RIGHTcircle}$ & $\tib{\Circle}$ & $\tib{\RIGHTcircle}$ & $\tib{\RIGHTcircle}$ & $\tib{\Circle}$\\
	\cline{2-8}
	Datenmenge & $\tib{\CIRCLE}$ & $\tib{\CIRCLE}$ & $\tib{\CIRCLE}$ & $\tib{\CIRCLE}$ & $\tib{\RIGHTcircle}$ & $\tib{\Circle}$ & $\tib{\Circle}$\\
	\cline{2-8}
	Dateispeicherung und -sicherung\tib{*} & $\tib{\RIGHTcircle}$ & $\tib{\CIRCLE}$ & $\tib{\CIRCLE}$ & $\tib{\CIRCLE}$ & $\tib{\Circle}$ & $\tib{\Circle}$ & $\tib{\Circle}$\\
	\cline{2-8}
	Dateiverwaltung & $\tib{\RIGHTcircle}$ & $\tib{\CIRCLE}$ & $\tib{\CIRCLE}$ & $\tib{\CIRCLE}$ & $\tib{\RIGHTcircle}$ & $\tib{\Circle}$ & $\tib{\Circle}$\\
	\cline{2-8}
	Dokumentation\tib{*} & $\tib{\RIGHTcircle}$ & $\tib{\CIRCLE}$ & $\tib{\CIRCLE}$ & $\tib{\CIRCLE}$ & $\tib{\CIRCLE}$ & $\tib{\RIGHTcircle}$ & $\tib{\Circle}$\\
	\cline{2-8}
	Qualitätssicherung & $\tib{\RIGHTcircle}$ & $\tib{\CIRCLE}$ & $\tib{\CIRCLE}$ & $\tib{\CIRCLE}$ & $\tib{\CIRCLE}$ & $\tib{\Circle}$ & $\tib{\CIRCLE}$\\
	\cline{2-8}
	Datenaustausch & $\tib{\RIGHTcircle}$ & $\tib{\RIGHTcircle}$ & $\tib{\CIRCLE}$ & $\tib{\CIRCLE}$ & $\tib{\Circle}$ & $\tib{\CIRCLE}$ & $\tib{\Circle}$\\
	\cline{2-8}
	Mittelfristige Datenaufbewahrung & $\tib{\RIGHTcircle}$ & $\tib{\CIRCLE}$ & $\tib{\CIRCLE}$ & $\tib{\CIRCLE}$ & $\tib{\Circle}$ & $\tib{\Circle}$ & $\tib{\Circle}$\\
	\cline{2-8}
	Langfristige Archivierung\tib{*} & $\tib{\RIGHTcircle}$ & $\tib{\RIGHTcircle}$ & $\tib{\RIGHTcircle}$ & $\tib{\RIGHTcircle}$ & $\tib{\CIRCLE}$ & $\tib{\Circle}$ & $\tib{\CIRCLE}$ \\
	\cline{2-8}
	Zugänglichkeit und Nachnutzung & $\tib{\CIRCLE}$ & $\tib{\RIGHTcircle}$ & $\tib{\RIGHTcircle}$ & $\tib{\RIGHTcircle}$ & $\tib{\CIRCLE}$ & $\tib{\CIRCLE}$ & $\tib{\CIRCLE}$\\
	\cline{2-8}
	Projektabschluss & $\tib{\RIGHTcircle}$ & $\tib{\Circle}$ & $\tib{\Circle}$ & $\tib{\Circle}$ & $\tib{\RIGHTcircle}$ & $\tib{\RIGHTcircle}$ & $\tib{\Circle}$\\
	\cline{2-8}
\end{tabular}
\caption{Tabellarische Übersicht über die verschiedenen zu berücksichtigenden Aspekte eines Datenmanagementplans während unterschiedlicher Projektphasen und unter Beachtung des Lebenszyklus von Forschungsdaten. Besonders wichtige, als Minimalanforderung zu betrachtende Aspekte des Datenmanagementplans sind mit einem Stern gekennzeichnet.}
\label{tab:dmp-lebenszyklus}
\end{table}
\begin{table}[h!bt]
\begin{tabular}{cr}
	$\tib{\CIRCLE}$ & Aufgabe ist während der Phase relevant.\\
	$\tib{\RIGHTcircle}$ & Aufgabe ist während der Phase teilweise relevant.\\
	$\tib{\Circle}$ & Aufgabe ist während der Phase nicht relevant.\\
\end{tabular}
\end{table}
	
	\begin{flushleft}
		Das vollständige Kapitel finden Sie auf: \urllist{https://ianus-fdz.de/it-empfehlungen/datenmanagement}
		\label{lebenszyklus}
		Den Lebensdatenzyklus von Forschungsdaten finden Sie auf: \urllist{https://ianus-fdz.de/it-empfehlungen/lebenszyklus}
	\end{flushleft}

\newpage
\section{Dokumentation}\label{Metadaten-allgemein}
	\abschnittsautor{S. Jahn}
	Neben technischen Aspekten ist eine vollständige Dokumentation das wichtigste Kriterium für die Archivierung und Nachnutzbarkeit von Daten. Damit Forschungsdaten von Dritten gefunden und sinnvoll verwertet werden können, müssen sie verständlich und strukturiert beschrieben werden. Ohne zusätzliche Informationen sind die meisten digitalen Dateien, die in einem Forschungsprojekt entstehen, für die Wissenschaft verloren und der Aufwand einer langfristigen Datenkuratierung ist vergeblich. Die Dokumentation und Beschreibung von Daten ist als ein kontinuierlicher Prozess zu begreifen, der in allen Phasen eines Datenlebenszyklus adressiert werden sollte.
	
	Metadaten können sich auf folgende Aspekte beziehen:
	\begin{itemize}
		\item Deskriptive Angaben
		\item Strukturelle Angaben
		\item Administrativ-rechtliche Angaben
		\item Administrativ-erhaltungsbezogene Angaben
		\item Technische Angaben
	\end{itemize}
	
	Um Metadaten eindeutig zu vergeben und eine maschinelle Verarbeitung zu ermöglichen, sollten kontrollierte Vokabulare in Form von einfachen Wertelisten oder strukturierten Thesauri verwendet werden. Auch sollten standardisierte und etablierte Metadatenschemata, wie beispielsweise Dublin Core benutzt werden.
	
	Die Speicherung der Metadaten kann teilweise innerhalb der Datei selbst erfolgen, wobei die Speicherung der Informationen in einer eigenen Datei, wie etwa einer Fotoliste, empfohlen wird.
	
	\subsection{Metadaten in der Anwendung}
\label{Metadaten-anwendung}
\paragraph{Projektbezogene Metadaten}
Die wichtigsten Metadaten, die für die Beschreibung eines Projektes oder einer Dokumentensammlung erforderlich sind, werden in der folgenden Tabelle abgebildet und knapp definiert. Sie geben einen Überblick über einen größeren, zusammenhängenden Datenbestand und beschreiben den fachlichen Kontext in dem dieser entstanden ist. Vergleichbar einem Bibliothekskatalog liegt die Hauptfunktion dieser Metadaten darin, dass externe Personen ein Projekt oder eine dessen digitale Daten über Web-Portale und Suchmaschinen finden und einordnen können. Darüber hinaus enthalten sie rechtliche Informationen, die für den weiteren Umgang mit den Daten wichtig sind. 

Die hier vorgestellten Eigenschaften basieren auf dem Dublin Core Metadata Schema und den Angaben, wie sie vom ADS in den UK und tDAR in den USA erhoben werden, um einen zukünftigen Austausch zu vereinfachen. Ein darauf aufbauendes ausführliches Metadatenschema, das auch die Grundlage für die Archivierung von Projektdaten bei IANUS bildet, sowie ausgefüllte Beispielformulare sind in dem Kapitel Archivierung von Forschungsdaten in IANUS ab Seite \pageref{archivierungIANUS} zu finden.

\begin{center}
	\begin{longtable}{L{0.25\textwidth} p{0.68\textwidth}}
		\toprule
		Bezeichnung & Kurzdefinition\\ \midrule \endfirsthead
		\multicolumn{2}{l}{\footnotesize Fortsetzung der vorhergehenden Seite}\\
		\toprule
		Bezeichnung & Kurzdefinition\\ \midrule \endhead
		\bottomrule \multicolumn{2}{r}{{\footnotesize Fortsetzung auf der nächsten Seite}} \\
		\endfoot
		\bottomrule 
		\endlastfoot

		Identifizierung -- Projekttitel & Verbindliche Kurzbezeichnung des Projektes.\\
		Identifizierung -- Alternativtitel & Ggf. alternative Titel für ein Projekt.\\
		Identifizierung -- Projektnummer(n) & Nummern oder Kennungen, die z.B. innerhalb der durchführenden Organisation oder von Mittelgebern verwendet werden, um das Projekt eindeutig identifizieren zu können.\\
		Kurzbeschreibung & Knappe Angaben zur Fragestellung, zum Verlauf und Ergebnis des Projektes sowie Skizzierung der Datensammlung (insgesamt ca. 100-300 Worte)\\
		Schlagworte -- Fachdisziplinen & Stichworte, die die beteiligten Disziplinen und Fächer benennen. Sofern die Stichworte auf publizierten Standards oder internen Thesauri beruhen, müssen diese mitangegeben werden.\\
		Schlagworte -- Inhalt & Stichworte, die den Inhalt der Datensammlung benennen., z. B. zu Materialgruppen, Fundstellen-Klassifizierung, Quellenarten,  Kulturgruppen etc. Sofern die Stichworte auf publizierten Standards oder internen Thesauri beruhen, müssen diese mitangegeben werden.\\
		Schlagworte -- Methoden & Stichworte, die die eingesetzten Forschungsmethoden beschreiben. Sofern die Stichworte auf publizierten Standards oder internen Thesauri beruhen, müssen diese mitangegeben werden.\\
		Ausdehnung -- Geografisch-1 & Detaillierte Angaben zur räumlichen Ausdehnung oder zum Fundort des untersuchten Gegenstandes mittels geografischer Koordinaten. Die maximale Ausdehnung kann als Bounding Box angegeben werden.\\
		Ausdehnung -- Geografisch-2 & Sprachliche Beschreibung des untersuchten Gegenstandes mittels Ortsangaben mit Land, Stadt, Kreis, Straße, Gemarkung etc. Sofern Namen sich im Lauf der Zeit geändert haben, dies gesondert vermerken. Sofern eine Referenz zu einer Geo-Ressource oder einem Gazetteer existiert, sollte diese ebenfalls angegeben werden.\\
		Ausdehnung -- zeitlich & Chronologische Angaben zum untersuchten Gegenstand, entweder als Periodenbezeichnung und/oder mit groben/genauen Jahresangaben. Sofern die Stichworte auf publizierten Standards oder internen Thesauri beruhen, müssen diese mitangegeben werden.\\
		Primärforscher -- Person & Personen, die entweder für das Projekt als Ganzes, für das Datenmanagement oder für die Erzeugung bestimmter Datenarten zentral bzw. verantwortlich sind. Hier ist eine Kontaktadresse erforderlich und die aktuelle/letzte institutionelle Zugehörigkeit, damit die Personen bei Rückfragen erreicht werden kann.\\
		Eigentümer -- Organisation & Organisation, der die unter "`Primärforscher"' genannten Personen angehören, oder die nach Ausscheiden derselben für die Daten verantwortlich ist, im weitesten Sinne also Eigentümer der Daten ist. Hier ist eine Kontaktadresse erforderlich, damit die Organisation bei Rückfragen erreicht werden kann.\\
		Finanzierung & Nennung der Organisation(en) / (Dritt-)Mittelgeber, durch die das Projekt finanziert wurde. Es sollte jeweils der Zeitraum der Finanzierung angegeben werden.\\
		Veröffentlichung -- Projektdaten & Wenn die hier beschriebene Datensammlung des Projektes bereits an anderer Stelle veröffentlicht / online gestellt wurde, bitte entsprechende Angaben machen, z. B. durch Nennung der Organisationen, Datenarchive, Online-Ressourcen etc.\\
		Veröffentlichung -- Ergebnisse & Analoge oder digitale Publikationen zu Ergebnissen des Projektes oder zur Datensammlung des Projektes, ausführliche bibliographische Angaben (ohne fachspezifische Abkürzungen) unter Nennung des Verlages erforderlich.\\
		Dauer -- Projekt & Anfangs- und Enddatum des Projektes.\\
		Dauer -- Datenbestand & Anfangs- und Enddatum der Erzeugung oder Verarbeitung digitaler Daten im Rahmen des Projektes.\\
		Rechtliches -- Urheberrechte & Name des Inhabers der Urheber-, Nutzungs- und Verwertungsrechte; i. d. R. die Organisation, an der der Primärforscher beschäftigt war.\\
		Rechtliches -- Lizenzgeber & Angabe der Person, die i. d. R. als Vertretung für eine Organisation für die Lizenzierung von Daten zur Nachnutzung verantwortlich und berechtigt ist, einen Datenübergabevertrag abzuschließen.\\
		Rechtliches -- Datenschutz & Angaben, ob in der Datensammlung datenschutzrelevante Informationen enthalten sind. Wenn ja, in welchem Umfang.\\
		Quellen -- Ältere & Ältere Quellen oder existierende Ressourcen, auf denen die Daten aufbauen.\\
		Quellen -- Zugehörige & Sofern während des Projektes Informationen, Datensammlungen, (un-)publizierte Dokumente, Online-Ressourcen etc. verwendet oder erzeugt wurden, die nicht Teil der hier beschriebenen Datensammlung sind, aber für deren Verständnis wichtig sind, bitte entsprechende Angaben zu Art und Umfang dieser Quellen machen.\\
		Sprache & Die in den Dokumenten und Dateien verwendete(n) Sprache(n). Sprachkennungen nach ISO 639 angeben.\\
		Art der Daten & Kurzcharakterisierung der Daten, z. B. ob es sich um Rohdaten, verarbeitete Daten, Interpretationen, Ergebnisse, Abschlussberichte etc. handelt.\\
		Vollständigkeit & Aussagen zur Vollständigkeit der Projektdaten, z. B. ob bestimmte Datenarten noch fehlen und warum.\\
		Dateiformate & Auflistung der Dateiformate, die in der Datensammlung vorkommen, ggf. unter Nennung der verwendeten Programme und Zeichenkodierungen.\\
		Zugriffsrechte & Festlegung der gewünschten Zugriffsrechte für die Daten, sofern diese für den gesamten Projekt-Datenbestand gelten sollen; differenzierte Regelungen müssen auf Dateiebene vorgenommen werden.\\
		Signatur Metadaten & Angabe darüber, wer die o. g. Metadaten wann ausgefüllt hat.\\
		\bottomrule
	\end{longtable}
\end{center}

\paragraph{Dateibezogene Metadaten}
Bei dieser Art von Metadaten handelt es sich um technische und inhaltliche Informationen, die Nutzern verständlich machen, wie einzelne Dateien innerhalb eines Projektes oder einer Datensammlung beschaffen sind und welche Möglichkeiten der Nachnutzbarkeit sie beinhalten. 

Dateibezogene Metadaten sind abhängig von dem Format, dem Inhalt und der Methode, mit denen die Dateien erzeugt wurden. Beispielsweise sind für Rastergrafiken, die durch digitale Fotografie entstanden sind, andere Angaben erforderlich (Fotograf, Aufnahmedatum, Aufnahmeort, abgelichtetes Objekt etc.) als für Rasterdateien, die durch geophysikalische Messungen erzeugt wurden (Koordinaten, Messgerät, Genauigkeit, Datum etc.). Zusätzliche spezifische Angaben, die für bestimmte Dateiformate empfohlen werden, sind in den verschiedenen Kapiteln zu den jeweiligen Formaten ab Seite \pageref{dateiformate} beschrieben.

Es gibt jedoch dateibezogene Metadaten, die unabhängig von Format, Inhalt und Methode für alle Einzeldateien gleichermaßen relevant und notwendig sind. Zu diesen Metadaten gehören neben den Angaben zu Dateiname, Dateiformat, Dateiversion, Titel, Beschreibung und Ersteller auch Informationen zur verwendeten Soft- und Hardware, Versionierung, zu rechtlichen Aspekten und Verweise auf weitere relevante Dateien.

Auch wenn es in der Theorie wünschenswert ist, diese Angaben sowie die zugehörigen spezifischen Angaben zu Methoden und Dateiformaten für jede Datei einzeln zu erfassen, so zeigt die Praxis, dass es häufig ausreichend ist, einen Metadatensatz für Gruppen von Dateien anzulegen, wenn diese das gleiche Format oder die gleichen inhaltlichen Eigenschaften aufweisen. 

\begin{center}
	\begin{longtable}{L{0.25\textwidth} p{0.68\textwidth}}
		\toprule
		Bezeichnung & Kurzdefinition\\ \midrule \endfirsthead
		\multicolumn{2}{l}{\footnotesize Fortsetzung der vorhergehenden Seite}\\
		\toprule
		Bezeichnung & Kurzdefinition\\ \midrule \endhead
		\bottomrule \multicolumn{2}{r}{{\footnotesize Fortsetzung auf der nächsten Seite}} \\
		\endfoot
		\bottomrule 
		\endlastfoot
		
		Identifikator, Dateiname & Eindeutiger Name der Datei.\\
		Dateiformat & Format, in dem die Datei abgespeichert ist.\\
		Urheber & Name des Verfassers oder Erstellers der Datei.\\
		Titel & Titel der Datei, nicht der Dateiname.\\
		Beschreibung & Beschreibung des Inhalts der Datei. \\
		Schlagworte & Schlagworte, wie etwa Periode, Fundstelle oder charakteristische Merkmale. Wenn vorhanden, angemessene Thesauri verwenden.\\
		Software -- Dateierstellung & Software, mit der die Datei erstellt wurde.\\
		Hardware -- Dateierstellung & Hardware, mit der die Datei erstellt wurde, v. a. bei technischen Geräten wie Kameras, GPS-Geräten, Vermessungsinstrumenten, Laserscanner etc.\\
		Betriebssystem -- Dateierstellung & Betriebssystem, das verwendet wurde als die Datei erstellt wurde.\\
		Erstellungsdatum & Datum, an dem die Datei erstellt wurde. Datum und Zeit in UTC nach ISO 8601.\\
		Letzte Aktualisierung & Datum, an dem die Datei zuletzt bearbeitet wurde. Datum und Zeit in UTC nach ISO 8601.\\
		Dateiversion & Angabe der Versionsnummer der Datei.\\
		Weitere Dateien & Referenzen auf Dateien, die für das Verständnis einer anderen Datei zentral sind, insbesondere für zusammenhängende, komplexe Dateien oder wenn auf eine Ursprungsdatei verwiesen werden soll.\\
		Sprache & Sofern schriftliche Inhalte vorhanden sind, die Sprache angeben. Sprachkennungen nach ISO 639 angeben.\\
		Copyright-Angaben & Angaben zur Person oder Einrichtung, die das Copyright oder die Lizenzrechte an der Datei oder deren Inhalt besitzt.\\
		\bottomrule
	\end{longtable}
\end{center}

\paragraph{Methodenbezogene Metadaten}
Jede Fachdisziplin innerhalb der Altertumswissenschaften verfügt über spezifische Forschungsmethoden. Auch diese haben Einfluss auf Umfang, Art, Struktur und Inhalt digitaler Objekte, da sie bei verschiedenen Arbeitsweisen und technischen Geräten unterschiedlich ausfallen. Daher sollten methodenbezogene Metadaten ebenfalls dokumentiert werden, insbesondere wenn mehrere Zwischenstände einer Prozesskette archiviert und Dritten zur Verfügung gestellt werden sollen. Je nach Genauigkeit der Methodenbeschreibung kann sie sich sowohl auf eine Einzeldatei als auch auf mehrere Dateien gleichen Typs beziehen.

Die Angabe der methodenbezogenen Metadaten ist wichtig, um zu beschreiben, wie die Rohdaten in prozessierte Daten überführt wurden. Außerdem kann so verstanden werden, welchen Einfluss die angewandte Methode auf das Ergebnis hat, das Ergebnis folglich zu interpretieren ist und wo Fehler zu erwarten sind.

Zusätzliche spezifische Angaben, die für bestimmte Forschungsmethoden oder Prozesse empfohlen werden, sind in den einzelnen Kapiteln des Abschnittes Forschungsmethoden ab Seite \pageref{methoden} beschrieben.

\begin{center}
	\begin{tabular}{L{0.25\textwidth} p{0.68\textwidth}}
		\toprule
		Bezeichnung & Kurzdefinition\\ \midrule
		Prozessnummer & Eindeutige Nummer eines Prozesses oder einer Methode.\\
		Prozessbeschreibung & Beschreibung des Prozesses oder der Methode. Insbesondere Beschreibung der Ausgangssituation und der Zielvorstellung.\\
		Ausgangsformat(e) & Format der Dateien, die am Anfang eines gesamten Prozesses stehen und den Ausgangspunkt bilden.\\
		Zwischenformat(e) & Format der Dateien, die im Verlauf eines Prozesses erzeugt werden und den Ausgangspunkt für weitere Prozesse bilden.\\
		Zielformat(e) & Format der Dateien, die am Ende eines Prozesses erzeugt werden.\\
		Durchführender & Person(en), die den Prozess durchgeführt hat (haben).\\
		Prozessbeginn & Datum, an dem der Prozess begonnen wurde. Datum und Zeit in UTC nach ISO 8601.\\
		Prozessende & Datum, an dem der Prozess beendet wurde. Datum und Zeit in UTC nach ISO 8601.\\
		Software & Software, mit der der Prozess durchgeführt wurde.\\
		Hardware & Hardware, auf der der Prozess durchgeführt wurde.\\
 		\bottomrule
		\bottomrule
	\end{tabular}
\end{center}


%##################################################################################
\label{grabungsdokumentation}
\subsection{Grabungsdokumentation}
Speziell für die Dokumentation von Grabungen und anderen archäologischen Maßnahmen sind weitere Metadaten erforderlich, die in der folgenden Tabelle aufgelistet sind. Dabei sollten auch die verschiedenen angewandten Methoden berücksichtigt werden, da diese ebenfalls den Umfang und die Art der Dokumentation beeinflussen. Wichtig bei einer Grabungsdokumentation ist, dass nicht nur die digitalen Daten, sondern auch die analogen Daten mit einer Dokumentation versehen sind, um beispielsweise auch Abhängigkeiten zwischen den verschiedenen Daten festzuhalten. Zum Beispiel sollten Zeichnungen mit den Befundbeschreibungen, Fotos mit den Fundstellenplänen oder Objektbeschreibungen mit den jeweiligen Objekten verknüpft sein.

\begin{center}
	\begin{longtable}{L{0.25\textwidth} p{0.68\textwidth}}
		\toprule
		Bezeichnung & Kurzdefinition\\ \midrule \endfirsthead
		\multicolumn{2}{l}{\footnotesize Fortsetzung der vorhergehenden Seite}\\
		\toprule
		Bezeichnung & Kurzdefinition\\ \midrule \endhead
		\bottomrule \multicolumn{2}{r}{{\footnotesize Fortsetzung auf der nächsten Seite}} \\
		\endfoot
		\bottomrule 
		\endlastfoot

		Fundstellenart & Angabe über die Art der Fundstelle. Mehrfachangaben sind möglich, wenn beispielsweise ein sich mit einer Siedlung überlagerndes Gräberfeld beschrieben werden soll.\\
		Maßnahmenart & Angabe über die Art der Untersuchung. Werte können beispielsweise sein: Ausgrabung, Survey, Baustellenbegleitung etc.\\
		Anlass & Angabe des Anlasses, der zur Durchführung der Maßnahme führt. Werte können beispielsweise sein: Rettungsgrabung, Notgrabung, Forschungsgrabung, Lehrgrabung etc.\\
		Grabungsmethodik & Angabe der angewandten Grabungsmethodik. Werte können beispielsweise sein: Flächengrabung, Schichtengrabung, Wheeler-Kenyon-Methode etc.\\
		Verfahren & Angabe über die angewandten Arbeitsverfahren, die nicht zu Dokumentationsverfahren gehören. Werte können beispielsweise sein: Schlämmen, Bohrung, Sieben etc.\\
		Datierung & Datierung der Fundstelle, beispielsweise anhand des archäologischen Fundmaterials. Eingruppierung in eine Epoche als relative Zeitstellung und optional als absolute Zeitangabe. Bei mehrphasigen Fundstellen werden mehrere Zeiträume angegeben.\\
		Fundstellenstatus & Angaben zum Schutzstatus einer Fundstelle.\\
		Bodenbeschaffenheit & Angabe über die natürlichen Gegebenheiten und die Bodenbeschaffenheit, die Einfluss auf die Erhaltungsbedingungen von Objekten und Strukturen haben. Werte können beispielsweise sein: Feuchtboden, Mineralboden, Unter Wasser etc.\\
		Strukturen & Verschlagwortung von aufgefundenen archäologischen Strukturen, die sich aus den Befunden ergeben.\\
		Funde & Verschlagwortung von signifikanten Fundgattungen.\\
		Richtlinie & Angaben zur Grabungs- oder Dokumentationsrichtlinie, die für die Maßnahme vorgegeben war oder gewählt wurde.\\
		Dokumentationssystem & Angabe des verwendeten Dokumentationssystems. Werte können beispielsweise sein: Stellenkartensystem, Single Context Recording, spezielle institutionelle Systeme etc.\\
		Dokumentationsmethoden & Angabe der verwendeten Dokumentationsmethoden. Werte können beispielsweise sein: Text, Foto, Zeichnung, 3D-Scan, Vermessung etc. \\
		\bottomrule
	\end{longtable}
\end{center}

Die große Zahl an Dokumentationsverfahren und Forschungsmethoden führt dazu, dass eine umfassende Grabungsdokumentation aus einer großen Menge unterschiedlicher Dokumente besteht, die folgendes enthalten sollte:
\begin{itemize}
	\item Allgemeine Angaben
	\item Grabungsplan
	\item Grabungstagebuch und Grabungsprotokoll
	\item Befundblätter und Befundliste
	\item Fundzettel und/oder Fundmeldung
	\item Fund- und Probenliste
	\item Datenbanken
	\item Dokumentation der angewandten Methoden, wie:
	\begin{itemize}
		\item Vermessungsmethoden
		\item Fotografie
		\item Photogrammetrie
		\item 3D-Scans
		\item Luftbildaufnahmen
		\item Naturwissenschaftliche Beprobung
		\item Zeichnung
		\item Allgemeine Beschreibungen
	\end{itemize}
	\item Abschlussbericht
\end{itemize}

Wie eine Grabungsdokumentation aussehen und welchen Umfang sie haben soll, wird in zahlreichen Richtlinien und Vorgaben spezifiziert, die bei der Planung des Vorhabens bereits berücksichtigt werden sollten. Dabei handelt es sich meist um Vorgaben von Landesdenkmalämtern. Online verfügbare Vorgaben sind bei den weiterführenden Informationen unter Vorgaben zur Grabungsdokumentation ab Seite \pageref{Metadaten-ListeLDA} aufgelistet.

Um die Grabungsdokumentation zu erleichtern und auch sicher zu stellen, dass alle erforderlichen Metadaten angegeben werden, sollten Vorlagen für Formulare und Checklisten bereits vor der Durchführung der Maßnahme erstellt werden. In der online verfügbaren Fassung der IT-Empfehlungen sind zu diesem Kapitel passende Vorlagen zur freien Verwendung zu finden. Darunter befinden sich eine Befundliste, eine Fotoliste, eine Fundliste, eine Geräteliste, ein Probenverzeichnis, ein Restaurierungsverzeichnis, ein Zeichnungsverzeichnis und ein Dokument zur Urheberrechtsverwaltung. Ein weiterer Anhaltspunkt sind die bereits genannten Richtlinien, sowie das Werk "`Tabellen und Tafeln zur Grabungstechnik"' von Andreas Kinne und das Grabungstechnikerhandbuch des Verbandes der Landesarchäologen. 
	\begin{flushleft}
		Das vollständige Kapitel finden Sie auf: \urllist{https://ianus-fdz.de/it-empfehlungen/dokumentation}
		\label{archivierungIANUS}
		Hinweise zur Archivierung bei IANUS finden Sie auf: \urllist{https://ianus-fdz.de/it-empfehlungen/archivierung}
		\label{Metadaten-ListeLDA}
		Die Liste der Vorgaben der Landesdenkmalämter finden Sie auf: \urllist{https://www.ianus-fdz.de/it-empfehlungen/metadaten-quellen\#grabungsdokumentation_richtlinien}
		\label{dateibenennung}
		Empfehlungen zur Dateibenennung finden Sie auf: \urllist{https://www.ianus-fdz.de/it-empfehlungen/dateibenennung}
	\end{flushleft}


\newpage
\section{Dateiverwaltung}\label{dateiverwaltung}
	\abschnittsautor{M. Trognitz}
	\hyphenation{
Schnitt-stel-le
}
Der tägliche Umgang mit digitalen Daten wird durch eine effiziente Dateiverwaltung erheblich erleichtert. Aussagekräftige Dateinamen, die auch für Dritte verständlich sind, sorgen dafür, dass die Dateien gefunden und deren Inhalt verstanden wird. Einheitliche Dateinamensstrukturen erhöhen die Lesbarkeit. Die konsequente Einhaltung von Versionierungsangaben sorgt dafür, dass immer mit der richtigen Dateiversion gearbeitet wird und eine selbsterklärende Ordnerstruktur hilft dabei auch in großen Projekten bestimmte Dateien wiederzufinden.

Die Konzipierung der Dateiverwaltung muss schon zu Beginn des Projektes erfolgen, damit die Regeln von Anfang an angewendet werden können. Die Niederschrift der durch Beispiele ergänzten Benennungsregeln und deren Weitergabe dient im laufenden Betrieb als wichtiges Nachschlagewerk für die konsistente und konsequente Einhaltung derselben. Dies ermöglicht eine effizientere Arbeitsweise.

Im laufenden Projektbetrieb sollte die Einhaltung der Benennungsregeln kontrolliert und gegebenfalls angepasst werden.


\subsection{Dateiablage}
Mit Dateiablage ist hier vor allem die Ordnerstruktur gemeint. Für die Benennung der Ordner gelten die gleichen Regeln, wie für die Dateibenennung, die im Abschnitt Dateibenennung ab Seite \pageref{dateibenennung} thematisiert werden. Lediglich die Dateinamenserweiterung wird bei Ordnernamen nicht verwendet. Die Dateiablage sollte selbsterklärend sein und unpräzise Namen wie etwa \emph{"`In Arbeit"'} vermieden werden.

Wichtig ist, dass die Dateiablage logisch und hierarchisch aufgebaut ist, damit andere Nutzer die gewünschten Informationen finden und einordnen können. Dies bedeutet unter anderem, dass die Inhalte und Unterordner auch tatsächlich thematisch in den übergeordneten Ordner hineinpassen. Beispielsweise erwartet man in einem Ordner mit dem Namen \emph{"`Fotos"'} diverse Fotos, die bei einer entsprechenden Menge vielleicht noch auf verschiedene Unterordner, wie zum Beispiel \emph{"`Plana"'} und \emph{"`Profile"'}, verteilt sind.   

Ist aber in dem Ordner \emph{"`Fotos"'} ein Unterordner mit dem Namen \emph{"`Zeichnungen"'} enthalten, in dem verschiedene digitalisierte Zeichnungen abgelegt sind, führt das zu Schwierigkeiten. Der Unterordner wird wahrscheinlich nur schwer wiedergefunden werden, da der Name des übergeordneten Ordners einen anderen Inhalt verspricht.

Tritt der Fall ein, dass eine Datei oder ein Ordner thematisch in mehrere verschiedene übergeordnete Ordner passen würde, können verschiedene Lösungswege zur Anwendung kommen. Beispielsweise könnte eine Kopie abgelegt werden, was allerdings problematisch ist, wenn eine der Kopien verändert wird und diese nicht mit der anderen abgeglichen wird.

Auch von Dateiverknüpfungen ist abzuraten, da sie meist nicht betriebssystemübergreifend funktionieren und im schlechtesten Fall nur auf dem Rechner, mit dem sie erstellt wurden, funktionieren. Wird die Datei verschoben, umbenannt oder gelöscht, so wird die Dateiverknüpfung ins Leere führen.

Eine bessere Methode ist das Anlegen einer Textdatei mit einem Hinweis auf den Ablageort der Datei oder des Ordners. In diesem Fall muss aber darauf geachtet werden, dass die gesuchte Datei sich auch tatsächlich an dem angegebenen Ort befindet. Bei Verschieben oder Löschen müsste die Textdatei also auch immer berücksichtigt und angepasst werden. 

Am besten ist es, nur eine Instanz einer Datei oder eines Ordners zu haben. Bei Zuordnungsschwierigkeiten kann es helfen, die Datei oder den Ordner in der hierarchischen Struktur höher anzusiedeln.

Eine Textdatei kann auch verwendet werden, um die vorhandene Ordnerstruktur zu erklären oder auf Besonderheiten hinzuweisen. Beispielsweise kann in einem Ordner mit Fotos, deren Metadaten in einer Datenbank abgespeichert sind, mit Hilfe einer Textdatei verdeutlicht werden, wo die Metadaten zu finden sind. Damit die Textdatei auch als eine Hilfedatei erkannt wird, kann man ihr den Namen \emph{"`README"'} oder \emph{"`LIESMICH"'} geben. Die Schreibung mit Großbuchstaben erhöht die Auffälligkeit der Datei. Durch eine vorangestellte Null oder einen vorangestellten Unterstrich kann sie auch in der Sortierung an den Anfang gestellt werden.

Mittels eines Dokumentenmangaementsystems, kann eine Dateiablage auch datenbankgestützt Verwaltet werden, was jedoch einen höheren technischen Aufwand erfordert. Der Vorteil ist, dass Dateien nach beliebigen Kriterien sortiert und gesucht werden können. Jedoch kann eine mit einem Dokumentenmanagementsystem erstellte Dateiablage ohne dieses System unter Umständen nicht mehr verständlich sein, da für die Dateien abstrakte Bezeichnungen vom System vergeben werden.

\subsection{Empfehlungen für eine Ordnerstruktur}
Die Organisation der Ordnerstruktur kann sich anhand der angewandten Prozesse oder an den Ergebnissen orientieren. Im ersten Fall bildet die Struktur zeitliche und methodische Prozesse der Daten ab, was zu einer besseren Nachvollziehbarkeit der Arbeitsschritte führt. Im zweiten Fall orientiert sich die Ordnerstruktur an den fachlichen Ergebnissen, was die inhaltliche Nachvollziehbarkeit verbessert.

Bei der Planung einer Ordnerstruktur spielen weitere Kriterien eine Rolle. Abhängig von dem Umfang und der Dauer des Projektes, kann sich die Ordnerstruktur beispielsweise am Thema, dem Material, dem Jahr, dem Bearbeiter oder den einzelnen Arbeitsschritten orientieren. Dabei sollte beachtet werden, dass die Dateiablage nicht zu verzweigt ist, da die maximale Pfadlänge, die sich aus allen enthaltenden Ordnernamen und dem Dateinamen zusammensetzt, in Windows auf 260 Zeichen begrenzt ist.

Ganz allgemein kann die oberste Hierarchie der Dateiablage nach folgenden Kriterien unterteilt werden:
\begin{itemize}
	\item Ort
	\item Fundplatz, Monumente oder Denkmäler
	\item Aktivität
	\item Projekt
\end{itemize}

Für weitere Hierarchieebenen können folgende Kriterien berücksichtigt werden:
\begin{itemize}
	\item Verfahren, wie Prospektion, Voruntersuchung oder Hauptuntersuchung
	\item Arbeitsschritte mit Bezug auf Zeit, Typ und Kennung
	\item Fachliche Inahlte, wie Befunde, Funde, Proben, Bauwerke, Berichte oder Tagebuch
	\item Methodik, wie Vermessung, Foto, Zeichnung oder Photogrammetrie
	\item Räumliche Spezifizierung, wie Planum, Schnitt oder Surveyfläche
	\item Administration
\end{itemize}

Zusätzlich sollte eine Unterscheidung zwischen originalen Daten (Rohdaten), sekundären und finalen Daten erfolgen, um Prozesse transparent abzubilden. Dies ist auch für die zu archivierenden Daten relevant, da bei der Auswahl der Daten überholte oder temporäre Dateien und Dubletten in der Regel nicht berücksichtigt werden.

Anwendungsgebiete mit mehreren voneinander abhängigen Dateikomplexen, wie zum Beispiel 3D-Scanning oder RTI-Fotografie, erfordern eigene Dateiablagen. Diese werden in den entsprechenden Abschnitten im Kapitel "`Forschungsmethoden"' ab Seite \pageref{methoden} beschrieben.
	\begin{flushleft}
		Das vollständige Kapitel finden Sie auf: \urllist{https://ianus-fdz.de/it-empfehlungen/dateiverwaltung}
	\end{flushleft}

\newpage
\section{Dateispeicherung und -sicherung}\label{dateispeicherung}
	\abschnittsautor{M. Trognitz, R. Komp, R. Förtsch}
	Um Datenverlust vorzubeugen, ist es unerlässlich eine geeignete Sicherungsstrategie zu verwenden. Dabei muss zwischen einer kurzfristigen Speicherung und einer mittelfristigen Sicherung unterschieden werden. Ersteres meint, wie Daten während der Erstellung und Bearbeitung gespeichert werden. Letzteres bezieht sich auf einen längeren Speicherzeitraum, der durchaus auch mehrere Monate betragen kann, stellt also ein klassisches Backup der Daten dar.

Die Sicherungsstrategie legt fest, wie die Datensicherung erfolgen soll und berücksichtigt folgende Fragen:
\begin{itemize}
	\item Wer ist für die Datensicherung verantwortlich?
	\item Wer hat Zugriff auf die gesicherten Daten?
	\item Wann und wie oft soll die Datensicherung durchgeführt werden?
	\item Auf welche Weise soll gesichert werden?
	\item Welche Daten sollen gesichert werden?
	\item Welche Speichermedien sollen verwendet werden?
	\item Wie viele Sicherungskopien sollen angelegt werden?
	\item Wo sollen die Sicherungen aufbewahrt und wie sollen sie geschützt werden?
	\item Wie soll der Transport der Sicherungskopien erfolgen?
	\item Wie lange soll eine Sicherung aufbewahrt werden?
\end{itemize}

Die Sicherungsstrategie kann mit den zu verwendenden Richtlinien zur Dateiablage eng verzahnt sein, um etwa einen nahezu automatisierten Sicherungsvorgang zu ermöglichen.

Die richtige Speicherstrategie beugt zwar möglichen Datenverlusten durch Hardware-, Software- oder menschliche Fehler vor, stellt aber noch \emph{keine} Archivierung der Daten dar. Eine Archivierung ist auf Langfristigkeit ausgelegt und impliziert immer eine bewusste Auswahl und umfassende Dokumentation der Daten, da eine Nachnutzung derselben das Ziel ist. Während der Projektlaufzeit kann bereits ein eigener Archivordner angelegt werden, worin beispielsweise finale Dateien oder unprozessierte Rohdaten abgelegt werden können, um die spätere Auswahl der zu archivierenden Daten vorzubereiten und zu vereinfachen.

Weiterführende Hinweise zur Datensicherung sind auf den Seiten des Bundesamtes für Sicherheit in der Informationstechnik zu finden, die sich sowohl an Einsteiger\footnote{\url{https://www.bsi-fuer-buerger.de/BSIFB/DE/MeinPC/Datensicherung/Sicherungsmethoden/sicherungsmethoden\_node.html}} als auch an Experten\footnote{\url{https://www.bsi.bund.de/DE/Themen/ITGrundschutz/ITGrundschutzKataloge/Inhalt/\_content/baust/b01/b01004.html}} richten. 
	\begin{flushleft}
		Das vollständige Kapitel finden Sie auf: \urllist{https://ianus-fdz.de/it-empfehlungen/dateispeicherung}
	\end{flushleft}
	
%Dateiformate
\chapter{Dateiformate}
	\abschnittsautor{F. Schäfer}
	\label{dateiformate}
Im Folgenden werden konkrete technische Hinweise und Hintergrundinformationen zu den Dateiformaten gegeben, die nach heutigem Wissen für die Langzeitarchivierung geeignet sind. Des Weiteren werden die minimalen Anforderungen an die technische wie inhaltliche Dokumentation von einzelnen Dateiformaten genannt, damit diese nicht nur heute, sondern auch in Zukunft von Dritten verstanden und nachgenutzt werden können.

Es geht hier nur um die Dateiformate an sich, ihre Entstehungsweise wird dabei nicht berücksichtigt. Abhängig von der jeweiligen Generierung und Verwendung sind gegebenfalls zusätzliche methodische und fachspezifische Metadaten erforderlich. Diese werden in dem Kapitel Forschungsmethoden auf \href{https://www.ianus-fdz.de/it-empfehlungen/forschungsmethoden}{www.ianus-fdz.de/it-empfehlungen/forschungsmethoden} beschrieben.

Bereits die Auswahl der verwendeten Software beeinflusst das Dateiformat und dessen Nachhaltigkeit. Daher sollte bei der Entscheidung für oder gegen eine bestimmte Software darauf geachtet werden, dass Dateiformate unterstützt werden, die möglichst folgende Eigenschaften aufweisen:
\begin{itemize}
	\item Weit verbreitet und standardisiert
	\item Nicht proprietär, also nicht von einer Anwendung oder einem Hersteller abhängig und mit unterschiedlichen Programmen verwendbar
	\item Offen dokumentiert mit frei verfügbaren technischen Spezifikationen
	\item Verlustfreie oder keine Kompression
	\item Einfach dekodierbar oder sogar unmittelbar lesbar, also nicht durch Kodierung versteckt
\end{itemize} 

Werden während eines Arbeitsprozesses andere als die hier empfohlenen Dateiformate erzeugt oder verwendet, sollte darauf geachtet werden, dass diese leicht und mit möglichst geringem Verlust an Informationen und Funktion in eines der in den folgenden Kapiteln aufgeführten Langzeitformate überführt werden können. 

In den Übersichtstabellen mit den Formaten für bestimmte Dateitypen sind die präferierten Formate mit einem grünen Haken, die noch akzeptablen mit einer gelben Tilde und für die Langzeitarchivierung ungeeignete Formate mit einem roten Kreuz gekennzeichnet.

\newpage
\section{PDF-Dokumente}
\label{pdf-dokumente}
\abschnittsautor{S. Jahn; Mit Unterstützung von: D. von Seggern}
	Das PDF (Portable Document Format) wurde 1993 von Adobe Systems entwickelt, um den Datenaustausch zu erleichtern. Es ist ein plattformunabhängiges, offenes Dateiformat, das 2008 mit der Version 1.7 als ISO-Standard zertifiziert wurde und seitdem von der ISO weiter gepflegt wird.  

Der große Vorteil von PDFs liegt darin, dass Dateiinhalte unabhängig vom Betriebssystem, dem ursprünglichen Anwendungsprogramm und der Hardwareplattform unverändert dargestellt werden. Das Aussehen von Dokumenten wird, wie bei einem analogen Ausdruck, eingefroren und kann somit angezeigt werden, wie es ursprünglich vom Autor intendiert war.  Gleichzeitig sind die Möglichkeiten zur nachträglichen Bearbeitung begrenzt, wodurch eine große Authentizität gewährleistet wird.

Für das Öffnen einer PDF-Datei gibt es verschiedene freie Anwendungen, worin ein Grund für die große Verbreitung und Akzeptanz von PDFs liegt. Viele Programme können Dateien direkt im PDF-Format speichern oder exportieren. Darüber hinaus lassen sich mit Hilfe von zusätzlich installierten Druckertreibern PDF-Dateien aus allen Programmen heraus erzeugen.

\subparagraph{Langzeitformate} Nicht jede Datei mit der Dateierweiterung .pdf  ist gleichermaßen für die Langzeitarchivierung geeignet. Für diesen Zweck  wurde das PDF/A-Format entwickelt und als ISO-Standard zertifiziert. Es beschreibt in welcher Form bestimmte Elemente in einer PDF-Datei enthalten sein müssen und welche nicht erlaubt sind. Die erste Version der PDF/A-Norm wurde nachträglich um zwei weitere, aufeinander aufbauende Normteile ergänzt. Wird das PDF/A-Format für die Langzeitarchivierung einer Datei verwendet, die nicht nur textuelle Informationen enthält, sollten nach Möglichkeit auch die ursprünglichen Ausgangsdateien in archivierungstauglichen Formaten entweder als separate Dateien oder integriert in das Container-Format PDF/A-3 mitarchiviert werden.

\begin{center}
	\begin{tabular}{l p{0.2\textwidth} p{0.6\textwidth}}
		\toprule
		\multicolumn{2}{l}{Format} & Begründung \\ \midrule
		\multirow{3}{*}{\color{ForestGreen} \LARGE \checkmark} & PDF/A-1 & \multirow{2}{*}{\parbox{0.6\textwidth}{PDF/A ist gezielt als stabiles, offenes und standardisiertes Format für die Langzeitarchivierung unterschiedlicher Ausgangsdateien entwickelt worden.}}\\
		& PDF/A-2 & \\
		& &\\
		\cmidrule(r){1-3}
		\multirow{1}{*}{$\color{BurntOrange} \thicksim$} & PDF/A-3 & PDF/A-3 ist nur dann für die Langzeitarchivierung geeignet, wenn alle eingebetteten Dateien in einem anerkannten Archivformat vorliegen.\\
		\cmidrule(r){1-3}
		\multirow{1}{*}{\LARGE \boldmath$\color{BrickRed} \times$}& andere PDF-Varianten & Viele gängige PDF-Varianten sind nicht für die Langzeitarchivierung geeignet. Stattdessen sollten entweder die Ausgangsdateien in einem passenden Format archiviert oder eine Migration in ein PDF/A-Format vorgenommen werden.\\
		\bottomrule
		\bottomrule
	\end{tabular}
\end{center}

\newpage
\subparagraph{Dokumentation}
Welche Metadaten für ein PDF-Dokument relevant sind, hängt von dessen Inhalt ab. Bei PDF-Dateien, die nur Text enthalten, sind meist weniger Informationen erforderlich, als bei PDFs, die Bilder, GIS-Karten oder 3D-Modelle enthalten. Die Metadaten können unmittelbar in einem PDF-Dokument im XMP-Format gespeichert werden.

Die hier angegebenen Metadaten sind als minimale Angabe zu betrachten und ergänzen die angegebenen Metadaten für Projekte und Einzeldateien in dem Abschnitt Metadaten in der Anwendung ab Seite \pageref{Metadaten-anwendung}.

\begin{center}
	\begin{tabular}{l p{0.6\textwidth}}
		\toprule
		Metadatum & Beschreibung \\ \midrule
Titel & Titel des Dokuments, nicht der Dateiname\\
Autor & Name des Verfassers oder Erstellers der Datei; gegebenfalls Name der Einrichtung\\
Stichwörter & Schlagworte, wie z.B. Periode, Fundstelle oder charakteristische Merkmale. Wenn vorhanden, angemessene Thesauri verwenden\\
Beziehungen & Dateien oder Ressourcen, die mit dem PDF zusammenhängen, insbesondere der Name der Originaldatei, aus der heraus ein PDF erstellt wurde\\
Anwendung & Programm, mit dem der Inhalt ursprünglich erzeugt wurde\\
Datum & Datum der Erstellung oder letzten Änderung der Datei\\
Copyright-Angaben & Angaben zur Person oder Einrichtung, die das Copyright oder die Lizenzrechte an der Datei oder deren Inhalt besitzt\\
\multicolumn{2}{l}{Zusätzliche Metadaten}\\
Kurzbeschreibung & Kurzbeschreibung über den Inhalt des Dokuments\\
Sprache & Sofern schriftliche Inhalte vorhanden sind, die Sprache angeben. Sprachkennungen nach ISO 639 angeben.\\
Verfasser der Metadaten & Name der Person, welche die Metadaten ausgefüllt hat\\
 		\bottomrule
		\bottomrule
	\end{tabular}
\end{center}

Weitere Metadaten sind abhängig vom Inhalt und der Methoden und können in den jeweiligen Abschnitten im Kapitel Forschungsmethoden ab Seite \pageref{methoden} nachgelesen werden.
	\begin{flushleft}
		Das vollständige Kapitel finden Sie auf: \urllist{https://ianus-fdz.de/it-empfehlungen/pdf-dokumente}
	\end{flushleft}

\newpage		
\section{Textdokumente}\label{textdokumente}
\abschnittsautor{M. Trognitz}
	\input{../dateiformate/textdokumente-uebersicht}	
	\begin{flushleft}
		Das vollständige Kapitel finden Sie auf: \urllist{https://ianus-fdz.de/it-empfehlungen/textdokumente}
	\end{flushleft}

\newpage
\section{Bilder -- Rastergrafiken}\label{rastergrafiken}
\abschnittsautor{M. Trognitz, P. Grunwald}
	\hyphenation{
Wech-sel-ob-jek-ti-ven
}
Bei Rastergrafiken, auch Pixelgrafiken, handelt es sich um digitale Bilder, die mittels rasterförmig angeordneter Bildpunkte, den Pixeln, beschrieben werden. Jedem Pixel ist dabei ein Farbwert zugeordnet. Rastergrafiken haben eine fixe Größe und sind im Gegensatz zu Vektorgrafiken nicht beliebig skalierbar. 

Zu den Rastergrafiken gehören: Digitale Fotografien jeder Art, Satellitenbilder, digitalisierte Bilder (Scans), Screenshots sowie digitale Originalbilder und -grafiken.

\subparagraph{Langzeitformate} Alle als Rastergrafiken vorliegenden Roh- und Urfassungen (Master) von Bildern sind in angemessener Qualität und unkomprimiert im baseline TIFF- oder DNG-Format abzuspeichern. Für georeferenziertes digitales Bildmaterial ist zwecks Erhalt der Referenzdaten das Format GeoTIFF zu verwenden.

Nur für Grafiken, \emph{nicht} für Fotos, eignet sich auch das PNG-Format. Allerdings ist jederzeit das TIFF-Format vorzuziehen. JPEG, bzw. JPG, eignet sich nicht zur Langzeitarchivierung, da es keine verlustfreie Komprimierung anbietet.

Es muss darauf geachtet werden, dass die Bildgröße und -auflösung der originalen Datei erhalten bleibt, wenn in andere Formate umgewandelt wird. Außerdem muss bei der Konvertierung darauf geachtet werden, dass eine verlustfreie Komprimierung verwendet wird. Auch Farbtiefe und Farbraum sollten nach der Konvertierung erhalten bleiben.

Bilder, die Ebenen enthalten, müssen vorher auf eine Ebene reduziert werden. Bei Bedarf, sollten die verschiedenen Ebenen und Komponenten als einzelne Dateien abgespeichert werden. 

%Vorschlag einer Hinweisbox
%\begin{center}
%	\fboxrule0.5mm
% 	\fcolorbox{ianusGrau}{white}{\parbox[c]{0.95\textwidth}{{\scshape\large\color{ianusBlau}Achtung} Wiederholtes Bearbeiten und Abspeichern, führt zu einer allmählichen Abnahme der Qualität. Dieser sogenannte \gls{Generationsverlust} tritt insbesondere bei der verlustbehafteten Bildkomprimierung auf.}}
%\end{center}

Hinweis: Wiederholtes Bearbeiten und Abspeichern, führt zu einer allmählichen Abnahme der Qualität. Dieser sogenannte Generationsverlust tritt insbesondere bei der verlustbehafteten Bildkomprimierung auf.

\begin{center}
	\begin{longtable}{l L{0.2\textwidth} p{0.6\textwidth}}
			\toprule 
		\multicolumn{2}{l}{Format} & Begründung \\
		\midrule \endfirsthead
		\multicolumn{3}{l}{\footnotesize Fortsetzung der vorhergehenden Seite}\\
		\toprule
		\multicolumn{2}{l}{Format} & Begründung \\ \midrule \endhead
		\bottomrule \multicolumn{3}{r}{{\footnotesize Fortsetzung auf der nächsten Seite}} \\
		\endfoot
		\bottomrule 
		\endlastfoot
		
		\multirow{3}{*}{\color{ForestGreen} \LARGE \checkmark} & Baseline TIFF, unkomprimiert & TIFF ist quasi ein Standardformat für die digitale Langzeitarchivierung von Bilddateien und unterstützt auch die Speicherung von Metadaten im Exif-Format. Obwohl das Format auch Kompression verwenden kann, kommt für die Langzeitarchivierung nur die unkomprimierte Form in Frage. \\
		  & DNG & Das von Adobe entwickelte Digital Negative Format ist ein offenes Format, das für die Langzeitarchivierung geeignet ist. Damit können RAW-Dateien und deren Metadaten (Exif oder IPTC-NAA) gelesen und gespeichert werden. Außerdem können über XMP weitere Metadaten eingespeist werden.\\ 
		  & GeoTIFF & GeoTIFF basiert auf TIFF und ist besonders für georeferenzierte Bilddaten geeignet, da somit die Referenzdaten erhalten bleiben.\\ \cmidrule(r){1-3}
		\multirow{1}{*}{$\color{BurntOrange} \thicksim$} & PNG & PNG ist eine verlustfreie Alternative zu dem GIF-Format, welches eine verlustbehaftete Kompression verwendet. Es bietet eine Farbtiefe von 32 Bit, einen Alphakanal für Transparenz und verlustfreie Kompression. Allerdings wird nur RGB als Farbraum unterstützt und es können keine Exif-Daten gespeichert werden. Das Format eignet sich \emph{nicht} für digitale Fotos. \\ \cmidrule(r){1-3}
		\multirow{2}{*}{\LARGE \boldmath$\color{BrickRed} \times$}& JPEG & Trotz einiger Vorzüge, eignet sich das JPEG-Format \emph{nicht} für die Langzeitarchivierung, da es \emph{keine} verlustfreie Komprimierung bietet.\\
		 & GIF & GIF kann sowohl statische, als auch animierte Bilder speichern. Da es aber verlustbehaftet komprimiert, wird PNG als Alternativformat empfohlen. \\
 		\bottomrule    
	\end{longtable}
\end{center}


\subparagraph{Dokumentation}
\label{Metadaten-Rastergrafiken} Eingebettete Metadaten, wie beispielsweise Exif, IPTC-NAA oder XMP, sollten behalten und archiviert werden. Am besten werden sie in eine eigene Text- oder XML-Datei transferiert und getrennt gespeichert.

Neben technischen Informationen, die sich hauptsächlich mit der Erstellung der Rastergrafik befassen, sollten vor allem auch beschreibende und administrative Metadaten über das Bild gespeichert werden.

Hinweis: Werden digitale Aufnahmen mit Programmen bearbeitet, welche die eingebetteten Metadaten ignorieren, gehen diese verloren. 

Die hier angegebenen Metadaten sind als minimale Angabe zu betrachten und ergänzen die angegebenen Metadaten für Projekte und Einzeldateien in dem Abschnitt Metadaten in der Anwendung ab Seite \pageref{Metadaten-anwendung}.

\begin{center}
		\begin{longtable}{L{0.3\textwidth} p{0.6\textwidth}}
			\toprule 
		Metadatum & Beschreibung \\
		\midrule \endfirsthead
		\multicolumn{2}{l}{\footnotesize Fortsetzung der vorhergehenden Seite}\\
		\toprule
		Metadatum & Beschreibung \\ \midrule \endhead
		\bottomrule \multicolumn{2}{r}{{\footnotesize Fortsetzung auf der nächsten Seite}} \\
		\endfoot
		\bottomrule 
		\endlastfoot
		
		Identifikator & Name der Datei, z.B. grabung01.tif \\
		Bildunterschrift & Der Titel oder eine passende Bildunterschrift des Bildes \\
		Beschreibung & Beschreibung des Bildes  \\
		Urheber & Name des Fotografen oder Erstellers \\
		Datum & Datum der Erstellung oder letzten Änderung des Bildes\\
		Rechte & Details zum Urheberrecht \\
		Schlagworte & Schlagworte, wie z.B. Periode, Fundstelle oder charakteristische Merkmale. Wenn vorhanden, angemessene Thesauri verwenden\\
		Ort & Ortsinformationen zu dem Bild. Möglichst in einem standardisierten Format angeben, wie z.B. Lat/Long oder Schlagworte aus einem geeigneten Thesaurus, z.B. Getty Thesaurus of Geographic Names oder GeoNames \\
		Dateiformat \& Version & z.B. Baseline TIFF 6.0 \\
		Dateigröße & Größe der Datei in Bytes \\
		Bildgröße & Maße des Bildes gemessen in Pixeln, z.B. $400px \times 700px$ \\
		Auflösung & Bildauflösung, gemessen in Punkten pro Zoll (dpi) \\
		Farbraum & Der in dem Bild verwendete Farbraum, z.B. RGB oder Graustufen \\
		Farbtiefe & z.B. 24 bit oder 8 bit \\
		Aufnahmegerät & Beispielsweise Details zur Kamera oder dem Scanner \\
		Software & Software mit der das Bild aufgenommen, erstellt oder bearbeitet wurde, wie z.B. Adobe Photoshop CS3 \\ 
 		\bottomrule    
	\end{longtable}
\end{center}

Weitere Metadaten sind methodenabhängig und können in den jeweiligen Abschnitten nachgelesen werden.
	\begin{flushleft}
		Das vollständige Kapitel finden Sie auf: \urllist{https://ianus-fdz.de/it-empfehlungen/rastergrafiken}
	\end{flushleft}

\newpage
\section{Bilder -- Vektorgrafiken und CAD-Daten}\label{vektorgrafiken}
\abschnittsautor{M. Trognitz, I. Schwarzbach, F. Martin}	
	\hyphenation{pro-jekt-ü--ber-grei-fen-de
Spei-che-rung
}
Vektorgrafiken sind Grafiken, die mittels grafischer Primitiven, wie beispielsweise Linien, Punkte, Polygone, Kreise und Kurven beschrieben werden. Sie sind daher im Gegensatz zu Rastergrafiken ohne Qualitätsverlust beliebig skalierbar. Zeichnungen von Scherben, Isometrische Darstellungen von Objekten oder aus CAD- oder GIS-Anwendungen exportierte Pläne werden häufig als Vektorgrafik angelegt oder gespeichert. 

Auch bei Daten, die mittels CAD entstehen, handelt es sich um Vektordaten mit grafischen Primitiven, die in den entsprechenden Programmen zusätzliche Funktionalitäten bieten, die speziell auf architektonische und technische Zeichnungen zugeschnitten sind. CAD-Daten können sowohl in 2D als auch in 3D vorliegen und werden in den Altertumswissenschaften meist für Zeichnungen von Ausgrabungen, archäologische Karten oder architektonische Pläne verwendet.

Dieser Abschnitt beschäftigt sich mit der Archivierung von zweidimensionalen Vektorgrafiken und CAD-Daten. Die Archivierung von GIS-Daten wird in dem eigenen Abschnitt GIS ab Seite \pageref{GIS} beschrieben. Hinweise zur Archivierung von dreidimensionalen Objekten sind in dem Abschnitt 3D und Virtual Reality ab Seite \pageref{3D} zu finden.


\subparagraph{Langzeitformate}
Für zweidimensionale Vektorgrafiken ist das vom W3C entwickelte und seit 2001 auch empfohlene Format Scalable Vector Graphics (SVG) für die Langzeitarchivierung geeignet, da es ein XML-basiertes, offen dokumentiertes Format ist, eine weite Verbreitung gefunden hat und als MIME-Type image/svg+xml registriert ist. Die aktuelle Version ist 1.1. Neben grafischen Primitiven, geometrischen Formen, Text und eingebetteten Rastergrafiken können SVG-Dateien auch Skripte und Animationen enthalten. Mit Hilfe des Gruppierungselementes \emph{$<$d$>$} können Ebenen ausgezeichnet werden, wobei dieses Element in der Praxis auch für andere Zwecke genutzt wird. Umfassende Metadaten können in einem eigens dafür vorgesehenen Bereich gespeichert werden. Speziell für mobile Geräte gibt es die Variante SVG Tiny 1.2. Das W3C arbeitet seit 2012 an SVG 2, das in absehbarer Zeit die Version 1.1. ablösen soll. SVG-Dateien können auch komprimiert mittels gzip als SVGZ gespeichert werden, wobei dies nicht für die Archivierung zu empfehlen ist. Auch von der Verwendung von Skripten sollte für die Archivierung abgesehen werden.

Computer Graphics Metafile (CGM) ist unter ISO/IEC 8632 standardisiert, wurde 1986 entwickelt und zuletzt 1999 aktualisiert. CGM kann umfangreiche grafische Informationen und geometrische Primitiven speichern, ist inzwischen jedoch für viele Bereiche veraltet, weshalb SVG für die Archivierung verwendet werden sollte. Das W3C hat 2001 die Variante WebCGM veröffentlicht, die zusätzliche Funktionalitäten für die Verwendung im Internet enthält. Die aktuellste und empfohlene Fassung von WebCGM ist Version 2.1.

Speziell für CAD-Daten gibt es kein Format, das uneingeschränkt für die Langzeitarchivierung zu empfehlen ist. Ein Grund dafür ist die immerwährende und schnelle Entwicklung von CAD-Software, die sich auch auf die Ausgabeformate auswirkt und zu einer hohen Heterogenität und mangelnden Interoperabilität der Datenformate führt. Durch die weite Verbreitung des von der Firma Autodesk vertriebenen Programms AutoCAD haben die von Autodesk entwickelten Dateiformate DXF und DWG eine weite Verbreitung in der Archäologie und Bauforschung gefunden und sich de facto als Standard durchgesetzt.

Das Drawing Interchange Format (DXF, aktuelle Version v.u.28.1.01) wurde von Autodesk entwickelt, um den Austausch von CAD-Dateien zwischen AutoCAD und anderen Programmen zu ermöglichen. Dabei wurde das Format so gestaltet, dass es möglichst alle in einer Datei enthaltenen Informationen speichern kann, ohne dass dafür die Optimierungen des proprietären DWG-Formates von Autodesk enthüllt werden müssen. Neue Versionen von DXF werden fast jedes Jahr veröffentlicht und durch neue Funktionalitäten angereichert. Für die jüngsten DXF-Versionen stellt Autodesk die Spezifikationen frei zur Verfügung, wobei allerdings ab Release 13 (v.u13.1.01) manche Bereiche ungenügend spezifiziert sind, so dass unter Umständen nicht alle Funktionen in anderen Programmen umgesetzt und unterstützt werden. DXF gibt es sowohl in einer textbasierten ASCII-Variante, als auch als Binärformat, wobei für die Archivierung erstere bevorzugt werden sollte. Maße werden von DXF nicht unterstützt, weshalb die Angabe des Maßstabes und der Zeicheneinheiten erforderlich ist.

DWG (abgeleitet vom englischen \emph{drawing}) ist das native Speicherformat von AutoCAD und weder offen dokumentiert noch standardisiert. Mit jeder neuen Version von AutoCAD wird auch eine neue Version des DWG-Formates veröffentlicht. Da AutoCAD aber eine so weite Verbreitung gefunden hat, wird dieses Format auch von anderen CAD-Programmen unterstützt. Eine frei verfügbare nachkonstruierte Spezifikation wird von der Open Design Alliance zur Verfügung gestellt.

Für die Archivierung von CAD-Dateien sollten bis zur Etablierung von geeigneten Archivformaten neben dem Originalformat auch Versionen der Datei als DXF und als DWG vorgehalten werden. Um eine möglichst hohe Nachnutzbarkeit zu gewährleisten, sollte eine ältere verbreitete Version der Formate verwendet werden, wie beispielsweise DWG 2010 (AC1024) und DXF 2010 (AC1024). 

Für alle Vektorgrafiken und insbesondere für CAD-Daten gilt die zusätzliche Empfehlung, verschiedene Druckansichten in geeigneten Formaten, wie etwa PDF/A, zu speichern. Dadurch bleibt das intendierte Aussehen der Ursprungsgrafik erhalten, auch wenn die direkte Bearbeitbarkeit verloren geht. Jedoch können Vektorgrafiken größtenteils auch wieder aus einer PDF/A-Datei extrahiert werden. Nähere Informationen zu dem Format sind im Abschnitt über PDF-Dokumente ab Seite \pageref{pdf-dokumente} zu finden.

Externe zugehörige Dateien müssen ebenfalls in einem geeigneten Archivformat gespeichert werden.

PostScript (PS) ist eine Sprache zur Beschreibung von Seiten, die von Adobe entwickelt wurde, und einen Grundstein des PDF-Formates bildet. Darauf aufbauend wurde das Format Encapsulated PostScript (EPS) entwickelt, das neben der Beschreibung in PostScript auch eine eingebettete Datei enthält, die eine Vorschau auf den Inhalt der Datei ermöglicht. Die Vorschaudatei wird jedoch nicht einheitlich gespeichert, was zu einer eingeschränkten Interoperabilität von EPS führt. Ferner kann EPS keine Ebenen und Transparenz speichern. Für die Langzeitarchivierung sollte daher SVG oder PDF/A verwendet werden.

Proprietäre Formate für Vektorgrafiken, wie etwa die Formate AI von Adobe Illustrator und INDD von Adobe InDesign, sind nicht für die Langzeitarchivierung geeignet. Dies trifft ebenfalls auf das von Autodesk entwickelte Format DWF zu, da es primär für Visualisierungen gedacht ist und nicht für den vollständigen Datenaustausch. In der Industrie werden als alternative offene Austauschformate auch IGES und STEP für CAD-Daten und IFC für Gebäudedatenmodellierung verwendet.

\begin{center}
	\begin{longtable}{l L{0.2\textwidth} p{0.6\textwidth}}
			\toprule 
		\multicolumn{2}{l}{Format} & Begründung \\
		\midrule \endfirsthead
		\multicolumn{3}{l}{\footnotesize Fortsetzung der vorhergehenden Seite}\\
		\toprule
		\multicolumn{2}{l}{Format} & Begründung \\ \midrule \endhead
		\bottomrule \multicolumn{3}{r}{{\footnotesize Fortsetzung auf der nächsten Seite}} \\
		\endfoot
		\bottomrule 
		\endlastfoot
		
		\multirow{1}{*}{\color{ForestGreen} \LARGE \checkmark} & SVG & Ein XML-basierter offener Standard vom W3C. Für die Archivierung sollte unkomprimiertes SVG 1.1 ohne script bindings verwendet werden. \\ \cmidrule(r){1-3}
		\multirow{4}{*}{$\color{BurntOrange} \thicksim$} & WebCGM & Vom W3C 2001 als Variante von CGM veröffentlicht, um die Verwendung im Internet zu ermöglichen. Wenn es verwendet wird, dann  in Version 2.1, wobei SVG für die Langzeitarchivierung besser geeignet ist.\\
			& DXF & Das Drawing Interchange Format wurde von Autodesk entwickelt und ist offen dokumentiert. Da Maße nicht gespeichert werden, ist die Angabe des Maßstabes und der Zeicheneinheiten erforderlich. Um eine hohe Nachnutzbarkeit zu erlauben, sollte eine ältere Version, wie etwa DXF 2010 (AC1024), verwendet werden.\\
			& DWG & Ist ein proprietäres Format von Autodesk und sollte nur verwendet werden, wenn eine Speicherung in DXF einen erheblichen Funktionsverlust darstellen würde. Um eine hohe Nachnutzbarkeit zu erlauben, sollte eine ältere Version, wie etwa DWG 2010 (AC1024), verwendet werden.\\
			& PDF/A & PDF/A ist für die langfristige Speicherung von zweidimensionalen Vektorgrafiken und Druckansichten von CAD-Daten geeignet. Jedoch bleibt nur das Aussehen erhalten. Nähere Informationen zu dem Format sind im Abschnitt über PDF-Dokumente ab Seite \pageref{pdf-dokumente} zu finden.\\ \cmidrule(r){1-3}
		\multirow{4}{*}{\LARGE \boldmath$\color{BrickRed} \times$}& CGM & Ist unter ISO/IEC 8632 standardisiert, jedoch veraltet. Stattdessen sollte SVG verwendet werden.\\
			& PS, EPS & Encapsulated PostScript beruht auf PostScript, einer Sprache zur Beschreibung von Seiten für den Druck, die von Adobe entwickelt wurden. Für die Archivierung sollte SVG oder PDF/A verwendet werden.\\
			& AI, INDD & Proprietäre Formate von Adobe Illustrator und Adobe InDesign sind nicht für die Langzeitarchivierung geeignet.\\
			& DWF & Wurde von Autodesk für die Visualisierung von CAD-Daten entwickelt und ist nicht für die Langzeitarchivierung geeignet.\\
 		\bottomrule 
	\end{longtable}
\end{center}


\subparagraph{Dokumentation}
Neben den allgemeinen minimalen Angaben zu Einzeldateien, wie sie in dem Abschnitt Metadaten in der Anwendung ab Seite \pageref{Metadaten-anwendung} aufgelistet sind, werden für Vektorgrafiken und insbesondere für CAD-Daten weitere Angaben benötigt.

Neben technischen Informationen sollten vor allem auch beschreibende und administrative Metadaten über die Datei erfasst werden. Dabei können manche Metadaten auch Bestandteil der Grafik selbst sein, wie beispielsweise eine Legende, die über Zeichnungskonventionen für verwendete Strichstärken oder Farben Auskunft gibt.

Zur Dokumentation von CAD-Daten gehören unter anderem die Angabe der Quellen oder Messdaten, die zur Erstellung der Zeichnung herangezogen wurden, die Dokumentation der wesentlichen Arbeitsschritte, die im Verlauf der Erstellung durchlaufen wurden, die Dokumentation der angewendeten Methoden und eine Beschreibung der Abhängigkeitsverhältnisse zwischen den unterschiedlichen Bestandteilen wie etwa den Ebenen.

\begin{center}
		\begin{longtable}{L{0.3\textwidth} p{0.6\textwidth}}
			\toprule 
		Metadatum & Beschreibung \\
		\midrule \endfirsthead
		\multicolumn{2}{l}{\footnotesize Fortsetzung der vorhergehenden Seite}\\
		\toprule
		Metadatum & Beschreibung \\ \midrule \endhead
		\bottomrule \multicolumn{2}{r}{{\footnotesize Fortsetzung auf der nächsten Seite}} \\
		\endfoot
		\bottomrule 
		\endlastfoot
		
		Identifikator & Name der Datei, z.B. grabung01.dxf \\
		Dateiformat \& Version & z.B. DXF Release 14 \\
		Farbraum & Der in dem Bild verwendete Farbraum, z.B. RGB oder Graustufen \\
		Farbtiefe & z.B. 24 bit oder 8 bit \\
		Bildunterschrift & Der Titel oder eine passende Unterschrift \\
		Beschreibung & Beschreibung der Datei  \\
		Urheber & Name des Erstellers oder der Bearbeiter \\
		Datum & Datum der Erstellung oder letzten Änderung der Datei\\
		Rechte & Details zum Urheberrecht \\
		Schlagworte & Schlagworte, wie z.B. Periode, Fundstelle oder charakteristische Merkmale. Wenn vorhanden, angemessene Thesauri verwenden\\
		Ort & Ortsinformationen zu der Datei. Möglichst in einem standardisierten Format angeben, wie z.B. Lat/Long oder Schlagworte aus einem geeigneten Thesaurus, z.B. Getty Thesaurus of Geographic Names oder GeoNames \\
		Software & Angaben zu Software (inklusive Plug-ins) und Version mit der die Datei erstellt oder bearbeitet wurde, wie z.B. Autodesk AutoCAD 2012\\ 
		Externe Dateien & Eine Liste aller externen zugehöriger Dateien, wie etwa Messdaten, Rastergrafiken oder eingebundener Datenbanken\\
		Maßstab & Angabe zum verwendeten Maßstab bzw. was eine Zeicheneinheit darstellt\\
		Koordinatensysteme & Falls vorhanden, verwendetes Koordinatensystem oder Kartenprojektion angeben\\
		Konventionen & Dokumentation der Bedeutung von Farben, Linienstärken, Linientypen, Füllarten etc.\\
		Ebenen & Angabe der Ebenen, deren Inhalt und der Konventionen für deren Benennung und Befüllung\\
		Aufnahmemethode & Angabe zur verwendeten Methode zur Datengewinnung \\
		Aufnahmegerät & Beispielsweise Details zum Laserscanner \\
		Messgenauigkeit & Angaben zur Messgenauigkeit des Aufnahmegerätes\\
		\bottomrule    
	\end{longtable}
\end{center}

Weitere Metadaten sind methodenabhängig und können in den jeweiligen Abschnitten nachgelesen werden.
	\begin{flushleft}
		Das vollständige Kapitel finden Sie auf: \urllist{https://ianus-fdz.de/it-empfehlungen/vektorgrafiken}
		\label{GIS}
		Verweis GIS: \urllist{https://ianus-fdz.de/it-empfehlungen/node/70}
	\end{flushleft}

\newpage	
\section{Tabellen}\label{tabellen}
\abschnittsautor{M. Trognitz}
	Tabellen werden verwendet, um Informationen strukturiert in Zellen zu speichern, die in Zeilen und Spalten angeordnet sind. Im Gegensatz zu analogen Tabellen bieten digitale Tabellen, die mit entsprechenden Programmen erstellt werden, eine Vielzahl an weiteren Funktionalitäten. Beispielsweise können Zellinhalte nach bestimmten Kriterien sortiert, dynamisch mittels Formeln erzeugt oder Grafiken aus den Daten generiert werden. Um solche und andere Funktionen zu erhalten, erfordert die Speicherung besondere Aufmerksamkeit.


\subparagraph{Langzeitformate} Für einfache Tabellen ohne interaktive Elemente wie Formeln oder für Tabellenkalkulationen bei denen es ausreicht die Ergebnisse der Formeln zu speichern, wird ein textbasiertes Format mit Trennzeichen für die Archivierung empfohlen. 

Dazu eignet sich beispielsweise das CSV-Format, wobei als Trennzeichen ein Komma (\emph{,}) und als Textbegrenzungszeichen das Anführungszeichen (\emph{''}) verwendet werden sollte, um den Vorgaben von \href{https://tools.ietf.org/html/rfc4180}{RFC 4180} gerecht zu werden. Andere Trennzeichen und Textbegrenzungszeichen können in begründeten Ausnahmefällen ebenfalls eingesetzt werden und müssen entsprechend dokumentiert werden. Als Zeichenkodierung sollte UTF-8 ohne BOM verwendet werden.

Ein alternatives textbasiertes Format für Tabellen ist das TSV-Format, das als Trennzeichen das Tabulator-Zeichen (U+0009) verwendet. Auch TSV"=Dateien sollten UTF-8 ohne BOM als Zeichenkodierung verwenden.

Bei Dateien mit mehr als einem Arbeitsblatt (Tabellenkalkulationen), die als CSV- oder TSV-Datei gespeichert werden sollen, muss jedes Arbeitsblatt gesondert gespeichert werden. Dabei gilt für die Dateinamen der Arbeitsblätter, dass der Name des Arbeitsblattes an den Namen der Tabellenkalkulation, am besten durch ein Unterstrich ({\_}) getrennt, angefügt wird (z. B. Tabellenname{\_}Blatt1.csv, Tabellenname{\_}Blatt2.csv usw.).

Tabellenkalkulationen, deren zusätzlichen Funktionalitäten erhalten bleiben sollen, werden am besten in einem offenen auf XML basierenden Format gespeichert, wie beispielsweise XLSX oder ODS. Ersteres ist das Standardformat, das in Microsoft Excel seit 2007 verwendet wird und auch von Microsoft entwickelt wurde. Letzteres ist das Format für Tabellenkalkulationen, welches in OpenOffice oder LibreOffice verwendet wird. ODS ist ein Teil vom OpenDocument Format (ODF) und wurde von einem technischen Komitee unter der Leitung der \emph{Organization for the Advancement of Structured Information Standards} (OASIS) entwickelt.

Grafiken, die in Tabellenkalkulationen anhand der Daten erstellt wurden, müssen zusätzlich exportiert und gesondert in einem geeigneten Format gespeichert werden. Dies gilt ebenfalls für eingebettete Bilder oder andere Medien. Passende Formate sind beispielsweise in dem Abschnitt Rastergrafiken ab Seite \pageref{rastergrafiken} oder in dem Abschnitt Vektorgrafiken ab Seite \pageref{vektorgrafiken} zu finden.

Tabellen können auch im XML-Format gespeichert werden. Es gibt eine ganze Reihe an DTDs oder XSDs, die hier als Grundlage dienen können, wie beispielsweise das Schema der TEI, das OASIS Exchange Table Model oder ISO 12083. Auch das Speichern von Tabellen im HTML-Format ist möglich. In jedem Fall muss die zugrundeliegende DTD oder XSD angegeben und gegebenenfalls mit archiviert werden. Die Dateien sollten UTF-8 ohne BOM als Zeichenkodierung verwenden.

Wenn neben den eigentlichen Daten in den einzelnen Zellen auch das Aussehen der Tabelle archiviert werden soll, kann neben einer CSV-, TSV-, XLSX- oder ODS-Datei zusätzlich eine Version der Tabelle im PDF/A-Format gespeichert werden. Bei der Erstellung von Tabellen sollte aber darauf geachtet werden, dass Informationen nicht nur durch Formatierungsangaben, wie beispielsweise die Farbe von Zellen, vermittelt werden, da je nach gewähltem Format die Formatierungsangaben verloren gehen können.

Hinweis: Obwohl Textverarbeitungsprogramme entsprechende Funktionen bieten, sollten Tabellen auch tatsächlich als Tabellen in einem der hier gelisteten Formate gespeichert werden.

\begin{center}
	\begin{tabular}{l p{0.2\textwidth} p{0.6\textwidth}}
		\toprule
		\multicolumn{2}{l}{Format} & Begründung \\ \midrule		
		\multirow{5}{*}{\color{ForestGreen} \LARGE \checkmark} & CSV & Das textbasierte CSV-Format sollte mit einem Komma als Trennzeichen und mit Anführungszeichen als Textbegrenzungszeichen verwendet werden. Ausnahmen müssen dokumentiert werden. Die Zeichen sollten in UTF-8 ohne BOM kodiert sein.\\
			& TSV & TSV (MIME-Type text/tab-separated-values) ist ein textbasiertes Format, welches das Tabulator-Zeichen (U+0009) als Trennzeichen verwendet. Die Zeichen sollten in UTF-8 ohne BOM kodiert sein.\\
		  & ODS & ODS basiert auf XML und ist Teil vom OpenDocument Format. Eingebettete Bilder und Medien müssen gesondert gespeichert werden.\\
		  & XLSX & XLSX ist das auf XML basierende Format von Microsoft. Eingebettete Bilder und Medien müssen gesondert gespeichert werden.\\ 
			& XML, HTML & Tabellen im textbasierten XML- oder HTML-Format können ebenfalls archiviert werden. XML-Dateien benötigen zusätzlich eine DTD-Datei oder das XML Schema. Die Zeichen sollten in UTF-8 ohne BOM kodiert sein.\\ \cmidrule(r){1-3}
		\multirow{2}{*}{$\color{BurntOrange} \thicksim$} & PDF/A & Wenn neben der Daten auch das Aussehen der Tabelle erhalten bleiben soll, eignet sich PDF/A am besten. Zusätzlich sollten die tabellarischen Daten in einem nachnutzbaren Format, wie etwa CSV, gespeichert werden.\\ 
			& SXC & SXC ist ein Vorgängerformat von ODS, weshalb letzteres auch bevorzugt werden sollte.\\ \cmidrule(r){1-3}
		\multirow{1}{*}{\LARGE \boldmath$\color{BrickRed} \times$}& XLS & Das XLS-Format von Microsoft eignet sich nicht zur Archivierung, da es proprietär ist und die Inhalte nicht textbasiert gespeichert werden.\\
		\bottomrule
		\bottomrule
	\end{tabular}
\end{center}

\subparagraph{Dokumentation} Neben den allgemeinen minimalen Angaben zu Einzeldateien, wie sie in dem Abschnitt Metadaten in der Anwendung ab Seite \pageref{Metadaten-anwendung} gelistet sind, werden für Tabellen und Tabellenkalkulationen weitere Angaben benötigt.

Um die Verständlichkeit einer Tabelle auch für Dritte zu gewährleisten, müssen Name und Zweck der jeweiligen Tabelle und der einzelnen Arbeitsblätter bekannt sein. Jede Spalte benötigt eine Überschrift, und zusätzlich müssen die verwendeten Formatvorgaben, Abkürzungen, Codes, Wertelisten und sonstige Terminologien dokumentiert werden. Um leere Zellen auch explizit als solche zu kennzeichnen, sollte ein vorher festgelegtes Zeichen (z.B. -) eingetragen und dokumentiert werden. Wenn Maßeinheiten nicht direkt aus der Tabelle ersichtlich sind, müssen diese ebenfalls gesondert dokumentiert werden.

Um sicher zu gehen, dass die Tabelle auch vollständig vorliegt, sollten die Anzahl der Spalten, Zeilen und der Arbeitsblätter angegeben werden.

Tabellen in textbasierten Formaten brauchen Angaben zu den verwendeten Trennzeichen, Textbegrenzungszeichen und der Zeichenkodierung. 

Für Tabellenkalkulationen müssen bei Bedarf weitere Informationen zu Relationen, Formeln und Makros dokumentiert werden. Eingebettete Medien, wie etwa Bilder, sollten separat gespeichert und archiviert werden und in einer Liste zugehöriger Dateien aufgeführt werden. Dies gilt ebenfalls für Grafiken, die aus den Daten in der Tabellenkalkulation erzeugt wurden.

\begin{center}
	\begin{longtable}{L{0.3\textwidth} p{0.6\textwidth}}
			\toprule 
		Metadatum & Beschreibung \\
		\midrule \endfirsthead
		\multicolumn{2}{l}{\footnotesize Fortsetzung der vorhergehenden Seite}\\
		\toprule
		Metadatum & Beschreibung \\ \midrule \endhead
		\bottomrule \multicolumn{2}{r}{{\footnotesize Fortsetzung auf der nächsten Seite}} \\
		\endfoot
		\bottomrule 
		\endlastfoot
		
		Beschreibung der Tabelle oder des Arbeitsblattes & Welchen Zweck verfolgt die Tabelle oder das Arbeitsblatt?\\
		Bezeichnung der Arbeitsblätter & Auflistung der Bezeichnungen der Arbeitsblätter.\\
		Spaltenüberschrift & Jede Spalte einer Tabelle muss einen Namen haben.\\
		Spaltenbeschreibung & Beschreibung und Auflistung der in der jeweiligen Spalte verwendeten Formatvorgaben, Abkürzungen, Codes, Wertelisten, Eingabekonventionen, Fachvokabulare, Zeichen für leere Zellen oder Maßeinheiten.\\
		Anzahl Spalten & Wie viele Spalten enthält die Tabelle?\\		
		Anzahl Zeilen & Wie viele Zeilen enthält die Tabelle?\\
		Anzahl Arbeitsblätter & Anzahl der Arbeitsblätter in einer Tabellenkalkulation.\\
		Trennzeichen & Angabe des verwendeten Trennzeichens bei textbasierten Speicherformaten wie CSV.\\
		Textbegrenzungszeichen & Angabe des verwendeten Textbegrenzungszeichens bei textbasierten Speicherformaten wie CSV.\\
		Zeichenkodierung & Angabe der verwendeten Zeichenkodierung bei textbasierten Speicherformaten wie CSV oder TSV.\\
		Relationen & Welche Querverweise gibt es innerhalb der Tabellenkalkulation?\\
		Formeln & Welche Formeln werden in der Tabellenkalkulation verwendet?\\
		Makros & Welche Makros gibt es in der Tabellenkalkulation?\\
		Abgeleitete Grafiken & Aus den Daten erzeugte Grafiken müssen zusätzlich separat gespeichert werden und in der Liste zugehöriger Dateien aufgenommen werden.\\
		Sprache & In welchen Sprachen ist das Dokument verfasst? Sprachkennungen nach ISO 639 angeben.\\
		Identifikator & Wenn das Dokument bereits veröffentlicht wurde und eine ISBN oder einen anderen persistenten Identifikator erhalten hat, sollte dieser angegeben werden.\\
		Weitere Dateien & Liste abgeleiteter Grafiken und eingebetteter Medien, wie Bilder, die zusätzlich separat gespeichert wurden. Liegt eine Dokumentationsdatei für das Dokument vor, muss diese ebenfalls genannt werden.\\
	  \bottomrule
	\end{longtable}
\end{center}

Weitere Metadaten sind methodenabhängig und können in den jeweiligen Abschnitten nachgelesen werden.
	\begin{flushleft}
		Das vollständige Kapitel finden Sie auf: \urllist{https://ianus-fdz.de/it-empfehlungen/tabellen}
	\end{flushleft}

\newpage
\section{Datenbanken}\label{Datenbanken}\label{datenbanken}
\abschnittsautor{P. Gerth, F. Schäfer}
	\hyphenation{
Mit-ar-bei-ter-da-tei
Fund-da-tei
Be-fund-da-tei
Pro-jekt-da-tei
}
\paragraph{Übersicht}
Datenbanksysteme (DBS) werden zur Strukturierung, Pflege und Verwaltung von Daten verwendet. Sie bestehen aus zwei Komponenten - einerseits aus der Datenbank, welche den logisch zusammengehörigen Datenbestand umfasst, sowie aus dem  Datenbankmanagementsystem (DBMS), der Datenbanksoftware, welche u.a. die Zugriffsrechte regelt und Vorkehrungen zur Datensicherheit gewährleistet. Es gibt verschiedene Datenbankmodelle, von denen relationale Datenbanksysteme am häufigsten verwendet und im Rahmen der IT Empfehlungen betrachtet werden. Dabei werden die Daten in Tabellen (Relationen) strukturiert, wobei Verweise von einer Tabelle auf eine andere möglich sind. Näheres zu Tabellen ist in dem Kapitel Tabellen ab Seite \pageref{tabellen} zu finden.

\subparagraph{Langzeitformate}
Für relationale Datenbanken wird eine Archivierung in Form eines Exports der Datenbankinhalte, der Strukturinformationen und weiterer Funktionalitäten in textbasierte Formate empfohlen, welche unabhängig vom verwendeten Datenbankmanagementsystem sind. Neben den Daten in den Tabellen müssen die Datenbankstrukturdefinitionen wie Attributdatentypen, Relationen zwischen den Tabellen und Formeln zwingend mit archiviert werden. Unabhängig vom gewählten Archivformat ist es nötig, die grafische Benutzeroberfläche zu dokumentieren, zum Beispiel in Form von Bildschirmfotos oder eines eventuell vorhandenen Benutzerhandbuches. Eine darüberhinaus gehende Erhaltung der Funktionalität ist zumeist nur mit Hilfe der technisch sehr aufwendigen Emulation möglich, bei der das ursprüngliche Datenbanksystem durch ein anderes System mit ähnlicher Funktionsweise ersetzt wird. Unabhängig von den vorgestellten Archivformaten ist der Umgang mit sogenannten Binary Large Objects (BLOBs). Dabei handelt es sich um intern gespeicherte Mediendateien, meistens Fotos, Zeichnungen, z. B. von Funden, oder PDF-Dokumente. Diese Medien müssen in jedem Fall exportiert, auf die dann externen Dateien muss referenziert und je nach Dateityp in für die Archivierung geeigneten Formaten gespeichert werden (siehe entsprechendes Dateiformatkapitel). Grundsätzlich wird von der Verwendung von BLOBs abgeraten und eine externe Speicherung von Medieninhalten empfohlen.

Ein geeignetes Archivformat ist SQL (Structured Query Language), welches seit 1986 bei der ANSI standardisiert ist und seit 1987 ebenfalls als ISO/IEC 9075 zertifiziert ist. Seitdem wurde der SQL-Standard in sieben Revisionen weiter ausgebaut, wobei SQL:2011 die neueste Version ist. Jedoch wird dieser Standard in den verschiedenen Datenbanksystemen wie MySQL, PostgreSQL oder OracleDB unterschiedlich umgesetzt und erweitert. Daher gibt es je nach Datenbanksystem mehrere spezifische Befehle, welche über die Spezifikation des Standards hinausgehen. In Folge dessen sind die verschiedenen DBMS nicht vollständig kompatibel, weswegen eine exportierte Datenbank im SQL-Format nur dann problemlos und ohne zusätzlichen Aufwand importiert werden kann, wenn das ursprüngliche DBMS verwendet wird. Daher kann SQL für die Archivierung nur dann empfohlen werden, wenn ein einheitlicher ANSI-/ISO-Standard, z. B. SQL:2008, verwendet wird, welcher von DBMS-spezifischen Befehlen bereinigt ist. Der verwendete Standard muss dokumentiert werden. Der Vorteil in der Speicherung in SQL liegt in der gleichzeitigen Erhaltung der kompletten Daten, der Tabellen und ihrer Relationen zueinander, der Attributspezifikationen sowie der kompletten Metadaten. Nach einem Export, erhält man eine singuläre textbasierte Datei, welche sich mit relativ geringem Aufwand wiederherstellen lässt. Die Speicherung einer Datenbank in einem SQL-Skript erfolgt unter Angabe der SQL-Befehle, welche zum Import der Datenbankinhalte benötigt und ggf. für jeden Datensatz wiederholt werden müssen. Die damit einhergehende Redundanz erhöht die Datenmenge. Datenbanken aus geisteswissenschaftlichen Forschungsprojekten erreichen jedoch nur selten eine Größe, bei welcher dieser Nachteil zum Tragen kommt.

XML eignet sich als Archivformat für Datenbanken aus Gründen der Systemunabhängigkeit, Standardisierung, Lesbarkeit und Erweiterbarkeit. Für die Speicherung der Datenbankinhalte kann XML uneingeschränkt empfohlen werden. Die Beschreibung der Datenbankstruktur sollte mithilfe einer entsprechenden XML-Schema-Definition (XSD) wie der Database Markup Language erfolgen. Alternativ kann die Datenbankstruktur im SQL-Format und als Entitäten-Relationen-Diagramm (ERD) gespeichert werden.

Das 2008 initial vom schweizerischen Bundesarchiv entwickelte und im Rahmen des e-ARK-Projekts weiterentwickelte Format SIARD (Software Independent Archiving of Relational Databases) in der Version 2.0 beruht im Wesentlichen auf den beiden ISO-Standards XML gemäß ISO/IEC 19503:2005 und SQL:2008. Es vereinigt damit die Vorteile einer Datenhaltung in XML mit einer Dokumentation der Datenbankstruktur in SQL. Es ist als eCH-Standard offen und wird unter einer freien Lizenz zur Verfügung gestellt.

Ein alternativer bewährter Ansatz zur Archivierung der Tabellen einer Datenbank ist der Export in das CSV-Format, welches besonders langlebig und weit verbreitet ist. Diese Methode ist technisch mit relativ geringem Aufwand verbunden, besitzt aber einen starken Fokus auf die Datenbankinhalte und geht daher mit einem vollständigen Verlust der Funktionalitäten und Datenbankstruktur einher. Infolgedessen ist es zwingend erforderlich, bei der Speicherung im CSV-Format zusätzlich die Datenbankstruktur in Form von strukturierten Diagrammen, wie dem Entitäten-Relationen-Diagramm (ERD) festzuhalten. Selbst dann ist die Datenbankstruktur nur teilweise und sehr aufwendig rekonstruierbar, weswegen das CSV-Format nur bei einer hybriden Lösung mit der Archivierung der Datenbankstruktur in einem geeigneten Format (z. B. SIARD, SQL, XML/DML) bei gleichzeitiger Speicherung der Daten im CSV-Format, empfohlen werden kann. Als Zeichenkodierung sollte UTF-8 verwendet werden, wobei RFC-4180-konform als Trennzeichen ein Komma (\emph{,}) und als Textbegrenzungszeichen das Anführungszeichen (\emph{''}) verwendet werden sollte. Mehr dazu ist in dem Kapitel Tabellen ab Seite \pageref{tabellen} zu finden.

Bei der JavaScript Object Notation (JSON) handelt es sich um ein einfaches textbasiertes Datenformat, welches für einen schnellen Datentransfer im Internet entwickelt wurde und als RFC 7159 und ECMA-404 standardisiert ist. Es besitzt bei der Speicherung von Daten vergleichbare Vorteile wie XML, ist dabei aber wesentlich kompakter. Zur Zeit gibt es allerdings keinen Standard zur Beschreibung von Datenbankstrukturen in JSON, weswegen auch hier auf eine hybride Archivierungslösung mit der Speicherung der Datenbankinhalte in JSON und gleichzeitiger Speicherung der Datenbankstruktur in einem anderen Format (z. B. SIARD, SQL, XML/DML) zurückgegriffen werden muss.

Die proprietären Dateiformate von Desktop-DBMS, wie Microsoft Access, dBASE oder FileMaker eignen sich nicht für die Archivierung, da die Formatspezifikationen nicht offen vorliegen und verschiedene Versionen oft nicht untereinander kompatibel sind. Ebenso verhält es sich mit den binären Exportformaten serverbasierter, relationaler DBMS, wie Oracles DMP-Format oder das komprimierte BAK-Format von PostgreSQL. Auch diese folgen keinem allgemeingültigen Standard, liegen nur als Binärdatei vor und sind teilweise proprietär. Eine spätere Nachnutzung ist eng an das jeweilige für alle Nutzer kostenpflichtige Produkt sowie spezielle Tools gebunden.

ODB ist ein Container-Dateiformat des Programms, welche in LibreOffice und OpenOffice enthalten ist. ODB ist ein Teil vom OpenDocument Format (ODF) und wurde von einem technischen Komitee unter der Leitung der Organization for the Advancement of Structured Information Standards (OASIS) entwickelt und unter ISO/IEC 26300 standardisiert. Der ZIP-Container enthält verschiedene Dateien, die zum Betrieb der Datenbank erforderlich sind, wie persönliche Konfigurationsdateien, eine binäre Datei des Datenbankmanagementsystems HSQLDB mit den  Datenbankinhalten sowie der Datenbankstruktur und die XML-basierten Formulare und Reports. Aufgrund der binären Datenstruktur kann ODB nicht für die Langzeitarchivierung empfohlen werden.

Keines der genannten und empfohlenen Formate unterstützt und enthält alle notwendigen Informationen und signifikanten Eigenschaften, die für die Archivierung und Nachnutzung einer Datenbank relevant sind. Und einige Informationen, etwa ein Handbuch zur Benutzung, müssen ohnehin gesondert gespeichert werden. Daher ist eine hybride Archivierungsstrategie erforderlich, welche verschiedene Dateien mit unterschiedlichen Dateiformaten umfasst. In der folgenden Tabelle werden die Kategorien von Informationen, welche von den für die Archivierung geeigneten Datenbankformate prinzipiell gespeichert werden können, durch Abkürzungen kenntlich gemacht: I = Datenbankinhalt, S = Datenbankstruktur, F = Funktionalitäten, B = Benutzung. Eine zusätzliche Übersicht ist in dem Abschnitt Vertiefung zu finden.

\begin{center}
	\begin{longtable}{l L{0.2\textwidth} p{0.6\textwidth}}
			\toprule 
		\multicolumn{2}{l}{Format} & Begründung \\
		\midrule \endfirsthead
		\multicolumn{3}{l}{\footnotesize Fortsetzung der vorhergehenden Seite}\\
		\toprule
		\multicolumn{2}{l}{Format} & Begründung \\ \midrule \endhead
		\bottomrule \multicolumn{3}{r}{{\footnotesize Fortsetzung auf der nächsten Seite}} \\
		\endfoot
		\bottomrule 
		\endlastfoot
		
		\multirow{3}{*}{\color{ForestGreen} \LARGE \checkmark} & SIARD, SIARD2 & SIARD ist offen, frei verfügbar und setzt auf die Standards SQL und XML. Der wesentliche Vorteil von SIARD liegt darin, dass sowohl die Datenbankstruktur als auch die Datenbankinhalte gemeinsam beschrieben und archiviert werden. Für die Archivierung sollte Version 2.0 verwendet werden. Speichert: I, S, F.\\
			& SQL & Geeignet für die Langzeitarchivierung bei Verwendung eines offiziellen ISO/IEC 9075 Standards, z.B. SQL:2008. Daten, Datenbankstruktur und ein Großteil der Funktionalität bleiben erhalten. Der verwendete Standard muss dokumentiert sein. Speichert: I, S, F.\\
		  & XML & XML bietet sich insbesondere zur Speicherung der Daten einer Datenbank an. Die Datenbankstruktur sollte mit Hilfe eines XSD-Schemas, wie DBML, beschrieben werden. Alternativ ist dies auch mithilfe einer SQL-Datei oder SIARD möglich. Speichert I und teilweise S.\\ \cmidrule(r){1-3}
		\multirow{2}{*}{$\color{BurntOrange} \thicksim$} & CSV & Das CSV Format ist ein langlebiges, textbasiertes und weit verbreitetes Format. Bei dem Export von Datenbankinhalten in das CSV-Format gehen allerdings Informationen zur Datenbankstruktur und zu Funktionalitäten verloren, weshalb es nur bei einer gleichzeitigen Speicherung der Datenbankstruktur in einem anderen Format (z. B. SIARD, SQL, XML/DBML) zu empfehlen ist. Speichert: I.\\ 
			& JSON & JSON bietet sich als Format zur strukturierten Speicherung von Datenbankinhalten an. Da es keine Standards zur Beschreibung der Datenbankstruktur im JSON Format gibt, muss diese gesondert in einem geeigneten Format (z. B. SIARD, SQL, XML/DBML) gespeichert werden. JSON bietet sich insbesondere für die Archivierung von Daten aus NoSQL Datenbanken an. Speichert: I.\\ \cmidrule(r){1-3}
		\multirow{5}{*}{\LARGE \boldmath$\color{BrickRed} \times$}& MDB, ACCDB & Binäre, proprietäre Formate von Microsoft Access, die nicht zur Archivierung geeignet sind.\\
		& FP5, FP7, FMP12 & Binäre, proprietäre Formate von FileMaker, die nicht zur Archivierung geeignet sind.\\
		& ODB & Containerformat von OpenOffice, welches verschiedene XML-basierte Dateien, wie z.B. Formulare und Abfragen, und die binären Dateien mit den Inhalten und der Struktur der Datenbank enthält. ODB ist nicht zur Archivierung geeignet. Speichert: teilweise B.\\
		& DBF & Binäres, proprietäres Format von dBASE, das nicht zur Archivierung geeignet ist.\\
		& BAK, DB, DMP & Die verschiedenen, binären Exportformate von relationalen Datenbanksystemen sind für die Archivierung nicht geeignet.\\
		\bottomrule    
	\end{longtable}
\end{center}


\subparagraph{Dokumentation}
Die Empfehlungen zur Dokumentation von Datenbanken orientieren sich an verschiedenen Standards, darunter SIARD Metadata Schema, Archival Data Description Mark-up Language (ADDML) und Database Markup Language (DBML). Allen Schemata gemein ist eine differenzierte Beschreibung der verschiedenen Ebenen eines Datenbanksystems. Dazu gehören Metadaten zur allgemeinen Beschreibung der Datenbank als auch spezifische Metadaten zur Beschreibung der hierarchisch aufgebauten Ebenen einer Datenbank: den in SQL Datenbanken vorkommenden Schemen, die mehrere Tabellen beinhalten, welche wiederum mit Hilfe von mehreren Attributen definiert werden können. 

Weitergehende Dokumente, die zum Verständnis einer Datenbank, ihrer spezifischen Nutzung oder zur Eigenart der Inhalte notwendig sind, wie beispielsweise Benutzerhandbuch, Bildschirmfotos, ERD-Diagramme, Beschreibungen semantischer Konventionen (wie Wertelisten), Anforderungsanalysen oder Rechte- und Rollendokumentation sollten auf alle Fälle mit archiviert werden.

Ergänzend zu den Metadaten, wie sie in dem Abschnitt Metadaten in der Anwendung ab Seite \pageref{Metadaten-anwendung} gelistet sind, sollten folgende minimale Metadaten zur nachhaltigen Beschreibung von Datenbanken aufgenommen werden.

\begin{center}
	\begin{longtable}{L{0.3\textwidth} p{0.6\textwidth}}
			\toprule 
		Metadatum & Beschreibung \\
		\midrule \endfirsthead
		\multicolumn{2}{l}{\footnotesize Fortsetzung der vorhergehenden Seite}\\
		\toprule
		Metadatum & Beschreibung \\ \midrule \endhead
		\bottomrule \multicolumn{2}{r}{{\footnotesize Fortsetzung auf der nächsten Seite}} \\
		\endfoot
		\bottomrule 
		\endlastfoot
		
		Datenbankname & Ansprache/Name der Datenbank\\
		Beschreibung der Datenbank & Zusammenfassung über Zweck, Bedeutung und Inhalt der Datenbank\\
		Sprache & Liste der Sprachen, in welchen die Daten eingegeben wurden sowie Sprachkennungen nach ISO 639 angeben.\\
		Identifikator & Falls die Datenbank online öffentlich zugänglich ist, sollte ein persistenter Identifikator (z.B. DOI, URI) oder eine eindeutige Adresse (z.B. URL) angegeben werden.\\
		Rechteinhaber & Personen, Institutionen oder Unternehmen, die Rechte an der Datenbankstruktur und/oder den Datenbankinhalten besitzen.\\
		DBMS & Name des Datenbankmanagementsystems mit Versionsnummer mit der die Datenbank betrieben wurde.\\
		Schemata Liste & Liste der Schemata innerhalb der Datenbank sofern vorhanden.\\
		Benutzer & Liste der Benutzer und ihrer Rolle innerhalb der Datenbank.\\
		Rolle & Liste der verschiedenen Rollen und Gruppen mit ihren definierten Zugriffsrechten.\\
		Standard & Name und Version verwendeter Standards, etwa zur Definition des Datentyps eines Attributs, zur Dokumentation von Abfragen oder zu dem Datenbankformat, z.B. SQL:2008 oder SIARD 2.0.\\
		Abgeleitete Dateien & Aus den Datenbankinhalten erzeugte Grafiken, Abbildungen, Diagramme müssen zusätzlich separat archiviert und gelistet werden.\\
		Weitere Dateien & Liste weiterer Dokumente, die für das Verständnis und die Dokumentation einer Datenbank notwendig sind, wie beispielsweise ein ERD oder ein Benutzerhandbuch.\\
		\midrule
		Schema Metadatum & Beschreibung\\
		Schema Name & Name des Schemas.\\
		Schema Beschreibung & Bedeutung und Inhalt des Schemas.\\
		Tabellenliste & Liste der Tabellen im Schema.\\
		Funktion & Name und Art der Funktion (Gespeicherte Funktionen, Sichten).\\
		Funktion Beschreibung & Beschreibung der Funktion hinsichtlich Sinn und Zweck.\\
		Funktion Befehl & Befehle der Funktion nach einem einheitlich festgelegten Standard (Syntax Standard).\\
		\midrule
		Tabelle Metadatum & Beschreibung\\
		Tabelle Name & Name der Tabelle.\\
		Tabelle Beschreibung & Bedeutung und Inhalt der Tabelle.\\
		Anzahl Datensätze & Anzahl der Datensätze in der Tabelle.\\
		Attributliste & Liste der Attribute in der Tabelle.\\
		\midrule
		Attribut Metadatum & Beschreibung\\
		Attribut Name & Name des Attributs.\\
		Attribut Beschreibung & Bedeutung und Inhalt des Attributs. Wird das Attribut mit Hilfe von Einheiten, z. B. metrische Maßeinheiten, ausgedrückt, ist die Angabe der Einheit erforderlich.\\
		Attribut Typ & Datentyp des Attributs nach einem einheitlich festgelegtem Standard (Syntax Standard) z. B. "`Integer"' nach SQL:2008.\\
		Kontrolliertes Vokabular & Sofern für das Attribut eine Werteliste vorliegt oder ein Thesaurus verwendet wurde, ist dieses zu dokumentieren.\\
		Attribut Schlüssel & Sofern es sich um ein Schlüsselattribut handelt, Benennung der Schlüsselart z. B. Primär- oder Fremdschlüssel. \\
		Fremdschlüssel Referenz & Im Falle eines Fremdschlüssels Angabe des referenzierten Attributs mit Angabe der Tabelle und des Schemas. \\

	  \bottomrule
	\end{longtable}
\end{center}
	\begin{flushleft}
		Das vollständige Kapitel finden Sie auf: \urllist{https://ianus-fdz.de/it-empfehlungen/datenbanken}
	\end{flushleft}

%\newpage
%\section{GIS}\label{GIS}
%\abschnittsautor{K. Rassman}
%Bitte beachten Sie, dass die Inhalte dieses Abschnittes den Inhalten des IT-Leitfadens des DAIs von 2011 entsprechen. Eine aktuellere Empfehlung können Sie den Guides to Good Practice des ADS entnehmen: \urllist{http://guides.archaeologydataservice.ac.uk/g2gp/Gis_Toc}
%\begin{center}
%\tib{\rule{0.9\textwidth}{0.2mm}}\vspace{3mm}
%\end{center}

%Für die Langzeitarchivierung von GIS-Daten ist derzeit noch kein allgemeingültiger Standard etabliert, weswegen auf die Vorgaben zu den in das GIS importierten Daten (z.B. Luftbilder, Geländemodell etc.) verwiesen wird. Es sollen jedoch möglichst keine programminternen Formate, die nicht von anderen GIS Systemen importiert werden können, genutzt werden. Bevorzugt werden sollen:
%\begin{itemize}
	%\item Vektordaten als Esri Shapefile (shp + shx + dbf)
	%\item Rasterdaten als GeoTiff
%\end{itemize}
%
%Die Haltung von Geodaten sollte vorzugsweise in Geodatenbanken erfolgen. Falls dies nicht möglich ist, müssen die Metadaten ausgefüllt werden. Folgende Informationen sind in der Geodatenbank / den Metadaten bereitzustellen:
%\begin{itemize}
	%\item Datenquelle (z.B. SRTM, Tachymeter)
	%\item Verarbeitung: Programm, Algorithmus, Parameter (z.B. QGIS , r.shaded.relief, altitude, azimuth etc.)
	%\item Projektion (WGS 1984, EPSG: 4326)
	%\item Datenqualität (Auflösung 90m, Ungenauigkeit)
	%\item Information, was mit den Layern ausgedrückt wird, da es nicht immer aus dem Dateinamen hervorgeht (z.B. Höhenmodell)
%\end{itemize}

\newpage
\section{Video}\label{video}
\abschnittsautor{M. Trognitz; Mit Unterstützung von: H. Lewetz}
	\hyphenation{
Au-dio-stream
Bit-ra-te
FFmpeg
}
Videos sind in der altertumswissenschaftlichen Forschung vergleichsweise wenig verbreitet und dienen vor allem der Dokumentation von Arbeitsabläufen, Visualisierung von Ergebnissen oder publikumswirksamen Medienauftritten.

Eine digitale Videodatei enthält sowohl visuelle als auch auditive Inhalte. Diese sind jeweils in einem eigenen Format, dem sogenannten Codec, gespeichert und werden in einem Containerformat zusammengeführt. Je nachdem welches Containerformat verwendet wird, können weitere Informationen, wie Metadaten oder Untertitel gespeichert werden. Zu beachten ist, dass es sich bei manchen Containerformaten gleichzeitig um Codecs handeln kann. Die Digitalisierung von analogem Film- und Videomaterial ist nicht Gegenstand dieser Empfehlungen.

\subparagraph{Langzeitformate} Bei der Auswahl für die Langzeitarchivierung von digitalen Videos muss nicht nur ein passendes Format, sondern auch ein geeigneter Codec gefunden werden. Dabei gelten für die Auswahl des Codecs und des Containerformats die Kriterien, die für die Wahl von Dateiformaten ab Seite \pageref{dateiformate} beschrieben werden: Es sollte sich um einen weit verbreiteten, möglichst nicht proprietären und offen dokumentierten Standard handeln, der verlustfreie oder gar keine Kompression anwendet. Außerdem sollte darauf geachtet werden, dass das gewählte Format die Speicherung aller relevanten Elemente des Videos, wie beispielsweise Untertitel, unterstützt.

Das Containerformat Matroska (MKV) erfüllt alle Anforderungen an ein Archivformat. Es wird seit 2003 explizit als offenes Containerformat entwickelt, das modernen Ansprüchen genügt und viele verschiedene Codecs enthalten sowie zusätzliche Elemente, wie Untertitel, speichern kann. Das Format basiert auf einer binären Variante von XML, nämlich EBML (Extensible Binary Meta Language), was eine zukünftige flexible Erweiterung erlaubt, jedoch auch sicherstellt, dass ältere Programme weiterhin damit umgehen können. Zusätzlich ist das Format fehlertolerant und kann bis zu einem gewissen Grad auch beschädigte Dateien wiedergeben. Zusammen mit den Codecs FFV1 für Video und FLAC für Audio wird eine Standardisierung bei der IETF angestrebt, weshalb diese Kombination empfohlen wird.

Motion JPEG 2000 (MJ2) ist ein speziell für die Archivierung entwickeltes Containerformat, das unter ISO/IEC 15444-3:2007 zertifiziert und als MIME-Type video/mj2 registriert ist. Es verwendet einen eigenen Codec, JPEG 2000, der jedes Einzelbild verlustfrei mit JPEG 2000 komprimiert. Zugehörige auditive Inhalte werden von Motion JPEG 2000 ebenfalls unterstützt. Das Format ist rechenintensiv und nicht vollständig lizenzfrei, weshalb es besser sein kann den JPEG 2000 Codec in einem anderen Containerformat, wie beispielsweise MXF zu verwenden. Außerdem ist zu Beachten, dass MJ2 auch verlustbehaftete Varianten kennt, die für die Archivierung nicht verwendet werden sollten. Motion JPEG 2000 darf nicht mit dem MPEG-Containerformat oder dem verlustbehafteten Motion JPEG (M-JPEG) verwechselt werden.

Die Familie der MPEG-Containerformate wird von der Moving Picture Experts Group (MPEG) entwickelt und ISO/IEC zertifiziert. Es gibt mehrere Generationen von Standards, darunter MPEG-1 (MPG), MPEG-2 (MPG) oder MPEG-4 (MP4), die entsprechend unterschiedliche Eigenschaften aufweisen. Die Standards beschreiben nicht nur Containerformate, sondern auch Codecs für Video und Audio. MPEG-1 ist seit 1991 unter ISO/IEC 11172 zertifiziert und beschreibt einen Standard für verlustbehaftete Komprimierung von Audio- und Videodaten, der eine gute Qualität und weite Verbreitung hat. MPEG-1 ist insbesondere für das Audioformat MP3 bekannt, eignet sich aber nicht für die Langzeitarchivierung. Der ebenfalls verlustbehaftete Nachfolger, MPEG-2, ist unter ISO/IEC 13818 zertifiziert und hat eine weite Verbreitung in dem Containerformat MPEG-TS (TS) für Rundfunkübertragungen oder auf DVDs gefunden. MPEG-2 kann nur als Langzeitformat für Dateien verwendet werden, die ursprünglich in diesem Format entstanden sind und nicht mehr bearbeitet werden. 

Der unter ISO/IEC 14496 zertifizierte MPEG-4-Standard weist eine höhere Effizienz in der Komprimierung und eine höhere Qualität der Videodaten als seine Vorgänger auf. Er verwendet den Codec H.264/MPEG-4 AVC, der meistens verlustbehaftet verwendet wird, jedoch auch eine kaum verwendete verlustfreie Variante hat. Dieser Codec bietet eine hohe Bildqualität, weshalb er eine weite Verbreitung auf Blu-ray Discs, als Aufnahmeformat oder für das Streaming über das Internet gefunden hat. Teil 14 der MPEG-4-Spezifikation definiert das Containerformat MP4, das auch als MIME-Type video/mp4 registriert ist. Wird dieses Format mit dem H.264-Codec verwendet, kann es zur Langzeitarchivierung verwendet werden, wenn dies entweder dem Ursprungsformat der Videodatei entspricht oder verlustfreie Kompression verwendet wird.

Das Containerformat Material eXchange Format (MXF) ist ein offenes Format, das für den Austausch von auditiven und visuellen Inhalten entwickelt wurde und durch eine Reihe von Standards der Society of Motion Picture and Television Engineers (SMPTE) beschrieben wird. MXF wird vor allem im Bereich Kino und Fernsehen verwendet, wo jeweils spezialisierte Varianten verwendet werden, die nicht immer vollständig untereinander kompatibel sind. MXF ist als MIME-Type application/mxf registriert. Zusammen mit der verlustfreien Variante des Codecs JPEG 2000 stellt MXF eine für die Langzeitarchivierung geeignete Variante dar.

Es gibt eine große Zahl an weiteren Containerformaten, wie etwa MOV und ASF/WMV. Sie eignen sich aber nicht für die langfristige Videoarchivierung, da es sich dabei um proprietäre Formate handelt. Bei Audio Video Interleave (AVI) handelt es sich ebenfalls um ein proprietäres Format, es wird jedoch von einigen Archiven als Archivformat verwendet, da es sich um ein einfaches und robustes Format mit einer großen Verbreitung handelt. Die Formate Ogg und Flash wurden vor allem für das Streaming von Videos entwickelt und sind aufgrund der verwendeten verlustbehafteten Codecs nicht als Langzeitformate geeignet.

Die Wahl des Containerformates hängt von dem zu verwendenden Codec ab, da nicht jeder Codec in jedem Container verwendet werden kann. Bekannte und verbreitete Codecs werden nachfolgend beschrieben. 

Der Nachteil bei unkomprimierten Videodateien ist, dass sie sehr viel Speicherplatz (teilweise mehrere GB pro Filmminute) beanspruchen und auch für die Wiedergabe entsprechend leistungsfähige Hardware benötigen. Eine Alternative ist die Verwendung von Codecs mit verlustfreier Kompression, wie etwa FFV1, oder HuffYUV und Lagarith. Der FFmpeg Video Codec 1 (FFV1) wurde im Rahmen des FFmpeg-Projektes entwickelt, ist offen dokumentiert, hat eine weite Verbreitung gefunden und komprimiert sehr gut und schnell. Diese Eigenschaften machen ihn zu einem empfehlenswerten Codec für die Langzeitarchivierung. HuffYUV und dessen Abspaltung Lagarith wurden eigens für Windows-Systeme entwickelt, weshalb diese nicht für die langfristige Archivierung empfohlen werden können. 

Einige Codecs können sowohl verlustfrei als auch verlustbehaftet komprimieren. Dazu gehören: Dirac/Schroedinger, JPEG 2000, H.264/MPEG-4 AVC und H.265/MPEG-H (HEVC). Dirac/Schroedinger wurde beim BBC entwickelt, ist jedoch nicht sehr performant und wird von nur wenigen Programmen unterstützt, weshalb er aktuell nicht für die Archivierung empfohlen werden kann. Bei der Verwendung von JPEG 2000 muss darauf geachtet werden, dass die verlustfrei komprimierende Variante verwendet wird, da dieser Standard auch verlustbehaftet komprimieren kann. H.264/MPEG-4 AVC kann in einer verlustfreien Variante verwendet werden, wenn das entsprechende Programm dies unterstützt. Die verlustbehaftete Variante von H.264/MPEG-4 AVC kann nur für die Archivierung von Dateien verwendet werden, die in diesem Codec entstanden sind und nicht mehr bearbeitet werden. Seit 2013 gibt es den Codec High Efficiency Video Coding (HEVC, auch H.265/MPEG-H), der als Nachfolger für H.264 gedacht ist, jedoch noch nicht vollständig spezifiziert ist. Daher und aus patentrechtlichen Gründen ist dieser Codec nicht für die Archivierung zu empfehlen.

Neben den oben erwähnten Codecs gibt es weitere, jedoch verlustbehaftete Codecs, wie M-JPEG, Theora und DV. Im Gegensatz zu JPEG 2000 ist der Codec Motion JPEG (M-JPEG) nicht für die langfristige Archivierung von Videodaten geeignet, da dieser keine verlustfreie Komprimierung unterstützt. Theora ist der für Ogg-Dateien entwickelte Codec. DV ist ein Codec für digitale Videos auf Videokasetten, der mit der Ablösung von Kasetten durch andere Speichermedien obsolet wird. Theora und DV sind keine geeigneten Codecs für die Langzeitarchivierung.

Hinweise auf geeignete Audioformate und dazugehöriger Metadaten sind in dem Kapitel Audio ab Seite \pageref{audio} zu finden.

\pagebreak
\begin{center}
	\begin{longtable}{l L{0.2\textwidth} p{0.6\textwidth}}
			\toprule 
		\multicolumn{2}{l}{Format} & Begründung \\
		\midrule \endfirsthead
		\multicolumn{3}{l}{\footnotesize Fortsetzung der vorhergehenden Seite}\\
		\toprule
		\multicolumn{2}{l}{Format} & Begründung \\ \midrule \endhead
		\bottomrule \multicolumn{3}{r}{{\footnotesize Fortsetzung auf der nächsten Seite}} \\
		\endfoot
		\bottomrule 
		\endlastfoot
		
		{\color{ForestGreen} \LARGE \checkmark} & Matroska (MKV) & Ein offenes Containerformat, das eine große Bandbreite von Codecs und ergänzenden Inhalten unterstützt. Für die Archivierung können die Codecs FFV1 für Video und FLAC für Audio empfohlen werden. Weitere geeignete Codecs für Matroska sind H.264/MPEG-4 AVC und MPEG-2. \\ \cmidrule(r){1-3}
		\multirow{4}{*}{$\color{BurntOrange} \thicksim$} & Motion JPEG 2000 (MJ2) & Motion JPEG 2000 verwendet den Codec JPEG 2000 und ist unter ISO/IEC 15444-3:2007 zertifiziert. \\
			& MP4 & Der unter ISO/IEC 14496 zertifizierte MPEG-4-Standard verwendet den Codec H.264/MPEG-4 AVC, der meistens verlustbehaftet verwendet wird, jedoch auch verlustfrei komprimieren kann. Das Containerformat MP4 ist als MIME-Type video/mp4 registriert. Wird dieses Format mit dem H.264-Codec verwendet, kann es zur Langzeitarchivierung verwendet werden, wenn dies entweder dem Ursprungsformat der Videodatei entspricht oder verlustfreie Kompression verwendet wird.\\ 
			& MXF & Das Containerformat Material eXchange Format (MXF) ist ein offenes Format, das durch eine Reihe von Standards der SMPTE beschrieben wird. MXF ist als MIME-Type application/mxf registriert. Zusammen mit der verlustfreien Variante des Codecs JPEG 2000 stellt MXF eine für die Langzeitarchivierung geeignete Variante dar.\\ 
			& MPEG-2 & MPEG-2 ist unter ISO/IEC 13818 zertifiziert und komprimiert verlustbehaftet. MPEG-2 kann nur als Langzeitformat für Dateien verwendet werden, die ursprünglich in diesem Format entstanden sind und nicht mehr bearbeitet werden. \\ \cmidrule(r){1-3}
		\multirow{6}{*}{\LARGE \boldmath$\color{BrickRed} \times$}& MPEG (weitere Varianten) & Mit Ausnahme von MPEG-4 und MPEG-2 können andere MPEG-Varianten nicht für die Langzeitarchivierung empfohlen werden.\\
			& AVI & Audio Video Interleave ist ein von Microsoft entwickeltes proprietäres Format mit einer weiten Verbreitung. Es eignet sich nicht für die Langzeitarchivierung.\\
		  & MOV & Ein verbreitetes proprietäres Format von Apple, das in QuickTime verwendet wird und nicht für die Langzeitarchivierung geeignet ist.\\ 
			& ASF/WMV & Ein proprietäres Format von Microsoft, das nicht für die Langzeitarchivierung geeignet ist.\\ 
			& Ogg & Ein von Xiph entwickeltes und offenes Format für das Streaming von Videos, das jedoch nicht für die Archivierung geeignet ist.\\
			& Flash & Ein von Macromedia und dann von Adobe entwickeltes Format für das Streaming von Videos. Es ist nicht für die Langzeitarchivierung geeignet. \\
 		\bottomrule    
	\end{longtable}
\end{center}


\subparagraph{Dokumentation} Neben den allgemeinen minimalen Angaben zu Einzeldateien, wie sie in dem Abschnitt Metadaten in der Anwendung ab Seite \pageref{Metadaten-anwendung} gelistet sind, werden für Videos weitere Angaben benötigt, die insbesondere technische Details dokumentieren.

Die technischen Angaben zu Bildformat, Seitenverhältnis, Bildfrequenz, Bitrate und Codec werden zur korrekten Wiedergabe der Datei benötigt. Angaben zu Länge, Tonkanälen, Profilen und weiteren Inhalten sind zur Prüfung auf Vollständigkeit der Datei erforderlich.

Bereits eingebettete Metadaten, wie beispielsweise Exif, oder Bestandteile im Containerformat sollten behalten und archiviert werden. Am besten werden sie in eine eigene Text- oder XML-Datei transferiert und getrennt gespeichert.

\begin{center}
	\begin{longtable}{L{0.3\textwidth} p{0.6\textwidth}}
			\toprule 
		Metadatum & Beschreibung \\
		\midrule \endfirsthead
		\multicolumn{2}{l}{\footnotesize Fortsetzung der vorhergehenden Seite}\\
		\toprule
		Metadatum & Beschreibung \\ \midrule \endhead
		\bottomrule \multicolumn{2}{r}{{\footnotesize Fortsetzung auf der nächsten Seite}} \\
		\endfoot
		\bottomrule 
		\endlastfoot

		Länge & Dauer des Videos. Diese Angaben sollten konform zu ISO 8601 erfolgen. Beispiel: P3Y6M4DT12H30M5S (3 Jahre, 6 Monate, 4 Tage, 12 Stunden, 30 Minuten und 5 Sekunden) oder T2H2M (2 Stunden und 2 Minuten)\\
		Beteiligte & Angabe der Beteiligten, wie etwa Autor, Regisseur, Darsteller, Interviewpartner etc.\\
		Bildgröße und Seitenverhältnis & Maße des Bildes gemessen in Pixeln, z.B. $1280px \times 720px$, und Angabe des Seitenverhältnisses für die korrekte Darstellung im Verhältnis Breite zu Höhe, z.B.: 16:9\\ 
		Bildfrequenz & Angabe der Bildfrequenz in Bildern pro Sekunde, z.B. 25\\
		Bitrate Video & Angabe der Datenrate in Bits pro Sekunde, z.B. 863 kbps\\
		Farbraum & Angabe des verwendeten Farbraums, z.B. YUV \\
		Farbtiefe & z.B. 8 bit oder 10 bit \\
		Farbunterabtastung & Angabe der verwendeten Farbunterabtastung, z.B. 4:2:0 \\
		Videocodec & Name und Version des verwendeten Videocodecs und welche Kompression verwendet wird\\
		Audiostreams & Anzahl der enthaltenen Audiostreams \\
		Audiocodec & Angaben zu den verwendeten Audiocodecs je enthaltenen Audiostreams\\
		Bitrate Audio &  Angabe der Datenrate in Bits pro Sekunde je enthaltenen Audiostreams, z.B. 666 kbps\\
		Abtastrate & Angabe der Abtastrate in Hertz je enthaltenen Audiostreams, z.B. 44.1 kHz\\
		Tonkanäle & Angabe über Anzahl der Tonkanäle je enthaltenen Audiostreams \\
		Profil & Wenn ein bestimmtes Videoprofil für ein Containerformat verwendet wurde, bitte angeben.\\
		Weitere Inhalte & Angabe über weitere Inhalte, die in dem Containerformat enthalten sind oder als zusätzliche Datei vorliegen, wie beispielsweise Untertitel\\
		Aufnahmegerät & Herstellername und Modell des Aufnahmegeräts (z.B. eine Kamera)\\
		Software & Name und Versionsnummer der Software, mit der das Video aufgenommen, erstellt oder bearbeitet wurde, wie z.B. Adobe Premiere Pro CC (2015.1)\\
	  \bottomrule
	\end{longtable}
\end{center}

Weitere Metadaten sind methodenabhängig und können in den jeweiligen Abschnitten nachgelesen werden.
	\begin{flushleft}
		Das vollständige Kapitel finden Sie auf: \urllist{https://ianus-fdz.de/it-empfehlungen/video}
	\end{flushleft}

\newpage
\section{Audio}\label{audio}
\abschnittsautor{M. Trognitz; Mit Unterstützung von: S. Rohde-Enslin, A. Romeyke}
	\hyphenation{
Au-dio-stream
Bit-ra-te
FFmpeg
WAVE
}
Tonaufnahmen werden in der altertumswissenschaftlichen Forschung eher selten erzeugt. Sie werden für die Aufnahme von Interviews, rekonstruierten Musikinstrumenten im Rahmen der Musikarchäologie oder akustischen Eigenschaften von archäologischen Orten erstellt. Sogar Hörbücher altertumswissenschaftlichen Inhalts existieren mittlerweile.

Eine Audiodatei enthält auditive Inhalte, die durch zusätzliche Komponenten wie Kapiteleinteilungen angereichert sein können. Die zusammenfassende Speicherung aller Inhalte erfolgt in einem Containerformat. Die auditiven Inhalte selbst werden nochmals in einem eigenen Format, dem sogenannten Codec, gespeichert. Je nachdem welches Containerformat verwendet wird, können weitere Informationen, wie Metadaten oder Transkriptionen gespeichert werden.


\subparagraph{Langzeitformate} Da es sich bei den Dateiformaten für digitale Audiodateien um Containerformate handelt, muss bei der Auswahl für die Langzeitarchivierung nicht nur ein passendes Format, sondern auch ein geeigneter Codec gefunden werden. Dabei gelten für die Auswahl des Codecs und des Containerformats die Kriterien, die für die Wahl von Dateiformaten ab Seite \pageref{dateiformate} beschrieben werden: Es sollte sich um einen weit verbreiteten, möglichst nicht proprietären und offen dokumentierten Standard handeln, der verlustfreie oder gar keine Kompression anwendet. Außerdem sollte darauf geachtet werden, dass das gewählte Format die Speicherung aller relevanten Elemente der Audiodatei, wie etwa Kapiteleinteilungen, unterstützt.

Da viele Containerformate für Audio nur ein Codec-Format speichern können, werden diese in der Regel unter einem Namen zusammengefasst.

Für die Speicherung von unkomprimierten Audiodaten hat sich lineares PCM (Lineare Puls-Code-Modulation, auch LPCM) als Standard durchgesetzt, das von verschiedenen Containerformaten unterstützt und auch von der IASA (International Association of Sound and Audiovisual Archives) empfohlen wird. Der einzige Nachteil ist der große Speicherplatzbedarf mit etwa 10 MB pro Minute.

Eine verlustfrei komprimierende Alternative für die Langzeitarchivierung stellt der Free Lossless Audio Codec (FLAC) dar, der offen dokumentiert und frei verfügbar ist. FLAC wird beispielsweise von dem Containerformat Matroska unterstützt oder kann als eigenständiges Format verwendet werden. 

Das Waveform Audio File Format (WAVE) wurde von Microsoft und IBM als Teil des Resource Interchange File Format (RIFF) entwickelt. Es ist ein offen dokumentiertes aber proprietäres Format, das mehrere Audio-Codecs unterstützt. Aufgrund der weiten Verbreitung wird dieses Format jedoch zusammen mit linearem PCM von der IASA empfohlen und hat sich als Standard de facto durchgesetzt.

Die European Broadcast Union (EBU) hat auf WAVE aufbauend das Format BWF entwickelt, das zusätzlich die Einbettung von Metadaten unterstützt. Für die Langzeitarchivierung sollte dieses Format mit linearem PCM verwendet werden. Für Dateien, die größer als 4 GB sind, hat die EBU das Format RF64/MBWF entwickelt, das zusätzlich die Speicherung von mehr als zwei Tonkanälen erlaubt.

Das Containerformat Matroska (MKA) wird seit 2003 explizit als offenes Containerformat entwickelt, das modernen Ansprüchen genügt und viele verschiedene Codecs enthalten sowie zusätzliche Elemente, wie Kapiteleinteilungen, speichern kann. Das Format basiert auf einer binären Variante von XML, nämlich EBML (Extensible Binary Meta Language), was eine zukünftige flexible Erweiterung erlaubt, jedoch auch sicher stellt, dass ältere Programme weiterhin damit umgehen können. Zusätzlich ist das Format fehlertolerant und kann bis zu einem gewissen Grad auch beschädigte Dateien wiedergeben. Wenn die Audiodaten in linearem PCM oder FLAC gespeichert werden, kann Matroska für die Langzeitarchivierung empfohlen werden.

Aus der Familie der MPEG-Containerformate, die von der Moving Picture Experts Group (MPEG) entwickelt werden und ISO/IEC zertifiziert sind, stammen das MP3- und das AAC-Format. MP3 (MPEG-1 Audio Layer III) wurde schon in MPEG-1 spezifiziert, das seit 1991 unter ISO/IEC 11172 zertifiziert ist. Es handelt sich dabei um einen Audiocodec mit verlustbehafteter Komprimierung, der jedoch eine weite Verbreitung gefunden hat und in einem gleichnamigen Containerformat gespeichert wird.

Advanced Audio Coding (AAC) ist ein Audiocodec, der als Nachfolger von MP3 im Rahmen von MPEG-2 und MPEG-4 spezifiziert wurde und unter ISO/IEC 13818-7 und 14496-3 standardisiert ist. Auch AAC komprimiert verlustbehaftet und kann in einem gleichnamigen Container gespeichert werden oder beispielsweise auch in dem Containerformat MP4. MP3 und AAC sollten nur dann als Langzeitformat für Dateien verwendet werden, wenn diese ursprünglich in dem Format entstanden sind.

Das von Apple entwickelte Audio Interchange File Format (AIFF) ist nicht für die Langzeitarchivierung geeignet, weil es proprietär ist und hauptsächlich nur auf Apple-Systemen Verbreitung gefunden hat. Das von Microsoft entwickelte Windows Media Audio (WMA) ist ebenfalls ein proprietäres Format, das sich wegen der verwendeten verlustbehafteten Kompression nicht für die Langzeitarchivierung eignet. Das Format Ogg wurde vor allem für das Streaming entwickelt und ist aufgrund der verwendeten verlustbehafteten Codecs Opus und Vorbis nicht als Langzeitformat geeignet.
		

\begin{center}
	\begin{tabular}{l L{0.2\textwidth} p{0.6\textwidth}}
		\toprule
		\multicolumn{2}{l}{Format} & Begründung \\ \midrule
		\multirow{3}{*}{\color{ForestGreen} \LARGE \checkmark} & FLAC & Free Lossless Audio Codec ist ein verlustfrei komprimierender Codec, der offen dokumentiert und frei verfügbar ist.\\
			& WAVE & Waveform Audio File Format wurde von Microsoft und IBM entwickelt und ist offen dokumentiert aber proprietär. Die Audiodaten sollten als lineares PCM gespeichert werden.\\
			& BWF & Das Format BWF baut auf WAVE auf und wurde von der EBU entwickelt, um zusätzlich die Einbettung von Metadaten zu unterstützen. Die Audiodaten sollten als lineares PCM gespeichert werden.\\ 
		\cmidrule(r){1-3}
		\multirow{4}{*}{$\color{BurntOrange} \thicksim$} & Matroska & Ein offenes Containerformat, das eine große Bandbreite von Codecs und ergänzenden Inhalten unterstützt. Die Audiodaten sollten als lineares PCM oder FLAC gespeichert werden.\\
			& RF64/MBWF & Wurde von der EBU aus BWF entwickelt, um Dateien zu speichern, die größer als 4 GB sind oder mehr als zwei Tonkanäle beinhalten. Die Audiodaten sollten als lineares PCM gespeichert werden.\\
			& AAC/MP4 & Advanced Audio Coding ist der Nachfolger von MP3 und unter ISO/IEC 13818-7 und 14496-3 standardisiert. Er kann unter anderem in den Formaten AAC oder MP4 gespeichert werden. Die Daten werden verlustbehaftet komprimiert, weshalb er nur für Dateien verwendet werden darf, die ursprünglich in diesem Format entstanden sind.\\
			& MP3 & MPEG-1 Audio Layer III ist unter ISO/IEC 11172 zertifiziert und verwendet verlustbehaftete Kompression. MP3 kann nur als Langzeitformat für Dateien verwendet werden, die ursprünglich in diesem Format entstanden sind.\\
		\cmidrule(r){1-3}
		\multirow{3}{*}{\LARGE \boldmath$\color{BrickRed} \times$} & AIFF & Das Audio Interchange File Format (AIFF) von Apple ist nicht für die Langzeitarchivierung geeignet, weil es proprietär ist. \\
			& WMA & Windows Media Audio ist ein von Microsoft entwickeltes Format mit verlustbehafteter Kompression.\\
			& Ogg & Ein von Xiph entwickeltes und offenes Format, das jedoch nicht für die Archivierung geeignet ist, da die Codecs Opus und Vorbis verlustbehaftet komprimieren.\\
 		\bottomrule
		\bottomrule
	\end{tabular}
\end{center}


\subparagraph{Dokumentation} Neben den allgemeinen minimalen Angaben zu Einzeldateien, wie sie in dem Abschnitt Metadaten in der Anwendung ab Seite \pageref{Metadaten-anwendung} gelistet sind, werden für Audiodateien weitere Angaben benötigt, die insbesondere technische Details dokumentieren. 

Die technischen Angaben zu Codec, Bitrate, Abtastrate und Abtasttiefe werden zur korrekten Wiedergabe der Datei benötigt und vermitteln einen ersten Eindruck über die Qualität der Datei. Angaben zu Länge, Tonkanälen und weiteren Inhalten sind zur Prüfung auf Vollständigkeit der Datei erforderlich.

Bereits eingebettete Metadaten, wie beispielsweise Exif, oder Bestandteile im Containerformat sollten behalten und archiviert werden. Am besten werden sie in eine eigene Text- oder XML-Datei transferiert und getrennt gespeichert.

\begin{center}
	\begin{tabular}{L{0.3\textwidth} p{0.6\textwidth}} 
		\toprule
		Metadatum & Beschreibung \\
		\midrule
		Beteiligte & Angabe der Beteiligten, wie etwa Autor, Komponist, Interpret, Interviewpartner etc.\\
		Länge & Dauer der Audiodatei. Diese Angaben sollten konform zu ISO 8601 erfolgen. Beispiel: P3Y6M4DT12H30M5S (3 Jahre, 6 Monate, 4 Tage, 12 Stunden, 30 Minuten und 5 Sekunden) oder T2H2M (2 Stunden und 2 Minuten)\\
		Audiocodec & Angabe des verwendeten Audiocodecs\\
		Tonkanäle & Angabe der Anzahl der Tonspuren und Benennung des Systems, z.B. 5 (Dolby 5.1)\\
		Bitrate & Angabe der Datenrate in Bits pro Sekunde, z.B. 666 kbps\\
		Abtastrate & Angabe der Abtastrate in Hertz, z.B. 44.1 kHz \\
		Abtasttiefe & Angabe der Anzahl der Quantisierungsstufen als Bittiefe, z.B. 16 bit\\
		Weitere Inhalte & Angabe über weitere Inhalte die in dem Containerformat enthalten sind oder als zusätzliche Datei vorliegen, wie beispielsweise Transkriptionen\\
		Aufnahmegerät & Herstellername und Modell des Aufnahmegeräts (z.B. eines Analog-Digital-Umsetzers oder einer Kamera)\\
		Software & Name und Versionsnummer der Software mit der die Audiodatei aufgenommen, erstellt oder bearbeitet wurde, wie z.B. Audacity 2.1.2\\
	  \bottomrule
		\bottomrule
	\end{tabular}
\end{center}

Weitere Metadaten sind methodenabhängig und können in den jeweiligen Abschnitten nachgelesen werden.
	\begin{flushleft}
		Das vollständige Kapitel finden Sie auf: \urllist{https://ianus-fdz.de/it-empfehlungen/audio}
	\end{flushleft}

\newpage
\section{3D und Virtual Reality}\label{3D}
\abschnittsautor{M. Trognitz; Mit Unterstützung von: D. Lengyel, K. Niven, V. Gilissen}
	Im Gegensatz zu statischen zweidimensionalen Bildern können dreidimensionale Repräsentationen von Objekten aus jeder Richtung betrachtet, skaliert und rotiert werden. Ein Punkt in einem 3D-Modell wird von seiner Lage auf der x-, der y- und der z-Achse eines kartesischen Koordinatensystems beschrieben, wobei die z-Achse in diesem Zusammenhang üblicherweise die Tiefe, seltener die Höhe, angibt.

Virtual Reality (Virtuelle Realität) bezeichnet digitale dreidimensionale Welten, mit welchen in Echtzeit interagiert werden kann.

3D-Inhalte können auf unterschiedliche Weise entstehen: durch manuelle Modellierung, wie Rekonstruktionen von Gebäuden, durch Aufnahme, wie etwa einem 3D-Scan von Objekten, oder durch automatisierte Berechnung aus Fotos, wie etwa Photogrammetrie oder Structure from Motion. Von der Entstehungsweise hängen weitere Angaben für die Dokumentation ab, die über die hier angegebenen hinausgehen. Zusätzlichen Angaben sind in den jeweiligen Abschnitten in dem Kapitel Forschungsmethoden ab Seite \pageref{methoden} zu finden, wobei vor allem die Abschnitte Bauforschung, Geodäsie, Geodatenanalyse und Materialaufnahme von Interesse sind. Außerdem bieten die Ergebnisse des Projektes 3D ICONS umfangreiche Informationen zur Dokumentation von 3D-Aufnahmemethoden und die anschließende Verarbeitung. 

\subparagraph{Langzeitformate} 3D-Inhalte sollten in einem offen dokumentierten, textbasierten Format (ASCII) gespeichert werden. Dies ermöglicht bei Bedarf die Rückentwicklung unabhängig von einem bestimmten Programm.

Das zu verwendende Format hängt von der Entstehungsweise und dem Zweck des 3D-Inhaltes ab, da unterschiedliche Formate unterschiedliche Eigenschaften und Elemente speichern, wie beispielsweise Geometrie, Texturen, Lichtquellen oder Standpunkt und Bildausschnitt (auch Viewport genannt). Eine Übersicht über die Speichereigenschaften von den hier empfohlenen 3D-Formaten wird in Tabelle \ref{tab:3Deigenschaften} auf Seite \pageref{tab:3Deigenschaften} gegeben.

Das vom Web3D Consortium entwickelte Format X3D (eXtensible 3D Graphics) ist seit 2006 unter ISO/IEC 19775/19776/19777 zertifiziert und eignet sich sowohl zur Speicherung von einzelnen 3D-Modellen, als auch komplexer 3D-Inhalte, wie etwa Virtual Reality. Es ist das Nachfolgeformat von dem seit 2007 unter ISO/IEC 14772-1 zertifizierten VRML-Format und sollte diesem daher vorgezogen werden. 
	
Ein weiteres für komplexe 3D-Inhalte geeignetes Format ist das von der Khronos Group entwickelte COLLADA (collaborative design activity, DAE), das seit 2012 unter ISO 17506 standardisiert ist.

Das U3D-Format bietet einen ähnlichen Funktionsumfang wie X3D und COLLADA und ist insbesondere für das Teilen von 3D-Inhalten in PDF"=Dokumenten gedacht. Es ist nicht für die Langzeitarchivierung geeignet. 

Die Formate OBJ, PLY und STL eignen sich nicht für komplexe Szenen mit Lichtquellen oder gar Animationen, bieten aber alle Eigenschaften, um die visuellen Oberflächeneigenschaften eines 3D-Objektes zu speichern.

Aus dem CAD-Bereich stammt das Format DXF, welches neben 2D-Inhalten auch 3D-Inhalte speichern kann. Dieses Format sollte nur verwendet werden, wenn die 3D-Inhalte mit CAD-Software erstellt wurden. In der Industrie werden als alternative Formate auch IGES und STEP verwendet.

Nicht für die Langzeitarchivierung geeignet sind programmspezifische oder binäre Formate, wie beispielsweise FBX, 3DS, MAX, SKP, BLEND, PRC oder NXS.

Um zukünftigen Nutzern einen schnellen Überblick über die 3D-Inhalte zu bieten, ist die zusätzliche Speicherung von Bild- oder Videodateien, die einen ersten Eindruck des Modells vermitteln, zu empfehlen.

Hinweis: Bei der Konvertierung von einem 3D-Format in ein anderes können bestimmte Informationen verloren gehen, wenn sie von dem Zielformat nicht unterstützt werden. Zusätzlich zu dem gewählten Archivierungsformat sollten die originalen Quelldateien, verwendete Texturdateien, Visualisierungen und alle weiteren Dateien, die für die Benutzung und das Verständnis des 3D-Modells relevant sind aufgehoben werden. In den entsprechenden Abschnitten in dem Kapitel Forschungsmethoden ab Seite \pageref{methoden} sind nähere Details zu finden.

Die Abkürzungen in der folgenden Tabelle geben die unterstützten Eigenschaften der 3D-Formate an: DG~= Drahtgittermodell, P~= Parametrisch, F~= Farbe, X~= Textur mittels Bild, B~= Bumpmapping, M~= Material, V~= Viewport und Kamera, L~= Lichtquellen, T~= Transformationen, G~= Gruppierung. Diese Begriffe werden in dem Abschnitt Vertiefung ab Seite \pageref{3d:vertiefung} erläutert.

\begin{center}
	\begin{longtable}{l L{0.2\textwidth} p{0.6\textwidth}}
			\toprule 
		\multicolumn{2}{l}{Format} & Begründung \\
		\midrule \endfirsthead
		\multicolumn{3}{l}{\footnotesize Fortsetzung der vorhergehenden Seite}\\
		\toprule
		\multicolumn{2}{l}{Format} & Begründung \\ \midrule \endhead
		\bottomrule \multicolumn{3}{r}{{\footnotesize Fortsetzung auf der nächsten Seite}} \\
		\endfoot
		\bottomrule 
		\endlastfoot
		
		\multirow{4}{*}{\color{ForestGreen} \LARGE \checkmark} & X3D & Das X3D-Format wurde vom Web3D Consortium entwickelt und ist seit 2006 ISO-zertifiziert. Es darf nicht mit dem proprietären Format 3DXML verwechselt werden. Speichert: DG, P, F, X, B, M, V, L, T, G und Animationen.\\
		  & COLLADA & COLLADA (\emph{.dae}) wurde von der Khronos Group entwickelt und ist seit 2012 ISO-zertifiziert. Es basiert auf XML und speichert: DG, P, B-Rep, F, X, B, M, V, L, T, G und Animationen. \\
		  & OBJ & Das offene OBJ-Format wurde von Wavefront Technologies entwickelt und hat eine weite Verbreitung gefunden. Material oder Texturen werden in gesonderten MTL- oder JPG-Dateien gespeichert, die auch archiviert werden müssen. Speichert: DG, P, F, X, B, M und G.\\ 
		  & PLY & Das Polygon File Format (acuh Stanford Triangle Format) ist ein einfaches Format, das an der Universität Stanford für Daten von 3D-Scannern entwickelt wurde. Mittels Erweiterungen kännten weitere Eigenschaften gespeichert werden, die aber nicht von jedem Programm unterstützt werden. Für die Langzeitarchivierung sollte die ASCII-Version verwendet werden. Speichert: DG, F, X, B und M.\\
		%\cmidrule(r){1-3}
		\multirow{3}{*}{$\color{BurntOrange} \thicksim$} & VRML & Die Virtual Reality Modeling Language ist das Vorgängerformat von X3D. Die jüngste Version wurde 1997 unter dem Namen VRML97 veröffentlicht. Speichert: DG, P, F, X, B, M, V, L, T, G und Animationen.\\ 
		  & STL & Das Format STL wurde von 3D Systems entwickelt. Es steht für \emph{stereolithography} oder \emph{Standard Tessellation Language} und findet weite Verbreitung im Bereich von 3D-Druckern und digitaler Fabrikation. Die ASCII-Variante des STL-Formats kann nur die Geometrie eines 3D-Modells speichern. Die Binärvariante des Formates ist weniger speicherintensiv und kann mit einer entsprechenden Erweiterung auch Farben des 3D-Modells speichern, ist aber nicht für die Langzeitarchivierung geeignet.\\
		  & DXF & Das von Autodesk entwickelte Drawing Interchange Format stammt aus dem CAD-Bereich und sollte nur für 3D-Inhalte verwendet werden, die mit CAD-Software erstellt wurden. Speichert: DG, P, CSG, B-Rep, F und G.\\
		\cmidrule(r){1-3}
		\multirow{8}{*}{\LARGE \boldmath$\color{BrickRed} \times$}& U3D & Das Universal 3D Format ist ein 2005 von der Ecma standardisiertes 3D-Format, das vom 3D Industry Forum mit Mitgliedern wie Intel und Adobe Systems entwickelt wurde. Dieses Format ist nur für die Integration von 3D-Modellen in ein 3D-PDF relevant. Speichert: DG, F, X, B, V, L, T, G und Animationen.\\ 
		 & FBX & Ein proprietäres Format von Autodesk, für den Datenaustausch mit anderen kommerziellen 3D-Programmen.\\
		 & 3DS, MAX  & Proprietäre binäre Formate von AutoDesk.\\
		 & SKP & Natives Format von Google SketchUp.\\
		 & BLEND & Natives binäres Format von Blender.\\
		 & PRC & PRC (Product Representation Compact) ist ein stark komprimiertes Format für den CAD-Bereich und für die Verwendung in 3D-PDFs, das unter ISO 14739-1 standardisiert ist.\\
		 & PDF & 3D-PDFs mit eingebetteten Modellen in U3D oder PRC eignen sich für einen unkomplizierten Datenaustausch, jedoch nicht für die Langzeitarchivierung. \\
		 & NXS & Nexus ist ein von CNR-ISTI offen entwickeltes Format, für die Web-Visualisierung von 3D-Modellen.\\
		\bottomrule    
	\end{longtable}
\end{center}


%PRC is a highly compressed format that facilitates the storage of different representations of a 3D model. For example, you can save only a visual representation that consists of polygons (a tessellation), or you can save the model's exact geometry (B-rep data). Varying levels of compression can be applied to the 3D CAD data when it is converted to the PRC format using the 3D PDF Converter plugin for Acrobat. The 3D data stored in PRC format in a PDF is interoperable with Computer-Aided Manufacturing (CAM) and Computer-Aided Engineering (CAE) applications.


\subparagraph{Dokumentation} Für die Dokumentation von 3D-Inhalten sollten die Leitsätze der \emph{\href{www.londoncharter.org}{Londoner Charta}} berücksichtigt und eingehalten werden. Dabei ist insbesondere \emph{"`Leitsatz 4: Dokumentation"'} wichtig, dessen erster Absatz lautet:  

"`\emph{Es sollen genügend Informationen dokumentiert und weitergegeben werden, um das Verstehen und Bewerten der computergestützten Visualisierungsmethoden und -ergebnisse in Bezug auf die Zusammenhänge und Absichten, für die sie eingesetzt werden, zu ermöglichen}"'

Dazu gehören unter anderem die Dokumentation der Forschungsquellen, die zur Erstellung des 3D-Modells herangezogen wurden, die Dokumentation der Prozesse, die im Verlauf der Modellerstellung durchlaufen wurden, die Dokumentation der angewendeten Methoden und eine Beschreibung der Abhängigkeitsverhältnisse zwischen den unterschiedlichen Bestandteilen. 

Wenn das Dateiformat es erlaubt, sollte ein Teil der Metadaten zusätzlich dort integriert werden. Allerdings kann der Großteil der Metadaten nur extern abgelegt werden.

Die hier angegebenen Metadaten sind als minimale Angabe zu betrachten und ergänzen die angegebenen Metadaten für Projekte und Einzeldateien in dem Abschnitt Metadaten in der Anwendung ab Seite \pageref{Metadaten-anwendung}.

\begin{center}
	\begin{longtable}{L{0.3\textwidth} p{0.6\textwidth}}
			\toprule 
		Metadatum & Beschreibung \\
		\midrule \endfirsthead
		\multicolumn{2}{l}{\footnotesize Fortsetzung der vorhergehenden Seite}\\
		\toprule
		Metadatum & Beschreibung \\ \midrule \endhead
		\bottomrule \multicolumn{2}{r}{{\footnotesize Fortsetzung auf der nächsten Seite}} \\
		\endfoot
		\bottomrule 
		\endlastfoot
		
		Anzahl Eckpunkte & Aus wie vielen Eckpunkten besteht das 3D-Modell? \\
		Anzahl Polygone & Wie viele Polygone hat das 3D-Modell?  \\
		Geometrietyp & Welcher Geometrietyp wird verwendet (Drahtgittermodell, parametrisch, CSG, B-Rep etc.)? \\
		Maßstab & Welcher Maßstab liegt vor bzw. was stellt eine Einheit dar? \\
		Koordinatensystem & Wird ein geographisches Koordinatensystem oder ein beliebiges verwendet? \\
		Modellstatus & Handelt es sich bei dem Modell um das ursprünglich erzeugte und unverarbeitete Modell (den Master) oder ist es ein Modell, das mittels weiterverarbeitenden Schritten, wie Füllen von Löchern, Vereinfachung oder Glättung aus dem Master erzeugt wurde?\\
		Detaillierungsgrad (LOD) oder Auflösung & Wie detailliert ist das Modell oder welche Auflösung wurde beim 3D-Scan verwendet? \\
		Ebenen & Werden Ebenen verwendet? Wie viele? \\
		Farbe und Textur & Werden Farben oder Texturen verwendet? Wie werden diese gespeichert? Müssen zusätzliche Texturdateien archiviert werden?\\
		Material & Informationen über die Materialeigenschaften des Modells und inwieweit sie physikalisch korrekt sind. \\
		Licht & Informationen über die Lichtquellen und inwieweit sie physikalisch korrekt sind.\\
		Shader & Werden spezielle, erweiterte Shader verwendet? \\
		Animation & Ist das Modell animiert? Welche Art von Animation wird verwendet (Keyframe, motion capture)? \\
		Externe Dateien & Eine Liste aller externen Dateien, die für das Öffnen der Datei notwendig sind (z.B. Texturen)\\

	  \bottomrule
	\end{longtable}
\end{center}
Weitere Metadaten sind methodenabhängig und können in den jeweiligen Abschnitten nachgelesen werden.
	\begin{flushleft}
		Das vollständige Kapitel finden Sie auf: \urllist{https://ianus-fdz.de/it-empfehlungen/3d}
		\label{tab:3Deigenschaften}
		\label{3d:vertiefung}
    Verweis 3D und Virtual Reality -- Vertiefung: \urllist{https://ianus-fdz.de/it-empfehlungen/3d?qt-3d_und_virtual_reality=1}
		Verweis 3D und Virtual Reality -- Vertiefung, Speicherung der verschiedenen Eigenschaften: \urllist{https://ianus-fdz.de/it-empfehlungen/3d?qt-3d_und_virtual_reality=1\#3d_tab-speichereigenschaften}
	\end{flushleft}
	

%\section{Präsentationen}

\newpage
\section{Webseiten}
\abschnittsautor{D. Hagmann}
	\hyphenation{
}

Eine Webseite stellt eine Ressource aus strukturiertem Text im World Wide Web (WWW) dar und besteht in ihrer einfachsten Form aus einer HTML-Datei. Sie kann via Hyperlinks mit beliebig vielen weiteren Ressourcen vernetzt sein. Webseiten sind ein integraler Bestandteil des WWW im Internet. In der Regel ist eine Webseite Teil einer Website, bzw. eines Webauftrittes, also eines zusammengehörenden Paketes von miteinander vernetzten Webseiten und weiteren Ressourcen.

In der altertumswissenschaftlichen Forschung können Webseiten für die Öffentlichkeit zugängliche Informationen beinhalten, wie etwa Blogbeiträge oder ausführliche Projektbeschreibungen.

Der vorliegende Artikel beschäftigt sich vornehmlich mit der Archivierung einzelner Webseiten und nicht mit der Archivierung ganzer Websitesysteme. Um umfangreiche Websites mit mehreren Webseiten zu archivieren, empfehlen sich Online-Speicherdienste, spezialisierte Internetarchive oder dezidierte Softwarelösungen. 

\subparagraph{Langzeitformate} 
Webseiten können archiviert werden, wenn die nötigen Nutzungsrechte der Inhalte vorliegen. Die Archivierung kann dabei auf verschiedene Arten erfolgen:
\begin{itemize}
	\item Speicherung als statisches PDF mit zugehörigen Ressourcen
	\item lokale Speicherung als HTML oder MHTML mit zugehörigen Ressourcen
	\item Speicherung bei einem spezialisierten Internetarchiv
\end{itemize}

Als Optimum ist die Langzeitarchivierung einer Webseite in einer Form, die möglichst wenig Informationsverlust garantiert und einfach umzusetzen ist, anzustreben. Webseiten bestehen zum einen aus mindestens einer strukturierten HTML-Datei, zum anderen aus beliebig vielen via Hyperlinks mit der HTML-Datei verbundenen Ressourcen, die teilweise auf demselben Webserver gespeichert sind, aber auch von jedem anderen Ort im Internet bezogen werden können. Dies stellt jedoch nur den theoretisch sehr einfachen Aufbau dar, praktisch bestehen Webseiten aus einer Vielzahl an weiteren strukturierten Textdateien, die etwa das Design der Website regeln (CSS-Dateien) und können zudem über verschiedenste von anderen Websites bezogene und auf der Webseite eingebettete Inhalte (Videos, 3D-Modelle, interaktive Karten etc.) verfügen. Eine der Hauptintentionen jedweder Webseite ist es verschiedene Informationen nach einem vorgegebenen Design in einer bestimmten Abfolge und einem bestimmten Layout dem Nutzer zu vermitteln, vergleichbar zu einer gedruckten Seite in einem Buch. Bei der Archivierung muss beachtet werden, dass für Webseiten teilweise Dateiformate verwendet werden,die für die Langzeitarchivierung dezidiert ungeeignet sind, etwa JPEGs. 

Generell empfiehlt es sich, die der Webseite zugrundeliegenden Daten (z.B. Text und Bilder) als Einzeldateien jeweils separat in einem geeigneten Archivformat zu archivieren. Nähere Informationen zu den Archivierungsformaten sind in den entsprechenden Kapiteln zu finden. Auch werden nicht alle multimedialen Inhalte, Webanwendungen (z.B. Web-GIS) oder über externe Dienste eingebettete Inhalte mit jeder Archivierungsmethode gespeichert, weshalb in solchen Fällen besonderes Augenmerk auf die Auswahl der zu verwendenden Methode gelegt werden muss.

Eine Webseite kann als PDF mit Hilfe des Webbrowsers und eines PDF-Generators als PDF-Datei gespeichert und anschließend mit entsprechender Software in ein archivierbares PDF/A-Datei konvertiert werden. Informationen dazu finden sich im Abschnitt PDF-Dokumente ab Seite \pageref{pdf-dokumente}. Diese Methode führt praktisch immer zu Änderungen des ursprünglichen Layouts. Jedoch können mittels Plug-ins im Webbrowser oder bestimmten Softwareprogrammen Webseiten unter großteiliger Wahrung des Layouts als PDF gespeichert werden. Abschließend muss auch hier eine Konvertierung in das PDF/A-Format vorgenommen werden. Multimediale Inhalte (Videos, 3D-Objekte etc.) werden mit dieser Methode nicht gespeichert.

Die lokale Speicherung einer Webseite aus dem WWW mittels eines Webbrowsers ist einfach möglich und wird durch alle aktuellen Webbrowser unterstützt. Hierbei gilt es jedoch, bestimmte Speicherformate zu beachten, da nicht alle in den Webbrowsern verfügbaren Formate für die Archivierung geeignet sind. Für die Strukturierung und Formatierung von Webseiten werden üblicherweise die Hypertext Markup Language (HTML) oder die  Extensible Hypertext Markup Language (XHTML), sowie Cascading Style Sheets (CSS) verwendet. Es handelt sich dabei um Standards, die vom W3C entwickelt und empfohlen werden, weshalb diese in den Versionen HTML5, XHTML5 und CSS 3 für die Archivierung empfohlen werden können.

Es bietet sich hier also die Möglichkeit der Speicherung der Webseite als HTML- oder XHTML-Datei an. HTML-Dateien (und XHTML) archivieren den strukturierten Text und die Hyperlinks, jedoch nicht die verknüpften Ressourcen (z.B. Bilder, multimediale Inhalte oder externe Inhalte), zudem wird hierdurch nicht das Design der Webseite, welches durch CSS geregelt wird, übernommen, da die entsprechenden Dateien nicht gespeichert werden. Um auch die verknüpften und für das Design benötigte Ressourcen zu speichern, können diese automatisch in einen zusätzlichen lokalen Ordner geladen werden. In der Regel handelt es sich dabei um HTML/XHTML- und CSS-Dateien, Grafiken, JavaScript-Dateien sowie gegebenenfalls Java-Applets und Multimedia-Dateien. 

Die lokale Speicherung einer Webseite in einer einzigen Datei wird mittels MIME HTML (MHTML) ermöglicht. Es handelt sich um ein textbasiertes Format, das in  RFC 2557 spezifiziert wird. In der Regel werden bei MHTML-Dateien das Layout und alle Hyperlinks vollständig übernommen. Auch hier muss das Speichern von eingebetteten Inhalten gegebenenfalls gesondert vorgenommen werden.

Das offen dokumentierte Mozilla Archive Format (MAFF) ermöglicht ebenfalls die Speicherung einer Webseite in Form einer einzelnen Datei. Dabei werden die einzelnen Bestandteile in einem ZIP-Container gespeichert. Da dieses Format aktuell nur von Mozilla Firefox unterstützt wird, sollte für die Archivierung jedoch ein anderes Format vorgezogen werden. Ähnlich verhält es sich mit dem Format Webarchive, das derzeit jedoch nur durch Appels Safari Webbrowser unterstützt wird und daher nicht empfohlen werden kann.

Auch HTML-Dateien mit Data-URIs ermöglichen die Speicherung einer gesamten Webseite meist unter Beibehaltung des Layouts in einer einzigen Datei. Data-URIs ermöglichen es, Ressourcen in HTML einzubetten und sind in RFC 2397 definiert. Es handelt sich dabei um eine spezielle Syntax, mit der binäre Daten als ASCII-Zeichenketten kodiert werden. Da Ressourcen als Data-URIs, wie beispielsweise Bilder,  direkt und in menschenunlesbarer Form in die Datei integriert werden, können diese nicht nachgenutzt werden, weshalb von einer Speicherung als HTML mit Data-URIs für die Archivierung abgesehen werden sollte.

Eine andere häufig praktizierte, jedoch eindeutig nicht empfohlene Möglichkeit, stellt die Speicherung von Webseiten in der Form von Screenshots dar. Screenshots werden in der Regel im PNG- oder JPEG-Format gespeichert. Dies hat drei Nachteile: (1) Die Konvertierung erfolgt oft in das JPEG-Format, das zur Archivierung nicht geeignet ist. (2) Die Speicherung als Rastergrafik kann in manchen Fällen aufgrund einer zu niedrigen Auflösung zu Qualitätsverlusten führen. Außerdem wird Text nicht mehr als solcher erkannt und gespeichert. (3) Die Konvertierung der Webseite in eine Grafik führt dazu, dass sämtliche Hyperlinks desintegriert werden.

Es besteht zwar hinsichtlich der Punkte (1) und (2) die Möglichkeit, mit entsprechender Software eine Texterkennung und anschließende Speicherung als PDF/A durchzuführen, jedoch können hinsichtlich Punkt (3) dadurch keine Hyperlinks wiederhergestellt werden. 

Ein anderer Ansatz ist die Archivierung einer Webseite über einen spezialisierten Archivierungsdienst. Solche werden etwa durch die Bayerische Staatsbibliothek (mit Anmeldung) oder die Organisation Internet Archive angeboten. Hier werden die Webseiten auf einem Server des Archivierungsdienstes gespeichert und können auf diesen Plattformen wiederum über das WWW abgerufen werden. Diese Dienste sind auch zur Archivierung ganzer Websites geeignet. Für die Archivierung ganzer Websites gibt es das Format Web ARChive (WARC), das seit 2009 als ISO 28500 standardisiert ist und von dem International Internet Preservation Consortium aufbauend auf dem Format ARC entwickelt wurde. In einer WARC-Datei werden alle Seiten, Ressourcen und weitere Komponenten einer Website gespeichert.

Hinweis: Angaben zur Archivierung von Programmen in JavaScript sowie Java (Java-Applets) finden sich im Kapitel Eigene Programme und Skripte, Ausführungen zu multimedialen Inhalten (z.B. 3D-Objekte, Audio oder Video) in den entsprechenden Kapiteln. 

\begin{center}
	\begin{longtable}{l L{0.2\textwidth} p{0.6\textwidth}}
			\toprule 
		\multicolumn{2}{l}{Format} & Begründung \\
		\midrule \endfirsthead
		\multicolumn{3}{l}{\footnotesize Fortsetzung der vorhergehenden Seite}\\
		\toprule
		\multicolumn{2}{l}{Format} & Begründung \\ \midrule \endhead
		\bottomrule \multicolumn{3}{r}{{\footnotesize Fortsetzung auf der nächsten Seite}} \\
		\endfoot
		\bottomrule 
		\endlastfoot
		
		\multirow{4}{*}{\color{ForestGreen} \LARGE \checkmark} & PDF/A-1, PDF/A-2 & PDF/A ist gezielt als stabiles, offenes und standardisiertes Format für die Langzeitarchivierung unterschiedlicher Ausgangsdateien entwickelt worden. \\
			& HTML und XHTML & HTML- und XHTML-Dateien können als streng strukturierte Textdokumente, die vom W3C standardisiert sind, problemlos archiviert werden. Die Datei sollte wohlgeformt und in UTF-8 ohne BOM kodiert sein. Es sollte möglichst HTML5 verwendet werden. Zusätzliche Dateien, wie CSS, JavaScript oder andere strukturierte Textformate müssen ebenfalls archiviert werden. Eingebettete Ressourcen müssen gesondert archiviert werden.\\
			& MHTML & MHTML-Dateien können als strukturierte Textdokumente mit genauen Spezifikationen für die Archivierung verwendet werden. Die Archivierung von eingebetteten Inhalten muss gegebenenfalls gesondert erfolgen.\\
			& WARC & Web ARChive ist als ISO 28500 standardisiert und dient als Containerformat für mehrere Webseiten einer Website.\\ \cmidrule(r){1-3}
		\multirow{2}{*}{$\color{BurntOrange} \thicksim$} & MAFF & Das Format ermöglicht die Speicherung einer ganzen Webseite samt aller zugehöriger Ressourcen komprimiert und verlustfrei in einem ZIP Container und eignet sich zur Langzeitarchivierung, solange die einzelnen Ressourcen selbst in archivfähigen Formaten vorliegen und Hyperlinks entsprechend aktualisiert werden. \\
			& HTML mit Data URIs & HTML-Dateien können als strukturierte Textdokumente, die weit verbreiteten Konventionen folgen und aufgrund der integrierten DTD, die die verwendete Struktur beschreibt, problemlos archiviert werden. Data URIs sind ebenso spezifiziert. \\ \cmidrule(r){1-3}
		\multirow{3}{*}{\LARGE \boldmath$\color{BrickRed} \times$}& andere PDF-Varianten & Viele gängige PDF-Varianten sind nicht für die Langzeitarchivierung geeignet. Stattdessen sollten entweder die Ausgangsdateien in einem passenden Format archiviert oder eine Migration in ein PDF/A-Format vorgenommen werden.\\
			& Screenshots & Screenshots eignen sich nur für die Dokumentation der Optik der Webseite, jedoch nicht für die Archivierung der Inhalte, da diese als Rastergrafik gespeichert werden und so kaum nachnutzbar sind. \\
		  & Webarchive & Ist ein Format von Apple, das derzeit auch nur von Safari unterstützt wird. Es ist nicht für die Archivierung geeignet.\\ 
 		\bottomrule    
	\end{longtable}
\end{center}


\subparagraph{Dokumentation} 
HTML, XHTML und MHTML verfügen über einen eigenen Dokumentenkopf, in dem verschiedene Metadaten eingebettet werden können. Es sollten Angaben zur verwendeten Zeichenkodierung, dem Titel des Dokumentes, dem/der AutorIn sowie Stichwörter gemacht werden. Ergänzende Metadaten können zusätzlich mit Hilfe eines Kommentars in den Kopfdaten der Datei eingefügt werden. Auch in CSS-Dateien können Metadaten als Kommentar eingetragen werden.
 
Die hier angegebenen Metadaten sind als minimale Angabe zu betrachten und ergänzen die angegebenen Metadaten für Projekte und Einzeldateien in dem Abschnitt Metadaten in der Anwendung.

\begin{center}
	\begin{longtable}{L{0.3\textwidth} p{0.6\textwidth}}
			\toprule 
		Metadatum & Beschreibung \\
		\midrule \endfirsthead
		\multicolumn{2}{l}{\footnotesize Fortsetzung der vorhergehenden Seite}\\
		\toprule
		Metadatum & Beschreibung \\ \midrule \endhead
		\bottomrule \multicolumn{2}{r}{{\footnotesize Fortsetzung auf der nächsten Seite}} \\
		\endfoot
		\bottomrule 
		\endlastfoot

Titel & Titel der Webseite \\
Kurzbeschreibung & Kurze Beschreibung des Inhaltes. \\
Stichwörter & Schlagworte, die den Inhalt beschreiben. \\
Autor & Name des Verfassers oder Erstellers der Datei. \\
Erstellungsdatum & Datum der Erstellung der Datei, also der Archivierung der Webseite. \\
Bearbeitungsdatum & Datum der letzten Bearbeitung der Webseite. \\
Abschaltung Webserver & Datum an dem die Webseite zum letzten Mal online verfügbar war \\
URI & Internetadresse der archivierten Webseite \\
Identifikator & Wenn das Dokument bereits veröffentlicht wurde und einen Persistent Identifier erhalten hat, sollte dieser angegeben werden. \\
Sprache & Angabe der im Dokument verwendeten  Sprachen. Sprachkennungen nach ISO639 angeben. \\
Rechte & Details zum Urheberrecht. \\
Standard & Name und Version des verwendeten Standards, z.B. HTML5 und CSS 3. \\
Zeichenkodierung & Angabe der verwendeten Zeichenkodierung, z.B. UTF-8 ohne BOM. \\
Beziehungen & Dateien oder Ressourcen, die mit der Datei zusammenhängen, wozu auch frühere Versionen gehören. Bei der Archivierung einer Website mit mehreren Webseiten müssen die Beziehungen der einzelnen Seiten untereinander dokumentiert werden, beispielsweise mit einer Sitemap. \\
Versionsnummer & Angabe der Dateiversion, bezogen auf den Inhalt. z.B. 1.3. \\
Software & Name und Version der für die Archivierung der Seite verwendeten Programme \\
weitere Dateien & Liste von eingebetteten Medien, die zusätzlich separat gespeichert wurden. Liegt eine Dokumentationsdatei für das Dokument vor, muss diese ebenfalls genannt werden. \\
  \bottomrule
	\end{longtable}
\end{center}
	\begin{flushleft}
		Das vollständige Kapitel finden Sie auf: \urllist{https://ianus-fdz.de/it-empfehlungen/webseiten}
	\end{flushleft}
	
%\section{Eigene Programme und Skripte}


%Forschungsmethoden
\chapter{Weitere Informationen}
	\abschnittsautor{M. Trognitz}\vspace{-0.5cm}
	\label{methoden}
	\section{Forschungsmethoden}
Die Ergebnisse aus verschiedenen in den Altertumswissenschaften angewandten Methoden setzen sich oft aus mehreren verschiedenen Dateien unterschiedlicher Formate zusammen. Dadurch gehen die Ansprüche an die Dokumentation und die erforderlichen Metadaten zu den einzelnen Dateiformaten über die des Kapitels zu Dateiformaten hinaus und unterscheiden sich von Methode zu Methode.

Zu folgenden Methoden finden Sie weitere Informationen auf: \url{https://ianus-fdz.de/it-empfehlungen/forschungsmethoden}

Geodäsie · Georeferenzierung · Oberflächen- (DOM) und Geländemodellierung (DGM) · Reflectance Transformation Imaging (RTI) · Satellitenmessungen

Die vorhandenen Inhalte decken nur einen kleinen Teil der angewandten Forschungsmethoden in den Archäologien und Altertumswissenschaften ab. Weitere Inhalte sind geplant, wobei dafür auch Autoren gesucht werden. Die folgende Auflistung soll einen Eindruck der möglichen Inhalte vermitteln, wobei weitere Vorschläge und Ergänzungen willkommen sind.

3D-Scanning · Anthropologie · Archäobotanik · Archäometrie · Archäozoologie · Ausgrabung · Bauforschung · Datierungsmethoden · Geoarchäologie · Numismatik · Oberflächenbegehung (Survey) · Textanalyse

Bis wir Ihnen ausführlichere Hinweise zur Verfügung stellen können, können Sie sich auf den Seiten des \href{http://guides.archaeologydataservice.ac.uk/g2gp}{Archaeology Data Service informieren}, wo insbesondere folgende Inhalte zu Forschungsmethoden zur Verfügung stehen.

Dendrochronolgie · Fernerkundung · GIS · Geophysik · Laserscanning · Photogrammetrie · UAV Survey · Unterwassersurvey


\section{Weitere Verweise}
Weitere Informationen zur Relevanz und Kuratierung von digitalen Forschungsdaten in den Altertumswissenschaften sind:
\begin{itemize}
	\item Archaeology Data Service, Guides to Good Practice: \urllist{http://guides.archaeologydataservice.ac.uk/g2gp} \vspace{-0.5cm}
	\item Verband der Landesarchäologen, Ratgeber-Archivierung: \urllist{http://www.landesarchaeologen.de/fileadmin/Dokumente/Dokumente_Kommissionen/Dokumente_Archaeologie-Informationssysteme/Dokumente_AIS_Archivierung/Ratgeber_Archivierung_V1.0.pdf}
\end{itemize}

\end{document}